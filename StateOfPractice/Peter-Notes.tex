\documentclass{article}

\usepackage{tabularx}
\usepackage{booktabs}
\usepackage{hyperref}

\title{Project Notes}

\author{Peter Michalski}

\date{}


\begin{document}

\maketitle

~\newpage


\section{List of LBM solutions}
\begin{enumerate}
	\item aromanro/LatticeBoltzmann\\
	\href{https://github.com/aromanro/LatticeBoltzmann}{https://github.com/aromanro/LatticeBoltzmann}\\
	Found from list at \href{https://github.com/topics/lattice-boltzmann}{https://github.com/topics/lattice-boltzmann}\\
	A 2D Lattice Boltzmann program
	
	\item Atruszkowska/LBM\_MATLAB\\
	\href{https://github.com/atruszkowska/LBM_MATLAB}{https://github.com/atruszkowska/LBM\_MATLAB}\\
		Found from list at \href{https://github.com/topics/lattice-boltzmann}{https://github.com/topics/lattice-boltzmann}\\
	MPI-style parallelized Shan and Chen LBM with multiscale modeling extension
	
	\item ch4-project\\
	\href{https://github.com/ecalzavarini/ch4-project}{https://github.com/ecalzavarini/ch4-project}\\  
	Found using Google search.\\
	Eulerian-Lagrangian fluid dynamics platform
	A general purpose Lattice-Boltzmann code for fluid-dynamics simulations. It includes :
	fluid dynamics (with several volume forcing terms for Channel flow, Homogeneous Isotropic Turbulence, buoyancy)
	temperature dynamics (advection, diffusion , sink/source or reaction terms)
	phase change (enthalpy formulation for solid/liquid systems)
	scalar transport (same functionalities as temperature)
	lagrangian dynamics (tracers, heavy/light and active point-like particles; non-shperical Jeffery rotation, gyrotaxis)
	large eddy simulation (Smagorinsky, Shear Improved Samgorinsky with Kalman Filter)
	
	\item CUDA-LBM-simulator\\
	\href{https://github.com/henryfriedlander/CUDA-LBM-simulator}{https://github.com/henryfriedlander/CUDA-LBM-simulator}\\
		Found from list at \href{https://github.com/topics/lattice-boltzmann}{https://github.com/topics/lattice-boltzmann}\\
	This is a Lattice-Boltzmann simulation using CUDA GPU graphics optimization.
	
	\item CudneLB (TCLB)\\
	\href{https://github.com/CFD-GO/TCLB}{https://github.com/CFD-GO/TCLB}\\
		Found from list at \href{https://github.com/topics/lattice-boltzmann}{https://github.com/topics/lattice-boltzmann}\\
	CudneLB is a MPI+CUDA or MPI+CPU high-performance CFD simulation code, based on Lattice Boltzmann Method.
	
	\item DL\_Meso \\
	\href{https://www.scd.stfc.ac.uk/Pages/DL_MESO.aspx}{https://www.scd.stfc.ac.uk/Pages/DL\_MESO.aspx} \\
	Found using Google search.\\
	DL\_MESO is a general purpose mesoscale simulation package developed by Michael Seaton for CCP5 and UKCOMES​ under a grant provided by EPSRC. It is written in Fortran 2003 and C++ and supports both Lattice Boltzmann Equation (LBE) and Dissipative Particle Dynamics (DPD) methods. It is supplied with its own Java-based Graphical User Interface (GUI) and is capable of both serial and parallel execution.
	
	\item ESPResSo
\\
	\href{http://espressomd.org/html/doc/index.html}{http://espressomd.org/html/doc/index.html}
\\
		Found from list at \href{https://github.com/topics/lattice-boltzmann}{https://github.com/topics/lattice-boltzmann}\\
	ESPResSo is a simulation package designed to perform Molecular Dynamics (MD) and Monte Carlo (MC) simulations. It is meant to be a universal tool for simulations of a variety of soft matter systems. It features a broad range of interaction potentials which opens up possibilities for performing simulations using models with different levels of coarse-graining. It also includes modern and efficient algorithms for treatment of Electrostatics (P3M, MMM-type algorithms, constant potential simulations, dielectric interfaces, …), hydrodynamic interactions (DPD, Lattice-Boltzmann), and magnetic interactions, only to name a few. It is designed to exploit the capabilities of parallel computational environments. The program is being continuously extended to keep the pace with current developments both in the algorithms and software.
	
	\item ESPResSo++
\\
	\href{http://www.espresso-pp.de/}{http://www.espresso-pp.de/}
\\
			Found from list at \href{https://github.com/topics/lattice-boltzmann}{https://github.com/topics/lattice-boltzmann}\\
	ESPResSo++ is a software package for the scientific simulation and analysis of coarse-grained atomistic or bead-spring models as they are used in soft matter research.\\
	ESPResSo++ has a modern C++ core and flexible Python user interface.\\
	ESPResSo and ESPResSo++ have common roots however their development is independent and they are different software packages.\\
	ESPResSo++ is free, open-source software published under the GNU General Public License (GPL).
	
	\item firesim \\ 
	\href{https://github.com/kynan/firesim}{https://github.com/kynan/firesim}
\\
	Found using Google search.\\
	Lattice Boltzmann Method (LBM) fluid solver driving a particle engine for the simulation and real-time visualization of fire
	
	\item fvLBM
\\
	\href{https://github.com/zhulianhua/fvLBM}{https://github.com/zhulianhua/fvLBM}
\\
	Found using Google search.\\
	finite volume lattice Boltzmann method
	
	\item HemeLB \\
	\href{https://github.com/UCL/hemelb}{https://github.com/UCL/hemelb}
\\
	Found using Google search.\\
	A software pipeline that simulates the blood flow through a stent (or other flow diverting device) inserted in a patient’s brain.
	
	\item JavaCFD
\\
	\href{https://github.com/SihaoHuang/JavaCFD}{https://github.com/SihaoHuang/JavaCFD}
\\
		Found from list at \href{https://github.com/topics/lattice-boltzmann}{https://github.com/topics/lattice-boltzmann}\\
	Computational fluid dynamics software written in Java using the Lattice-Boltzmann method. Allows custom-defined, arbitrary geometries in 2D incompressible flow field.
	
	\item JFlowSim \\
	\href{https://github.com/ChristianFJanssen/jflowsim}{https://github.com/ChristianFJanssen/jflowsim} \\
		Found from list at \href{https://github.com/topics/lattice-boltzmann}{https://github.com/topics/lattice-boltzmann}\\
	jFlowSim is an interactive, thread-parallel Lattice Boltzmann solver in two dimensions.
	
	\item laboetie
\\
	\href{https://github.com/maxlevesque/laboetie}{https://github.com/maxlevesque/laboetie}
\\
		Found from list at \href{https://github.com/topics/lattice-boltzmann}{https://github.com/topics/lattice-boltzmann}\\
	laboetie is a computational fluid dynamics code for chemical applications.
	It uses the Lattice-Boltzmann algorithm.
	
	\item LatBo.jl \\
	\href{https://github.com/UCL/LatBo.jl}{https://github.com/UCL/LatBo.jl}
\\
	Found using Google search.\\
	Lattice-Boltzmann implementation in Julia
	
	\item lettuce
\\
	\href{https://github.com/Olllom/lettuce}{https://github.com/Olllom/lettuce}
\\
	Found using Google search.\\
	GPU-acclerated Lattice Boltzmann in Python
	
	\item LB2D\_Prime
\\
	\href{http://faculty.fiu.edu/~sukopm/LBnD_Prime/LBnD_Prime.html}{http://faculty.fiu.edu/~sukopm/LBnD\_Prime/LBnD\_Prime.html} \\
	Found using Google search.\\
	LB2D\_Prime is a lattice Boltzmann (LB) code capable of simulating single and multi-phase flows and solute/heat transport in geometrically complex domains.
	
	\item LB3D \\ 
	\href{http://ccs.chem.ucl.ac.uk/lb3d}{http://ccs.chem.ucl.ac.uk/lb3d}
\\
	Found using Google search.\\
	A parallel implementation of the Lattice-Boltzmann method for simulation of interacting amphiphilic fluids. LB3D provides functionality to simulate three-dimensional simple, binary oil/water and ternary oil/water/amphiphile fluids using the Shan-Chen model for binary fluid interactions.
	
	\item LBDEMcoupling-public
\\
	\href{https://github.com/ParticulateFlow/LBDEMcoupling-public}{https://github.com/ParticulateFlow/LBDEMcoupling-public}
\\
	Found using Google search.\\
	Coupling between the Lattice-Boltzmann code Palabos and the DEM code LIGGGHTS
	
	\item LBSim
\\
	\href{https://github.com/noirb/lbsim}{https://github.com/noirb/lbsim}
\\
	Found using Google search.\\
	A small and simple Lattice-Boltzmann Method fluid simulator supporting complex boundaries.
	
	\item Limbes
\\
	\href{https://code.google.com/archive/p/limbes/}{https://code.google.com/archive/p/limbes/} \\
	Found using Google search.\\
	Open source (GPL) code in 2D based on Gauss-Hermite quadrature, parallel (openmp), fortran 90. LIMBES is the recursive acronym for LIMBES Is May be a Boltzmann Equation Solver. Version 1.0 solves numerically by a Lattice Boltzmann like method the BGK-Boltzmann equation for gas in two dimensions.
	
	\item listLBM
\\
	\href{https://github.com/sorush-khajepor/listLBM}{https://github.com/sorush-khajepor/listLBM} \\
		Found from list at \href{https://github.com/topics/lattice-boltzmann}{https://github.com/topics/lattice-boltzmann}\\
	ListLBM is a sparse lattice Boltzmann solver for multiphase flow in porous media
	
	\item LUMA
\\
	\href{https://github.com/ElsevierSoftwareX/SOFTX-D-18-00007}{https://github.com/ElsevierSoftwareX/SOFTX-D-18-00007}\\
		Found from list at \href{https://github.com/topics/lattice-boltzmann}{https://github.com/topics/lattice-boltzmann}\\
	LUMA: A many-core, Fluid–Structure Interaction solver based on the Lattice-Boltzmann Method
	
	\item MP-LABS
\\
	\href{https://github.com/carlosrosales/mplabs}{https://github.com/carlosrosales/mplabs}
\\
	Found using Google search.\\
	MP-LABS is a suite of numerical simulation tools for multiphase flows based on the free energy Lattice Boltzmann Method (LBM). The code allows for the simulation of quasi-incompressible two-phase flows, and uses multiphase models that allow for large density ratios. MP-LABS provides implementations that use periodic boundary conditions, but it is written in a way that allows for easy inclusion of different boundary conditions. The output from MP-LABS is in plain ASCII and VTK format, and can be analyzed using other Open Source tools such as Gnuplot and Paraview.
	
	The objective of the MP-LABS project is to provide a core set of routines that are well documented, highly portable, and have proven to perform well in a variety of systems. The source code is written in Fortran 90 and MPI and uses separate subroutines for most tasks in order to make modifications easier.
	
	\item Openlb \\  
	\href{https://www.openlb.net/}{https://www.openlb.net/}
\\
		Found from list at \href{https://github.com/topics/lattice-boltzmann}{https://github.com/topics/lattice-boltzmann}\\
	The OpenLB project provides a C++ package for the implementation of lattice Boltzmann methods that is general enough to address a vast range of tansport problems, e.g. in computational fluid dynamics. The source code is publicly available and constructed in a well readable, modular way. 
	
	\item Palabos \\
	\href{https://palabos.unige.ch/}{https://palabos.unige.ch/}
\\
	Found using Google search.\\
	The Palabos library is a framework for general-purpose computational fluid dynamics (CFD), with a kernel based on the lattice Boltzmann (LB) method. It is used both as a research and an engineering tool: its programming interface is straightforward and makes it possible to set up fluid flow simulations with relative ease, or, if you are knowledgeable of the lattice Boltzmann method, to extend the library with your own models. Palabos stands for Parallel Lattice Boltzmann Solver.
	The library’s native programming interface in written in C++. 
	
	\item pyLBM
\\
	\href{https://github.com/pylbm/pylbm}{https://github.com/pylbm/pylbm}
\\
		Found from list at \href{https://github.com/topics/lattice-boltzmann}{https://github.com/topics/lattice-boltzmann}\\
	pylbm is an all-in-one package for numerical simulations using Lattice Boltzmann solvers.
	This package gives all the tools to describe your lattice Boltzmann scheme in 1D, 2D and 3D problems.
	
	\item Sailfish \\
	\href{https://github.com/sailfish-team/sailfish}{https://github.com/sailfish-team/sailfish}
\\
	Found using Google search.\\
	Lattice Boltzmann (LBM) simulation package for GPUs (CUDA, OpenCL)
	
	\item siramirsaman/LBM
\\
	\href{https://github.com/siramirsaman/LBM}{https://github.com/siramirsaman/LBM}
\\
		Found from list at \href{https://github.com/topics/lattice-boltzmann}{https://github.com/topics/lattice-boltzmann}\\
	Lattice Boltzmann Method Implementation in MATLAB for Curved Boundaries
	
	\item SunlightLB \\
	\href{http://sunlightlb.sourceforge.net/}{http://sunlightlb.sourceforge.net/}
\\
	Found using Google search.\\
	SunlightLB is an open-source 3D lattice Boltzmann code which can be used to solve a variety of hydrodynamics problems, including passive scalar transport problems.
	
	\item Taxila-LBM
\\
	\href{https://github.com/ecoon/Taxila-LBM}{https://github.com/ecoon/Taxila-LBM}
\\
	Found using Google search.\\
	Taxila LBM is a parallel implementation of the Lattice Boltzmann Method for simulation of flow in porous and geometrically complex media.
	
	\item  loliverhennigh /
Lattice-Boltzmann-fluid-flow-in-Tensorflow 
\\
	\href{https://github.com/loliverhennigh/Lattice-Boltzmann-fluid-flow-in-Tensorflow}{https://github.com/loliverhennigh/Lattice-Boltzmann-fluid-flow-in-Tensorflow}
\\
	Found using Google search.\\
	A Lattice Boltzmann fluid flow simulation written in Tensorflow. 
	
	\item turbulent\_lbm\_multigpu
\\
	\href{https://github.com/arashb/turbulent_lbm_multigpu}{https://github.com/arashb/turbulent\_lbm\_multigpu}
\\
	Found using Google search.\\
	Lattice Boltzmann simulation of turbulent fluid flow on GPU Cluster
	
	\item waLBerla
\\
	\href{https://www.walberla.net/}{https://www.walberla.net/}
\\
	Found using Google search.\\
	waLBerla uses the lattice Boltzmann method (LBM), which is an alternative to classical Navier-Stokes solvers for computational fluid dynamics simulations. All of the common LBM collision models are implemented (SRT, TRT, MRT). Additionally, a coupling to the rigid body physics engine pe is available. 
	
	\item wlb
\\
	\href{https://github.com/weierstrass/wlb}{https://github.com/weierstrass/wlb}
\\
	Found using Google search.\\
	A Lattice-Boltzmann code for solving coupled equations in electrohydrodynamics. 
	Three collision operators are implemented for the (incompressible) Navier-Stokes, 
	Nernst-Planck (advection-diffusion) and Poission's equation for electrostatics 
	respectively. Various implementations of Dirichlet/Neumann boundary conditions 
	are also available. The code deals (so far) only with 2D systems.
	This code is part of a  master thesis project carried out at Chalmers University, 
	Gothenburg.
	
	\item Zmhhaha/LBM-Cplusplus-A.A.Mohamad
\\
	\href{https://github.com/zmhhaha/LBM-Cplusplus-A.A.Mohamad}{https://github.com/zmhhaha/LBM-Cplusplus-A.A.Mohamad} \\
		Found from list at \href{https://github.com/topics/lattice-boltzmann}{https://github.com/topics/lattice-boltzmann}\\
	The C++ version code of "Lattice Boltzmann Method Fundamentals and Engineering Applications with Computer Codes".
	
\end{enumerate}

~\newpage
\section{Variabilities}

Input Variabilities
\begin{enumerate}
	\item boundary parameters
	\item dimension
	\item number of velocity directions
\end{enumerate}

\noindent Calculation Variabilities
\begin{enumerate}
	\item computational model
	\item decomposition technique
	\item coefficient weights
	\item input check
	\item encoding of output
	\item exception check
\end{enumerate}

\noindent Other Variabilities
\begin{enumerate}
	\item language
	\item license
\end{enumerate}

~\newpage
\section{Quality Measurement}
\subsection{Robustness}
According to our definition, we must place the system in a state not assumed or anticipated in its requirements specification. Some ideas are providing invalid input, using the software on different hardware or OS software than required, or not providing (or deleting) additional libraries or directories. Such changes should of course be analyzed within reason - a different OS version as opposed to a fundamentally different OS, for example.\\ 

Provide invalid input and observe behaviour: Does the system crash - is it recoverable? Does the system prevent you from providing invalid input? Does the system return an error message?\\

Using software on different hardware/OS software than required (would need to make a reasonable list of variance): Ask similar questions as above. \\

Altering/removing/not providing external libraries and/or directories: Ask similar questions as above. \\

Furthermore, in regard to these questions, is software that crashes easily but provides an error more robust than a program that rarely crash but does not provide an error message?\\ 


We can additionally measure mean time to recovery from failures\\  


DOI: 10.1007/978-3-642-29032-9\_16:\\ 
The robustness failures are typically classified according to the CRASH criteria [540]: Catastrophic (the whole system crashes or reboots), Restart (the application has to be restarted), Abort (the application terminates abnormally), Silent (invalid operation is performed without error signal), and Hindering (incorrect error code is returned–note that returning a proper error code is considered as robust operation). The measure of robustness can be given as the ratio of test cases that exposes robustness faults, or, from the system’s point of view, as the number of robustness faults exposed by a given test suite.\\ 

\begin{enumerate}
	\item Injecting Random input 
	\item Using invalid input 
	\item Invalid inputs for each function in interface (valid and invalid inputs are type specific) 
	\item applying mutation techniques 
\end{enumerate}
\subsection{Performance}
\begin{enumerate}
	\item How many concurrent loads (this may also require hardware parallelism) - Try to run the multiple tests in one instance of the application (pass/fail)?
	\item How many application instances can be run? - Try to open and run multiple instances of the application (yes/no)?
	\item CPU usage (find a tool to monitor the applications use of CPU), calculate average over the run of tests. Compare CPU usage between solutions - keep in mind the solutions vary significantly.
	\item Memory usage - similar to CPU usage
	\item Average Response Time (compared to other solutions) - run a pre determined set of test
	\item Error rate (exceptions) - would need to automate tests
	\item MTBF - would need to automate tests
\end{enumerate}

\subsection{Maintainability}
How much time it takes to add a new function (requirement)\\ 

Metrics for Assessing a Software System's Maintainability (Oman, Hagemeister)\\ 

Software Metrics for Predicting Maintainability (Frappier, Matwin, Mili) 

\subsection{Reusability}
Measuring Software Reusability (Poulin)\\  

Chidamber and Kemerer object-oriented metrics:\\ 

https://www.aivosto.com/project/help/pm-oo-ck.html:\\ 

eg weighted methods per class, number of children, coupling per class\\ 

Software reusability metrics estimation: Algorithms, models and optimization techniques (Padhy)\\ 

Cyclomatic complexity: independent paths through source code\\ 

Software Reuse and Reusability Metrics and Models (Frakes, Terry)\\ 

A new reusability metric for object-oriented software (Barnard)\\ 

A metrics suite for measuring reusability of software components (Washizaki)\\ 

Reusability Index: A Measure for Assessing Software Assets Reusability (Ampatzoglou) 
\subsection{Portability}
compare effort to port to effort to redevelop\\ 

determine if the system can be ported: ram, processor, resolution, OS, browser\\ 

Issues in the Specification and Measurement of Software Portability (Mooney)\\ 

Designing a Measurement Method for the Portability Non-functional Requirement (Talib)\\ 

ISO 9126 breaks down portability testing: installability, compatability, adapatability, and replaceability.\\ 

https://www.softwaretestinghelp.com/what-is-portability-testing/ 
\end{document}