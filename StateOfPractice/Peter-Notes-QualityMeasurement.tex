\documentclass{article}

\usepackage{tabularx}
\usepackage{booktabs}
\usepackage{hyperref}
\hypersetup{%
	colorlinks = true,
	linkcolor  = black
}


\title{Project Notes - Quality Measurement}

\author{Peter Michalski}

\date{}


\begin{document}

\maketitle


\newpage

\tableofcontents
\addtocontents{toc}{\protect\thispagestyle{empty}}

~\newpage
\section{Variabilities}

Input Variabilities
\begin{enumerate}
	\item boundary parameters
	\item dimension
	\item number of velocity directions
\end{enumerate}

\noindent Calculation Variabilities
\begin{enumerate}
	\item computational model
	\item decomposition technique
	\item coefficient weights
	\item input check
	\item encoding of output
	\item exception check
\end{enumerate}

\noindent Other Variabilities
\begin{enumerate}
	\item language
	\item license
\end{enumerate}

~\newpage
\section{Quality Measurement}
\subsection{Robustness}
\subsubsection{Definition}
Software possesses the characteristic of robustness if it behaves ``reasonably'' in two situations: i) when it encounters circumstances not anticipated in the requirements specification; and ii) when the assumptions in its requirements specification are violated.

\subsubsection{Metrics}
Using CRASH criteria: (DOI: 10.1007/978-3-642-29032-9\_16)
\begin{enumerate}
	\item inject random input at interface (automated testing (selenium?) - generate list of valid input and observe if CRASH)
	\item use invalid input at interface (testing - generate list of invalid input and observe if CRASH)
	\item not provide necessary libraries (find if external libraries are needed, not provide, observe if CRASH)
	\item removing directories (mutation) (for example removal of output directory - find directories for mutation, remove, observe if CRASH)
	\item using OS and hardware outside of those in the reqs: (find req OS and hardware, make list outside of reqs, attempt to run the software and observe if CRASH)
\end{enumerate}
\subsubsection{Notes}
DOI: 10.1007/978-3-642-29032-9\_16:\\ 
The robustness failures are typically classified according to the CRASH criteria [540]: Catastrophic (the whole system crashes or reboots), Restart (the application has to be restarted), Abort (the application terminates abnormally), Silent (invalid operation is performed without error signal), and Hindering (incorrect error code is returned–note that returning a proper error code is considered as robust operation). The measure of robustness can be given as the ratio of test cases that exposes robustness faults, or, from the system’s point of view, as the number of robustness faults exposed by a given test suite.\\ 

~\newpage

\subsection{Performance}
\subsubsection{Definition}
The degree to which a system or component accomplishes its designated functions within given constraints, such as speed (database response times, for instance), throughput (transactions per second), capacity (concurrent usage loads), and timing (hard real-time demands).

\subsubsection{Metrics}

\begin{enumerate}
	\item CPU usage (find a tool to monitor the applications use of CPU), calculate average over the run of tests. Compare CPU usage between solutions - keep in mind the solutions vary significantly.
	\item Memory usage
	\item Concurrent usage loads
	\item Timing
	\item Run-time errors
	\item errors in solution
\end{enumerate}

\subsubsection{Notes}

~\newpage

\subsection{Maintainability}
\subsubsection{Definition}
The effort with which a software system or component can be modified to:

\begin{enumerate}
	\item correct faults
	\item improve performance or other
	attributes
	\item satisfy new requirements
\end{enumerate}

\subsubsection{Metrics}
\begin{enumerate}
	\item Open and closed issues on Git
	\item Time to close issue avg
	\item Update frequency
	\item Number of maintainers
\end{enumerate}

Metrics for Assessing a Software System's Maintainability (Oman, Hagemeister):
\begin{enumerate}
\item Age - since release in months
\item Size - thousands of non commented source statements (TNCSS)
\item Stability = 1 - Change Factor -> CF is ( e$\land$number of months * average percentage change of lines of code in number of months)
\item Defect Intensity =  e$\land$number of months * average percentage of defective lines of code per month
\item Subjective Product Appraisals = 5 point (very low to very high) for language complexity, application complexity, requirements volatility, product dependencies, complexity of build, installation complexity, intensity of product use, efficiency of the software system
\item modularity: number of modules, average module size
\item consistency: std deviation of module size (TNCSS)
\item cyclomatic complexity
\item global data types: the number of global data types divided by the total number of defined data types
\item global data structures: the number of global data structures divided by the total number of defined data structures
\item Data type consistency = 1 - percentage of data structures that undergo type conversion during assignment operations
\item I/O complexity = the number of lines of code devoted to I/O divided by TNCSS
\item Local data types = the number of local data types divided by the total number of defined data types averaged over all modules
\item Local data types = the number of local data types divided by the total number of defined data types averaged over all modules
\item Local data structures = the number of local data structures divided by the total number of defined data structures averaged over all modules
\item Initialization integrity = percentage of variables initilaized prior to use averaged over all modules
\item Overall Formatting =  percentage of blank lines in the whole program, percentage of modules with blank lines
\item Commenting =  percentage of comment lines in program, percentage of modules with header comments
\item Statement formatting =  percentage of uncrowded statements (no more than one statement per line) per module averaged over all modules
\item intramodule commenting = percentage lines of comments in module, averaged over all modules
\item subjective evaluation of document descriptiveness = accuracy, consistency, unambiguous
\item subjective evaluation of document completeness = extent of document set, contents
\item subjective evaluation of document correctness = traceability, verifiability
\item subjective evaluation of document readability = organization, accessibility via indices and table of contents, consistency of the writing style, typography, and comprehensibility
\item subjective evaluation of document modifiability = document set redundancies
\end{enumerate}
\subsubsection{Notes}


Metrics for Assessing a Software System's Maintainability (Oman, Hagemeister):\\
Software system metrics divided into 3 categories:\\

1. server maturity attributes (age since release, size, stability, maintenance intensity, defect intensity, reliability, reuse, subjective product appraisals)\\

2. source code (control structure, information structure, typography and naming and commenting) - each of these is broken into system and component subcategories\\

system control structure: modularity, complexity, consistency, nesting, control coupling, encapsulation, module reuse, control flow consistency\\

component control structure: complexity, use of structured constructs, use of unconditional branching, nesting, span of control structures, cohesion\\

system information structure: global data types, global data structures, system coupling, data flow consistency, data type consistency, nesting, I/O complexity, I/O integrity\\

component information structure: local data types, local data structures, data coupling, initialization integrity, span of data \\

system typography, naming and commenting: overall program formatting, overall program commenting, module separation, naming, symbols and case\\

component typography, naming and commenting: statement formatting, vertical spacing, horizontal spacing, intramodule commenting\\

3. supporting documentation (abstraction, physical attributes)\\
 
supporting documentation abstraction: descriptiveness appraisals, completeness appraisals, correctness appraisals\\

supporting documentation physical attributes: readability appraisals, modifiability appraisals\\



\noindent Software Metrics for Predicting Maintainability (Frappier, Matwin, Mili): \\

Documentation criterion (metrics): readability, traceability (NR, UR), coupling (M-MC, HK-IF, R-IF, KPL-IF, IF4, CA-DC), cohesion (M-MS, HK-IF, R-IF, KPL-IF, IF4, CA-DC), size\\

Source code criterion (metrics): traceability(NR, UR), control structure(v(g), knots, RLC), independence, readability(Vcd, Vcs, Ls), size (LOC, E, RLC), doc accuracy (DAR), consistency (SCC)\\

UR - Unreferenced Requirements - The  number  of  original  requirements  not  referenced  by  a  lower  document  in  the  documentation hierarchy (Formula: UR = number of R1 not referenced by a R2, where..)\\

NR - Non-referenced requirements - The number of items not referencing an original requirement.(NR = number of R2 not referencing any R1, where..)\\

Module Coupling (M-MC) - A measure of the strength of the relationships between modules. (Formula: Myers defines six levels of coupling which are, in order of decreasing strength: coupling, common, external, control, stamp, data)\\

Module Strength (M-MS) - A measure of how strongly related are the elements within a module.(Formula: Module  strength  is  a  subjective  metric.    Myers  defines  seven  levels  of  strength  which are, in order of increasing strength: coincidental, logical, classical, procedural, communicational, informational, functional)\\

Information Flow (HK-IF) -A measure of the control flow and data flow between modules. (Formula: IFm = (fan-inm*fan-outm)2*int-compm)\\

Integrated Information Flow of Rombach (R-IF) - A  measure  of  intermodule  and  intramodule  complexity  based  on  information  flow. \\

 Information Flow by Kitchenham et al (KPL-IF) - A   measure   of   intermodule   complexity   inspired   from   Henry   and   Kafura's   information flow metric.  Since Kitchenham et al experienced some difficulties in understanding   the   definition   of   flows   provided   by   Henry   and   Kafura,   they   formulated a new set of definitions.\\
 
  IF4 -  A measure of intermodule complexity based on information flow.\\

Design Complexity of Card and Agresti (CA-DC) - A  measure  of  intermodule  and  intramodule  complexity  of  a  system  based  on  fan-out, number of modules and input/output variables \\

Cocomo Inspired Metric (COCO) - A selection of appropriate adjustment factors of the intermediate Cocomo metric. \\

Cyclomatic Complexity Number (v(G)) -   The number of independent basic paths in a program. \\

Knots Description -  The  number  of  crossing  lines  (unstructured  goto  statements)  in  a  control  flow  graph\\

Relative Logical Complexity (RLC) -  The number of binary decisions divided by the number of statements \\

Comments Volume of Declarations (Vcd) -  Total number of characters found in the comments of the declaration section of a module.  The declaration section comprises comments before the module heading up to the first executable statement of the module body. \\

 Comments Volume of Structures (Vcs) Description:    Total number of characters in the comments found anywhere in the module except in  the  declaration  section.    The  declaration  section  comprises  comments  before  the module heading up to the first executable statement of the module body.\\
 
 Average Length of Variable Names (Ls) Description:     Mean  number  of  characters  of  all  variables  used  in  a  module.    Unused  declared  variables are not included. \\
 
  Lines of  Code (LOC) Description: The number of lines in the source code excluding blank lines or comment lines.\\
  
  Software Science Effort (E) Description:    An  estimation  of  programming  effort  based  on  the  number  of  operators  and  operands.  It is a combination of other Software Science metrics.\\
  
  Documentation Accuracy Ratio (DAR) Description:     A  verification  of  the  accuracy  of  the  CEI Spec, RS and SDD with respect to the source code.\\
  
  Source Code Consistency (SCC) Description:     The  extent  to  which  the  source  code  contains uniform notation, terminology and symbology within itself. \\
  
~\newpage

\subsection{Reusability}
\subsubsection{Definition}
The extent to which a software component can be used with or without adaptation in a problem solution other than the one for which it was originally developed.

\subsubsection{Metrics}
\subsubsection{Notes}
Measuring Software Reusability (Poulin)\\  

Taxonomy of reusability mctrics
1. Empirical methods
-
-
Module oriented
- Complexity based
- Size based
- Reliability based
Component oriented

 2.Qualitative methods
-
-Module oriented
- Style guidelines
Component oriented
- Certification guidelines
- Quality guidelines\\

3.1 Prieto-Diaz and Freeman:
Their process model encourages
white-box reuse and consists of finding candidate
reusable modules, evaluating each, deciding which
module the programmer can modify the easiest, then
adapting the module. In this model they identify
four module-oricntcd metrics and a fifth metric used
to modify the fust four.

Program size. Reuse depends on a small module
size, as indicated by lines of source code.
Program structure. Reuse depends o n a simple
program structure as indicated by fewer links to
other modules (low coupling) and low
cyclomatic complexity.
Program documentation.
Reuse depends on
excellent documentation as indicated by a sub-
jective overall rating o n a scale of 1 to 10.
Programming language. Reuse depends on pro-
gamming language to the extent that it helps to
reuse a module written in the same programming
language. If a reusable module in the same lan-
guage does not exist, the degree of similarity
between the target language and the one used in
the module affects the difficulty of modifying the
module to meet the new requirement.
Reuse experience. The experience of the reuser
in the programming language and in the appliea-
tion domain affects the previous metrics because
every programmer views a module from their
own perspective. For example, programmers
wtll have diffcrent views of what makes a
“small” module, depending on their background.
?’lis fifth metric serves to modify the values of
the other mctrics.


-----

Chidamber and Kemerer object-oriented metrics:\\ 

https://www.aivosto.com/project/help/pm-oo-ck.html:\\ 

eg weighted methods per class, number of children, coupling per class\\ 

Software reusability metrics estimation: Algorithms, models and optimization techniques (Padhy)\\ 

Cyclomatic complexity: independent paths through source code\\ 

Software Reuse and Reusability Metrics and Models (Frakes, Terry)\\ 

A new reusability metric for object-oriented software (Barnard)\\ 

A metrics suite for measuring reusability of software components (Washizaki)\\ 

Reusability Index: A Measure for Assessing Software Assets Reusability (Ampatzoglou) 

~\newpage
\subsection{Portability}
\subsubsection{Definition}
Effort required to transfer a program between system environments (including hardware and software).\\
\subsubsection{Metrics}
\subsubsection{Notes}
compare effort to port to effort to redevelop\\ 

determine if the system can be ported: ram, processor, resolution, OS, browser\\ 

Issues in the Specification and Measurement of Software Portability (Mooney):\\
The term portability
 refers to the ability of a software
 unit to be ported (to a given environment).
 A program is
 portable if and to the degree that the cost of porting is less than
 the cost of redevelopment. A software unit would be perfectly
 portable if it could be ported at zero cost, but this is never
 possible in practice.
 Instead, a software unit may be
 characterized by its degree ofportability,
 which is a function of
 the porting and development costs, with respect to a specific
 target environment.\\   
 
 Some of these issues have been examined by the author
 [Mooney 931. This work has proposed as a metric the degree of
 portability
 of a software unit with respect to a target
 environment, defined as
 DPfsu) = 1 - (Cport(su,q) I Crd$\&$req.e2)).\\
 This metric relates portability to a ratio between the cost
 of porting (which depends on the properties of the existing
 software unit and on the target environment), and the cost of
 redevelopment (which depends on the requirements and the
 target environment). A series of experiments is underway to
 refine and validate this metric and to determine how to measure
 or estimate Cpoyf and Crdev.
 One study by Sheets [94]
 suggests that the metric can be badly skewed by secondary
 elements such as inadequate documentation for the existing
 software.\\
   

Designing a Measurement Method for the Portability Non-functional Requirement (Talib):\\
 EffortNew is the total effort needed in working hours units
 to develop the software functionalities on a new
 environment\\
 
 
-----------\\
https://www.softwaretestinghelp.com/what-is-portability-testing/:\\

ISO 9126 breaks down portability testing: installability, compatability, adapatability, and replaceability.\\ 

check Installability, Adaptability, Replaceability, Compatibility or Coexistence\\

Installability: validate OS reqs, memory and RAM reqs, clear installation and uninstallation procedures, additional prerequisites\\

Adapatability: hardware and software dependency, language dependency and communication system\\



\end{document}