\documentclass{article}
\usepackage{natbib}
\usepackage{tabularx}
\usepackage{booktabs}
\usepackage{hyperref}
\usepackage{amsmath,amsthm}

\title{Quality measure notes}

\author{Olu O}

\date{}


\begin{document}

\maketitle

~\newpage
\section{Quality Measurement}
\subsection{Installability}

It depends on a lot of factors:
\begin{itemize}
	\item How experienced is the installer?
	\item How fast the destination computer?
	\item What is the medium from which the software is being installed (Internet download, local network, CD/DVD)?
	\item Are there any manual steps needed during the installation?
\end{itemize}
\subsubitem\citep{SmithEtAl2018}
\begin{itemize} 

	\item Are there installation instructions?
	\item Are the installation instructions linear?
	\item Is there something in place to automate the installation?
	\item Is there a means given to validate the installation?
	\item How many steps were involved in the installation?
	\item How many software packages need to be installed?
	\item Run uninstall, if available. Any obvious problems?


\subsubitem
 \url{www.testingstandards.co.uk/installability_guidelines.htm}
\begin{itemize}
	\item Can an untrained user successfully perform an initial installation of the application in an average of 10 minutes?
	\item When installing an upgraded version of the application, are all customizations in the user’s profile retained and converted to the new version’s data format if needed?
	\item Does installation program verify the correctness of the download before beginning the installation process???
	\item Following successful installation, does the installation program delete all temporary, backup, obsolete, and unneeded files associated with the application.
	\item What is the maximum number of installation steps
	\item Time that the installation process takes compared to the expected time it takes
	\item Install instructions available?
	\item Uninstall instructions available?
	\item Does the system handle an incomplete installation, such as one interrupted by a power failure or aborted by the user?
	\item Do applications need to be shut down before performing the installation?
	\item Does the user need the capability to install, uninstall, reinstall, or repair just selected portions of the application if required? (Can this be performed?)
	\item Is the user aware of successful, or unsuccessful, installation?
\end{itemize}

\end{itemize}
Installability addresses the following activities:
\begin{itemize}

\item Initial installation
\item Recovery from an incomplete, incorrect, or user-aborted installation
\item Reinstallation of the same version
\item Installation of a new version
\item Reverting to a previous version
\item Installation of additional components or updates
\item Uninstallation
\end{itemize}

\subsection{Verifiability}

\citep{SmithEtAl2018}
This was for Correctness and Verifiability
\begin{itemize}
	\item Are external libraries used? 
	\item Does the community have confidence in this library? 
	\item Any reference to the requirements specifications of the program?
	\item What tools or techniques are used to build confidence of correctness? (string)
	\item Is there a getting started tutorial, if yes, is the output as expected?
\end{itemize}
\citep{wiegers2003softreq}
\begin{itemize}
	\item What reference reports or other outputs can we use to verify that the system is producing its outputs correctly?
	\item Is the maximum cyclomatic complexity of a module more than 20?
	\item Are there any portions of the system that do not yield deterministic outputs, such that it could be difficult to determine if they were working correctly?
	\item Is it possible to come up with test data sets that have a high probability of revealing any errors in the requirements or in their implementation?
\end{itemize}


\bibliographystyle {plainnat}
\bibliography {../CommonFiles/ResearchProposal}
\end{document}