\documentclass[letterpaper,cleveref]{lipics-v2019}

\usepackage[round]{natbib}
\usepackage{booktabs}
\usepackage{amsmath,amsthm}

\usepackage{hyperref}
\hypersetup{
    colorlinks=true,       % false: boxed links; true: colored links
    linkcolor=red,          % color of internal links (change box color with linkbordercolor)
    citecolor=blue,       % color of links to bibliography
    filecolor=magenta,   % color of file links
    urlcolor=cyan           % color of external links
}

%% Comments
\newif\ifcomments\commentstrue

\ifcomments
\newcommand{\authornote}[3]{\textcolor{#1}{[#3 ---#2]}}
\newcommand{\todo}[1]{\textcolor{red}{[TODO: #1]}}
\else
\newcommand{\authornote}[3]{}
\newcommand{\todo}[1]{}
\fi

\newcommand{\wss}[1]{\authornote{blue}{SS}{#1}} %Spencer Smith
\newcommand{\jc}[1]{\authornote{red}{JC}{#1}} %Jacques Carette
\newcommand{\oo}[1]{\authornote{magenta}{OO}{#1}} %Olu Owojaiye
\newcommand{\pmi}[1]{\authornote{green}{PM}{#1}} %Peter Michalski
\newcommand{\ad}[1]{\authornote{cyan}{AD}{#1}} %Ao Dong

%\oddsidemargin 0mm
%\evensidemargin 0mm
%\textwidth 160mm
%\textheight 200mm

\theoremstyle{definition}
\newtheorem{defn}{Definition}

\title{Quality Definitions of Qualities} 
\author{Spencer Smith}{McMaster University, Canada}{smiths@mcmaster.ca}{}{}
\author{Jacques Carette}{McMaster University, Canada}{carette@mcmaster.ca}{}{}
\author{Olu Owojaiye}{McMaster University, Canada}{owojaiyo@mcmaster.ca}{}{}
\author{Peter Michalski}{McMaster University, Canada}{michap@mcmaster.ca}{}{}
\author{Ao Dong}{McMaster University, Canada}{}{}{}

\authorrunning{Smith et al.}  \Copyright{Spencer Smith and Jacques Carette and
  Olu Owojaiye and Peter Michalski and Ao Dong}

\date{\today}
	
\hideLIPIcs
\nolinenumbers

\begin{document}
\maketitle

\begin{abstract}
  ...
\end{abstract}

\tableofcontents

\section{Introduction} \label{SecIntroduction}

Purpose and scope of the document.  \wss{Needs to be filled in.  Should
  reference the overall research proposal, and the ``state of the practice''
  exercise in particular.}

The presentation is divided into two main sections: i) qualities that apply to
software products, software artifacts and software development processes, and
ii) qualities that are considered important for good specifications.  The
specification could be a specification of requirements, design or a test plan.

\section{Qualities of Software Products, Artifacts and
  Processes} \label{SecQualities}

To assess the current state of software development, and to understand how
future changes impact software development, we need a clear definition of what
we mean by quality.  The concept of quality is decomposed into a set of separate
components that together make up ``quality''. Unfortunately, these are
called \emph{qualities}. These are associated to the software product,
the software artifacts (documentation, test cases, etc) and to the software
development process itself, and combinations thereof.

Our analysis is centred around a set of software qualities.  Quality is not
considered as a single measure, but a collection of different qualities, often
called ``ilities.''  These qualities highlight the desirable nonfunctional
properties for software artifacts, which include both documentation and
code. Some qualities, such as visibility and productivity, apply to the process
used for developing the software. The following list of qualities is based on
\cite{GhezziEtAl2003}. To the list from \cite{GhezziEtAl2003}, we have added
three qualities important for SC: installability, reproducibility and
sustainability.

\subsection{Installability \oo{owner}}

\begin{defn}
  A measure of the ease of installation.  \wss{Where does this definition come
    from? Please add a citation.  I believe it is from \citet{McCallEtAl1977},
    but you should verify this.}
\end{defn}

\begin{defn} \label{DefnInstallability}
  Installability is related to the amount of human effort required to install a
  software in a designated environment \citep{berander2005software}.
\end{defn}

\begin{defn}
  ``The capability of a software product to be installed in a specified
  environment'' \citep{berander2005software}.
\end{defn}

\begin{defn}
  ``The efficacy of the installation, uninstallation or reinstallation process
  of a software product in a specified environment'' \citep{iso201017043}
\end{defn}

\noindent \textbf{Proposed Definition}

Definition~\ref{DefnInstallability}.  \wss{This definition does not allow for
  measuring the machine time.}

\noindent \textbf{Reasoning}

\wss{The rationale is missing.  Please add.}

\subsection{Correctness \oo{owner}}

\begin{defn}
  The degree to which a software's specification is satisfied
  \citep{berander2005software}.
\end{defn}

\begin{defn} \label{CorrectDefnSelected} 
  A software is correct if it functions according to its specified functional
  and non-functional specification \citep{GhezziEtAl2003}.  \wss{This is not how
    \citet{GhezziEtAl2003} defines correctness.  They only require satisfaction
    of the functional requirements.}

\end{defn}

\begin{defn}
  According to \citet{wilson2009quality}, ``the degree to which a system is free
  from defects in its specification, design, and implementation, is its
  correctness.''
\end{defn}

\begin{defn}
  ``The ability of software products to perform their exact tasks, as defined by
  their specification'' \citep{meyer1988object}.
\end{defn}

\begin{defn}
  The extent to which a program satisfies its specifications and fulfills the
  user's mission objectives \citep{McCallEtAl1977}. (As summarized in
  \citet{VanVliet2000}.)
\end{defn}

\noindent \textbf{Proposed Definition}

Definition~\ref{CorrectDefnSelected}
	
\noindent \textbf{Reasoning}

There is no direct tool or method for measuring correctness. One way of building
confidence in correctness is by reviewing to ensure that each requirement stated
is one that the stakeholders and experts desire.  By maintaining traceability,
consistency and unambiguity, we can reduce the occurrence of errors and make the
goal of reviewing for correctness easier.

The quality of a software's operation is dependent on the degree of correctness
\citep{berander2005software}. Correctness and reliability are said to have
dependencies, such that if a system exhibits a high degree of correctness then
it tends to be reliable \citep{GhezziEtAl2003}.

\subsection{Verifiability \oo{owner}}

\begin{defn}
  A software is verifiable if its properties can be verified
  \citep{GhezziEtAl2003}.
\end{defn}

\begin{defn}
  Attributes of software that relate to the effort needed for validating the
  modified software \citep{berander2005software}.
\end{defn}

\begin{defn}
  The degree to which a system or component facilitates the establishment of
  test criteria and the performance of tests to determine whether those criteria
  have been met \citep{IEEEStdGlossarySET1990}.
\end{defn}

\begin{defn} \label{Defn_Verifiability1}
  The degree to which a requirement is stated in terms that permit establishment
  of test criteria and performance \citep{IEEEStdGlossarySET1990}.
\end{defn}

\begin{defn} \label{Defn_Verifiability2}
  Ease of performing testing on a software product or system
  \citep{IEEEStdGlossarySET1990}.
\end{defn}

\noindent \textbf{Proposed Definition}

Combo of definitions~\ref{Defn_Verifiability1} and~\ref{Defn_Verifiability2} ?

\noindent \textbf{Reasoning}

Verifiability involves ``solving the equations right'' \citep[p.~23]{Roache1998};
it benefits from rational documentation that systematically shows, with explicit
traceability, how the governing equations are transformed into code.
\oo{Verifiability is sometimes referred to as testability, so I culled some
  testability definitions here}

\subsection{Validatability \oo{owner}}

\begin{defn}
  Validatability of a software is the degree of ease in validating (checking)
  that software meets user needs.
\end{defn}

\begin{defn}
  Validatability means ``solving the right equations'' \citep[p.~23]{Roache1998}.
  \oo{question about this definition}
\end{defn}

\begin{defn} \label{DefnValidatability}
  The degree of ease of evaluating software products or system to determine
  whether it satisfies specified business
  requirements. \url{http://softwaretestingfundamentals.com/verification-vs-validation/}
  \oo{Here I tweaked the meaning of validation to suit validatability} \oo{Only
    a few resource on validatability}
\end{defn}

\noindent \textbf{Proposed Definition}

Definition~\ref{DefnValidatability}.

\noindent \textbf{Reasoning}

Validatability is improved by a rational process via clear documentation of the
theory and assumptions, along with an explicit statement of the systematic steps
required for experimental validation.

\subsection{Reliability \oo{owner}}

\begin{defn}
  The probability that the software will operate as expected over a specified
  time interval \citep{GhezziEtAl2003}.
\end{defn}

\begin{defn}
  A set of attributes that relate to the capability of software to maintain its
  level of performance under stated conditions for a stated period of time
  \citep{berander2005software}.
\end{defn}

\begin{defn}
  ``The capability of the software product to maintain a specified level of
  performance when used under specified conditions''
  \citep{international2001iso}.
\end{defn}

\begin{defn}
  Code possesses the characteristic reliability to the extent that it can be
  expected to perform its intended functions satisfactorily
  \citep{boehm2007software}.
\end{defn}

\begin{defn}
  A concern encompassing correctness and robustness \citep{meyer1988object}.
\end{defn}

\noindent \textbf{Proposed Definition}

\wss{Needs to be completed.}

\noindent \textbf{Reasoning}

Reliability is a critical quality for scientific software, since the results of
computations are meaningless, if they are not dependable.  Reliability is
closely tied to verifiability, since the key quality to verify is reliability,
while the act of verification itself improves reliability.

Reliability models can be used to predict reliability of a software product. For
example measuring Mean Time to Fail (MTTF) can be a good measure of reliability
\citep{berander2005software}.

\subsection{Robustness \pmi{owner}}
\begin{defn}
  The degree to which a system or component can function correctly in the
  presence of invalid inputs or stressful environmental conditions
  \citep{IEEEStdGlossarySET1990}.
\end{defn}
\begin{defn}
  The quality can be further informally refined as the ability of a software to
  keep an acceptable behaviour, expressed in terms of robustness requirements, in
  spite of exceptional or unforeseen execution conditions (such as the
  unavailability of system resources, communication failures, invalid or
  stressful inputs, etc.) \citep{fernandez2005model}.
\end{defn}
\begin{defn}
  Code possesses the characteristic of robustness to the extent that it can
  continue to perform despite some violation of the assumptions in its
  specification \citep{boehm2007software}.
\end{defn}
\begin{defn} \label{RobustnessDefnSelected}
  A program is robust if it behaves "reasonably", even in circumstances that
  were not anticipated in the requirements specification - for example, when it
  encounters incorrect input data or some hardware malfunction
  \citep{ghezzi1991fundamentals}.
\end{defn}

\noindent \textbf{Proposed Definition}

Definition \ref{RobustnessDefnSelected} rephrased: Software possesses the
characteristic of robustness if it fulfills its requirements correctly or
behaves reasonably in circumstances not anticipated in the requirements
specification. \wss{I think we might want to separate the definition of
  robustness and correctness.  I'm thinking we just want the definition of
  robustness to be how the software behaves in unanticipated circumstances.}

\noindent \textbf{Reasoning}

This definition indicates that robustness is related to the quality of
correctness (both within and outside of it). 

\pmi{I think this is done.}

\subsection{Performance \pmi{owner}}

\begin{defn}
The degree to which a system or component accomplishes its designated functions
within given constraints, such as speed, accuracy, or memory usage
\citep{IEEEStdGlossarySET1990}.
\end{defn}
\begin{defn} \label{PerformanceDefnSelected}
How well or how rapidly the system must perform specific functions. Performance
requirements encompass speed (database response times, for instance), throughput
( transactions per second), capacity (con-current usage loads), and timing (hard
real-time demands) \citep{wiegers2003softreq}.
\end{defn}
\begin{defn}
In software engineering we often equate performance with efficiency. A software
system is efficient if it uses computing resources economically
\citep{ghezzi1991fundamentals}.
\end{defn}

\noindent \textbf{Proposed Definition}

Definition \ref{PerformanceDefnSelected}: How well or how rapidly the system
must perform specific functions. Performance requirements encompass speed
(database response times, for instance), throughput ( transactions per second),
capacity (con-current usage loads), and timing (hard real-time demands).

\noindent \textbf{Reasoning}

This definition includes all important categories of performance. 

\pmi{I think this is done.}

\subsection{Usability \jc{owner}} 

ISO defines usability as
\begin{quote}
The extent to which a product can be used by specified users to achieve
specified goals with effectiveness, efficiency, and satisfaction in a specified
context of use.
\end{quote}

Nielsen and (separately) Schneidermann have defined usability as part of usefulness and
is composed of:
\begin{itemize}
\item Learnability: How easy is it for users to accomplish basic tasks the
  first time they encounter the design?
\item Efficiency: Once users have learned the design, how quickly can they perform tasks?
\item Memorability: When users return to the design after a period of not using
  it, how easily can they re-establish proficiency?
\item Errors: How many errors do users make, how severe are these errors, and
  how easily can they recover from the errors?
\item Satisfaction: How pleasant is it to use the design?
\end{itemize}
In that context, it makes sense to separate \emph{usefulness} into
\emph{usability} (purely an interface concern) and \emph{utility} (in the economics
sense of the word).

There are two ISO standards covering this, namely ISO/TR 16982:202 and ISO 9241. 

The Interaction Design Foundation~\url{https://www.interaction-design.org/literature/topics/usability}
further lists the following desirable outcomes:

\begin{enumerate}
\item It should be easy for the user to become familiar with and competent in using
the user interface during the first contact with the website. For example, if a
travel agent’s website is a well-designed one, the user should be able to move
through the sequence of actions to book a ticket quickly.
\item It should be easy for users to achieve their objective through using the
website. If a user has the goal of booking a flight, a good design will guide
him/her through the easiest process to purchase that ticket.
\item It should be easy to recall the user interface and how to use it on
subsequent visits. So, a good design on the travel agent’s site means the user
should learn from the first time and book a second ticket just as easily.
\end{enumerate}

One core reference, for definitions and metrics, is
\citet{bevan1995measuring}.

\noindent \textbf{Proposed Definition}

\wss{Still needs to be completed}

\noindent \textbf{Reasoning}

\wss{Needs to be completed}

\subsection{Maintainability \pmi{owner}}

\begin{defn} \label{MaintainabilityDefnSelected1}
The ease with which a software system or component can be modified to correct
faults, improve performance or other attributes, or adapt to a changed
environment \citep{IEEEStdGlossarySET1990}. 
\end{defn}\
\begin{defn}
ISO/IEC 25010 refers to maintainability as the degree of effectiveness and
efficiency with which a product or system can be modified by the intended
maintainers \citep{ISO/IEC25010}.
\end{defn}
\begin{defn}
Effort required to locate and fix an error in a program
\citep{pressman2005software}.
\end{defn}
\begin{defn}
A set of attributes that bear on the effort needed to make specified
modifications (which may include corrections, improvements, or adaptations of
software to environmental changes and changes in the requirements and functional
specifications)\citep{pfleeger2006software}.
\end{defn}
\begin{defn}
We will view maintainability as two separate qualities: repairability and
evolvability. Software is repairable if it allows the fixing of defects; it is
evolvable if it allows changes that enable it to satisfy new requirements
\citep{ghezzi1991fundamentals}.
\end{defn}
\begin{defn} \label{MaintainabilityDefnSelected2}
Code possesses the characteristic of maintainability to the extent that it
facilitates updating to satisfy new requirements or to correct deficiencies
\citep{boehm2007software}.
\end{defn}

\noindent \textbf{Proposed Definition}

Combined Definition \ref{MaintainabilityDefnSelected1} and Definition
\ref{MaintainabilityDefnSelected2}: The ease with which a software system or
component can be modified to correct faults, improve performance or other
attributes, or satisfy new requirements.

\noindent \textbf{Reasoning}

This definition includes all potential reasons to modify the software. 

\pmi{I think this is done.}

\subsection{Reusability \pmi{owner}}

\begin{defn} 
The degree to which a software module or other work product can be used in more
than one software system \citep{IEEEStdGlossarySET1990}. 
\end{defn}
\begin{defn}
Extent to which a program [or parts of a program] can be reused in other
applications - related to the packaging and scope of the functions that the
program performs \citep{pressman2005software}.
\end{defn}
\begin{defn} \label{ReusabilityDefnSelected}
The extent to which a software component can be used with or without adaptation
in a problem solution other than the one for which it was originally developed
\citep{kalagiakos2003non}.
\end{defn}
\begin{defn}
Reusability is the likelihood a segment of source code that can be used again to
add new functionalities with slight or no modification \citep{sandhu2010survey}.
\end{defn}

\noindent \textbf{Proposed Definition}

Definition \ref{ReusabilityDefnSelected}: The extent to which a software
component can be used with or without adaptation in a problem solution other
than the one for which it was originally developed.

\noindent \textbf{Reasoning}

This definition highlights the possible but not necessary adaptation of the
software component(s) being transferred.

\pmi{I think this is done.}

\subsection{Portability \pmi{owner}}
\begin{defn}
The ease with which a system or component can be transferred from one hardware
or software environment to another \citep{IEEEStdGlossarySET1990}.
\end{defn}
\begin{defn}
An application is portable across a class of environments to the degree that the
effort required to transport and adapt it to a new environment in the class is
less than the effort of redevelopment \citep{mooney1990strategies}.
\end{defn}
\begin{defn} \label{PortabilityDefnSelected}
Effort required to transfer the program from one hardware and/or software system
environment to another \citep{pressman2005software}.
\end{defn}
\begin{defn}
A set of attributes that bear on the ability of software to be transferred from
one environment to another (including the organizational, hardware, of software
environment)\citep{pfleeger2006software}.
\end{defn}
\begin{defn}
Code possesses the characteristic of portability to the extent that it can be
operated easily and well on computer configurations other than its current
one. This implies that special function features, not easily available at other
facilities, are not used, that standard library functions and subroutines are
selected for universal applicability, and so on \citep{boehm2007software}.
\end{defn}
\begin{defn}
Portability refers to the ability to run a system on different hardware
platforms \citep{ghezzi1991fundamentals}.
\end{defn}

\noindent \textbf{Proposed Definition}

Definition \ref{PortabilityDefnSelected} rephrased: Effort required to transfer
a program between system environments.

\noindent \textbf{Reasoning}

This is measurable and succinct. 

\pmi{I think this is done.}

\subsection{Understandability \jc{owner}}

Understandability is artifact-dependent. What it means for a user-interface (graphical
or otherwise) to be understandable is wildly different than what it means for the code,
and even the user documentation.

The literature here is thin and scattered.  More work will need to be done to find
something useful.

Interestingly, the business literature seems to have taken more care to define this.
Here we encounter
\begin{quote}
Understandability is the concept that X should be presented
so that a reader can easily comprehend it.
\end{quote}
At least this brings in the idea that the \emph{reader} is actively involved, and
indirectly that the reader's knowledge may be relevant, as well as the
``clarity of exposition'' of X.

Section 11.2 of \citet{adams2015nonfunctional} does have a full definition.

\noindent \textbf{Proposed Definition}

\wss{Still needs to be completed}

\noindent \textbf{Reasoning}

\wss{Needs to be completed}

\subsection{Interoperability \ad{owner}}

\begin{defn}
Interoperability is the ability of two or more systems or components to exchange
information and to use the information that has been exchanged
\citep{IEEEComputerDictionary1991}.
\end{defn}
\begin{defn}
The degree to which two or more systems, products or components can exchange
information and use the information that has been exchanged
\citep{ISO/IEC25010}.
\end{defn}
\begin{defn}
The capability to communicate, execute programs, and transfer data among various
functional units in a manner that requires the user to have little or no
knowledge of the unique characteristics of those units
\citep{ISO/IEC/IEEE24765}.
\end{defn}
\begin{defn}Interoperability is a characteristic of a product or system, whose
interfaces are completely understood, to work with other products or systems,
present or future, in either implementation or access, without any restrictions
\citep{AFUL2019}.
\end{defn}
\begin{defn}
\label{InteroperabilitySelected}
Interoperability is the ability of different information systems, devices and
applications (‘systems’) to access, exchange, integrate and cooperatively use
data in a coordinated manner, within and across organizational, regional and
national boundaries, to provide timely and seamless portability of information
and optimize the health of individuals and populations globally. Health data
exchange architectures, application interfaces and standards enable data to be
accessed and shared appropriately and securely across the complete spectrum of
care, within all applicable settings and with relevant stakeholders, including
by the individual \citep{HIMSS2019}.
        
Four Levels of Interoperability:
\begin{itemize}
\item Foundational (Level 1) – establishes the inter-connectivity requirements
needed for one system or application to securely communicate data to and receive
data from another

\item Structural (Level 2) – defines the format, syntax, and organization of
data exchange including at the data field level for interpretation

\item Semantic (Level 3) – provides for common underlying models and
codification of the data including the use of data elements with standardized
definitions from publicly available value sets and coding vocabularies,
providing shared understanding and meaning to the user

\item Organizational (Level 4) – includes governance, policy, social, legal and
organizational considerations to facilitate the secure, seamless and timely
communication and use of data both within and between organizations, entities
and individuals. These components enable shared consent, trust and integrated
end-user processes and workflows
\end{itemize}
\end{defn}

\noindent \textbf{Proposed Definition}

Definition \ref{InteroperabilitySelected}
rephrased: Interoperability is the ability of different information systems,
devices and applications (‘systems’) to access, exchange, integrate and
cooperatively use data in a coordinated manner, within and across
organizational, regional and national boundaries, to provide timely and seamless
portability of information.

\wss{I moved the proposed definition to the end, to match the other sections.
  Hopefully I did not change the meaning.}

\noindent \textbf{Reasoning}

\wss{Needs to be completed}

\subsection{Visibility/Transparency \ad{owner}}

\begin{defn}
\label{VisibilitySelected}
A software development process is visible if all of its steps and its current
status are documented clearly. Another term used to characterize this property
is transparency \citep{ghezzi1991fundamentals}.
\end{defn}
      
\noindent \textbf{Proposed Definition} 

Definition \ref{VisibilitySelected}.

\noindent \textbf{Reasoning}

\wss{Needs to be completed}

\subsection{Reproducibility \wss{owner}}

Reproducibility is a required component of the scientific
method \citep{Davison2012}.  Although QA has, ``a bad name among creative
scientists and engineers'' \citep[p.~352]{Roache1998}, the community need to
recognize that participating in QA management also improves reproducibility.
Reproducibility, like QA, benefits from a consistent and repeatable computing
environment, version control and separating code from
configuration/parameters \citep{Davison2012}.

Reproducibility is defined as:

\begin{defn}
A result is said to be reproducible if another researcher can take the original
code and input data, execute it, and re-obtain the same result (Peng, Dominici,
and Zeger, 2006), as cited in \citet{BenureauAndRougier2017}.
\end{defn}

The related concept of replicable is defined as:

\begin{defn}
Documentation achieves replicability if the description it provides of the
algorithms is sufficiently precise and complete for an independent researcher to
re-obtain the results it presents.  \citep{BenureauAndRougier2017}
\end{defn}

It would be worthwhile to look for some additional definitions.

\noindent \textbf{Proposed Definition} 

\wss{Needs to be completed}

\noindent \textbf{Reasoning}

\wss{Needs to be completed}

\subsection{Productivity \ad{owner}}

\begin{defn}
The best definition of the productivity of a process is
\[\text{Productivity} = \dfrac{\text{Outputs produced by the
process}}{\text{Inputs consumed by the process}}\]
Defining inputs. For the software process, providing a meaningful definition of
inputs is a nontrivial but generally workable problem. Inputs to the software
process generally comprise labor, computers, supplies, and other support
facilities and equipment. Defining outputs. The big problem in defining software
productivity is defining outputs. Here we find a defining delivered source
instructions (DSI) or lines of code as the output of the software process is
totally inadequate, and they argue that there are a number of deficiencies in
using DSI. However, most organizations doing practical productivity measurement
still use DSI as their primary metric \citep{Boehm1987}.
\end{defn}
\begin{defn}
\label{ProductivitySelected}
Productivity is the amount of output (what is produced) per unit of input
used.If we can measure the size of the software product and the effort required
to develop the product, we have:
\begin{align}
\text{productivity} = \text{size}/\text{effort}
\end{align}
Equation (1) assumes that size is the output of the software production process
and effort is the input to the process. This can be contrasted with the
viewpoint of software cost models where we use size as an independent variable
(i.e., an input) to predict effort which is treated as an output. Equation (1)
is simple to operationalize if we have a single dominant size measure, for
example, product size measured in lines of code \citep{Kitchenham2004}.
\end{defn}

\noindent \textbf{Proposed Definition} 

The first sentence of Definition \ref{ProductivitySelected}: Productivity is the
amount of output (what is produced) per unit of input used.

\noindent \textbf{Reasoning}

\wss{Needs to be completed}

\subsection{Sustainability \wss{owner}}

One of the original definitions of sustainability (for systems, not software
specific), and still often quoted, is:

\begin{defn}
\noindent The ability to meet the needs of the present without compromising the ability of
future generations to meet their own needs \citep{Brundtland1987}.
\end{defn}

This is the definition used by \citet{IISD2019}.

To make it more useful, this definition is often split into three dimensions:
social, economic and environmental. \wss{cite UN paper [9] in
  \citet{PenzenstadlerAndHenning2013}}  To this list Penzenstadler and Henning
(2013) have added technical sustainability \citep{PenzenstadlerAndHenning2013}.
Where technical sustainability for software is defined as:

\begin{defn}
\noindent Technical sustainability has the central objective of long-time
usage of systems and their adequate evolution with changing surrounding
conditions and respective requirements \citep{PenzenstadlerAndHenning2013}.
\end{defn}

The fourth dimension of technical sustainability is also added
by \citep{WolframEtAl2017}.  Technical sustainability is the focus on the thesis
by \citet{Hygerth2016}.

\begin{defn}
  \noindent Sustainable development is a mindset (principles) and an
  accompanying set of practices that enable a team to achieve and maintain an
  optimal development pace indefinitely \citep{Tate2005}.
\end{defn}

Parnas discusses as software aging \citep{Parnas1994a}.

SCS specific definitions:

\begin{defn}
  The concept of sustainability is based on three pillars: the
  ecological, the economical and the social. This means that for a software to
  be sustainable, we must take all of its effects -- direct and indirect -- on
  the environment, the economy and the society into account. In addition, the
  entire life cycle of a software has to be considered: from planning and
  conception to programming, distribution, installation, usage and
  disposal \citep{Heine2017}.
\end{defn}

\begin{defn}
  \noindent The capacity of the software to endure. In other words,
  sustainability means that the software will continue to be available in the
  future, on new platforms, meeting new needs \citep{Katz2016}.
\end{defn}

Definition from Neil Chue Hong:
\begin{defn}
Sustainable software is software which is:
-- Easy to evolve and maintain
-- Fulfils its intent over time
-- Survives uncertainty
-- Supports relevant concerns (Political, Economic, Social, Technical,
Legal, Environmental) \citep{Katz2016}.
\end{defn}

Paper critical of a lack of a definition \citep{VentersEtAl2014}.

Sounds like definition of maintainability.

Find paper that combines nonfunctional qualities into sustainability.

Sustainability depends on the software artifacts AND the software team AND the
development process.

\noindent \textbf{Proposed Definition} 

\wss{Needs to be completed}

\noindent \textbf{Reasoning}

\wss{Needs to be completed}

\section{Desirable Qualities of Good Specifications} \label{SecDesirableQs}

To achieve the qualities listed in Section~\ref{SecQualities}, the documentation
should achieve the qualities listed in this section.  All but the final quality
listed (abstraction), are adapted from the IEEE recommended practise for
producing good software requirements \citep{IEEE1998}.  Abstraction means only
revealing relevant details, which in a requirements document means stating what
is to be achieved, but remaining silent on how it is to be achieved.
Abstraction is an important software development principle for dealing with
complexity \citep[p.~40]{GhezziEtAl2003}.  Correctness was in the above list, so
it is not repeated here.  \citet{SmithAndKoothoor2016} present further details
on the qualities of documentation for SCS.

\subsection{Completeness \ad{owner}}

\begin{defn}
\label{CompletenessSelected}
A specification is complete to the extent that all of its parts are present and
each part is fully developed. A software specification must exhibit several
properties to assure its completeness \citep{Boehm1984}:
\begin{itemize}
\item No TBDs. TBDs are places in the specification where decisions have been
postponed by writing "To be Determined" or "TBD."
\item No nonexistent references. These are references in the specification to
functions, inputs, or outputs (including databases) not defined in the
specification.
\item No missing specification items. These are items that should be present as
part of the standard format of the specification, but are not present.
\item No missing functions. These are functions that should be part of the
software product but are not called for in the specification.
\item No missing products. These are products that should be part of the
delivered software but are not called for in the specification.
\end{itemize}
\end{defn}

\noindent \textbf{Proposed Definition} 

The first sentence of Definition \ref{CompletenessSelected}: A specification is
complete to the extent that all of its parts are present and each part is fully
developed.

\noindent \textbf{Reasoning}

\wss{Needs to be completed}

\subsection{Consistency \ad{owner}}

\begin{defn}
A specification is consistent to the extent that its provisions do not conflict
with each other or with governing specifications and objectives. Specifications
require consistency in several ways \citep{Boehm1984}.
\begin{itemize}
\item Internal consistency. Items within the specification do not conflict with
each other.
\item External consistency. Items in the specification do not conflict with
external specifications or entities.
\item Traceability. Items in the specification have clear antecedents in earlier
specifications or statements of system objectives.
\end{itemize}
\end{defn}
\begin{defn}
Consistency requires that no two or more requirements in a specification
contradict each other. It is also often regarded as the case where words and
terms have the same meaning throughout the requirements specifications
(consistent use of terminology). These two views of consistency imply that
mutually exclusive statements and clashes in terminology should be avoided
\citep{ZOWGHI2003}.
\end{defn}
\begin{defn}
\label{ConsistencySelected}
Consistency: 1. the degree of uniformity, standardization, and freedom from
contradiction among the documents or parts of a system or component 2. software
attributes that provide uniform design and implementation techniques and
notations \citep{ISO/IEC/IEEE24765}.
\end{defn}

\noindent \textbf{Proposed Definition} 

Definition \ref{ConsistencySelected}.

\noindent \textbf{Reasoning}

\wss{Needs to be completed}

\subsection{Modifiability \jc{owner}}

Here we do seem to have a simple, if somewhat uninformative, definition:

\begin{defn}
Modifiability is the degree of ease at which changes can be made to a system,
and the flexibility with which the system adapts to such changes.
\end{defn}

IEEE Standard 610 seems to speak about this. (which is superseded?)

\noindent \textbf{Reasoning}

\wss{Needs to be completed}

\subsection{Traceability \jc{owner}}

Here the Wikipedia page~\url{https://en.wikipedia.org/wiki/Traceability} is
actually rather informative, especially as it also lists how this concept is
used in other domains.  A generic definition that is still quite useful is
\begin{defn}
The capability (and implementation) of keeping track of a given set or type of
information to a given degree, or the ability to chronologically interrelate
uniquely identifiable entities in a way that is verifiable.
\end{defn}
By specializing the above to software artifacts, ``interrelate'' to 
``why is this here'' (for forward tracing from requirements), this does
indeed give what is meant in SE.

Various standards (DO178C, ISO 26262, and IEC61508) explicitly mention it.

24765-2017 - ISO/IEC/IEEE International Standard - Systems and software engineering--Vocabulary
has a full definition, namely
\begin{enumerate}
\item the degree to which a relationship can be established between two or more
products of the development process, especially products having a
predecessor-successor or master-subordinate relationship to one another;
\item
the identification and documentation of derivation paths (upward) and
allocation or flowdown paths (downward) of work products in the work product
hierarchy;
\item the degree to which each element in a software development
product establishes its reason for existing; and discernible association
among two or more logical entities, such as requirements, system elements,
verifications, or tasks.
\end{enumerate}

\noindent \textbf{Proposed Definition} 

\wss{Needs to be completed.}

\noindent \textbf{Reasoning}

\wss{Needs to be completed}

\subsection{Unambiguity \wss{owner}}

A specification is unambiguous when it has a unique interpretation.  If there is
a possibility that two readers will have two different interpretations, than the
specification is ambiguous.  \wss{When I get the Ghezzi text back from Olu, I'll
  check to see if they have anything to add to this definition.}

A Software Requirements Specification (SRS) is unambiguous if, and only if,
every requirement stated therein has only one interpretation \citep{IEEE1998}.

\noindent \textbf{Proposed Definition} 

\wss{Needs to be completed.}

\noindent \textbf{Reasoning}

\wss{Needs to be completed}

\subsection{Verifiability \wss{owner}}

\begin{itemize}

\item Verification - Are we building the product right?  Are we implementing the
  requirements correctly (internal)
\item Validation - Are we building the right product? Are we getting the right
  requirements (external)
\item According to
  \href{https://en.wikipedia.org/wiki/Software_verification_and_validation}{Capability
    Maturity Model (CMM)}
\begin{itemize}
\item 
    Software Verification: The process of evaluating software to determine
    whether the products of a given development phase satisfy the conditions
    imposed at the start of that phase. [IEEE-STD-610]
  \item Software Validation: The process of evaluating software during or at the
    end of the development process to determine whether it satisfies specified
    requirements. [IEEE-STD-610] 
\end{itemize}
\end{itemize}

``An SRS is verifiable if, and only if, every requirement stated therein is
verifiable. A requirement is verifiable if, and only if, there exists some
finite cost-effective process with which a person or machine can check that the
software product meets the requirement. In general any ambiguous requirement is
not verifiable.'' \citep{IEEE1998}

Verifiability is related to testability, which is defined by McCall et al.\ as
``The effort required to test a program to ensure that it performs its intended
function'' \citep{VanVliet2000}.  

\wss{When I get the Ghezzi text back from Olu,
  I'll check to see if they have anything to add to this definition.}

\noindent \textbf{Proposed Definition} 

\wss{Needs to be completed}

\noindent \textbf{Reasoning}

\wss{Needs to be completed}

\subsection{Abstract \wss{owner}}

\begin{defn}
Documented requirements are said to be abstract if they state what the software
must do and the properties it must possess, but do not speak about how these are
to be achieved \citep{GhezziEtAl2003}.  
\end{defn}

\begin{defn}
``An abstraction for a software artifact is a succinct description that suppresses
the details that are unimportant to a software developer and emphasizes the
information that is important.'' \citep{Krueger1992}
\end{defn}

\begin{defn}
``Abstraction means that we concentrate on the essential features and ignore,
abstract from, details that are not relevant at the level we are currently
working.''  \citep[p.\ 296]{VanVliet2000}
\end{defn}

\begin{defn}
``Abstraction in mathematics is the process of extracting the underlying essence
of a mathematical concept, removing any dependence on real world objects with
which it might originally have been connected, and generalizing it so that it
has wider applications or matching among other abstract descriptions of
equivalent phenomena.''
\href{https://en.wikipedia.org/wiki/Abstraction_(mathematics)} {Wikipedia
  Definition}
\end{defn}

Abstraction is related to reusability (and other qualities).

\wss{When I get the Ghezzi text back from Olu, I'll check to see if they have
  anything to add to this definition.}

\noindent \textbf{Proposed Definition} 

\wss{Needs to be completed}

\noindent \textbf{Reasoning}

\wss{Needs to be completed}

\newpage

\bibliographystyle {plainnat}
\bibliography {../../CommonFiles/ResearchProposal}

\end{document}