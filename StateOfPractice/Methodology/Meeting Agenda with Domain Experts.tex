\documentclass[12pt]{article}

\usepackage{natbib}
\usepackage{booktabs}
\usepackage{longtable}
\usepackage{xspace}
\usepackage{xcolor}
\usepackage{fullpage}
\usepackage{hyperref}

\hypersetup{
colorlinks=true,       % false: boxed links; true: colored links
linkcolor=red,          % color of internal links (change box color with
%linkbordercolor)
citecolor=blue,       % color of links to bibliography
filecolor=magenta,   % color of file links
urlcolor=cyan           % color of external links
}

%% Comments
\newif\ifcomments\commentstrue

\ifcomments
\newcommand{\authornote}[3]{\textcolor{#1}{[#3 ---#2]}}
\newcommand{\todo}[1]{\textcolor{red}{[TODO: #1]}}
\else
\newcommand{\authornote}[3]{}
\newcommand{\todo}[1]{}
\fi

\newcommand{\wss}[1]{\authornote{blue}{SS}{#1}} %Spencer Smith
\newcommand{\jc}[1]{\authornote{red}{JC}{#1}} %Jacques Carette
\newcommand{\oo}[1]{\authornote{magenta}{OO}{#1}} %Olu Owojaiye
\newcommand{\pmi}[1]{\authornote{green}{PM}{#1}} %Peter Michalski
\newcommand{\ad}[1]{\authornote{brown}{AD}{#1}} %Ao Dong

%% Global noindent
% \setlength\parindent{0pt}

\title{Meeting Agenda with Domain Experts} 
\author{Ao Dong}
\date{\today}

\begin{document}

\maketitle

\section{Introduction}
Presenter: \ad{TBD}

A brief introduction to the project. \ad{To be written}

\subsection{About this Meeting}
Presenter: \ad{TBD}

The meeting will be roughly one hour. We will follow this document and go
through all the steps.

In the interest of time, some topics will only be discussed briefly during this
meeting, and tasks will be assigned to domain experts to finish later. For
example, Task \ref{task_usability_tests} in Section \ref{usability_tests}.

We will create issues on GitHub to track the progress of the tasks. Details can
be found in Appendix \ref{proj_mgmt_tools}. The tasks can also be found in
Appendix \ref{tasks_domain_experts}.

\subsection{Project Objectives}
\label{project_objectives}
Presenter: Dr.\ Spencer Smith and Dr.\ Jacques Carette \ad{TBD}

The overall objectives are documented in the section Overall Objective of
\href{https://github.com/smiths/AIMSS/blob/master/OverallResearchProposal/ObjectivesAndResearchQuestions.pdf}{Objectives
And Research Questions}.

A brief introduction. \ad{To be written}

\subsection{Research Proposal}
\label{research_proposal}
Presenter: Dr.\ Spencer Smith \ad{TBD}

The research proposal is documented in
\href{https://github.com/smiths/AIMSS/blob/master/OverallResearchProposal/ResearchProposal.pdf}{Research
	Proposal}.

A brief introduction. \ad{To be written}

\subsection{Research Questions}
\label{research_questions}
Presenter: Dr.\ Spencer Smith \ad{TBD}

The research questions are documented in the section Current State of the
Practice Research Questions of
\href{https://github.com/smiths/AIMSS/blob/master/OverallResearchProposal/ObjectivesAndResearchQuestions.pdf}{Objectives
And Research Questions}.

A brief introduction. \ad{To be written}

Do you have any thoughts on one of the questions - What are the ``pain points''
for developers working on SCS projects? What aspects of the existing processes,
methodologies and tools do they consider could potentially be improved?

\section{Qualities \& Measurements}

\subsection{Measurement Template}
\label{measurement_template}
Presenter: \ad{TBD}

The template for software quality measurements can be found at
\href{https://github.com/smiths/AIMSS/blob/master/StateOfPractice/Combined_MeasurementTemplate_EmpiricalMeasures.xlsx}{Measurement
	Template}.

\subsection{Survey for Short List Projects}
\label{survey_short_list_projects}
Presenter: \ad{TBD}

After the first round of measurement, we would like to select a short list of
projects (about 5 of them for each domain) and directly ask questions to the
developers.

We would like to do the selection based on the results of the first round
measurement, such as selecting the top 5 ones with the highest scores. However, we
can also include 1 or 2 most recommended ones from you. What do you think of
this selection process?

The survey questions to the developers can be found at
\href{https://github.com/smiths/AIMSS/blob/master/StateOfPractice/Ao-Notes/Questions%20to%20Developers.pdf}{Questions
	to Developers}.

\subsection{Usability Tests}
\label{usability_tests}
Presenter: Olu \ad{TBD}

A brief introduction. \ad{To be written}

The usability test can be found at ... \ad{TBD}

Do you have any suggestions? Feedback can be provided later in Task
\ref{task_usability_tests}

\subsection{Modifiability Experiment}
\label{modifiability_experiment}
Presenter:

A brief introduction. 

The usability test can be found..

\section{Software List}
Presenter: Ao or Peter \ad{TBD}

In this section, we would like to discuss the list of software packages and the
process of creating and shortening it.

\subsection{Top 10 List}
\label{top_10_list}
What will be the top 10 software of your choice? Feedback can be provided later
in Task \ref{task_top_10_list}.

\subsection{Software List}
\label{software_list}
The software list is
\href{https://docs.google.com/spreadsheets/d/122wU0v3ZtvDcqy8C4zKJ89kU-8fXAbo3Mzn6vcVXOi0/edit?usp=sharing}{Medical
	Imaging Software List} or Lattice Boltzmann Solver List. \ad{TBD}

What do you think of the list of the selected software?

Is there any software you would like to remove from or add to the list?

Feedback can be provided later in Task \ref{task_software_list}.

\subsection{Domain Analysis}
\label{domain_analysis}
The domain analysis can be found in Section 3.2 of
\href{https://github.com/Ao99/MISEG/blob/master/docs/SRS/SRS.pdf}{SRS for
Medical Imaging Software} or
\href{https://github.com/peter-michalski/LatticeBoltzmannSolvers/blob/master/docs/SRS/CA.pdf}{CA
of Lattice Boltzmann Solvers}.

What do you think of our analysis?  \wss{This is too big a question for the
  domain experts to answer in a reasonable amount of time.  Moreover, without
  more guidance their feedback is likely to be shallow.  I think we are better
  off with a task based inspection, using issues assigned via a GitHub project
  page.  We also will likely be more successful asking for reviews from grad
  students, rather than from faculty members.  We should also consider
  summarizing the commonalities, variabilities and parameters of variation in
  tables, or possibly graphs.}

\subsection{First AHP Ranking}
\label{first_AHP}
Presenter:

A brief introduction. 

Do you have any feedback?

\subsection{Final AHP Ranking}
\label{final_AHP}
Presenter:

A brief introduction. 

Do you have any feedback?

\section{Publication}
Presenter: Dr.\ Spencer Smith and Dr.\ Jacques Carette \ad{TBD}

\begin{itemize}
\item Where should we publish this paper?
\item Who are our targeted readers?
\end{itemize}

\section{Schedule \& Team}
Presenter: \ad{TBD}

\begin{itemize}
\item We propose to start this project from ... and finish it before ... . What
do you think?
\item Our meeting schedule is ... does it fit your plan?
\item We need ... extra members from your team, each working ... hours per week
for ... weeks. What do you think?
\item Are there any other experts you think we should keep them involved?
\end{itemize}

\appendix
\section{Project Management Tools}
\label{proj_mgmt_tools}
We would like to use GitHub for task tracking and document version control. The
repository for this project is
\href{https://github.com/smiths/AIMSS}{https://github.com/smiths/AIMSS}.

Tasks will be created as issues on the repository and assigned to individual
stakeholders. An open issue usually means a task to be fulfilled, and it can be
closed once it is finished. Here is an example of an issue
\href{https://github.com/smiths/AIMSS/issues/19}{https://github.com/smiths/AIMSS/issues/19}.

\newpage

\section{Tasks for Domain Experts}
\label{tasks_domain_experts}

\subsection{Review the Usability Tests}
\label{task_usability_tests}
GitHub Issue:
\href{https://github.com/smiths/AIMSS/issues}{https://github.com/smiths/AIMSS/issues}
\ad{To be created}

\noindent Refer to Section \ref{usability_tests}

\subsection{Create a Top 10 List}
\label{task_top_10_list}
GitHub Issue:
\href{https://github.com/smiths/AIMSS/issues}{https://github.com/smiths/AIMSS/issues}
\ad{To be created}

\noindent Refer to Section \ref{top_10_list}

\subsection{Review the Software List}
\label{task_software_list}
GitHub Issue:
\href{https://github.com/smiths/AIMSS/issues}{https://github.com/smiths/AIMSS/issues}
\ad{To be created}

\noindent Refer to Section \ref{software_list}

\subsection{Review the Domain Analysis}
GitHub Issue:
\href{https://github.com/smiths/AIMSS/issues}{https://github.com/smiths/AIMSS/issues}
\pmi{To be created}

\noindent Refer to Section \ref{domain_analysis}

\subsection{Review the AHP Ranking and Short List of Software Packages}
GitHub Issue:
\href{https://github.com/smiths/AIMSS/issues}{https://github.com/smiths/AIMSS/issues}
\pmi{To be created}

\noindent Refer to Section \ref{first_AHP}

\subsection{Comment on Final Short List Ranking}
\label{task_final_ranking}
GitHub Issue:
\href{https://github.com/smiths/AIMSS/issues}{https://github.com/smiths/AIMSS/issues}
\pmi{To be created}

\noindent Refer to Section \ref{final_AHP}

%
%\section{Deleted Notes}
%\subsection{Selecting Process}
%The process of creating and shortening the list of software packages is
%documented in
%\href{https://github.com/smiths/AIMSS/blob/master/StateOfPractice/Ao-Notes/Ao-N%otes.pdf}{Ao's
%	Notes} or Peter's Notes \ad{TBD}
%
%What do you think of our process of selecting software candidates?
%

\end{document}