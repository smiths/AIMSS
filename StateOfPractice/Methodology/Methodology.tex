\documentclass[letterpaper,cleveref]{lipics-v2019}

\usepackage{natbib}
\usepackage{booktabs}
\usepackage{amsmath,amsthm}
\usepackage{hyperref}

\usepackage{hyperref}
\hypersetup{
colorlinks=true,       % false: boxed links; true: colored links
linkcolor=red,          % color of internal links (change box color with
%linkbordercolor)
citecolor=blue,       % color of links to bibliography
filecolor=magenta,   % color of file links
urlcolor=cyan           % color of external links
}

%% Comments
\newif\ifcomments\commentstrue

\ifcomments
\newcommand{\authornote}[3]{\textcolor{#1}{[#3 ---#2]}}
\newcommand{\todo}[1]{\textcolor{red}{[TODO: #1]}}
\else
\newcommand{\authornote}[3]{}
\newcommand{\todo}[1]{}
\fi

\newcommand{\wss}[1]{\authornote{blue}{SS}{#1}} %Spencer Smith
\newcommand{\jc}[1]{\authornote{red}{JC}{#1}} %Jacques Carette
\newcommand{\oo}[1]{\authornote{magenta}{OO}{#1}} %Olu Owojaiye
\newcommand{\pmi}[1]{\authornote{green}{PM}{#1}} %Peter Michalski
\newcommand{\ad}[1]{\authornote{cyan}{AD}{#1}} %Ao Dong

\newcommand{\notdone}[1]{\textcolor{red}{#1}}
\newcommand{\done}[1]{\textcolor{black}{#1}}

%\oddsidemargin 0mm
%\evensidemargin 0mm
%\textwidth 160mm
%\textheight 200mm

\theoremstyle{definition}
\newtheorem{defn}{Definition}

\title{Methodology for Assessing the State of the Practice for Domain X} 
\author{Spencer Smith}{McMaster University, Canada}{smiths@mcmaster.ca}{}{}
\author{Jacques Carette}{McMaster University, Canada}{carette@mcmaster.ca}{}{}
\author{Olu Owojaiye}{McMaster University, Canada}{owojaiyo@mcmaster.ca}{}{}
\author{Peter Michalski}{McMaster University, Canada}{michap@mcmaster.ca}{}{}
\author{Ao Dong}{McMaster University, Canada}{donga9@mcmaster.ca}{}{}

\authorrunning{Smith et al.}  \Copyright{Spencer Smith and Jacques Carette and
Olu Owojaiye and Peter Michalski and Ao Dong}

\date{\today}

\hideLIPIcs
\nolinenumbers

\begin{document}
\maketitle

\begin{abstract}
	...
\end{abstract}

\tableofcontents

\section{Introduction} \label{SecIntroduction}

Purpose and scope of the document.  \wss{Needs to be filled in.  Should
	reference the overall research proposal, and the ``state of the practice''
	exercise in particular.}

\section{Overview of Steps in Assessing Quality of the Domain Software}

\begin{enumerate}
\item Identify domain.  (Provide criteria on a candidate domain.)
\item 
\end{enumerate}

\section{Identify Candidate Software}

\begin{enumerate}
	\item Must be open source.
	\item Must have GitHub repository.
\end{enumerate}

\section{Domain Analysis} \label{SecDomainAnalysis}

Commonality analysis.  Follow as for mesh generator (likely with less detail).

Commonality analysis document Steps:
\begin{enumerate}
	\item Introduction
	\item Overview of Domain
	\item Add Commonalities - Split into simulation, input, output, and nonfunctional requirements
	\item Add Variabilites - Split into simulation, input, output, system constraints, and nonfunctional requirements
	\item Add Parameters of Variation - Split into simulation, input, output, system constraints, and nonfunctional requirements
	\item Add Terminology, Definitions, Acronyms
\end{enumerate}


Commonality analysis for Lattice Boltzmann Solvers can be found \href{run:../Peter-Notes/Commonality-Analysis-LB-Systems.pdf}{here}.

\section{Empirical Measures}

\subsection{Raw Data}
Measures that can be extracted from on-line repos.

\ad{Still at brainstorm stage.}
\begin{itemize}
\item number of contributors
\item number of watches
\item number of stars
\item number of forks
\item number of clones
\item number of commits
\item number of total/code/document files
\item lines of total/logical/comment code
\item lines/pages of documents (can pdf be extracted?)
\item number of total/open/closed/merged pull requests
\item number of total/open/closed issues
\item number of total/open/closed issues with assignees
\end{itemize}

\subsection{Processed Data}
Metrics that can be calculated from the raw data.

\ad{Still at brainstorm stage.}
\begin{itemize}
\item percentage of total/open/closed issues with assignees -
Visibility/Transparency
\item lines of new code produced per person-day - Productivity
\item lines/pages of new documents produced per person-day - Productivity
\item number of issues closed per person-day - Productivity
\item percentage of comment lines in the code - maintainability \ad{Not Ao's
qualities}
\end{itemize}

\subsection{Tool Tests}
\ad{This section is currently a note of unorganized contents. Most parts will beremoved or relocated.}

\ad{This citation needs to be deleted later. It's here because my compiler
doesn't work with 0 citations}
\cite{Emms2019}

Most tests were done targeting to the repo of 3D Slicer
\href{https://github.com/tomgi/git_stats}{GitHub repo}

\subsubsection{git-stats}
\href{https://github.com/tomgi/git_stats}{GitHub repo}

Test results:
\href{http://git-stats-slicer.ao9.io/}{http://git-stats-slicer.ao9.io/} the
results are output as webpages, so I hosted for you to check. Data can be
downloaded as spreadsheets.

\subsubsection{git-of-theseus}
\href{https://github.com/erikbern/git-of-theseus}{GitHub repo}

Test results: It took about 100 minutes for one repo on a 8 core 16G ram Linux
machine. It only outputs graphs.

\subsubsection{hercules}
\href{https://github.com/src-d/hercules}{GitHub repo}

Test results: this one seems to be promising, but the installation is
complicated with various errors.

\subsubsection{git-repo-analysis}
\href{https://github.com/larsxschneider/git-repo-analysis}{GitHub repo}

\subsubsection{HubListener}
\href{https://github.com/pjmc-oliveira/HubListener}{GitHub repo}

The data that HubListener can extract.

Raw:
\begin{itemize}
\item Number of Files
\item Number of Lines
\item Number of Logical Lines
\item Number of Comments
\end{itemize}

Cyclomatic:
\href{https://www.geeksforgeeks.org/cyclomatic-complexity/}{Intro}
\begin{itemize}
\item Cyclomatic Complexity
\end{itemize}
 
Halstead:
\href{https://www.geeksforgeeks.org/software-engineering-halsteads-software-metrics/}{Intro}
\begin{itemize}
\item Halstead Effort
\item Halstead Bugs
\item Halstead Length
\item Halstead Difficulty
\item Halstead Time
\item Halstead Vocabulary
\item Halstead Volume
\end{itemize}

Test results: HubListener works well on the repo of itself, but it did not work
well on some other repos.

\subsubsection{gitinspector}
\href{https://github.com/ejwa/gitinspector}{GitHub repo}

Test results: it doesn't work well. Instead of creating output results, it
prints the results directly in the console.

\section{User Experiments}

Describe experiments with users to assess usability, performance etc.

\section{Analytic Hierarchy Process}

Describe process.  Domain expert review.

\section{Quality Specific Measures}

\subsection{\notdone{Installability} \oo{owner}}

\subsection{\notdone{Correctness} \oo{owner}}

\subsection{\notdone{Verifiability/Testability} \oo{owner}}

\subsection{\notdone{Validatability} \oo{owner}}

\subsection{\notdone{Reliability} \oo{owner}}

\subsection{\notdone{Robustness} \pmi{owner}}

\subsection{\notdone{Performance} \pmi{owner}}

\subsection{\notdone{Usability} \jc{owner}} 

\subsection{\notdone{Maintainability} \pmi{owner}}

\subsection{\notdone{Reusability} \pmi{owner}}

\subsection{\notdone{Portability} \pmi{owner}}

\subsection{\notdone{Understandability} \jc{owner}}

\subsection{\notdone{Interoperability} \ad{owner}}

\subsection{\notdone{Visibility/Transparency} \ad{owner}}

\subsection{\notdone{Reproducibility} \wss{owner}}

\subsection{\notdone{Productivity} \ad{owner}}

\subsection{\notdone{Sustainability} \wss{owner}}

\subsection{\notdone{Completeness} \ad{owner}}

\subsection{\notdone{Consistency} \ad{owner}}

\subsection{\notdone{Modifiability} \jc{owner}}

\subsection{\notdone{Traceability} \jc{owner}}

\subsection{\notdone{Unambiguity} \wss{owner}}

\subsection{\notdone{Verifiability} \wss{owner}}

\subsection{\notdone{Abstract} \wss{owner}}

\section{Using Data to Rank Family Members}

Describe AHP process (or similar).

\newpage

\bibliographystyle {plainnat}
\bibliography {../../CommonFiles/ResearchProposal}

\end{document}