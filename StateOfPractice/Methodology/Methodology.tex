\documentclass[letterpaper,cleveref]{lipics-v2019}

\usepackage{natbib}
\usepackage{booktabs}
\usepackage{amsmath,amsthm}
\usepackage{hyperref}
\usepackage{xcolor}

\hypersetup{
colorlinks=true,       % false: boxed links; true: colored links
linkcolor=red,          % color of internal links (change box color with
%linkbordercolor)
citecolor=blue,       % color of links to bibliography
filecolor=magenta,   % color of file links
urlcolor=cyan           % color of external links
}

%% Comments
\newif\ifcomments\commentstrue

\ifcomments
\newcommand{\authornote}[3]{\textcolor{#1}{[#3 ---#2]}}
\newcommand{\todo}[1]{\textcolor{red}{[TODO: #1]}}
\else
\newcommand{\authornote}[3]{}
\newcommand{\todo}[1]{}
\fi

\newcommand{\wss}[1]{\authornote{blue}{SS}{#1}} %Spencer Smith
\newcommand{\jc}[1]{\authornote{red}{JC}{#1}} %Jacques Carette
\newcommand{\oo}[1]{\authornote{magenta}{OO}{#1}} %Olu Owojaiye
\newcommand{\pmi}[1]{\authornote{green}{PM}{#1}} %Peter Michalski
\newcommand{\ad}[1]{\authornote{brown}{AD}{#1}} %Ao Dong

\newcommand{\notdone}[1]{\textcolor{red}{#1}}
\newcommand{\done}[1]{\textcolor{black}{#1}}

%\oddsidemargin 0mm
%\evensidemargin 0mm
%\textwidth 160mm
%\textheight 200mm

\theoremstyle{definition}
\newtheorem{defn}{Definition}

\title{Methodology for Assessing the State of the Practice for Domain X} 
\author{Spencer Smith}{McMaster University, Canada}{smiths@mcmaster.ca}{}{}
\author{Jacques Carette}{McMaster University, Canada}{carette@mcmaster.ca}{}{}
\author{Olu Owojaiye}{McMaster University, Canada}{owojaiyo@mcmaster.ca}{}{}
\author{Peter Michalski}{McMaster University, Canada}{michap@mcmaster.ca}{}{}
\author{Ao Dong}{McMaster University, Canada}{donga9@mcmaster.ca}{}{}

\authorrunning{Smith et al.}  \Copyright{Spencer Smith and Jacques Carette and
Olu Owojaiye and Peter Michalski and Ao Dong}

\date{\today}

\hideLIPIcs
\nolinenumbers

\begin{document}
\maketitle

\begin{abstract}
	...
\end{abstract}

\tableofcontents

\section{Introduction} \label{SecIntroduction}

Purpose and scope of the document.  \wss{Needs to be filled in.  Should
	reference the overall research proposal, and the ``state of the practice''
	exercise in particular.  Reference questions we are trying to answer.}

Note the following formatting conventions of this document. Red text denotes a link to information within the document. Purple text denotes a link to supplementary information that is external to this document. Blue text denotes a URL link.\pmi{needs to be added to intro to explain convention} 

\section{Research Questions}\label{ResearchQuestions}

Questions from Olu's Research Proposal:
	\begin{enumerate} 
		\item What artifacts are recommended for research software development?
\pmi{remove}
		\item What artifacts do current software projects produce?
\pmi{keep}
\item What information is captured by current software project artifacts?\pmi{keep}
\item What information is necessary to capture for sustainability?\pmi{remove}
\item What information is necessary to capture for productivity?\pmi{remove}
\item What commonalities exist among software artifacts?\pmi{need to review}
\item How can we improve sustainability in research softifacts development practices?\pmi{remove}
\item What are the consequences of technology tools towards sustainability and productivity?\pmi{remove}
\item What are attitudes of research software developers and engineer towards sustainability and productivity with respect to producing softifacts? \pmi{keep}
\item What are the factors that lead to sustainability and productivity?\pmi{remove}
\item What relationships exist between `high rated' software projects and sustainability with respect to research softifacts production? \pmi{remove}
\end{enumerate}

Questions from Overall Research Project folder
\begin{enumerate}
	\item What knowledge is currently captured in the artifacts generated by
	existing open source SCS projects?\pmi{keep}
	\item What principles, processes, methodologies and tools are used by existing open source SCS projects?\pmi{keep}
	\item What are the pain points" for developers working on SCS projects?
	What aspects of the existing processes, methodologies and tools do
	they consider could potentially be improved?\pmi{keep}
	\item For a given software product (softifacts), how can we produce a reasonable measure of the software qualities (correctness, reliability, usability etc.) with a few hours of effort?\pmi{remove}
	\item For a given software product (softifacts), how can we produce a meaningful measure of usability with a few days worth of effort?\pmi{remove}
	\item For a given software product (softifacts), how can we produce a meaningful measure of maintainability (with respect to likely changes) with
	a few days worth of effort?\pmi{remove}
	\item For a given software product (softifacts), how can we produce a meaningful measure of reproducibility/replicability with a few days worth of
	effort?\pmi{remove}
	\item With a reasonable effort and time, how can we compare the productivity between different processes, methodologies and tools?\pmi{remove}
	\item Can a correlation be identified between software projects that use
	best practices for the development process and methodologies and improvement in the qualities of usability, performance and reproducibility/replicability?\pmi{remove}
	\item Ignoring the time and effort required, what knowledge is necessary to
	capture for sustainable SCS? How can this knowledge best be documented in software artifacts?\pmi{remove}
	\item How can Model Driven Engineering and Code Generation be t in the
	development process for SCS to remove as many repetitive, tedious and
	error prone tasks as possible? We will call this the ideal process."\pmi{remove}
	\item How well do existing modelling and code generation tools do at implementing the ideal process"? Candidates for comparison include Drasil
	and Epsilon.\pmi{remove}
	\item How well does a model driven approach to compared to a traditional
	approach for usability, productivity and sustainability?\pmi{remove}
	\item What are the attitudes and preferences of typical users towards an
	implementation of the ideal process"?\pmi{remove}
\end{enumerate}

%In general questions:
%
%\begin{enumerate}
%\item Comparison between domains
%\item How to measure qualities
%\item How does the quality compare for projects with the most resources to %those
%  with the fewest?
%\item What skills/knowledge are needed by future developers?
%\item How can the development process be improved?
%\item What are the common pain points?
%\end{enumerate}
%
%For each domain questions''
%
%\begin{enumerate}
%\item Best examples within the domain
%\item What software artifacts?
%\item What are the pain points?
%\item Any advice on what can be done about the pain points?
%\end{enumerate}
%
%Measure the effort invested and the reward.  Related to sustainability.
%
%Collect the data and see what conclusions follow.  For an individual domain,
%between domains.  The process isn't so much about ranking the software as it is
%about looking at the software closely and see what conclusions arise.  The
%measurements are intended to force scrutiny, from different perspectives.

\section{Overview of Steps in Assessing Quality of the Domain Software}\label{StepsAQDS}

\begin{enumerate}
\item Start with state of practice research questions. (Section~\ref{ResearchQuestions}) \pmi{To be completed}
\item Identify the domain. (Section~\ref{SecIdentifyDomain}) \pmi{To be completed based on suggestions}
\item \emph{Domain Experts}: Create a top ten list of software packages in the domain. (\href{run:Meeting Agenda with Domain Experts.pdf}{Meeting Agenda with Domain Experts})
\item Brief the Domain Experts on the overall objective, research proposal, research questions, measurement template, survey for short list projects, usability tests, performance benchmarks, maintainability experiments. (\href{run:Meeting Agenda with Domain Experts.pdf}{Meeting Agenda with Domain Experts})
\item Identify broad list of candidate software packages in the domain. (Section~\ref{SecIdentifyCandSoft}) \pmi{To be reviewed}
\item Preliminary filter of software packages list. (Section~\ref{SecInitialFilter}) \pmi{To be reviewed}
\item \emph{Domain Experts}: Review domain software list. (\href{run:Meeting Agenda with Domain Experts.pdf}{Meeting Agenda with Domain Experts})
\item Domain Analysis. (Section~\ref{SecDomainAnalysis})\pmi{To be reviewed}
\item \emph{Domain Experts}: Vet domain analysis. (\href{run:Meeting Agenda with Domain Experts.pdf}{Meeting Agenda with Domain Experts}) \pmi{This was part of the original Steps in Assessing Quality list. We need to add it to the Meeting Agenda}
\item Gather source code and documentation for each prospective software package.
\item Collect empirical measures. (Section~\ref{SecEmpiricalMeasures}) \pmi{To be completed - go over how to clean this up in meeting, or through an issue}
\item Measure using measurement template. (Section~\ref{SecShallowMeasure}) \pmi{To be reviewed}
\item Use AHP process to rank the software packages. (Section~\ref{SecAHP}) \pmi{To be reviewed}
\item Identify a short list of top software packages, typically four to six, for deeper exploration according to the AHP rankings of the measurements.
\item \emph{Domain Experts}: Vet AHP ranking and short list. (\href{run:Meeting Agenda with Domain Experts.pdf}{Meeting Agenda with Domain Experts})
\item With short list:
\begin{enumerate}
\item Survey developers (\href{run:Questions to Developers.pdf}{Questions to Developers})
\item Usability experiments (\href{run:User Experiments.pdf}{User Experiments})
\item Performance benchmarks\pmi{note: this is still in consideration}
\item Maintainability experiments\pmi{note: these experiments need to be completed - to be discussed at next meeting}
\end{enumerate}
\item Rank short list. (Section~\ref{SecRankShortList}) \pmi{To be completed}
\item Document answers for research questions.
\end{enumerate}

\wss{The domain expert is involved in multiple steps in the process.  How best
  to get their feedback?  The domain experts are busy and are unlikely to devote
  significant time to the project.  We need to quickly get to the point.  Maybe
  something around task based inspection?  Directed interview?}


\section{How to Identify the Domain} \label{SecIdentifyDomain}
\pmi{add introduction}
\pmi{add a definition in the introduction for what we mean by research software}
\begin{enumerate}	
	\item The domain must fall within the research software scope.
	\item The domain must be well understood.
	\item There must be a community of people studying the domain.
	\item A preliminary search, or discussion with experts, suggests that there will be numerous, at least about 15, candidate software packages to study.
	\item The software packages must have open source options. 
\end{enumerate}
\pmi{SS:"Add examples of well understood domains. We can include pointers to previous "state of the practice" projects as examples."}

\section{How to Identify Candidate Software} \label{SecIdentifyCandSoft}
The candidate software can be found through search engine queries targeting authoritative lists of software from the domain, such as lists on GitHub and swMATH, and through searching domain related publications. Domain experts are also asked for their suggestions and are asked to review the list. The candidate software should have the following properties:

\begin{enumerate}
	\item Major function(s) must fall within the identified domain.
	\item Must have viewable source code.
	\item Ideally have a git repository or ability to gather empirical measures found in Section \ref{SecEmpiricalMeasures}.
	\item \ad{``Ideally have the latest release or source code commit within the last 5 years.'' How about this one? Do we care about the ``vintage'' software?}\pmi{I tested some software that has not been updated in about 8 years, as this software was mentioned in multiple places and looked professional. That being said, age should be noted as something that is considered along with other qualities. "Ideally the latest release or source code commit should be relatively recent, such as within the last 5 years, unless the candidate software appears to be well recommended and currently in use."}
\end{enumerate}

\section{How to Initially Filter the Software List} \label{SecInitialFilter}
The initial list of candidate software packages should first be filtered to a manageable size. Our target is about 30 packages. We want the list to include packages that are not marked as incomplete and are at least somewhat organized and understandable. We also want the list to be a good representation of the software domain, with packages having enough in common with respect to functionality that it makes sense to compare the software. The initial list of candidate software should be filtered using the following properties:

\begin{enumerate}
	\item Organization - The software and any related documentation should appear to be easy to gather and understand.
	\item Available documentation - The purpose of the software and the installation and usage procedures should appear to be moderately clear or easy to find.
	\ad{What if installation or usage procedures are unavailable? I think we're measuring that if the software have installation instructions and user manuals,
	so perhaps they should be kept even if these docs are missing.}
    \pmi{How are we installing the software if such documents and procedures are missing? Should we expect users to be able to install? I did not encounter a situation where even basic installation steps were missing.}
	\item Status - The software cannot be marked as incomplete or in an initial development phase.
	\item \ad{What if there are still more candidates than we need? Should we also filter out the ``older'' or ``less popular'' ones if needed? }\pmi{Good question. Let's discuss during meeting.}
\end{enumerate}


Copies of both the initial and filtered lists should be kept for traceability purposes.

\section{Domain Analysis} \label{SecDomainAnalysis}
\pmi{SS:"We should have more of an introduction on what is a commonality analysis, and more generally, what is a program family. We should also have more of a literature review, and a pointer to examples. Some publications to look at include ArdisAndWeiss1997, Chen2003, SmithAndChen2004, SmithEtAl2008, SmithMcCutchanAndCarette2017, Weiss1998"}
The domain analysis consists of a commonality analysis of the software packages. Its purpose is to show the relationships between these packages, and to facilitate an understanding of the informal specification and development of them. The final result of the analysis will be tables of commonalities, variabilities and parameters of variation.

\noindent Steps to produce a commonality analysis:
\begin{enumerate}
\item Write an Introduction
\item Write the Overview of Domain
\item List Commonalities
\item List Variabilities
\item List Parameters of Variation
\item Add Terminology, Definitions, Acronyms
\end{enumerate}

A sample commonality analysis for Lattice Boltzmann Solvers can be found
\href{https://github.com/smiths/AIMSS/blob/master/StateOfPractice/Peter-Notes/Commonality-Analysis-LB-Systems.pdf}{here}.

\section{Empirical Measures} \label{SecEmpiricalMeasures}
\pmi{Add introduction.  We should orient the reader on the purpose of this section, and provide a brief roadmap of the subsections.}

\subsection{Raw Data}
The following raw data measures are extracted from repositories:
\pmi{SS:"we should say why we are collecting this data. It is okay to say that we are collecting data that is reasonably easy to collect. We will then put the data together and analyze it. However, we should also relate the measures we are collecting to questions we want to answer. Ideally, we will relate the data to our research questions. For instance, we are collecting some of the data to see how large a project is. More files, more lines of code, etc, means a larger project. The number of developers is another measure of size. Other measures are intended to ascertain a project's popularity, like the number of stars and forks. Other measures are intended to see how active the project is, like the measures related to the history of commits."}
\begin{itemize}
\item Number of stars.
\item Number of forks.
\item Number of people watching the repository.
\item Number of open pull requests.
\item Number of closed pull requests.	
\item Number of developers.	
\item Number of open issues.
\item Number of closed issues.
\item Initial release date.
\item Last commit date.
\item Programming languages used.
\item Number of text-based files.
\item Number of total lines in text-based files.
\item Number of code lines in text-based files.
\item Number of comment lines in text-based files.
\item Number of blank lines in text-based files.
\item Number of binary files.  
\item Number of total lines added to text-based files.
\item Number of total lines deleted from text-based files.
\item Number of total commits.
\item Numbers of commits by year in the last 5 years. (Count from as early as possible if the project is younger than 5 years.) 
\item Numbers of commits by month in the last 12 months.
\end{itemize}


\subsection{Processed Data}
The following measures are calculated from the raw data:

\begin{itemize}
\item Status of software package (dead or alive). Alive is defined as the presence of repository commits or software package version releases in the last 18 months.
\item Percentage of identified issues that are closed.
\item Percentage of code that is comments.
\end{itemize}

\subsection{Tools}
\href{https://github.com/tomgi/git_stats}{GitStats} is used to measure the number of binary files as well as the number of added and deleted lines in a repository. The tool is also used to measuring the number of commits over different intervals of time. \href{https://github.com/boyter/scc}{Sloc Cloc and Code (scc)} is used to measure the number of text based files as well as the number of total, code, comment, and blank lines in a repository. These tools were selected due to their installability, usability, and ability to gather the above empirical measures. A guide for installing and running them can be found \href{run:A Guide to Empirical Measures.pdf}{here}.

\section{Measure Using Measurement Template} \label{SecShallowMeasure}
For each software package fill out one column of the \href{run:Combined_MeasurementTemplate_EmpiricalMeasures.xlsx}{Measurement Template} spreadsheet found in our repository. Follow these steps:
\ad{Maybe a one or two sentences introduction to the template?} \pmi{SS:"A screenshot of a portion of the template would be very helpful here"}

\pmi{SS:"please add an estimate of how long each measurement typically takes"}

\pmi{SS:"explain any tricky parts with interpreting the parameters of variation for measuring a particular indicator. A few example would help."}


\begin{enumerate} 
	\item Gather the summary information into the top section of the document
	\item Using the GitStats tool found in Section \ref{gitstats} gather the measurements for the Repo Metrics - GitStats section found near the bottom of the document
	\item Using the SCC tool found in Section \ref{scc} gather the measurements for the Repo Metrics - SCC section found near the bottom of the document
	\item If the software package is found on git, gather the measurements for the Repo Metrics - GitHub section found near the bottom of the document
	\item Review installation documentation and attempt to install the software package on a virtual machine
	\item Gather the measurements for installability
	\pmi{SS:"we should mention that the project owners will be contacted for help with installation, if necessary, but that there is a cap on the amount of time spent on installation"}
	\item Gather the measurements for correctness and verifiability
	\item Gather the measurements for surface reliability
	\item Gather the measurements for surface robustness
	\item Gather the measurements for surface usability
	\item Gather the measurements for maintainability
	\item Gather the measurements for reusability
	\item Gather the measurements for surface understandability
	\item Gather the measurements for visibility and transparency
	\item Assign a score out of ten for each quality \pmi{SS:"add some text about how the measure out of 10 is assigned for each quality", link to scoring document}
\end{enumerate}
\pmi{SS:"add something about the sensitivity analysis tool and its purpose", explain what the tool is and its purpose, indicate that it will be used in the next section to verify rankings, add it to the next section}

\section{Analytic Hierarchy Process} \label{SecAHP}
The Analytical Hierarchy Process (AHP) is a decision-making technique that can be used when comparing multiple options by multiple criteria. In our work AHP is used for comparing and ranking the software packages of a domain using the quality scores that are gathered in the \href{run:Combined_MeasurementTemplate_EmpiricalMeasures.xlsx}{Measurement Template}. AHP performs a pairwise analysis between each of the quality options using a matrix which is then used to generate an overall score for each software package for the given criteria. \cite{SmithEtAl2016} shows how AHP is applied to ranking software based on quality measures. We have developed a tool for conducting this process. The tool includes an AHP JAR script and a sensitivity analysis JAR script that is used to ensure that the software package rankings are appropriate with respect to the uncertainty of the quality scores. The README file of the tool is found \href{run:../AHP2020/LBM/README.txt}{here}. This file outlines the requirements for, and configuration and usage of, the JAR scripts. The JAR scripts, source code, and required libraries are located in the same folder as the README file.

\section{Rank Short List} \label{SecRankShortList}
Rank using pairwise comparison of short list software packages with respect to usability survey results.

\pmi{SS: let's ask the people that do the usability/performance/maintainability experiments to do an AHP pair-wise comparison between the short-list software packages}

\section{Quality Specific Measures}

\subsection{\notdone{Installability} \oo{owner}}

\subsection{\notdone{Correctness} \oo{owner}}

\subsection{\notdone{Verifiability/Testability} \oo{owner}}

\subsection{\notdone{Validatability} \oo{owner}}

\subsection{\notdone{Reliability} \oo{owner}}

\subsection{\notdone{Robustness} \pmi{owner}}

\subsection{\notdone{Performance} \pmi{owner}}

\subsection{\notdone{Usability} \jc{owner}} 

\subsection{\notdone{Maintainability} \pmi{owner}}

\subsection{\notdone{Reusability} \pmi{owner}}

\subsection{\notdone{Portability} \pmi{owner}}

\subsection{\notdone{Understandability} \jc{owner}}

\subsection{\notdone{Interoperability} \ad{owner}}

\subsection{\notdone{Visibility/Transparency} \ad{owner}}

\subsection{\notdone{Reproducibility} \wss{owner}}

\subsection{\notdone{Productivity} \ad{owner}}

\subsection{\notdone{Sustainability} \wss{owner}}

\subsection{\notdone{Completeness} \ad{owner}}

\subsection{\notdone{Consistency} \ad{owner}}

\subsection{\notdone{Modifiability} \jc{owner}}

\subsection{\notdone{Traceability} \jc{owner}}

\subsection{\notdone{Unambiguity} \wss{owner}}

\subsection{\notdone{Verifiability} \wss{owner}}

\subsection{\notdone{Abstract} \wss{owner}}

\section{Using Data to Rank Family Members}

Describe AHP process (or similar).

\appendix
\section{Appendix}
\subsection{Survey for the Selected Projects}
\ad{Several questions are borrowed from \href{https://gitlab.cas.mcmaster.ca/smiths/pub/-/blob/master/Jegatheesan2016.pdf}{Jegatheesan2016}, and needed to be cited later.}
\subsubsection{Information about the developers and users}
\begin{enumerate}
\item Interviewees' current position/title? degrees?
\item Interviewees' contribution to/relationship with the software?
\item Length of time the interviewee has been involved with this software?
\item How large is the development group?
\item What is the typical background of a developer?
\item How large is the user group?
\item What is the typical background of a user?
\end{enumerate}

\subsubsection{Information about the software}

\begin{enumerate}
\item \ad{General} What is the most important software quality(ies) to your work? (set of selected qualities plus "else")
\item \ad{General} Are there any examples where the documentation helped? If yes, how it helped. ({yes$^*$, no})
\item \ad{General} Is there any documentation you feel you should produce and do not? If yes, what is it and why? ({yes$^*$, no})
\item \ad{Completeness} Do you address any of your quality concerns using documentation? If yes, what are the qualities and the documents. ({yes$^*$, no})
\item \ad{Visibility/Transparency} Is there a certain type of development methodologies used during the development? (\{Waterfall, Scrum, Kanban, else\})
\item \ad{Visibility/Transparency} Is there a clearly defined development process? If yes, what is it. (\{yes$^*$, no\})
\item \ad{Visibility/Transparency} Are there any project management tools used during the development? If yes, what are they. (\{yes$^*$, no\})
\item \ad{Visibility/Transparency} Going forward, will your approach to documentation of requirements and design
change? If not, why not. (\{yes, no$^*$\})
\item \ad{Correctness and Verifiability} During the process of development, what tools or techniques are used to build confidence of correctness? (string)
\item \ad{Correctness and Verifiability} Do you use any tools to support testing? If yes, what are they. (e.g. unit testing tools, regression testing suites) (\{yes$^*$, no\})
\item \ad{Correctness and Verifiability} Is there any document about the requirements specifications of the program? If yes, what is it. (\{yes$^*$, no\})
\item \ad{Portability} Do you think that portability has been achieved? If yes, how? (\{yes$^*$, no\})
\item \ad{Maintainability} How was maintainability considered in the design? (string)
\item \ad{Maintainability} What is the maintenance type? (set of \{corrective, adaptive, perfective,
unclear\})
\item \ad{Reusability} How was reusability considered in the design? (string)
\item \ad{Reusability} Are any portions of the software used by another package? If yes, how they are used. ({yes$^*$, no})
\item \ad{Reproducibility} Is reproducibility important to you? ({yes$^*$, no})
\item \ad{Reproducibility} Do you use tools to help reproduce previous software results? If yes, what are they. (e.g. version control, configuration management) ({yes$^*$, no})
\item \ad{Completeness} Is any of the following documents used during the development? ({yes$^*$, no})
\item \ad{General} Will this experience influence how you develop software? Do you see yourself maintaining the same level of documentation, tool support as you go forward? (string)
\begin{itemize}
\item Module Guide
\item Module Interface Specification
\item Verification and Validation Plan
\item Verification and Validation Report
\end{itemize}
\end{enumerate}

\subsection{Empirical Measures Considerations - Raw Data}

Measures that can be extracted from on-line repos.

\ad{Still at brainstorm stage.}
\begin{itemize}
	\item number of contributors
	\item number of watches
	\item number of stars
	\item number of forks
	\item number of clones
	\item number of commits
	\item number of total/code/document files
	\item lines of total/logical/comment code
	\item lines/pages of documents (can pdf be extracted?)
	\item number of total/open/closed/merged pull requests
	\item number of total/open/closed issues
	\item number of total/open/closed issues with assignees
\end{itemize}

Instead of only focus on the current status of the above numbers, we may find
the time history of them to be more valuable. For example, the number of
contributors over time, the number of lines of code over time, the number of
open issues over time, etc.

\subsection{Empirical Measures Considerations - Processed Data}
Metrics that can be calculated from the raw data.

\ad{Still at brainstorm stage.}
\begin{itemize}
	\item percentage of total/open/closed issues with assignees -
	Visibility/Transparency
	\item lines of new code produced per person-month - Productivity
	\item lines/pages of new documents produced per person-month - Productivity
	\item number of issues closed per person-month - Productivity
	\item percentage of comment lines in the code - maintainability \ad{Not Ao's
		qualities}
\end{itemize}

In the above calculations, a month can be determined to be 30 days.

\subsection{Empirical Measures Considerations - Tool Tests}
\ad{This section is currently a note of unorganized contents. Most parts will beremoved or relocated.}

\ad{This citation needs to be deleted later. It's here because my compiler
	doesn't work with 0 citations}
\cite{Emms2019}

Most tests were done targeting to the repo of 3D Slicer
\href{https://github.com/tomgi/git_stats}{GitHub repo}

\subsubsection{git-stats}
\href{https://github.com/tomgi/git_stats}{GitHub repo}

Test results:
\href{http://git-stats-slicer.ao9.io/}{http://git-stats-slicer.ao9.io/} the
results are output as webpages, so I hosted for you to check. Data can be
downloaded as spreadsheets.


\subsubsection{scc}
\href{https://github.com/boyter/scc}{GitHub repo}


\subsubsection{git-of-theseus}
\href{https://github.com/erikbern/git-of-theseus}{GitHub repo}

Test results: It took about 100 minutes for one repo on a 8 core 16G ram Linux
machine. It only outputs graphs.

\subsubsection{hercules}
\href{https://github.com/src-d/hercules}{GitHub repo}

Test results: this one seems to be promising, but the installation is
complicated with various errors.

\subsubsection{git-repo-analysis}
\href{https://github.com/larsxschneider/git-repo-analysis}{GitHub repo}

\subsubsection{HubListener}
\href{https://github.com/pjmc-oliveira/HubListener}{GitHub repo}

The data that HubListener can extract.

Raw:
\begin{itemize}
	\item Number of Files
	\item Number of Lines
	\item Number of Logical Lines
	\item Number of Comments
\end{itemize}

Cyclomatic:
\href{https://www.geeksforgeeks.org/cyclomatic-complexity/}{Intro}
\begin{itemize}
	\item Cyclomatic Complexity
\end{itemize}

Halstead:
\href{https://www.geeksforgeeks.org/software-engineering-halsteads-software-metrics/}{Intro}
\begin{itemize}
	\item Halstead Effort
	\item Halstead Bugs
	\item Halstead Length
	\item Halstead Difficulty
	\item Halstead Time
	\item Halstead Vocabulary
	\item Halstead Volume
\end{itemize}

Test results: HubListener works well on the repo of itself, but it did not work
well on some other repos.

\subsubsection{gitinspector}
\href{https://github.com/ejwa/gitinspector}{GitHub repo}

Test results: it doesn't work well. Instead of creating output results, it
prints the results directly in the console.





\newpage

\bibliographystyle {plainnat}
\bibliography {../../CommonFiles/ResearchProposal}

\end{document}