\documentclass[12pt]{article}

\usepackage{natbib}
\usepackage{booktabs}
\usepackage{longtable}
\usepackage{xspace}
\usepackage{xcolor}
\usepackage{fullpage}
\usepackage{hyperref}

\hypersetup{
colorlinks=true,       % false: boxed links; true: colored links
linkcolor=red,          % color of internal links (change box color with
%linkbordercolor)
citecolor=blue,       % color of links to bibliography
filecolor=magenta,   % color of file links
urlcolor=cyan           % color of external links
}

%% Comments
\newif\ifcomments\commentstrue

\ifcomments
\newcommand{\authornote}[3]{\textcolor{#1}{[#3 ---#2]}}
\newcommand{\todo}[1]{\textcolor{red}{[TODO: #1]}}
\else
\newcommand{\authornote}[3]{}
\newcommand{\todo}[1]{}
\fi

\newcommand{\wss}[1]{\authornote{blue}{SS}{#1}} %Spencer Smith
\newcommand{\jc}[1]{\authornote{red}{JC}{#1}} %Jacques Carette
\newcommand{\oo}[1]{\authornote{magenta}{OO}{#1}} %Olu Owojaiye
\newcommand{\pmi}[1]{\authornote{green}{PM}{#1}} %Peter Michalski
\newcommand{\ad}[1]{\authornote{brown}{AD}{#1}} %Ao Dong

%% Global noindent
% \setlength\parindent{0pt}

\title{Interactions with Domain Experts} 
\author{Ao Dong}
\date{\today}

\begin{document}

\maketitle

The purpose of the document is to describe the interactions with the domain
experts.  For each domain that is measured, a domain expert is a member of the
assessment team.  Their role is to provide their expertise on the domain and its
existing software.   The domain expert will typically be a faculty member.
Ideally their graduate students will also assist with the measurement process.

The interactions with the domain experts and their graduate students are divided
into 3 stages:

\begin{itemize}
\item Stage 1 - Introductory Meeting in Section \ref{introductory_meeting}. In
this meeting, we will be focusing on introducing the background and objectives
of the project and vetting the initial software list.  We will also discuss the
resources that the domain expert can make available for Stage 2.
\item Stage 2 - This stage isn't isolated to a single meeting, but is rather a
  series of measurement and assessment tasks, as described in Section
  \ref{qualities_measurements}.  During this stage the domain expert and their
  students will help with vetting the Domain Analysis, usability
  experiments, and modifiability experiments.  We will also interview the domain
  expert in this step, using similar questions to those that we are asking the
  developers.  (By developers, we mean the programmers that have created the
  programs in the domain that are being measured.)
\item Stage 3 - The domain expert interaction concludes with a final meeting, as
  detailed in Section \ref{concluding_meeting}.  In this meeting we will discuss
  the results of Stage 2 and go over the final ranking.
\end{itemize}

\section{Introductory Meeting}
\label{introductory_meeting}

\subsection{Prerequisites}
This section contains the tasks of preparing for the introductory meeting. These
tasks need to be finished before the meeting.

\subsubsection{Top 10 Software List}
\label{top_10_list}

The domain experts need to prepare the top 10 software packages of their choice.
Feedback can be provided in Task \ref{task_top_10_list}.

\subsection{About this Meeting}
Presenter: \ad{TBD}

The meeting will be roughly one hour. We will go
through all the steps in Section \ref{introductory_meeting}.

In the interest of time, some topics will only be discussed briefly during this
meeting, and tasks will be assigned to domain experts to finish later. For
example, Task \ref{task_usability_tests} in Section \ref{usability_tests}.

We will create issues on GitHub to track the progress of the tasks. Details can
be found in Appendix \ref{proj_mgmt_tools}. The tasks can also be found in
Appendix \ref{tasks_domain_experts}.

\subsection{Overall Objective}
\label{overall_objective}
Presenter: Dr.\ Spencer Smith and Dr.\ Jacques Carette \ad{TBD}

As described in the document 
\href{https://github.com/smiths/AIMSS/blob/master/OverallResearchProposal/ObjectivesAndResearchQuestions.pdf}{Objectives
And Research Questions}, the overall objective asks ``What is needed to produce
\{software + artifacts\} sustainably?'' The definitions of \emph{software},
\emph{artifacts} and \emph{sustainably} can be found in the same document. The
term \emph{softifacts} will be used to refer the combination of software and
artifacts.

According to the above document, to answer the above question, we will need to
find answers for these two questions,
\begin{enumerate}
\item ``What softifacts, in addition to the code, are needed?''
\item ``What software engineering principles, processes and methodologies should
be employed?''
\end{enumerate}

\subsection{Research Proposal}
\label{research_proposal}
Presenter: Dr.\ Spencer Smith \ad{TBD}

The research proposal is documented in
\href{https://github.com/smiths/AIMSS/blob/master/OverallResearchProposal/ResearchProposal.pdf}{Research
Proposal}.

\subsubsection{State of the Practice}

MEng students will perform assessments to the state of the practice in several
Scientific Computing Software (SCS) domains. Domain analysis will be conducted
(related to Section \ref{domain_analysis}). About 20 to 30 software projects
will be chosen for the measurements, and the software list is in Section
\ref{software_list}. The questions and metrics (including empirical
measurements) to assess the projects are documented in the Section
\ref{measurement_template}.

Additionally, we will also perform Usability Tests (Section
\ref{usability_tests}) and Modifiability Experiment (Section
\ref{modifiability_experiment})

To compare each software quality and the overall quality, we will use the
Analytic Hierarchy Process (AHP) to rank software projects in each domain. The
first ranking will be for all projects, and the final ranking will be for the
short list projects (more details in Section \ref{survey_short_list_projects}).
The ranking results will be the topic for our last meeting - the concluding
meeting in Section \ref{concluding_meeting}.

\subsubsection{Impact of MDE on the Sustainability of SCS}

\subsection{Research Questions}
\label{research_questions}
Presenter: Dr.\ Spencer Smith \ad{TBD}

The research questions are documented in the section Current State of the
Practice Research Questions of
\href{https://github.com/smiths/AIMSS/blob/master/OverallResearchProposal/ObjectivesAndResearchQuestions.pdf}{Objectives
And Research Questions}. Guided by overall objective in Section
\ref{overall_objective}, these questions focus on researching on ``i) knowledge,
ii) principles, processes
and methodologies, iii) software qualities and iv) the necessary investment of
time and energy''.

Do you have any thoughts on one of the questions - What are the ``pain points''
for developers working on SCS projects? What aspects of the existing processes,
methodologies and tools do they consider could potentially be improved?

\subsection{Vetting the List of Candidate Software}
\label{software_list}

The software list is
\href{https://docs.google.com/spreadsheets/d/122wU0v3ZtvDcqy8C4zKJ89kU-8fXAbo3Mzn6vcVXOi0/edit?usp=sharing}{Medical
	Imaging Software List} or Lattice Boltzmann Solver List. \ad{TBD}

What do you think of the list of the selected software?

Is there any software you would like to remove from or add to the list?

Feedback can be provided later in Task \ref{task_software_list}.

\subsection{Publication}
Presenter: Dr.\ Spencer Smith and Dr.\ Jacques Carette \ad{TBD}

\begin{itemize}
\item Where should we publish this paper?
\item Who are our targeted readers?
\end{itemize}

\subsection{Schedule \& Team}
Presenter: \ad{TBD}

\begin{itemize}
\item We propose to start this project from ... and finish it before ... . What
do you think?
\item Our meeting schedule is ... does it fit your plan?
\item We need ... extra members from your team, each working ... hours per week
for ... weeks. What do you think?
\item Are there any other experts you think we should keep them involved?
\end{itemize}

\subsection{Introducing the Next Steps}

\section{Qualities \& Measurements}
\label{qualities_measurements}

\subsection{Vetting the Domain Analysis}
\label{domain_analysis}
Task leader: Ao and Peter

The domain analysis can be found in Section 3.2 of
\href{https://github.com/Ao99/MISEG/blob/master/docs/SRS/SRS.pdf}{SRS for
Medical Imaging Software} or
\href{https://github.com/peter-michalski/LatticeBoltzmannSolvers/blob/master/docs/SRS/CA.pdf}{CA
of Lattice Boltzmann Solvers}.

Please review the Figure 1 in \href{https://github.com/Ao99/MISEG/blob/master/docs/SRS/SRS.pdf}{SRS for
Medical Imaging Software} or Figure/Table ? \ad{TBD. It can be a summary of the
commonalities, variabilities and parameters of variation in tables, or possibly
graphs.} in
\href{https://github.com/peter-michalski/LatticeBoltzmannSolvers/blob/master/docs/SRS/CA.pdf}{CA
of Lattice Boltzmann Solvers}.
What do you think of our analysis for the scope of the family, commonalities, and variabilities  for this domain?

Our preference is to have feedback from several graduate students. Feedback can be provided in Task \ref{task_domain_analysis}.

\subsection{Measurement Template}
\label{measurement_template}
Task leader: Ao or Peter

The template for software quality measurements can be found at
\href{https://github.com/smiths/AIMSS/blob/master/StateOfPractice/Combined_MeasurementTemplate_EmpiricalMeasures.xlsx}{Measurement
	Template}.

As described in Task \ref{task_measurement_template}, domain experts are
expected to take a look at this template. Feedback are welcome but not required.

\subsection{Survey for Short List Projects}
\label{survey_short_list_projects}
Task leader: Ao or Peter

After the first round of measurement, we would like to select a short list of
projects (about 5 to 10 of them for each domain) and directly ask questions to the
developers.

Considering the fact that some development teams may ignore our requests for
this interview, We would like to send invitations to all of the teams developing
the software packages in the list. We estimate that there will be a smaller
proportion of them accepting our request. What do you think of
this process? Feedback can be provided in Task \ref{task_review_short_list}.

The survey questions to the developers can be found at
\href{https://github.com/smiths/AIMSS/blob/master/StateOfPractice/Methodology/Questions%20to%20Developers.pdf}{Questions
	to Developers}.

We also would like to ask the domain experts a modified version of the above
questions. The purpose will be to learn how our domain experts and their
students create software. If we time it right, the interview with the domain
experts is also a chance to do a dry run with our questions. We can use Task
\ref{task_short_list_survey} to manage the progress of the survey with domain
experts.

\subsection{Usability Tests}
\label{usability_tests}
Task leader: Olu

A brief introduction. \ad{To be written}

The usability test can be found at ... \ad{TBD}

Feedback can be provided later in Task
\ref{task_usability_tests}. We can also use the same task to manage the progress of the tests.

\subsection{Modifiability Experiment}
\label{modifiability_experiment}
Task leader: \ad{TBD}

A brief introduction. \ad{To be written} 

The modifiability experiment can be found at.. \ad{TBD}

Feedback can be provided later in Task
\ref{task_modifiability_experiment}. We can also use the same task to manage the
progress of the tests.

\section{Concluding Meeting}
\label{concluding_meeting}

\subsection{First AHP Ranking}
\label{first_AHP}
Presenter: \ad{TBD}

A brief introduction about the ranking results.

Do you have any feedback? Feedback can be provided in Task \ref{task_first_ranking}.

\subsection{Final AHP Ranking}
\label{final_AHP}
Presenter: \ad{TBD}

A brief introduction about the ranking results for the short list software packages.

Do you have any feedback? Feedback can be provided in Task \ref{task_final_ranking}.

\appendix
\section{Project Management Tools}
\label{proj_mgmt_tools}
We would like to use GitHub for task tracking and document version control. The
repository for this project is
\href{https://github.com/smiths/AIMSS}{https://github.com/smiths/AIMSS}.

Tasks will be created as issues on the repository and assigned to individual
stakeholders. An open issue usually means a task to be fulfilled, and it can be
closed once it is finished. Here is an example of an issue
\href{https://github.com/smiths/AIMSS/issues/19}{https://github.com/smiths/AIMSS/issues/19}.

\newpage

\section{Tasks for Domain Experts}
\label{tasks_domain_experts}

\subsection{Create a Top 10 List}
\label{task_top_10_list}
GitHub Issue:
\href{https://github.com/smiths/AIMSS/issues}{https://github.com/smiths/AIMSS/issues}
\ad{To be created}

\noindent Refer to Section \ref{top_10_list}

\subsection{Vet the Software List}
\label{task_software_list}
GitHub Issue:
\href{https://github.com/smiths/AIMSS/issues}{https://github.com/smiths/AIMSS/issues}
\ad{To be created}

\noindent Refer to Section \ref{software_list}

\subsection{Vet the Domain Analysis}
\label{task_domain_analysis}
GitHub Issue:
\href{https://github.com/smiths/AIMSS/issues}{https://github.com/smiths/AIMSS/issues}
\pmi{To be created}

\noindent Refer to Section \ref{domain_analysis}

\subsection{Read the Measurement Template}
\label{task_measurement_template}
GitHub Issue:
\href{https://github.com/smiths/AIMSS/issues}{https://github.com/smiths/AIMSS/issues}
\ad{To be created}

\noindent Refer to Section \ref{measurement_template}

\subsection{Review the Process of Creating a Short List}
\label{task_review_short_list}
GitHub Issue:
\href{https://github.com/smiths/AIMSS/issues}{https://github.com/smiths/AIMSS/issues}
\ad{To be created}

\noindent Refer to Section \ref{survey_short_list_projects}

\subsection{Take the Survey for the Short List Projects}
\label{task_short_list_survey}
GitHub Issue:
\href{https://github.com/smiths/AIMSS/issues}{https://github.com/smiths/AIMSS/issues}
\ad{To be created}

\noindent Refer to Section \ref{survey_short_list_projects}

\subsection{Participate in the Usability Tests}
\label{task_usability_tests}
GitHub Issue:
\href{https://github.com/smiths/AIMSS/issues}{https://github.com/smiths/AIMSS/issues}
\ad{To be created}

\noindent Refer to Section \ref{usability_tests}

\subsection{Participate in the Modifiability Experiment}
\label{task_modifiability_experiment}
GitHub Issue:
\href{https://github.com/smiths/AIMSS/issues}{https://github.com/smiths/AIMSS/issues}
\ad{To be created}

\noindent Refer to Section \ref{modifiability_experiment}

\subsection{Review the First AHP Ranking}
\label{task_first_ranking}
GitHub Issue:
\href{https://github.com/smiths/AIMSS/issues}{https://github.com/smiths/AIMSS/issues}
\pmi{To be created}

\noindent Refer to Section \ref{first_AHP}

\subsection{Review the Final Short List Ranking}
\label{task_final_ranking}
GitHub Issue:
\href{https://github.com/smiths/AIMSS/issues}{https://github.com/smiths/AIMSS/issues}
\pmi{To be created}

\noindent Refer to Section \ref{final_AHP}

%
%\section{Deleted Notes}
%\subsection{Selecting Process}
%The process of creating and shortening the list of software packages is
%documented in
%\href{https://github.com/smiths/AIMSS/blob/master/StateOfPractice/Ao-Notes/Ao-N%otes.pdf}{Ao's
%	Notes} or Peter's Notes \ad{TBD}
%
%What do you think of our process of selecting software candidates?
%

\end{document}