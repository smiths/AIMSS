\documentclass[letterpaper,cleveref]{lipics-v2019}

\usepackage{natbib}
\usepackage{booktabs}
\usepackage{amsmath,amsthm}
\usepackage{hyperref}

\usepackage{hyperref}
\hypersetup{
colorlinks=true,       % false: boxed links; true: colored links
linkcolor=red,          % color of internal links (change box color with
%linkbordercolor)
citecolor=blue,       % color of links to bibliography
filecolor=magenta,   % color of file links
urlcolor=cyan           % color of external links
}

%% Comments
\newif\ifcomments\commentstrue

\ifcomments
\newcommand{\authornote}[3]{\textcolor{#1}{[#3 ---#2]}}
\newcommand{\todo}[1]{\textcolor{red}{[TODO: #1]}}
\else
\newcommand{\authornote}[3]{}
\newcommand{\todo}[1]{}
\fi

\newcommand{\wss}[1]{\authornote{blue}{SS}{#1}} %Spencer Smith
\newcommand{\jc}[1]{\authornote{red}{JC}{#1}} %Jacques Carette
\newcommand{\oo}[1]{\authornote{magenta}{OO}{#1}} %Olu Owojaiye
\newcommand{\pmi}[1]{\authornote{green}{PM}{#1}} %Peter Michalski
\newcommand{\ad}[1]{\authornote{cyan}{AD}{#1}} %Ao Dong

\newcommand{\notdone}[1]{\textcolor{red}{#1}}
\newcommand{\done}[1]{\textcolor{black}{#1}}

%\oddsidemargin 0mm
%\evensidemargin 0mm
%\textwidth 160mm
%\textheight 200mm

\theoremstyle{definition}
\newtheorem{defn}{Definition}

\title{Experiments} 
\author{Spencer Smith}{McMaster University, Canada}{smiths@mcmaster.ca}{}{}
\author{Jacques Carette}{McMaster University, Canada}{carette@mcmaster.ca}{}{}
\author{Olu Owojaiye}{McMaster University, Canada}{owojaiyo@mcmaster.ca}{}{}
\author{Peter Michalski}{McMaster University, Canada}{michap@mcmaster.ca}{}{}
\author{Ao Dong}{McMaster University, Canada}{donga9@mcmaster.ca}{}{}

\authorrunning{Smith et al.}  \Copyright{Spencer Smith and Jacques Carette and
Olu Owojaiye and Peter Michalski and Ao Dong}

\date{\today}

\hideLIPIcs
\nolinenumbers

\begin{document}
\maketitle

~\newpage
\section{User Experiments}
\subsection{Usability Experiment}
The purpose of this experiment is to assess the usability of each software package and collect qualitative data. An initial step is required to define a set of tasks for the experiment. The guideline below would assist domain experts in defining a list usability tasks for each domain.
This list will then be used to evaluate the usability of each selected software package. It is expected that the list would allow experimenters perform tasks in  different aspects of the software package and provide their evaluation using a standardized usability questionnaire ( \url{https://www.usabilitest.com/sus-pdf-generator}- 20-29 , \url{https://uiuxtrend.com/pssuq-post-study-system-usability-questionnaire/} - PSSUQ) - (we need to decide on the questionnaire to use eventually)


\subsubsection{Task selection criteria for domain experts}
The following criteria will be considered by the domain expert when defining the tasks for each software package.

\begin {enumerate}
\item Tasks should be executable for subjects with novice to intermediate experience.
\item All tasks should take no more than one hour.
\item Tasks should include the basic/common use cases of the software package.
\item Include tasks that require sequential or hierarchical steps for completion

\end {enumerate}

\subsubsection {Procedure}
\begin {enumerate}

\item Survey participants to collect pre-experiment data (background, experience of subjects)
\item Participants perform tasks based on task defined by domain experts.
\item Observe the study subjects (take notes, record sessions(OBS screen recorder), watch out for body languages and verbal cues)
\item Survey the study subjects to collect feedback (post-experiment interview)
\item Prepare a summary report of experiment

\end {enumerate}






~\newpage
\section{Modifiability Experiment}

This experiment is designed to gather qualitative data regarding the modifiability of each software package. The initial step outlined below produces a list of likely software changes for members of a scientific computing software domain. This list is then used to analyze whether specific software packages have been designed to accommodate these expected modifications. 

Parnas and Clements' definition of likely changes has informed this analysis: ``If the system is required to be easy to change, the requirements should contain a definition of the areas that are considered likely to change. You cannot design a system so that everything is equally easy to change. Programmers should not have to decide which changes are most likely'' \cite{parnas1986rational}.



\subsection {Procedure}

\begin{enumerate}
	\item \textit{Domain Expert}: List all likely changes that a domain expert might make in a software package in the domain.
	\item Using the short list of software packages listed in the  \href{https://github.com/smiths/AIMSS/blob/master/StateOfPractice/Methodology/Methodology.pdf}{Methodology document}:
	\begin{enumerate}
		\item Identify which of the above changes each package is likely designed to accommodate.
		\item Prepare a short report outlining the likely changes for software in the domain, and which changes each software package is likely designed to accommodate.
	\end{enumerate}
\end{enumerate}

\newpage

\bibliographystyle {plainnat}
\bibliography {../../CommonFiles/ResearchProposal}

\end{document}