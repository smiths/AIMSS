%\documentclass[handout]{beamer} 
\documentclass[t,12pt,numbers,fleqn]{beamer}
%\documentclass[ignorenonframetext]{beamer}

\newif\ifquestions
%\questionstrue
\questionsfalse

\usepackage{pgfpages} 
\usepackage{hyperref}
\hypersetup{colorlinks=true,
    linkcolor=blue,
    citecolor=blue,
    filecolor=blue,
    urlcolor=blue,
    unicode=false}
\urlstyle{same}

\usepackage{multicol}
\usepackage{booktabs}
\usepackage{bibentry}
\usepackage[round, authoryear]{natbib}
\bibliographystyle{plainnat}

%\usetheme{Iimenau}

\useoutertheme{split} %so the footline can be seen, without needing pgfpages

%\pgfpagesuselayout{resize to}[letterpaper,border shrink=5mm,landscape]  %if this is uncommented, the hyperref links do not work

\mode<presentation>{}

\input{./def-beamer}

\newcommand{\topic}{State of the Practice for Lattice Boltzmann Solvers}

%Title page for slides

\newcommand{\presenters}{Spencer Smith, Zahra Motamed, Peter Michalski} %use to switch presenters
\newcommand{\projectTitle}{SOP Discussion with Domain Expert}
\newcommand{\department}{Department of Computing and Software}
\newcommand{\instituteName}{Faculty of Engineering, McMaster University}

%\setbeamerfont{structure}{series=\bfseries}
%\usefonttheme[stillsansseriftext,stillsansserifmath]{serif}
\setbeamertemplate{navigation symbols}{} 
\setbeamertemplate{itemize item}[ball]

\title{
  {\normalsize \bf 
    \borange{\projectTitle}}\\[2ex]
  {\Large \bf \topic}}

\author{\presenters}

\institute{\instituteName}

\date{
\today
\bc
  \includegraphics[scale = 0.2, keepaspectratio]
  {./mcmaster-logo-full-color.jpg}
\ec
}

\renewcommand{\borange}[1] %orange is too hard to read
{
   \bred{#1}
}


\begin{document}

%\nobibliography{../../../CommonFiles/ResearchProposal}

% Footline for slides

% Display title page and displays footers

\setbeamertemplate{footline}{} %so the title screen does not have a footline

%%%%%%%%%%%%%%%%%%%%%%%%%%%%%%%%%%%%%%%%%%%%%%%%%%%%%%%%%%%%

\begin{frame}
\titlepage
\end{frame}

%%%%%%%%%%%%%%%%%%%%%%%%%%%%%%%%%%%%%%%%%%%%%%%%%%%%%%

\setbeamertemplate{footline}{
\begin{beamercolorbox}{sectioninhead/foot}
\hspace{1ex}\bblue{\hrulefill}\hspace{1ex}

\vspace{1ex}
\hspace{1ex}
{\tiny \department \hfill 
\topic \hfill 
\insertframenumber/\inserttotalframenumber~~}

\vspace{1ex}
\end{beamercolorbox}}

%%%%%%%%%%%%%%%%%%%%%%%%%%%%%%%%%%%%%%%%%%%%%%%%%%%%%%



%%%%%%%%%%%%%%%%%%%%%%%%%%%%%%%%%%%%%%%%%%%%%%%%%%%%%%

\begin{frame}
\frametitle{Overview}

\bi
\item Goals
  \bi
  \item Understand the state of the software development practices for
    LBM Software
  \item Make recommendations for improvements
  \item A publication that is useful to the community
  \ei  
\item We have developed a standard methdology for assessing SOP for Domain X
\item The methodology requires a \textbf{domain expert} to:
  \bi
  \item Vet the preliminary results
  \item Assess the feasibility of the recommendations
  \item Navigate the publication process
  \item Answer \textbf{developer} interview questions on pain points  
  \ei
\item Today's meeting
  \bi
\item Informal
\item Questions do not have to be answered in real time, or by domain expert
  \ei
\ei

\end{frame}

%%%%%%%%%%%%%%%%%%%%%%%%%%%%%%%%%%%%%%%%%%%%%%%%%%%%%%%%%%%%%%%%%%%%%%%%%%%%%

\begin{frame}
\frametitle{Overall Process}

\begin{enumerate}
%\item Identify the domain
\item \emph{Domain Expert}: Create a top ten list
\item Brief \emph{Domain Expert}
\item Initial list of candidate software packages
\item \emph{Domain Expert}: Vet domain software list
\item Domain Analysis
\item \emph{Domain Expert}: Vet domain analysis
%\item Gather source code and documentation for software
\item Collect empirical measures
\item Measure using measurement template
\item Interview developers
\item Use AHP process to rank the software packages
\item \emph{Domain Expert}: Vet AHP ranking
\item \emph{Domain Expert}: Review recommendations
\end{enumerate}

\end{frame}

%%%%%%%%%%%%%%%%%%%%%%%%%%%%%%%%%%%%%%%%%%%%%%%%%%%%%%%%%%%%%%%%%%%%%%%%%%%%%

\begin{frame}
\frametitle{Vet Software List}

\bi
\item How does our list compare to the domain expert's list?
\item Is any software missing?
\item Is there software that should be included?
\item Any other questions/comments or concerns?  
\ei
  
\end{frame}

%%%%%%%%%%%%%%%%%%%%%%%%%%%%%%%%%%%%%%%%%%%%%%%%%%%%%%%%%%%%%%%%%%%%%%%%%%%%%

\begin{frame}
\frametitle{Software List}

\begin{multicols}{2}	
	\bi
			\item \href{https://ludwig.epcc.ed.ac.uk/}{Ludwig}
			\item \href{http://espressomd.org/wordpress/}{ESPResSo}
			\item \href{https://palabos.unige.ch/}{Palabos}
			\item \href{https://www.openlb.net/}{OpenLB}
			\item \href{https://github.com/aharwood2/LUMA/}{LUMA}
			\item \href{https://pypi.org/project/pylbm/}{pyLBM}
			\item \href{https://www.scd.stfc.ac.uk/Pages/DL_MESO.aspx}{DL\_MESO (LBE)}
			\item \href{https://www.walberla.net}{waLBerla}
			\item \href{http://sailfish.us.edu.pl/}{Sailfish}
			\item \href{https://github.com/maxlevesque/laboetie}{laboetie}
			\item \href{https://docs.tclb.io/}{TCLB}
			\item \href{http://mechsys.nongnu.org/}{MechSys}
			\item \href{https://github.com/Olllom/lettuce}{lettuce}
			\item \href{https://github.com/espressopp/espressopp}{ESPResSo++} 
			\item \href{https://github.com/carlosrosales/mplabs}{MP-LABS}		
			\item \href{http://sunlightlb.sourceforge.net/}{SunlightLB}
			\item \href{http://ccs.chem.ucl.ac.uk/lb3d}{LB3D} - 			
			\href{https://www.sciencedirect.com/science/article/pii/S0010465517301017}{paper}
			\item \href{https://code.google.com/archive/p/limbes/}{LIMBES}
			\item \href{http://faculty.fiu.edu/~sukopm/LBnD_Prime/LBnD_Prime.html}{LB2D-Prime}	
			\item \href{https://github.com/UCL/hemelb}{HemeLB} 
			\item \href{https://pypi.org/project/lbmpy/}{lbmpy}	
			\item \href{http://faculty.fiu.edu/~sukopm/LBnD_Prime/LBnD_Prime.html}{LB3D-Prime}	
			\item \href{https://github.com/UCL/LatBo.jl}{LatBo.jl}
	\ei
\end{multicols}

\end{frame}

%%%%%%%%%%%%%%%%%%%%%%%%%%%%%%%%%%%%%%%%%%%%%%%%%%%%%%%%%%%%%%%%%%%%%%%%%%%%%

\begin{frame}
\frametitle{Domain Expert Software List}


\end{frame}

%%%%%%%%%%%%%%%%%%%%%%%%%%%%%%%%%%%%%%%%%%%%%%%%%%%%%%%%%%%%%%%%%%%%%%%%%%%%%

\begin{frame}
\frametitle{Domain Analysis}

\bi
\item So users can pick the right software for their task
\item Think of the software as a family of related software products, like:
\bi
\item Family of automobiles
\item Computer hardware
\item ...
\ei
\item Commonalities
\item Variabilities
\item Parameters of variation
\item Do the following slides capture the commonalities and variabilities?
\item We can follow up over e-mail, if that is easier
\ei

\end{frame}

%%%%%%%%%%%%%%%%%%%%%%%%%%%%%%%%%%%%%%%%%%%%%%%%%%%%%%%%%%%%%%%%%%%%%%%%%%%%%

\begin{frame}
\frametitle{Commonalities}

\bi
	\item Lattice - discretized domain within a boundary
	\item Collision operator (The Bhatnagar-Gross-Krook Collision Operator is common)
	\item Probability density function
	\item Equilibrium distribution function
	\item Boltzmann transport equation
\ei

\end{frame}

%%%%%%%%%%%%%%%%%%%%%%%%%%%%%%%%%%%%%%%%%%%%%%%%%%%%%%%%%%%%%%%%%%%%%%%%%%%%%

\begin{frame}
\frametitle{Variabilities}

\bi
\item May use parallel processing
\item Different equilibrium distribution functions
\item Coefficients for equilibrium distribution function (based on velocity directions)
\item Lattice dimensions (1D, 2D, 3D)
\item Varied velocity directions, partially based on dimensions 
\item Collision operators (SRT, TRT, MRT, BGK)
\item Collision vs collision-free transport equations
\item Number of fluid that can be modeled simultaneously
\item Fluid parameters (Reynolds Num., density, viscosity, etc.)
\item Lattice boundary (reflective or non-reflective)
\ei

\end{frame}

%%%%%%%%%%%%%%%%%%%%%%%%%%%%%%%%%%%%%%%%%%%%%%%%%%%%%%%%%%%%%%%%%%%%%%%%%%%%%

\begin{frame}
\frametitle{Summary of Measures}

\bi
\item Empirical measures
  \bi
  \item Examples
  \ei
\item Measurement template
  \bi
  \item qualities
  \ei
\item Interview questions
  \bi
  \item Examples
  \ei
  \item AHP Process with pairwise comparisons
\ei
\end{frame}

%%%%%%%%%%%%%%%%%%%%%%%%%%%%%%%%%%%%%%%%%%%%%%%%%%%%%%%%%%%%%%%%%%%%%%%%%%%%%

\begin{frame}
\frametitle{Thoughts on Overall Ranking?}

[Peter and Ao, give the overall ranking graphic for your domain]

\end{frame}

%%%%%%%%%%%%%%%%%%%%%%%%%%%%%%%%%%%%%%%%%%%%%%%%%%%%%%%%%%%%%%%%%%%%%%%%%%%%%

\begin{frame}
\frametitle{Ranking Follow-Up}

\bi
\item We would like feedback on the ranking for each of the qualities
\item Is it feasible to review all 10 graphs, and the associated write-up?
\item Maybe there is a grad student that can review the rankings?
 \ei

\end{frame}

%%%%%%%%%%%%%%%%%%%%%%%%%%%%%%%%%%%%%%%%%%%%%%%%%%%%%%%%%%%%%%%%%%%%%%%%%%%%%

\begin{frame}
\frametitle{Discuss Recommendations}

\bi
\item 
\ei

\end{frame}

%%%%%%%%%%%%%%%%%%%%%%%%%%%%%%%%%%%%%%%%%%%%%%%%%%%%%%%%%%%%%%%%%%%%%%%%%%%%%

\begin{frame}
\frametitle{Publication}

\bi
\item Where should we publish this paper?
\item Who do you see as the targeted readers?
\ei

\end{frame}

%%%%%%%%%%%%%%%%%%%%%%%%%%%%%%%%%%%%%%%%%%%%%%%%%%%%%%%%%%%%%%%%%%%%%%%%%%%%%

\begin{frame}
\frametitle{Developer Questions (for next time)}

\bi
\item 
\ei

\end{frame}

%%%%%%%%%%%%%%%%%%%%%%%%%%%%%%%%%%%%%%%%%%%%%%%%%%%%%%%%%%%%%%%%%%%%%%%%%%%%%

\end{document}
