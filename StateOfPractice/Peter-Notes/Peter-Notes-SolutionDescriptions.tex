\documentclass{article}

\usepackage{tabularx}
\usepackage{booktabs}
\usepackage{hyperref}

\title{Project Notes - Solution Descriptions}

\author{Peter Michalski}

\date{}


\begin{document}

\maketitle

~\newpage


\section{List of LBM solutions}
\begin{enumerate}
	\item aromanro/LatticeBoltzmann\\
	\href{https://github.com/aromanro/LatticeBoltzmann}{https://github.com/aromanro/LatticeBoltzmann}\\
	Found from list at \href{https://github.com/topics/lattice-boltzmann}{https://github.com/topics/lattice-boltzmann}\\
	A 2D Lattice Boltzmann program
	
	\item Atruszkowska/LBM\_MATLAB\\
	\href{https://github.com/atruszkowska/LBM_MATLAB}{https://github.com/atruszkowska/LBM\_MATLAB}\\
	Found from list at \href{https://github.com/topics/lattice-boltzmann}{https://github.com/topics/lattice-boltzmann}\\
	MPI-style parallelized Shan and Chen LBM with multiscale modeling extension
	
	\item ch4-project\\
	\href{https://github.com/SoftwareImpacts/SIMPAC-2019-1}{https://github.com/SoftwareImpacts/SIMPAC-2019-1}\\  
	Found using Google search.\\
	Eulerian-Lagrangian fluid dynamics platform
	A general purpose Lattice-Boltzmann code for fluid-dynamics simulations. It includes :
	fluid dynamics (with several volume forcing terms for Channel flow, Homogeneous Isotropic Turbulence, buoyancy)
	temperature dynamics (advection, diffusion , sink/source or reaction terms)
	phase change (enthalpy formulation for solid/liquid systems)
	scalar transport (same functionalities as temperature)
	lagrangian dynamics (tracers, heavy/light and active point-like particles; non-shperical Jeffery rotation, gyrotaxis)
	large eddy simulation (Smagorinsky, Shear Improved Samgorinsky with Kalman Filter)
	
	\item CUDA-LBM-simulator\\
	\href{https://github.com/henryfriedlander/CUDA-LBM-simulator}{https://github.com/henryfriedlander/CUDA-LBM-simulator}\\
	Found from list at \href{https://github.com/topics/lattice-boltzmann}{https://github.com/topics/lattice-boltzmann}\\
	This is a Lattice-Boltzmann simulation using CUDA GPU graphics optimization.
	
	\item CudneLB (TCLB)\\
	\href{https://github.com/CFD-GO/TCLB}{https://github.com/CFD-GO/TCLB}\\
	Found from list at \href{https://github.com/topics/lattice-boltzmann}{https://github.com/topics/lattice-boltzmann}\\
	CudneLB is a MPI+CUDA or MPI+CPU high-performance CFD simulation code, based on Lattice Boltzmann Method.
	
	\item DL\_Meso \\
	\href{https://www.scd.stfc.ac.uk/Pages/DL_MESO.aspx}{https://www.scd.stfc.ac.uk/Pages/DL\_MESO.aspx} \\
	Found using Google search.\\
	DL\_MESO is a general purpose mesoscale simulation package developed by Michael Seaton for CCP5 and UKCOMES​ under a grant provided by EPSRC. It is written in Fortran 2003 and C++ and supports both Lattice Boltzmann Equation (LBE) and Dissipative Particle Dynamics (DPD) methods. It is supplied with its own Java-based Graphical User Interface (GUI) and is capable of both serial and parallel execution.
	
	\item elbe \\
	\href{https://www.tuhh.de/alt/elbe/home.html}{https://www.tuhh.de/alt/elbe/home.html}\\
	Found using www.swmath.org \\
	elbe is a highly efficient academic flow solver for the simulation of non-linear three-dimensional flow problems, such as violent impact, tank sloshing or Tsunami propagation, and highly turbulent singlephase flows for drag resistance, aerodynamic noise control and fluid-structure interactions. elbe, the efficient lattice boltzmann environment, is accelerated with recent graphics hardware (graphics processing units, GPUs). Thanks to the very efficient numerical backend, three-dimensional simulations of complex flows are possible in or near real-time. elbe can be considered as a prototype for next-generation CFD tools that allow for a real-time analysis of complex flow fields and simulation-based design. Being developed since late 2010, the main development efforts currently take place at the Institute for Fluid Dynamics and Ship Theory (FDS) at Hamburg University of Technology (TUHH), with external partners and contributors at University of Rhode Island and TU Braunschweig.
	
	\item eLBM \\
	No direct source\\
	Found using www.swmath.org \\
	Direct simulation of pore-scale two-phase visco-capillary flow on large digital rock images using a phase-field lattice Boltzmann method on general-purpose graphics processing units. 
	
	\item ESPResSo
	\\
	\href{http://espressomd.org/html/doc/index.html}{http://espressomd.org/html/doc/index.html}
	\\
	Found from list at \href{https://github.com/topics/lattice-boltzmann}{https://github.com/topics/lattice-boltzmann}\\
	ESPResSo is a simulation package designed to perform Molecular Dynamics (MD) and Monte Carlo (MC) simulations. It is meant to be a universal tool for simulations of a variety of soft matter systems. It features a broad range of interaction potentials which opens up possibilities for performing simulations using models with different levels of coarse-graining. It also includes modern and efficient algorithms for treatment of Electrostatics (P3M, MMM-type algorithms, constant potential simulations, dielectric interfaces, …), hydrodynamic interactions (DPD, Lattice-Boltzmann), and magnetic interactions, only to name a few. It is designed to exploit the capabilities of parallel computational environments. The program is being continuously extended to keep the pace with current developments both in the algorithms and software.
	
	\item ESPResSo++
	\\
	\href{http://www.espresso-pp.de/}{http://www.espresso-pp.de/}
	\\
	Found from list at \href{https://github.com/topics/lattice-boltzmann}{https://github.com/topics/lattice-boltzmann}\\
	ESPResSo++ is a software package for the scientific simulation and analysis of coarse-grained atomistic or bead-spring models as they are used in soft matter research.\\
	ESPResSo++ has a modern C++ core and flexible Python user interface.\\
	ESPResSo and ESPResSo++ have common roots however their development is independent and they are different software packages.\\
	ESPResSo++ is free, open-source software published under the GNU General Public License (GPL).
	
	\item firesim \\ 
	\href{https://github.com/kynan/firesim}{https://github.com/kynan/firesim}
	\\
	Found using Google search.\\
	Lattice Boltzmann Method (LBM) fluid solver driving a particle engine for the simulation and real-time visualization of fire
	
	\item fvLBM
	\\
	\href{https://github.com/zhulianhua/fvLBM}{https://github.com/zhulianhua/fvLBM}
	\\
	Found using Google search.\\
	finite volume lattice Boltzmann method
	
	\item HemeLB \\
	\href{https://github.com/UCL/hemelb}{https://github.com/UCL/hemelb}
	\\
	Found using Google search.\\
	A software pipeline that simulates the blood flow through a stent (or other flow diverting device) inserted in a patient’s brain.
	
	\item JFlowSim \\
	\href{https://github.com/ChristianFJanssen/jflowsim}{https://github.com/ChristianFJanssen/jflowsim} \\
	Found from list at \href{https://github.com/topics/lattice-boltzmann}{https://github.com/topics/lattice-boltzmann}\\
	jFlowSim is an interactive, thread-parallel Lattice Boltzmann solver in two dimensions.
	
	\item laboetie
	\\
	\href{https://github.com/maxlevesque/laboetie}{https://github.com/maxlevesque/laboetie}
	\\
	Found from list at \href{https://github.com/topics/lattice-boltzmann}{https://github.com/topics/lattice-boltzmann}\\
	laboetie is a computational fluid dynamics code for chemical applications.
	It uses the Lattice-Boltzmann algorithm.
	
	\item LatBo.jl \\
	\href{https://github.com/UCL/LatBo.jl}{https://github.com/UCL/LatBo.jl}
	\\
	Found using Google search.\\
	Lattice-Boltzmann implementation in Julia
	
	\item LB2D\_Prime
	\\
	\href{http://faculty.fiu.edu/~sukopm/LBnD_Prime/LBnD_Prime.html}{http://faculty.fiu.edu/~sukopm/LBnD\_Prime/LBnD\_Prime.html} \\
	Found using Google search.\\
	LB2D\_Prime is a lattice Boltzmann (LB) code capable of simulating single and multi-phase flows and solute/heat transport in geometrically complex domains.
	
	\item LB3D \\ 
	\href{http://ccs.chem.ucl.ac.uk/lb3d}{http://ccs.chem.ucl.ac.uk/lb3d}
	\\
	Found using Google search.\\
	A parallel implementation of the Lattice-Boltzmann method for simulation of interacting amphiphilic fluids. LB3D provides functionality to simulate three-dimensional simple, binary oil/water and ternary oil/water/amphiphile fluids using the Shan-Chen model for binary fluid interactions.
	
	\item LBDEMcoupling-public
	\\
	\href{https://github.com/ParticulateFlow/LBDEMcoupling-public}{https://github.com/ParticulateFlow/LBDEMcoupling-public}
	\\
	Found using Google search.\\
	Coupling between the Lattice-Boltzmann code Palabos and the DEM code LIGGGHTS
	
	\item LBM-EP \\
	No direct source\\
	Found using www.swmath.org \\	
	LBM-EP: lattice-Boltzmann method for fast cardiac electrophysiology simulation from 3D images. Current treatments of heart rhythm troubles require careful planning and guidance for optimal outcomes. Computational models of cardiac electrophysiology are being proposed for therapy planning but current approaches are either too simplified or too computationally intensive for patient-specific simulations in clinical practice. This paper presents a novel approach, LBM-EP, to solve any type of mono-domain cardiac electrophysiology models at near real-time that is especially tailored for patient-specific simulations. The domain is discretized on a Cartesian grid with a level-set representation of patient’s heart geometry, previously estimated from images automatically. The cell model is calculated node-wise, while the transmembrane potential is diffused using Lattice-Boltzmann method within the domain defined by the level-set. Experiments on synthetic cases, on a data set from CESC’10 and on one patient with myocardium scar showed that LBM-EP provides results comparable to an FEM implementation, while being 10 − 45 times faster. Fast, accurate, scalable and requiring no specific meshing, LBM-EP paves the way to efficient and detailed models of cardiac electrophysiology for therapy planning. 
	
	\item lbmpy \\
	\href{https://pypi.org/project/lbmpy/}{https://pypi.org/project/lbmpy/}\\
	Found using www.swmath.org \\
	Run fast fluid simulations based on the lattice Boltzmann method in Python on CPUs and GPUs. lbmpy creates highly optimized LB compute kernels in C or CUDA, for a wide variety of different collision operators, including MRT, entropic, and cumulant schemes.
	
	\item LBSim
	\\
	\href{https://github.com/noirb/lbsim}{https://github.com/noirb/lbsim}
	\\
	Found using Google search.\\
	A small and simple Lattice-Boltzmann Method fluid simulator supporting complex boundaries.
	
	\item lettuce
	\\
	\href{https://github.com/Olllom/lettuce}{https://github.com/Olllom/lettuce}
	\\
	Found using Google search.\\
	GPU-acclerated Lattice Boltzmann in Python
	
	\item Limbes
	\\
	\href{https://code.google.com/archive/p/limbes/}{https://code.google.com/archive/p/limbes/} \\
	Found using Google search.\\
	Open source (GPL) code in 2D based on Gauss-Hermite quadrature, parallel (openmp), fortran 90. LIMBES is the recursive acronym for LIMBES Is May be a Boltzmann Equation Solver. Version 1.0 solves numerically by a Lattice Boltzmann like method the BGK-Boltzmann equation for gas in two dimensions.
	
	\item listLBM
	\\
	\href{https://github.com/sorush-khajepor/listLBM}{https://github.com/sorush-khajepor/listLBM} \\
	Found from list at \href{https://github.com/topics/lattice-boltzmann}{https://github.com/topics/lattice-boltzmann}\\
	ListLBM is a sparse lattice Boltzmann solver for multiphase flow in porous media
	
	\item  loliverhennigh /
	Lattice-Boltzmann-fluid-flow-in-Tensorflow 
	\\
	\href{https://github.com/loliverhennigh/Lattice-Boltzmann-fluid-flow-in-Tensorflow}{https://github.com/loliverhennigh/Lattice-Boltzmann-fluid-flow-in-Tensorflow}
	\\
	Found using Google search.\\
	A Lattice Boltzmann fluid flow simulation written in Tensorflow. 
	
	\item Ludwig
	\\
	\href{https://github.com/ludwig-cf/ludwig}{https://github.com/ludwig-cf/ludwig}\\
	Found using Google search.\\
	Ludwig is a parallel code for the simulation of complex fluids, which include mixtures, colloidal suspensions, gels, and liquid crystals. It takes its name from Ludwig Boltzmann, as it uses a lattice Boltzmann method as a basis for numerical solution of the Navier Stokes equations for hydrodynamics. It typically combines hydrodynamics with a coarse-grained order parameter (or order parameters) to represent the "complex" part in a free energy picture. The code is written in standard ANSI C, and uses the Message Passing Interface for distributed memory parallelism. Threaded parallelism is also available via a lightweight abstraction layer ("Target Data Parallel" or "TargetDP") which currently supports either OpenMP or CUDA (NVIDIA GPUs) from a single source.
	
	\item LUMA
	\\
	\href{https://github.com/ElsevierSoftwareX/SOFTX-D-18-00007}{https://github.com/ElsevierSoftwareX/SOFTX-D-18-00007}\\
	Found from list at \href{https://github.com/topics/lattice-boltzmann}{https://github.com/topics/lattice-boltzmann}\\
	LUMA: A many-core, Fluid–Structure Interaction solver based on the Lattice-Boltzmann Method
	
	\item Mechsys \\
	\href{http://mechsys.nongnu.org/}{http://mechsys.nongnu.org/}\\
	Found using www.swmath.org \\
	MechSys is a programming library for the implementation of simulation tools in mechanics. Its source code is mainly written in C++ with easier to use templates for further customization.
	
	\item MP-LABS
	\\
	\href{https://github.com/carlosrosales/mplabs}{https://github.com/carlosrosales/mplabs}
	\\
	Found using Google search.\\
	MP-LABS is a suite of numerical simulation tools for multiphase flows based on the free energy Lattice Boltzmann Method (LBM). The code allows for the simulation of quasi-incompressible two-phase flows, and uses multiphase models that allow for large density ratios. MP-LABS provides implementations that use periodic boundary conditions, but it is written in a way that allows for easy inclusion of different boundary conditions. The output from MP-LABS is in plain ASCII and VTK format, and can be analyzed using other Open Source tools such as Gnuplot and Paraview.
	
	The objective of the MP-LABS project is to provide a core set of routines that are well documented, highly portable, and have proven to perform well in a variety of systems. The source code is written in Fortran 90 and MPI and uses separate subroutines for most tasks in order to make modifications easier.
	
	\item Openlb \\  
	\href{https://www.openlb.net/}{https://www.openlb.net/}
	\\
	Found from list at \href{https://github.com/topics/lattice-boltzmann}{https://github.com/topics/lattice-boltzmann}\\
	The OpenLB project provides a C++ package for the implementation of lattice Boltzmann methods that is general enough to address a vast range of tansport problems, e.g. in computational fluid dynamics. The source code is publicly available and constructed in a well readable, modular way. 
	
	\item openLBMflow \\
	\href{https://github.com/wme7/openLBMflow}{https://github.com/wme7/openLBMflow}\\
	Found using www.swmath.org \\
	The openLBMflow is an fast fluid flow solver based on Lattice Boltzmann Method. Main future are 2D and 3D code, single and multiphase models, Output data in VTK format can be directly open in Paraview. Download Windows or Linux binary version now.
	
	\item Palabos \\
	\href{https://palabos.unige.ch/}{https://palabos.unige.ch/}
	\\
	Found using Google search.\\
	The Palabos library is a framework for general-purpose computational fluid dynamics (CFD), with a kernel based on the lattice Boltzmann (LB) method. It is used both as a research and an engineering tool: its programming interface is straightforward and makes it possible to set up fluid flow simulations with relative ease, or, if you are knowledgeable of the lattice Boltzmann method, to extend the library with your own models. Palabos stands for Parallel Lattice Boltzmann Solver.
	The library’s native programming interface in written in C++. 
	
	\item ParallelLbmCranfield \\
	\href{https://github.com/mate-szoke/ParallelLbmCranfield}{https://github.com/mate-szoke/ParallelLbmCranfield}\\
	Found using www.swmath.org \\
	This software is a two dimensional parallel Lattice Boltzmann Method solver implemented in CUDA and PGAS UPC
	
	\item PowerFLOW \\
	\href{https://www.3ds.com/products-services/simulia/products/powerflow/}{https://www.3ds.com/products-services/simulia/products/powerflow/}\\
	Found using www.swmath.org \\
	PowerFLOW: Through the usage of our unique, inherently transient Lattice Boltzmann-based physics, Exa’s PowerFLOW® CFD solution performs simulations that accurately predict real world conditions. Using the PowerFLOW suite, engineers evaluate product performance early in the design process prior to any prototype being built — when the impact of change is most significant for design and budgets. PowerFLOW imports fully complex model geometry and accurately and efficiently performs aerodynamic, aeroacoustic and thermal management simulations.
	
	\item ProLB \\
	\href{http://www.prolb-cfd.com/}{http://www.prolb-cfd.com/}
	\\
	Found using Google search.\\
	ProLB is an innovative Computational Fluid Dynamics (CFD) software solution. Based on the Lattice-Boltzmann method, its successfully-validated solver performs inherently transient simulations of highly complex flows with a competitive turnaround time. ProLB‘s accurate aerodynamic and aeroacoustic modeling allows engineers to make early design decisions that optimize and shorten the product development process.
	
	\item pyLBM
	\\
	\href{https://github.com/pylbm/pylbm}{https://github.com/pylbm/pylbm}
	\\
	Found from list at \href{https://github.com/topics/lattice-boltzmann}{https://github.com/topics/lattice-boltzmann}\\
	pylbm is an all-in-one package for numerical simulations using Lattice Boltzmann solvers.
	This package gives all the tools to describe your lattice Boltzmann scheme in 1D, 2D and 3D problems.
	
	\item Sailfish \\
	\href{https://github.com/sailfish-team/sailfish}{https://github.com/sailfish-team/sailfish}
	\\
	Found using Google search.\\
	Lattice Boltzmann (LBM) simulation package for GPUs (CUDA, OpenCL)
	
	\item siramirsaman/LBM
	\\
	\href{https://github.com/siramirsaman/LBM}{https://github.com/siramirsaman/LBM}
	\\
	Found from list at \href{https://github.com/topics/lattice-boltzmann}{https://github.com/topics/lattice-boltzmann}\\
	Lattice Boltzmann Method Implementation in MATLAB for Curved Boundaries
	
	\item SunlightLB \\
	\href{http://sunlightlb.sourceforge.net/}{http://sunlightlb.sourceforge.net/}
	\\
	Found using Google search.\\
	SunlightLB is an open-source 3D lattice Boltzmann code which can be used to solve a variety of hydrodynamics problems, including passive scalar transport problems.
	
	\item Taxila-LBM
	\\
	\href{https://github.com/ecoon/Taxila-LBM}{https://github.com/ecoon/Taxila-LBM}
	\\
	Found using Google search.\\
	Taxila LBM is a parallel implementation of the Lattice Boltzmann Method for simulation of flow in porous and geometrically complex media.
	
	\item turbulent\_lbm\_multigpu
	\\
	\href{https://github.com/arashb/turbulent_lbm_multigpu}{https://github.com/arashb/turbulent\_lbm\_multigpu}
	\\
	Found using Google search.\\
	Lattice Boltzmann simulation of turbulent fluid flow on GPU Cluster
	
	\item waLBerla
	\\
	\href{https://www.walberla.net/}{https://www.walberla.net/}
	\\
	Found using Google search.\\
	waLBerla uses the lattice Boltzmann method (LBM), which is an alternative to classical Navier-Stokes solvers for computational fluid dynamics simulations. All of the common LBM collision models are implemented (SRT, TRT, MRT). Additionally, a coupling to the rigid body physics engine pe is available. 
	
	\item wlb
	\\
	\href{https://github.com/weierstrass/wlb}{https://github.com/weierstrass/wlb}
	\\
	Found using Google search.\\
	A Lattice-Boltzmann code for solving coupled equations in electrohydrodynamics. 
	Three collision operators are implemented for the (incompressible) Navier-Stokes, 
	Nernst-Planck (advection-diffusion) and Poission's equation for electrostatics 
	respectively. Various implementations of Dirichlet/Neumann boundary conditions 
	are also available. The code deals (so far) only with 2D systems.
	This code is part of a  master thesis project carried out at Chalmers University, 
	Gothenburg.
	
	\item Zmhhaha/LBM-Cplusplus-A.A.Mohamad
	\\
	\href{https://github.com/zmhhaha/LBM-Cplusplus-A.A.Mohamad}{https://github.com/zmhhaha/LBM-Cplusplus-A.A.Mohamad} \\
	Found from list at \href{https://github.com/topics/lattice-boltzmann}{https://github.com/topics/lattice-boltzmann}\\
	The C++ version code of "Lattice Boltzmann Method Fundamentals and Engineering Applications with Computer Codes".
	
\end{enumerate}

\end{document}