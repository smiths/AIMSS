\documentclass[12pt]{article}

\usepackage{titling}
\usepackage{blindtext}
\usepackage{fullpage}
\usepackage[round]{natbib}
\usepackage{multirow}
\usepackage{booktabs}
\usepackage{tabularx}
\usepackage{siunitx}
\usepackage{graphicx}
\usepackage{float}
\usepackage{hyperref}
\hypersetup{
    colorlinks,
    citecolor=blue,
    filecolor=black,
    linkcolor=red,
    urlcolor=blue
}
\newcommand{\colAwidth}{0.24\textwidth}
\newcommand{\colBwidth}{0.76\textwidth}

\newcommand{\colCwidth}{0.28\textwidth}
\newcommand{\colDwidth}{0.72\textwidth}

\newcommand{\colEwidth}{0.33\textwidth}
\newcommand{\colFwidth}{0.67\textwidth}

\newcounter{comnum} %Commonality Number
\newcommand{\cthecomnum}{C\thecomnum}
\newcommand{\cref}[1]{C\ref{#1}}

\newcounter{varnum} %Variability Number
\newcommand{\vthevarnum}{V\thevarnum}
\newcommand{\vref}[1]{V\ref{#1}}

\newcounter{parnum} %Parameter of Variation Number
\newcommand{\ptheparnum}{P\theparnum}
\newcommand{\pref}[1]{P\ref{#1}}

\newif\ifcomments\commentstrue

\ifcomments
\newcommand{\authornote}[3]{\textcolor{#1}{[#3 ---#2]}}
\newcommand{\todo}[1]{\textcolor{red}{[TODO: #1]}}
\else
\newcommand{\authornote}[3]{}
\newcommand{\todo}[1]{}
\fi

\newcommand{\wss}[1]{\authornote{blue}{SS}{#1}} %Spencer Smith
\newcommand{\jc}[1]{\authornote{red}{JC}{#1}} %Jacques Carette
\newcommand{\oo}[1]{\authornote{magenta}{OO}{#1}} %Olu Owojaiye
\newcommand{\pmi}[1]{\authornote{green}{PM}{#1}} %Peter Michalski
\newcommand{\ad}[1]{\authornote{cyan}{AD}{#1}} %Ao Dong

\begin{document}

\title{\bf{Commonality Analysis for Lattice Boltzmann Systems}} 
\author{Peter Michalski, \\Department of Computing and Software\\McMaster University}
\date{\today}

\begin{titlepage}
	\clearpage\maketitle
	\thispagestyle{empty}
	\begin{abstract}
		This report presents a commonality analysis for Lattice Boltzmann systems. The document reviews both the methodology of commonality analysis and the details of Lattice Boltzmann systems. The commonality analysis itself consists of the following:
		i) terminology and definitions; ii) commonalities, or features that are common to all potential family members; iii) variabilities, or features and characteristics that may vary among family members; and, iv) parameters of variation, or the potential values that can be assigned to the variabilities. 
		The documentation of the above items for Lattice Boltzmann systems is clarified by decomposing each item into subsections on Lattice Boltzmann methods, input, output, nonfunctional
		requirements, and, where appropriate, system constraints.\\
		
		\noindent\textbf{Keywords:} Commonality analysis, Lattice Boltzmann methods, program family.
		
	\end{abstract}
\end{titlepage}

\pagenumbering{roman}

\newpage

\tableofcontents

\newpage

\pagenumbering{arabic}

\section{Introduction}
Mesh generating systems are well suited to development as a program family because they
fit Parnas’s definition of a program family: “a set of programs whose common properties
are so extensive that it is advantageous to study the common properties of the programs
before analyzing individual members” (Parnas, 1976). Developing mesh generating sys-
tems as a program family is advantageous because mesh generators share many common
features, or commonalities. Furthermore, when there are differences between systems, the
variabilities between them can be systematically considered. The purpose of this document
is to record these commonalities and variabilities and show the relationships between them,
and thus facilitate the specification and development of mesh generator program family
members. This document will be valuable in all future phases of system development and
analysis. For instance, the requirements documentation for any mesh generator will use
the commonality analysis, since the requirements should refine the commonalities, which
are shared requirements of all mesh generators. Moreover, the design of any future system
will use this documentation to facilitate consideration of the variabilities, so that likely
changes can be made to the system with a minimal amount of work.
The scope of the commonality analysis presented here can be considered both from the
point of view of mesh generating systems and from the point of view of software engineering
methodologies. From the mesh generator perspective, the starting point for the current
document is the commonality analysis conducted by Chen (2003). However, the scope of
the current document is broader than that of Chen (2003), which was restricted to two-
dimensional meshes and did not consider non-conformal, hybrid or mixed meshes. The
current commonality analysis is intended to cover all mesh generators that are targeted
at finite element applications. Meshes for cartography or other uses are not explicitly
considered, although much of the information in this document does overlap with other
mesh uses. With respect to software engineering methodologies, the scope of the current
report is restricted to informal methods, with the intention that the informal requirements
will form a starting point for later development and refinement by formal methods.
The first section below provides an overview of the program family of mesh generators,
by reviewing what is involved in a commonality analysis and the basics of mesh generating
systems. After this, the basic terminology and definitions necessary for understanding the
remainder of the document are provided. The definitions include terms used in describing
a commonality analysis and terms that are used in defining the characteristics and prop-
erties of meshes. The next three sections consist of the lists of commonalities, variabilities
and parameters of variation, respectively. These three sections form the heart of the docu-
mentation and include an extensive set of cross-references to demonstrate the relationships
between the different items. The final section provides information on unresolved issues.

\newpage
\section{Overview}
This section provides both an overview of the process of commonality analysis and of mesh
generation. The subsection on commonality analysis briefly introduces this topic, along
with references that can be searched for further information. The subsection on mesh
generators outlines the following: i) the basics of mesh generation, ii) the scope of the
current analysis, iii) the overall philosophy that has been adopted for the commonality
analysis, and iv) the arguments in favour of the development of mesh generators as a
program family.

\subsection{Commonality Analysis}
In some situations it is advantageous to develop a collection of related software products
as a program family. The idea is that if the software products are similar enough, then
it should be possible to predict what the products have in common, what differs between
them and then reuse the common aspects and thus support rapid development of the
family members. The idea of program families was introduced by Dijkstra (1972) and later
investigated by Parnas (Parnas, 1976, 1979). More recently, Weiss (Weiss, 1997, 1998;
Ardis and Weiss, 1997) has considered the concept of a program family in the context of
what he terms Family oriented Abstraction, Specification and Translation (FAST) (Cuka
and Weiss, 1997; Weiss and Lai, 1999).
In the approach advocated by Weiss, the first step is a commonality analysis. This
analysis consists of systematically identifying and documenting the commonalities that all
program family members share, the variabilities between family members and the termi-
nology used in describing the family. A commonality analysis provides a systematic way of
gaining confidence that a family is worth building and of deciding what the family members
will be. A commonality analysis document provides the following benefits (Weiss, 1997,
1998):
1. A starting point for the design of a domain specific languages (DSL): Once a DSL,
or application modelling language, is developed the program family members can be
rapidly generated by specifying a given family member using the language.
2. A basis for a common design for all family members: When the software engineers
come to designing the individual family members, they can take advantage of the
commonalities to reuse code. Moreover, the variabilities can be considered in the
design so that they can be easily accommodated. One approach to the design may
be to decompose the system into components that can each be customized by speci-
fication of values for its various parameters, where the parameters correspond to the
parameters of variation identified in the commonality analysis document.
3. A historical reference: This document records the important issues concerning the
scope and the nature of the family (as well as some unsolved issues) to facilitate the
involvement of the participants in maintaining and evolving the family.
CAS-04-10-SS
4
4. A basis for reengineering a domain: Existing projects may not have been developed
using software engineering methodologies. The projects can be systematically reor-
ganized and redesigned with the aid of a commonality analysis to unify the existing
products.
5. A basic training reference for new software developers: This document provides the
necessary basic information for a new team-member to understand the family.
The next section will show that the above uses of the commonality analysis document
will be beneficial for the development of a program family of mesh generators. The com-
monality analysis will benefit all subsequent stages in the software development process.
For instance, as mentioned in the introduction, the commonalities will act as requirements
that will be the starting point for writing a software requirements specification. The com-
monalities will be refined into specific requirements by fixing the value of their associated
variabilities. The change in the values of the variabilities then corresponds to the change
from one program family member to another.
As commonalities and variabilities are requirements, they should express “What” func-
tionalities and qualities the system should have, and not mention “How” these requirements
are to be accomplished. That is, the commonalities and variabilities should not involve
design decisions. The design decisions will be made after the requirements for a family
member have been specified. The one exception to this is system constraints, which are
requirements that explicitly make design decisions.
Besides the “What” versus “How” test, there are other tests that can be used to review
commonalities and variabilities, as proposed by Weiss (1997). One such test is the “what
is ruled out” test. This test determines if a commonality or variability actually makes a
decision because if no alternatives are ruled out then no decision has really been made.
Another test is the “negation” test. If the negation of a decision represents a position
that someone might argue for, then the original decision is likely to be meaningful. For
instance, the statement that “the software should be reliable” has a negation that no one
would likely argue for and thus the statement does not represent a good characterization
of a goal for the system.
In Weiss (1997) the stages of a commonality analysis are described in a systematic way.
The stages include the following: prepare, plan, analyze, quantify and review. The stages
are completed through the aid of a moderator and a series of meeting and preliminary
documents and documentation reviews. Although the systematic approach advocated by
Weiss has its advantages, for the case of writing a commonality analysis document for mesh
generating systems it was decided that a less structured approach is feasible. Mesh genera-
tors are simpler than other software systems in the sense that they have fewer interactions
with the environment. Also, the theory of mesh generation has a solid mathematical ba-
sis that can be used to remove some of the ambiguity that Weiss’s approach is aimed
at reducing. Therefore, the approach adopted here is to revise the original commonality
analysis document produced by Chen (2003) and then make the new document available
to others for review. The new document will be maintained in a concurrent versioning
CAS-04-10-SS
5
system repository so that multiple authors can work on it, and more importantly, so that
the documentation’s revision history can be tracked and the documentation can be rolled
back to an earlier version if necessary.


\subsection{Lattice Boltzmann Systems}
\newpage
\section{Terminology and Definitions}
This section is divided into two subsections. The first discusses the terminology that comes
from the software engineering field, while the second presents the definitions used in mesh
generation. Common acronyms are also listed in this section. The lists are not intended
to be read sequentially, but rather to be consulted for reference purposes; therefore, the
terms are ordered alphabetically, with the consequence that some terms that appear early
in the list depend on the definitions of later terms.

\subsection{Software Engineering Related Definitions and Acronyms}
Commonality: A requirement or goal common to all family members.
Goal: “Goals capture, at different levels of abstraction, the various objectives the system
under consideration should achieve.” van Lamsweerde (2001)
Program Family: A set of programs that are analyzed and designed together starting
from the initial stages of the software development life-cycle.
Requirements: A software requirement is: i) a condition or capability needed by a user
to solve a problem or achieve an objective; ii) a condition or capability that must be
met or possessed by a system or system component to satisfy a contract, standard,
specification, or other formally imposed document; or, iii) a documented represen-
CAS-04-10-SS
10
tation of a condition or capability as in the above two definitions.
Dorfman, 2000)
Variability: A requirement or goal that varies between family members.
\subsection{Lattice Boltzmann Related Definitions and Acronyms}
\newpage
\section{Commonalities}
\subsection{Lattice Boltzmann Method Solvers}
\noindent
\begin{minipage}{\textwidth}
	\renewcommand*{\arraystretch}{1.5}
	\begin{tabular}{| p{\colAwidth} | p{\colBwidth}|}
		\hline
		\bf Item Number& C\refstepcounter{comnum}\thecomnum \\
		\hline
	\end{tabular}\\
	
	\begin{tabular}{| p{\colAwidth} | p{\colBwidth}|}		
		\hline
		\bf Description & \\
		\hline
		\bf Related Variability & \vref{}\\
		\hline
		\bf History & \\
		\hline
	\end{tabular}
\end{minipage}\\
~\newline\\
~\newline
\noindent
\begin{minipage}{\textwidth}
	\renewcommand*{\arraystretch}{1.5}
	\begin{tabular}{| p{\colAwidth} | p{\colBwidth}|}
		\hline
		\bf Item Number& C\refstepcounter{comnum}\thecomnum \\
		\hline
	\end{tabular}\\
	
	\begin{tabular}{| p{\colAwidth} | p{\colBwidth}|}		
		\hline
		\bf Description & \\
		\hline
		\bf Related Variability & \vref{}\\
		\hline
		\bf History & \\
		\hline
	\end{tabular}
\end{minipage}\\
\subsection{Input}
\noindent
\begin{minipage}{\textwidth}
	\renewcommand*{\arraystretch}{1.5}
	\begin{tabular}{| p{\colAwidth} | p{\colBwidth}|}
		\hline
		\bf Item Number& C\refstepcounter{comnum}\thecomnum \\
		\hline
	\end{tabular}\\	

	\begin{tabular}{| p{\colAwidth} | p{\colBwidth}|}		
		\hline
		\bf Description & \\
		\hline
		\bf Related Variability & \vref{}\\
		\hline
		\bf History & \\
		\hline
	\end{tabular}
\end{minipage}\\
~\newline\\
~\newline
\noindent
\begin{minipage}{\textwidth}
	\renewcommand*{\arraystretch}{1.5}
	\begin{tabular}{| p{\colAwidth} | p{\colBwidth}|}
		\hline
		\bf Item Number& C\refstepcounter{comnum}\thecomnum \\
		\hline
	\end{tabular}\\

	\begin{tabular}{| p{\colAwidth} | p{\colBwidth}|}		
		\hline
		\bf Description & \\
		\hline
		\bf Related Variability & \vref{}\\
		\hline
		\bf History & \\
		\hline
	\end{tabular}
\end{minipage}\\
\subsection{Output}
\noindent
\begin{minipage}{\textwidth}
	\renewcommand*{\arraystretch}{1.5}
	\begin{tabular}{| p{\colAwidth} | p{\colBwidth}|}
		\hline
		\bf Item Number& C\refstepcounter{comnum}\thecomnum \\
		\hline
	\end{tabular}\\
	
	\begin{tabular}{| p{\colAwidth} | p{\colBwidth}|}		
		\hline
		\bf Description & \\
		\hline
		\bf Related Variability & \vref{}\\
		\hline
		\bf History & \\
		\hline
	\end{tabular}
\end{minipage}\\
~\newline\\
~\newline
\noindent
\begin{minipage}{\textwidth}
	\renewcommand*{\arraystretch}{1.5}
	\begin{tabular}{| p{\colAwidth} | p{\colBwidth}|}
		\hline
		\bf Item Number& C\refstepcounter{comnum}\thecomnum \\
		\hline
	\end{tabular}\\
	
	\begin{tabular}{| p{\colAwidth} | p{\colBwidth}|}		
		\hline
		\bf Description & \\
		\hline
		\bf Related Variability & \vref{}\\
		\hline
		\bf History & \\
		\hline
	\end{tabular}
\end{minipage}\\
\subsection{Nonfunctional Requirements}
\noindent
\begin{minipage}{\textwidth}
	\renewcommand*{\arraystretch}{1.5}
	\begin{tabular}{| p{\colAwidth} | p{\colBwidth}|}
		\hline
		\bf Item Number& C\refstepcounter{comnum}\thecomnum \\
		\hline
	\end{tabular}\\
	
	\begin{tabular}{| p{\colAwidth} | p{\colBwidth}|}		
		\hline
		\bf Description & \\
		\hline
		\bf Related Variability & \vref{}\\
		\hline
		\bf History & \\
		\hline
	\end{tabular}
\end{minipage}\\
~\newline\\
~\newline

\noindent
\begin{minipage}{\textwidth}
	\renewcommand*{\arraystretch}{1.5}
	\begin{tabular}{| p{\colAwidth} | p{\colBwidth}|}
		\hline
		\bf Item Number& C\refstepcounter{comnum}\thecomnum \\
		\hline
	\end{tabular}\\
	
	\begin{tabular}{| p{\colAwidth} | p{\colBwidth}|}		
		\hline
		\bf Description & \\
		\hline
		\bf Related Variability & \vref{}\\
		\hline
		\bf History & \\
		\hline
	\end{tabular}
\end{minipage}\\
\newpage
\section{Variabilities}
\subsection{Lattice Boltzmann Method Solvers}
\noindent
\begin{minipage}{\textwidth}
	\renewcommand*{\arraystretch}{1.5}
	\begin{tabular}{| p{\colCwidth} | p{\colDwidth}|}
		\hline
		\bf Item Number& V\refstepcounter{varnum}\thevarnum \\
		\hline
	\end{tabular}\\
	
	\begin{tabular}{| p{\colCwidth} | p{\colDwidth}|}		
		\hline
		\bf Description & \\
		\hline
		\bf Related Commonality & \cref{}\\
		\hline
		\bf Related Parameter & \pref{}\\
		\hline
		\bf History & \\
		\hline
	\end{tabular}
\end{minipage}\\
~\newline\\
~\newline
\noindent
\begin{minipage}{\textwidth}
	\renewcommand*{\arraystretch}{1.5}
	\begin{tabular}{| p{\colCwidth} | p{\colDwidth}|}
		\hline
		\bf Item Number& V\refstepcounter{varnum}\thevarnum \\
		\hline
	\end{tabular}\\
	
	\begin{tabular}{| p{\colCwidth} | p{\colDwidth}|}		
		\hline
		\bf Description & \\
		\hline
		\bf Related Commonality & \cref{}\\
		\hline
		\bf Related Parameter & \pref{}\\
		\hline
		\bf History & \\
		\hline
	\end{tabular}
\end{minipage}\\
\subsection{Input}
\noindent
\begin{minipage}{\textwidth}
	\renewcommand*{\arraystretch}{1.5}
	\begin{tabular}{| p{\colCwidth} | p{\colDwidth}|}
		\hline
		\bf Item Number& V\refstepcounter{varnum}\thevarnum \\
		\hline
	\end{tabular}\\
	
	\begin{tabular}{| p{\colCwidth} | p{\colDwidth}|}		
		\hline
		\bf Description & \\
		\hline
		\bf Related Commonality & \cref{}\\
		\hline
		\bf Related Parameter & \pref{}\\
		\hline
		\bf History & \\
		\hline
	\end{tabular}
\end{minipage}\\
~\newline\\
~\newline
\noindent
\begin{minipage}{\textwidth}
	\renewcommand*{\arraystretch}{1.5}
	\begin{tabular}{| p{\colCwidth} | p{\colDwidth}|}
		\hline
		\bf Item Number& V\refstepcounter{varnum}\thevarnum \\
		\hline
	\end{tabular}\\
	
	\begin{tabular}{| p{\colCwidth} | p{\colDwidth}|}		
		\hline
		\bf Description & \\
		\hline
		\bf Related Commonality & \cref{}\\
		\hline
		\bf Related Parameter & \pref{}\\
		\hline
		\bf History & \\
		\hline
	\end{tabular}
\end{minipage}\\
\subsection{Output}
\noindent
\begin{minipage}{\textwidth}
	\renewcommand*{\arraystretch}{1.5}
	\begin{tabular}{| p{\colCwidth} | p{\colDwidth}|}
		\hline
		\bf Item Number& V\refstepcounter{varnum}\thevarnum \\
		\hline
	\end{tabular}\\
	
	\begin{tabular}{| p{\colCwidth} | p{\colDwidth}|}		
		\hline
		\bf Description & \\
		\hline
		\bf Related Commonality & \cref{}\\
		\hline
		\bf Related Parameter & \pref{}\\
		\hline
		\bf History & \\
		\hline
	\end{tabular}
\end{minipage}\\
~\newline\\
~\newline
\noindent
\begin{minipage}{\textwidth}
	\renewcommand*{\arraystretch}{1.5}
	\begin{tabular}{| p{\colCwidth} | p{\colDwidth}|}
		\hline
		\bf Item Number& V\refstepcounter{varnum}\thevarnum \\
		\hline
	\end{tabular}\\
	
	\begin{tabular}{| p{\colCwidth} | p{\colDwidth}|}		
		\hline
		\bf Description & \\
		\hline
		\bf Related Commonality & \cref{}\\
		\hline
		\bf Related Parameter & \pref{}\\
		\hline
		\bf History & \\
		\hline
	\end{tabular}
\end{minipage}\\
\subsection{System Constraints}
\noindent
\begin{minipage}{\textwidth}
	\renewcommand*{\arraystretch}{1.5}
	\begin{tabular}{| p{\colCwidth} | p{\colDwidth}|}
		\hline
		\bf Item Number& V\refstepcounter{varnum}\thevarnum \\
		\hline
	\end{tabular}\\
	
	\begin{tabular}{| p{\colCwidth} | p{\colDwidth}|}		
		\hline
		\bf Description & \\
		\hline
		\bf Related Commonality & \cref{}\\
		\hline
		\bf Related Parameter & \pref{}\\
		\hline
		\bf History & \\
		\hline
	\end{tabular}
\end{minipage}\\
~\newline\\
~\newline
\noindent
\begin{minipage}{\textwidth}
	\renewcommand*{\arraystretch}{1.5}
	\begin{tabular}{| p{\colCwidth} | p{\colDwidth}|}
		\hline
		\bf Item Number& V\refstepcounter{varnum}\thevarnum \\
		\hline
	\end{tabular}\\
	
	\begin{tabular}{| p{\colCwidth} | p{\colDwidth}|}		
		\hline
		\bf Description & \\
		\hline
		\bf Related Commonality & \cref{}\\
		\hline
		\bf Related Parameter & \pref{}\\
		\hline
		\bf History & \\
		\hline
	\end{tabular}
\end{minipage}\\
\subsection{Nonfunctional Requirements}
\noindent
\begin{minipage}{\textwidth}
	\renewcommand*{\arraystretch}{1.5}
	\begin{tabular}{| p{\colCwidth} | p{\colDwidth}|}
		\hline
		\bf Item Number& V\refstepcounter{varnum}\thevarnum \\
		\hline
	\end{tabular}\\
	
	\begin{tabular}{| p{\colCwidth} | p{\colDwidth}|}		
		\hline
		\bf Description & \\
		\hline
		\bf Related Commonality & \cref{}\\
		\hline
		\bf Related Parameter & \pref{}\\
		\hline
		\bf History & \\
		\hline
	\end{tabular}
\end{minipage}\\
~\newline\\
~\newline
\noindent
\begin{minipage}{\textwidth}
	\renewcommand*{\arraystretch}{1.5}
	\begin{tabular}{| p{\colCwidth} | p{\colDwidth}|}
		\hline
		\bf Item Number& V\refstepcounter{varnum}\thevarnum \\
		\hline
	\end{tabular}\\
	
	\begin{tabular}{| p{\colCwidth} | p{\colDwidth}|}		
		\hline
		\bf Description & \\
		\hline
		\bf Related Commonality & \cref{}\\
		\hline
		\bf Related Parameter & \pref{}\\
		\hline
		\bf History & \\
		\hline
	\end{tabular}
\end{minipage}\\
\newpage
\section{Parameters of Variation}
\subsection{Lattice Boltzmann Method Solvers}
\noindent
\begin{minipage}{\textwidth}
	\renewcommand*{\arraystretch}{1.5}
	\begin{tabular}{| p{\colEwidth} | p{\colFwidth}|}
		\hline
		\bf Item Number& P\refstepcounter{parnum}\theparnum \\
		\hline
	\end{tabular}\\
	
	\begin{tabular}{| p{\colEwidth} | p{\colFwidth}|}		
		\hline
		\bf Corresponding Variability & \vref{}\\
		\hline
		\bf Range of Parameters & \\
		\hline
		\bf Binding Time & \\
		\hline
	\end{tabular}
\end{minipage}\\
~\newline\\
~\newline
\noindent
\begin{minipage}{\textwidth}
	\renewcommand*{\arraystretch}{1.5}
	\begin{tabular}{| p{\colEwidth} | p{\colFwidth}|}
		\hline
		\bf Item Number& P\refstepcounter{parnum}\theparnum \\
		\hline
	\end{tabular}\\
	
	\begin{tabular}{| p{\colEwidth} | p{\colFwidth}|}		
		\hline
		\bf Corresponding Variability & \vref{}\\
		\hline
		\bf Range of Parameters & \\
		\hline
		\bf Binding Time & \\
		\hline
	\end{tabular}
\end{minipage}\\
\subsection{Input}
\noindent
\begin{minipage}{\textwidth}
	\renewcommand*{\arraystretch}{1.5}
	\begin{tabular}{| p{\colEwidth} | p{\colFwidth}|}
		\hline
		\bf Item Number& P\refstepcounter{parnum}\theparnum \\
		\hline
	\end{tabular}\\
	
	\begin{tabular}{| p{\colEwidth} | p{\colFwidth}|}		
		\hline
		\bf Corresponding Variability & \vref{}\\
		\hline
		\bf Range of Parameters & \\
		\hline
		\bf Binding Time & \\
		\hline
	\end{tabular}
\end{minipage}\\
~\newline\\
~\newline
\noindent
\begin{minipage}{\textwidth}
	\renewcommand*{\arraystretch}{1.5}
	\begin{tabular}{| p{\colEwidth} | p{\colFwidth}|}
		\hline
		\bf Item Number& P\refstepcounter{parnum}\theparnum \\
		\hline
	\end{tabular}\\
	
	\begin{tabular}{| p{\colEwidth} | p{\colFwidth}|}		
		\hline
		\bf Corresponding Variability & \vref{}\\
		\hline
		\bf Range of Parameters & \\
		\hline
		\bf Binding Time & \\
		\hline
	\end{tabular}
\end{minipage}\\
\subsection{Output}
\noindent
\begin{minipage}{\textwidth}
	\renewcommand*{\arraystretch}{1.5}
	\begin{tabular}{| p{\colEwidth} | p{\colFwidth}|}
		\hline
		\bf Item Number& P\refstepcounter{parnum}\theparnum \\
		\hline
	\end{tabular}\\
	
	\begin{tabular}{| p{\colEwidth} | p{\colFwidth}|}		
		\hline
		\bf Corresponding Variability & \vref{}\\
		\hline
		\bf Range of Parameters & \\
		\hline
		\bf Binding Time & \\
		\hline
	\end{tabular}
\end{minipage}\\
~\newline\\
~\newline
\noindent
\begin{minipage}{\textwidth}
	\renewcommand*{\arraystretch}{1.5}
	\begin{tabular}{| p{\colEwidth} | p{\colFwidth}|}
		\hline
		\bf Item Number& P\refstepcounter{parnum}\theparnum \\
		\hline
	\end{tabular}\\
	
	\begin{tabular}{| p{\colEwidth} | p{\colFwidth}|}		
		\hline
		\bf Corresponding Variability & \vref{}\\
		\hline
		\bf Range of Parameters & \\
		\hline
		\bf Binding Time & \\
		\hline
	\end{tabular}
\end{minipage}\\
\subsection{System Constraints}
\noindent
\begin{minipage}{\textwidth}
	\renewcommand*{\arraystretch}{1.5}
	\begin{tabular}{| p{\colEwidth} | p{\colFwidth}|}
		\hline
		\bf Item Number& P\refstepcounter{parnum}\theparnum \\
		\hline
	\end{tabular}\\
	
	\begin{tabular}{| p{\colEwidth} | p{\colFwidth}|}		
		\hline
		\bf Corresponding Variability & \vref{}\\
		\hline
		\bf Range of Parameters & \\
		\hline
		\bf Binding Time & \\
		\hline
	\end{tabular}
\end{minipage}\\
~\newline\\
~\newline
\noindent
\begin{minipage}{\textwidth}
	\renewcommand*{\arraystretch}{1.5}
	\begin{tabular}{| p{\colEwidth} | p{\colFwidth}|}
		\hline
		\bf Item Number& P\refstepcounter{parnum}\theparnum \\
		\hline
	\end{tabular}\\
	
	\begin{tabular}{| p{\colEwidth} | p{\colFwidth}|}		
		\hline
		\bf Corresponding Variability & \vref{}\\
		\hline
		\bf Range of Parameters & \\
		\hline
		\bf Binding Time & \\
		\hline
	\end{tabular}
\end{minipage}\\
\subsection{Nonfunctional Requirements}
\noindent
\begin{minipage}{\textwidth}
	\renewcommand*{\arraystretch}{1.5}
	\begin{tabular}{| p{\colEwidth} | p{\colFwidth}|}
		\hline
		\bf Item Number& P\refstepcounter{parnum}\theparnum \\
		\hline
	\end{tabular}\\
	
	\begin{tabular}{| p{\colEwidth} | p{\colFwidth}|}		
		\hline
		\bf Corresponding Variability & \vref{}\\
		\hline
		\bf Range of Parameters & \\
		\hline
		\bf Binding Time & \\
		\hline
	\end{tabular}
\end{minipage}\\
~\newline\\
~\newline
\noindent
\begin{minipage}{\textwidth}
	\renewcommand*{\arraystretch}{1.5}
	\begin{tabular}{| p{\colEwidth} | p{\colFwidth}|}
		\hline
		\bf Item Number& P\refstepcounter{parnum}\theparnum \\
		\hline
	\end{tabular}\\
	
	\begin{tabular}{| p{\colEwidth} | p{\colFwidth}|}		
		\hline
		\bf Corresponding Variability & \vref{}\\
		\hline
		\bf Range of Parameters & \\
		\hline
		\bf Binding Time & \\
		\hline
	\end{tabular}
\end{minipage}\\
\newpage
\section{Issues}
\newpage
\bibliographystyle {plainnat}
\bibliography {../../CommonFiles/ResearchProposal}
\newpage
\section{Appendix}
\end{document}