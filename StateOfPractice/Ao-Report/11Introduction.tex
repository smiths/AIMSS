\chapter{Introduction}
\label{ch_intro}
This paper analyzes the state of the practice for Medical Imaging (MI) software. MI is the clinical tool to image the interior of a body, providing information for diagnostic, analytic, and medical applications \cite{FDA2021} \cite{enwiki:1034887445}. MI is an essential part of collecting accurate information during clinical diagnosis \cite{Zhang2008}. MI computing and software aim to visualize and process medical images and produce clinically meaningful information \cite{enwiki:1034877594}.

Scientific computing (SC) is the technique and process of mathematically modeling real-world engineering or scientific systems with computing tools, as well as analysis and prediction with the models \cite{Smith2006}. MI software belongs to a specific domain of SC.

We aim to study the current status of SC software development in the MI domain; understand the current merits, drawbacks, and pain points during the development process, as well as the software qualities in the domain; provide guidelines and recommendations for future development.

Section \ref{sec_motivation} presents our motivation to start the research set the above goals, Section \ref{sec_research_questions} lists our research questions, and Section \ref{sec_scope} explains the domain analysis of MI software scope of our research.

\section{Motivation}
\label{sec_motivation}
Most scientists think developing and using SC software play significant roles in their research \cite{Hannay2009}. They spend a substantial proportion of their working hours on SC software development \cite{Hannay2009} \cite{Prabhu2011}, and this proportion of time has increased over the years \cite{Hannay2009}. 

Developing SC software requires solid knowledge in specific domains \cite{Wilson2014}. Many of them learn software engineering skills by themselves or from their peers, instead of proper training \cite{Hannay2009}. Hannay et al. \cite{Hannay2009} also pointed out that many scientists showed ignorance and indifference to standard software engineering concepts. According to a survey by Prabhu et al. \cite{Prabhu2011}, more than half of the 114 subjects did not use any proper debugger for their software.

Due to its nature, SC software born from one project can be part of many other projects in the future, with the potential to disproportionately causing damages to scientific researches \cite{Wilson2014}. 

As a result, the development process and quality of SC software concern us. We want to understand their status in SC domains and improve them.

\section{Research Questions}
\label{sec_research_questions}
To achieve our objectives, we designed a few research questions and tried to answer them by our research methods. The questions are as follows,

\begin{enumerate}
\item What are the pain points for developers working on MI software projects? What are the solutions for these pain points?
\item What artifacts the projects generated?
\item What role does documentation play in the projects? What are the developers' attitudes toward it?
\item What principles, processes, methodologies, and tools the projects used?
\item What is the current status of the following software qualities for the projects? What actions have the developers taken to address them?
\begin{itemize}
	\item Installability
	\item Correctness \& Verifiability
	\item Reliability
	\item Robustness
	\item Usability
	\item Maintainability
	\item Reusability
	\item Understandability
	\item Visibility/Transparency
\end{itemize}
\item How does software quality ranking generated by our methods compare with the ratings from the community?
\end{enumerate}

\section{Scope}
\label{sec_scope}
According to Bankman \cite{Bankman2000}, MI software deals with six different basic problems, while Angenentet et al. \cite{Angenent2006} pointed out that four fundamental problems are solved by MI software. While both mentioned Segmentation, Registration, and Visualization of medical images, Bankman also included Enhancement, Quantification, and a section covering some other functions \cite{Bankman2000}. On the other hand, Angenent's team included Simulation \cite{Angenent2006}. According to Wikipedia contributors \cite{enwiki:1034877594}, MI software has primary functions in categories such as Segmentation, Registration, Visualization (including the basic display, reformatted views, and 3D volume rendering), Statistical Analysis, Image-based Physiological Modelling, etc. As Kim et al. \cite{Kim2011} describe, the general steps of medical image analysis after obtaining digital data include Enhancement, Segmentation, Feature Extraction, Classification, and Interpretation. Besides the above major functions, some MI software provides supportive functions. For example, with Tool Kit libraries VTK \cite{SchroederEtAl2006} and ITK \cite{McCormick2014}, developers build software with Visualization and Analysis functions; Picture Archiving and Communication System (PACS) helps users to economically store and conveniently access images \cite{Choplin1992}. 

We divided MI software into five sub-groups and several sub-sub-groups by their major functions shown in Figure \ref{fig_mi_functions}.

\begin{figure}
\centering
\begin{tikzpicture}[mindmap, grow cyclic, every node/.style=concept, concept
color=orange!40,
level 1/.append style={level distance=5cm,sibling angle=72},
level 2/.append style={level distance=3cm,sibling angle=90},
level 3/.style={level distance=2.3cm,sibling angle=90}]
\node{MI software}
child[concept color=teal!40] { node {Enhancement}}
child[concept color=teal!40] { node {Analysis}
    child[concept color=blue!30] { node {Registration}}
    child[concept color=blue!30,] { node {Segmentation}}
    child[concept color=blue!30] { node {Statistical Analysis}
    child[concept color=blue!20] { node {Feature Extraction}}
    child[concept color=blue!20] { node {Classification}}
    child[concept color=blue!20] { node {Interpretation}}
    }}
child[concept color=teal!40] { node {Simulation/\\Modeling}}
child[concept color=teal!40] { node {Supporting}
    child[concept color=blue!30] { node {Tool Kit}}
    child[concept color=blue!30] { node {PACS}}
}
child[concept color=teal!40] { node {Visualization}
    child[concept color=blue!30] { node {2D Display}}
    child[concept color=blue!30] { node {3D Rendering}}
    child[concept color=blue!30] { node {Reformatted Views}}
};
\end{tikzpicture}
\caption{Major functions of MI software}
\label{fig_mi_functions}
\end{figure}

In this project, the scope of the software is limited to the software library providing the Visualization tools and functions.
