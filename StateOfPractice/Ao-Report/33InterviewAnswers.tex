\chapter{Other Interview Answers}
\label{ap_interview}

We asked 20 interview questions to the nine interviewees from eight software projects. We discuss the answers to interview questions 5, 9, 10, 11, 12, 13, 14, and 19 in Section \ref{ch_interview}, and summarize the answers to the other questions in this section.

\noindent\textbf{Q1. Interviewees’ current position/title? degrees?}

Six of the nine interviewees revealed their position/title, such as CEO of a company, endowed chair and professor in universities, software engineers in a commercial company and a hospital.

Most of them answered their backgrounds and degrees. Table \ref{tab_q1_degrees} shows the highest academic degrees the participants have, and Table \ref{tab_q1_majors} shows what majors they studied. Many of the interviewees studied in multiple majors.

\begin{table}[H]
\centering
\begin{tabular}{ll}
\hline
Highest degree & Number of interviewees \\ \hline
PHD & 4 \\
Master & 3 \\
Bachelor & 0 \\
Unspecified academic degree & 2 \\ \hline
\end{tabular}
\caption{\label{tab_q1_degrees}Interviewees' highest academic degrees}
\end{table}

\begin{table}[H]
\centering
\begin{tabular}{ll}
\hline
Major & Number of interviewees \\ \hline
Computer Science & 4 \\
Physics & 2 \\
Biomedical Engineering & 1 \\
Neuroimaging & 1 \\
Geology (image analysis) & 1 \\
Media Arts and Sciences & 1 \\
Mechanical Engineering & 1 \\
Materials Engineering & 1 \\
Psychology & 1 \\ \hline
\end{tabular}
\caption{\label{tab_q1_majors}Interviewees' majors at university}
\end{table}

\noindent\textbf{Q2. Interviewees’ contribution to/relationship with the software?}

Table \ref{tab_q2_roles} shows the interviewees’ roles and responsibilities in the projects. One of the participants did not explicitly mention his role, but implicitly revealed that he was a primary contributor to the project.

\begin{table}[H]
\centering
\begin{tabular}{ll}
\hline
Role in the projects & Number of interviewees \\ \hline
Chief Architect & 2 \\
Lead Developer & 1 \\
Core Developer & 5 \\
Unspecified & 1 \\ \hline
\end{tabular}
\caption{\label{tab_q2_roles}Interviewees' roles in the projects}
\end{table}

\noindent\textbf{Q3. Length of time the interviewee has been involved with this software?}

Table \ref{tab_q3_years} shows the distribution of the lengths of time the interviewees had worked on the projects.

\begin{table}[H]
\centering
\begin{tabular}{ll}
\hline
Length of time in the projects & Number of interviewees \\ \hline
0-1 & 1 \\
2-5 & 0 \\
6-10 & 2 \\
11-15 & 3 \\
16-20 & 2 \\
21-25 & 1 \\ \hline
\end{tabular}
\caption{\label{tab_q3_years}Lengths of time that the interviewees worked in the projects}
\end{table}

\noindent\textbf{Q4. How large is the development group?}

The size of each group grows and shrinks over the years. Most teams mentioned that the team members join and leave. Some teams said that when there was sufficient funding, they could afford more developers.

Table \ref{tab_q4_members} shows the numbers of active members at the time of interviews. The members include people working on development and project management.

\begin{table}[H]
\centering
\begin{tabular}{ll}
\hline
Number of current members & Number of projects \\ \hline
1-3 & 5 \\
4-6 & 3 \\ \hline
\end{tabular}
\caption{\label{tab_q4_members}Numbers of current members in the projects}
\end{table}

As shown in the table, no team had a vast number of members. Some projects had more developers, such as \textit{3D Slicer}; on the other hand, some teams such as \textit{dwv} had only one primary developer, plus a maximum of two or three developers occasionally.

\textit{3D Slicer} is a special case, because it supports third-party extensions. So there have been community members developing and maintaining these extensions. Table \ref{tab_q4_members} does not include these members.

\noindent\textbf{Q6. What is the typical background of a developer?}

Not all interviewees could clearly answer this question. Many of them talked about the backgrounds of members with who they were familiar. Table \ref{tab_q6_dev_backgrounds} shows the number of times all interviewees mentioned a background.

\begin{table}[H]
\centering
\begin{tabular}{ll}
\hline
Background of a developer & Number of interviewees with the answer \\ \hline
\begin{tabular}[c]{@{}l@{}}Computer Science, Information\\ Technology, and Software Development \end{tabular} & 6 \\
Imaging & 2 \\
Medical Imaging & 1 \\
Mathematics & 1 \\
Biomedical Engineering & 1 \\
Computer Aided Medical Procedures & 1 \\
Physician & 1 \\ \hline
\end{tabular}
\caption{\label{tab_q6_dev_backgrounds}Backgrounds of developers by the numbers of interviewees with the answers}
\end{table}

\noindent\textbf{Q7. What is your estimated number of users? How did you come up with that estimate?}

None of the interviewees knew the exact number of users. Some of them provided estimations based on different facts. However, we do not think these numbers are comparable to each other.

\begin{table}[H]
\centering
\begin{tabular}{lll}
\hline
Software & Rough estimation & Considered facts \\ \hline
3D Slicer & 100,000 & \begin{tabular}[c]{@{}l@{}} Search results on Google Scholar;\\ number of new posts per year on slicer.org; \\ number of downloads. \end{tabular} \\
INVESALIUS 3 & 75,000 & Number of random IDs created by new installation.\\
dwv & No estimation & About 20 companies integrated \textit{dwv} in their products. \\
BioImage Suite Web & 100 active users & \begin{tabular}[c]{@{}l@{}}  The  interviewee only counted the users from \\ several Universities who were active users. \end{tabular} \\
ITK-SNAP & 10,000 plus & Number of downloads. \\
MRIcroGL & No estimation & \begin{tabular}[c]{@{}l@{}}  It is the top 1 downloaded software on this NITRC list \\ \hyperlink{https://www.nitrc.org/top/toplist.php?type=downloads}{https://www.nitrc.org/top/toplist.php?type=downloads} \end{tabular} \\
Weasis & \begin{tabular}[c]{@{}l@{}}  10,000 user used \\ it as least once \end{tabular} & Number of profiles. \\
OHIF & About 5000 & \begin{tabular}[c]{@{}l@{}}   Some platforms integrated \textit{OHIF}, and it was hard \\ to know the number of end users. \end{tabular} \\ \hline
\end{tabular}
\caption{\label{tab_q7_num_users}Rough Estimations for the Number of Users}
\end{table}

Table \ref{tab_q7_num_users} shows the estimations and how the interviewees made them. It is clear that some estimated only the active users, and some counted users who had used only once. So we do not compare these numbers.

\noindent\textbf{Q8. What is the typical background of a user?}

All interviewees provided several different user backgrounds, and all of them mentioned medical researchers or medical professionals. Table \ref{tab_q8_user_backgrounds} shows the number of times all interviewees mentioned a background.

\begin{table}[H]
\centering
\begin{tabular}{ll}
\hline
Background of a user & Number of interviewees with the answer \\ \hline
Medical Researchers & 6 \\
Doctors/Health care professionals/Surgeons & 5 \\
Student Researchers & 4 \\
Patients & 3 \\
Paleontologist & 1 \\
Biomechanical Engineers & 1 \\
Imaging Researchers & 1 \\
Mechanical Engineers & 1 \\ \hline
\end{tabular}
\caption{\label{tab_q8_user_backgrounds}Backgrounds of users by the numbers of interviewees with the answers}
\end{table}

\noindent\textbf{Q15. Was it hard to ensure the correctness of the software? If there were any obstacles, what methods have been considered or practiced to improve the situation? If practiced, did it work?}

Table \ref{tab_q15_threats_correctness} shows the threats to \textit{correctness} by the numbers of interviewees with the answers.

\begin{table}[H]
\centering
\begin{tabular}{ll}
\hline
Threat to correctness & Number of interviewees with the answer \\ \hline
\begin{tabular}[c]{@{}l@{}}Clinical systems produce data in various formats \\ (e.g. DICOM), which have complicated standards.\\ The software needs to handle the complexity.\end{tabular} & 2 \\
\begin{tabular}[c]{@{}l@{}}Different types of MI machines can create data \\ in slightly different ways, adding more complexity\\ for the software to handle.\end{tabular} & 2 \\
\begin{tabular}[c]{@{}l@{}}Besides viewing, the software has additional \\ functions, which lead to extra complexity.\end{tabular} & 1 \\
There is lack of real word image data for testing. & 1 \\
\begin{tabular}[c]{@{}l@{}}The team cannot use private data for debugging,\\ even when the data cause problems.\end{tabular} & 1 \\
\begin{tabular}[c]{@{}l@{}}The software uses huge datasets for testing, so \\ the tests are expensive and time-consuming.\end{tabular} & 1 \\
It is hard to well manage releases. & 1 \\
The project has no unit tests. & 1 \\
The project has no dedicated quality assurance team. & 1 \\ \hline
\end{tabular}
\caption{\label{tab_q15_threats_correctness}Threats to correctness by the numbers of interviewees with the answers}
\end{table}

Table \ref{tab_q15_strategies_correctness} shows the strategies to ensure \textit{correctness} by the numbers of interviewees with the answers. The interviewees from the \textit{3D Slicer} and \textit{ITK-SNAP} teams thought that the self-tests and automated tests were beneficial and could significantly save time. The interviewee from the \textit{Weasis} team kept collecting medical images for more than ten years. These images have caused problems with the software. So he had samples to test specific problems.

\begin{table}[H]
\centering
\begin{tabular}{ll}
\hline
Strategy to ensure correctness & Number of interviewees with the answer \\ \hline
\begin{tabular}[c]{@{}l@{}}Test-driven development / component tests /\\ integration tests / smoke tests / regression tests.\end{tabular} & 4 \\
Self tests / automated tests. & 3 \\
\begin{tabular}[c]{@{}l@{}}Two stage development process / stable release \&\\ nightly builds.\end{tabular} & 3 \\
CI/CD. & 1 \\
\begin{tabular}[c]{@{}l@{}}Using deidentified copies of medical images for\\ debugging.\end{tabular} & 1 \\
\begin{tabular}[c]{@{}l@{}}Sending the beta versions of software to medical\\ workers who can access the data and do the tests.\end{tabular} & 1 \\
\begin{tabular}[c]{@{}l@{}}Collecting and maintaining a dataset of\\ problematic images.\end{tabular} & 1 \\ \hline
\end{tabular}
\caption{\label{tab_q15_strategies_correctness}Strategies to ensure correctness by the numbers of interviewees with the answers}
\end{table}

\noindent\textbf{Q16. When designing the software, did you consider the ease of future changes? For example, will it be hard to change the system’s structure, modules, or code blocks? What measures have been taken to ensure the ease of future changes and maintains?}

Table \ref{tab_q16_strategies_maintainability} shows the strategies to ensure \textit{maintainability} by the numbers of interviewees with the answers. The modular approach is the most talked-about solution to improve \textit{maintainability}. The \textit{3D Slicer} team used a well-defined structure for the software, which they named as ``event-driven MVC pattern''. Moreover, \textit{3D Slicer} discovers and loads necessary modules at runtime, according to the configuration and installed extensions. The \textit{BioImage Suite Web} team had designed and re-designed their software multiple times in the last 10+
years. They found that their modular approach effectively supported the maintainability \cite{Joshi2011}. 

\begin{table}[H]
\centering
\begin{tabular}{ll}
\hline
Strategy to ensure maintainability & Number of interviewees with the answer \\ \hline
\begin{tabular}[c]{@{}l@{}}Modular approach / thinking the software\\ as reusable Lego bricks / maintain \\ repetitive functions as libraries.\end{tabular} & 5 \\
Supporting third-party extensions. & 1 \\
Easy-to-understand architecture. & 1 \\
Dedicated architect. & 1 \\
Starting from simple solutions. & 1 \\
Documentation. & 1 \\ \hline
\end{tabular}
\caption{\label{tab_q16_strategies_maintainability}Strategies to ensure maintainability by the numbers of interviewees with the answers}
\end{table}

\noindent\textbf{Q17. Provide instances where users have misunderstood the software. What, if any, actions were taken to address understandability issues?}

Table \ref{tab_q17_threats_understandability} shows the threats to \textit{understandability} by the numbers of interviewees with the answers. It separates \textit{understandability} issues to users and developers by the horizontal dash line.

\begin{table}[H]
\centering
\begin{tabular}{ll}
\hline
Threat to understandability & Number of interviewees with the answer \\ \hline
\begin{tabular}[c]{@{}l@{}}Not all users understand how to use some\\ features of the software.\end{tabular} & 2 \\
\begin{tabular}[c]{@{}l@{}}The team has no dedicated user experience\\ (UX) designer.\end{tabular} & 1 \\
\begin{tabular}[c]{@{}l@{}}The software does not make some important\\ indicators noticeable (e.g. a progress bar).\end{tabular} & 1 \\
\begin{tabular}[c]{@{}l@{}}Not all users understand the purpose of the\\ software.\end{tabular} & 1 \\
\begin{tabular}[c]{@{}l@{}}Not all users know if the software includes\\ certain features.\end{tabular} & 1 \\
\begin{tabular}[c]{@{}l@{}}Not all users understand how to use the\\ command line tool.\end{tabular} & 1 \\
\begin{tabular}[c]{@{}l@{}}Not all users understand that the software is \\a web application.\end{tabular} & 1 \\\hdashline
\begin{tabular}[c]{@{}l@{}}Not all developers understand how to deploy\\ the software.\end{tabular} & 1 \\
\begin{tabular}[c]{@{}l@{}}The architecture is difficult for new\\ developers to understand.\end{tabular} & 1 \\ \hline
\end{tabular}
\caption{\label{tab_q17_threats_understandability}Threats to understandability by the numbers of interviewees with the answers}
\end{table}

Table \ref{tab_q17_strategies_understandability} shows the strategies to ensure \textit{understandability} by the numbers of interviewees with the answers.

\begin{table}[H]
\centering
\begin{tabular}{ll}
\hline
Strategy to ensure understandability & Number of interviewees with the answer \\ \hline
\begin{tabular}[c]{@{}l@{}}Documentation / user manual /\\ user mailing list / forum.\end{tabular} & 4 \\
Graphical user interface. & 2 \\
Testing every release with active users. & 1 \\
\begin{tabular}[c]{@{}l@{}}Making simple things simple and complicated\\ things possible.\end{tabular} & 1 \\
Icons with more clear visual expressions. & 1 \\
Designing the software to be intuitive. & 1 \\
Having a UX designer with the right experience. & 1 \\
Dialog windows for important notifications. & 1 \\
\begin{tabular}[c]{@{}l@{}}Providing an example if the users need to build\\ the software by themselves.\end{tabular} & 1 \\ \hline
\end{tabular}
\caption{\label{tab_q17_strategies_understandability}Strategies to ensure understandability by the numbers of interviewees with the answers}
\end{table}

\noindent\textbf{Q18. What, if any, actions were taken to address usability issues?}

Table \ref{tab_q18_strategies_usability } shows the strategies to ensure \textit{usability} by the numbers of interviewees with the answers.

\begin{table}[H]
\centering
\begin{tabular}{ll}
\hline
Strategy to ensure usability & Number of interviewees with the answer \\ \hline
Usability tests and interviews with end users. & 3 \\
Adjusting according to users' feedbacks. & 3 \\
\begin{tabular}[c]{@{}l@{}}Straightforward and intuitively designed interface /\\ professional UX designer.\end{tabular} & 2 \\
\begin{tabular}[c]{@{}l@{}}Providing step-by-step processes, and showing\\ the step numbers.\end{tabular} & 1 \\
\begin{tabular}[c]{@{}l@{}}Making the basic functions easy to use without\\ reading the documentation.\end{tabular} & 1 \\
Focusing on limited number of functions. & 1 \\
Making the software more streamlined. & 1 \\
Downsampling images to consume less memory. & 1 \\
\begin{tabular}[c]{@{}l@{}}An option to load only part of the data to boost\\ performance.\end{tabular} & 1 \\ \hline
\end{tabular}
\caption{\label{tab_q18_strategies_usability }Strategies to ensure usability by the numbers of interviewees with the answers}
\end{table}

\noindent\textbf{Q20. Do you have any concern that your computational results won’t be reproducible in the future? Have you taken any steps to ensure reproducibility?}

Table \ref{tab_q20_threats_reproducibility} shows the threats to \textit{reproducibility} by the numbers of interviewees with the answers.

\begin{table}[H]
\centering
\begin{tabular}{ll}
\hline
Threat to reproducibility & Number of interviewees with the answer \\ \hline
\begin{tabular}[c]{@{}l@{}}If the software is closed-source, the reproducibility\\is hard to achieve.\end{tabular} & 1 \\
The project has no user interaction tests. & 1 \\
The project has no unit tests. & 1 \\
\begin{tabular}[c]{@{}l@{}}Using different versions of some common libraries\\may cause problems.\end{tabular} & 1 \\
CPU variability can leads to non-reproducibility. & 1 \\
\begin{tabular}[c]{@{}l@{}}When reverse-engineering how manufacturers create\\medical images, the team may misinterpret it.\end{tabular} & 1 \\ \hline
\end{tabular}
\caption{\label{tab_q20_threats_reproducibility}Threats to reproducibility by the numbers of interviewees with the answers}
\end{table}

Table \ref{tab_q20_strategies_ reproducibility } shows the strategies to ensure \textit{ reproducibility} by the numbers of interviewees with the answers. The interviewee from the \textit{3D Slicer} team provided various suggestions. One interviewee from another team suggested that they used \textit{3D Slicer} as the benchmark to test their \textit{reproducibility}.

\begin{table}[H]
\centering
\begin{tabular}{ll}
\hline
Strategy to ensure reproducibility & Number of interviewees with the answer \\ \hline
Regression tests / unit tests / having good tests. & 6 \\
\begin{tabular}[c]{@{}l@{}}Making code, data, and documentation available / making\\ the software open-source / using open-source libraries.\end{tabular} & 5 \\
Running same tests on all platforms. & 1 \\
\begin{tabular}[c]{@{}l@{}}A dockerized version of the software,  insulating it from\\ the operating system environment.\end{tabular} & 1 \\
Using standard libraries. & 1 \\
Monitoring the upgrades of the libraries. & 1 \\
Clearly documenting the versions. & 1 \\
\begin{tabular}[c]{@{}l@{}}Bringing along the exact versions of all the dependencies\\ with the software.\end{tabular} & 1 \\
Providing checksums of the data. & 1 \\
\begin{tabular}[c]{@{}l@{}}Benchmark the software against other software with\\ similar purposes.\end{tabular} & 1 \\ \hline
\end{tabular}
\caption{\label{tab_q20_strategies_ reproducibility }Strategies to ensure  reproducibility by the numbers of interviewees with the answers}
\end{table}
