\chapter{Methods}
\label{ch_methods}

\ad{Where to include the domain experts? I plan to include them in  Domain Selection and Software Product Selection, and also mention that they'll participate in interviews.} 

In this project, we aim to study and research the SC software within scientific domains. However, we try to design the whole process of this project in a more general way, with the hope that it should not be limited to only one or several specific software categories.

The way how we choose a domain is listed in Section \ref{sec_domain_selection}. Within a specific domain, we use the process documented in Section \ref{sec_software_selection} to collect and filter the software packages for our study. Then, all the software is measured by using the grading template in Section \ref{sec_grading_template} and the empirical measurement method in Section \label{sec_empirical_measurements}. Section \ref{sec_measuring_qualities} presents more details about how we apply the above process.

\section{Domain Selection}
\label{sec_domain_selection}
Although the methods of this project may not have such limitations, we limit our selection within scientific domains to fulfill the objective of our research.

One major factor in properly choosing a candidate domain for the study is the ease to select software packages within it. As described in Section \ref{sec_software_selection}, the software product selection process is most likely to have a screening step, and the quantity of final-decision packages may not meet our initial expectation if we do not have enough software candidates to choose from. Section \ref{sec_software_selection} also explains why our process prioritize the OSS. Consequently, a scientific domain with a large number of active OSS is proffered. This also often indicates that the domain has an active community developing and using SC software, making it easier to conduct interviews mentioned in Section \ref{sec_interview_methods}.

Even if we can find an adequate number of software packages in a domain, we may still find the software products are developed to solve various problems within the domain and should be categorized into different sub-groups. So one question needs to be asked - do we prefer a group of software all providing similar functions and features, or do we aim to cross-compare several sub-sections within the same domain? With that answered, it should be easier to determine a favored domain.

Another aspect to consider is the team carrying out the research, more specifically, the domain experts - if there is any - in the team. In our team, the projects are often led by researchers in the software engineering field and supported by experts working in other scientific domains. Having domain experts in the team provides significant benefits in selecting software packages and designing interview questions.

In this project, Medical Imaging (MI) domain is selected. Numerous software products can be found in this domain, and a great number of domain experts use - and even develop - such software. We decide to focus on MI software with the viewing function, and more details about this filtering are in Section \ref{sec_software_selection}. Being able to include MI domain experts in our team also makes this domain more preferred.

\section{Software Product Selection}
\label{sec_software_selection}

The process of selecting software packages contains two steps: i) identify software candidates in the chosen domain, ii) filter the list according to needs \cite{SmithEtAl2021}.

\subsection{Identify Software Candidates}
The candidate software can be found from publications in the domain. Another source is to search various websites, such as \hyperlink{https://github.com/}{GitHub}, \hyperlink{https://swmath.org/}{swMATH} and the Google search results for software recommendation articles. Meanwhile, we should also include the suggested ones from domain experts \cite{SmithEtAl2021}.

As for this project, 48 MI software projects are identified as the candidates, and they are found from publications \cite{Bjorn2017} \cite{Bruhschwein2019} \cite{Haak2015}, online articles related to the domain \cite{Emms2019} \cite{Hasan2020} \cite{Mu2019}, forum discussions related to the domain \cite{Samala2014}, etc.

\subsection{Filter the Software List}
The goal is to build a software list with a length of about 30 \cite{SmithEtAl2021}.

The only mandatory requirement is that the software must be open source, as defined in Section \ref{sec_open_source_software}. This is due to the grading process defined int Section \ref{sec_grading_template}. In order to evaluate all aspects of some software qualities, the source code must be accessible.

The other factors to filter the list can be optional and should be considered according to the number of software candidates and the objectives of a research project.

One of the factors is the functions and purposes of the software. For example, we can choose a group of software with similar functions, so that the cross-comparison is between each individual of them. On the other hand, if our objective is comparing sub-categories in the domain, we should select from candidates in each of the categories.

The empirical measurement tools listed in Section \ref{sec_empirical_measurements} are limited to projects using \hyperlink{https://git-scm.com/}{Git} as the version control tool, so software with Git is preferred. Some manual steps in empirical measurement depend on a few metrics of GitHub, which makes projects hold on GitHub more favored \cite{SmithEtAl2021}.

Some of the OSS projects may experience a lack of recent maintenance. So packages that have not been updated for a long time can be eliminated, unless they are still popular and highly recommended by the users in the domain \cite{SmithEtAl2021}.

In this project, we focus on the MI software providing imaging viewing function. \ad{TBC}

\section{Grading Template}
\label{sec_grading_template}

\ad{The grading template for manual measurements}

\section{Empirical Measurements}
\label{sec_empirical_measurements}

\ad{Explain the empirical tools and how they affect the scores.}

\section{Measuring Qualities}
\label{sec_measuring_qualities}

\ad{e.g. virtual machines, time spent per software, where to look for docs, etc.}

\section{Interview Methods}
\label{sec_interview_methods}

\subsection{Interviewee Selection}

\ad{Generally speaking, ask all the teams which we can find contacts, and continue with the ones who are willing to participate.}

\subsection{Interview Question Selection}

\ad{The aspects we focused on.}

\subsection{Interview Methods}

\ad{The ethics approvals, the way to ask questions, the way to transcript answers, and the technologies used.}
