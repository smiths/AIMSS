\chapter{Methods}
\label{ch_methods}

We designed a general process for evaluating the state of the practice of domain-specific software, that we instantiate to SC software for specific scientific domains.

Our method involves: 1) choosing a software domain (Section \ref{sec_domain_selection}); 2) collecting and filtering software packages (Section \ref{sec_software_selection}); 3) grading the selected software (Section \ref{sec_grading_software}); 4) interviewing development teams for further information (Section \ref{sec_interview_methods}).

Details of how we applied the method on the MI domain are expanded upon in Section \ref{sec_applying_method}.

\section{Domain Selection}
\label{sec_domain_selection}
Our methods are generic, but we have only applied them to scientific domains due to the objective of our research.

When choosing a candidate domain, we prefer one with a large number of active OSS. The reason is that we aim to finalize a list of 30 software packages \cite{SmithEtAl2021} after the screening step. For example, we remove the ones without recent updates or specific functions. Thus, we need enough software candidates from the beginning. Besides, we prefer OSS projects because our grading method works better on them. Moreover, we prefer domains with active communities developing and using SC software, so it is easier to invite enough developers for interviews.

We prefer 30 software packages providing similar functions or falling into different sub-groups depending on our research purpose. So the domain needs to have enough candidates in one sub-group or enough sub-groups to cross-compare.

We also prefer domains in which our team has expertise. We invite domain experts to join and support our projects. They help us in many aspects, such as vetting the software list and interview questions.

\section{Software Product Selection}
\label{sec_software_selection}

The process of selecting software packages contains two steps: i) identify software candidates in the chosen domain, ii) filter the list according to needs \cite{SmithEtAl2021}.

\subsection{Identify Software Candidates}
\label{sec_identify_software_candidates}
We start with finding candidate software in publications of the domain. Then, we search various websites, such as \hyperlink{https://github.com/}{GitHub}, \hyperlink{https://swmath.org/}{swMATH} and the Google search results for software recommendation articles. Meanwhile, we include the suggested ones from domain experts \cite{SmithEtAl2021}.

\subsection{Filter the Software List}
\label{sec_filter_software_list}
The goal is to build a software list with a length of about 30 \cite{SmithEtAl2021}.

The only ``mandatory'' requirement is that the software must be OSS, as defined in Section \ref{sec_open_source_software}. Because to evaluate all software qualities, our methods need to access the source code.

The other factors to filter the list are optional, and we consider them according to the number of software candidates and the objectives of the research project.

One factor is the functions and purposes of the software. For example, we can choose a group of software with similar functions, so that the comparison is between the software; or we can cross-compare sub-groups in the domain, then we need to select candidates from each sub-group.

The empirical measurement tools listed in Section \ref{sec_empirical_measurements} only work on projects using \hyperlink{https://git-scm.com/}{Git}, so we prefer software with Git. Some manual steps in empirical measurement depend on a few metrics of GitHub, which makes projects held on GitHub more favored \cite{SmithEtAl2021}.

Some of the OSS projects may experience a lack of recent maintenance. So we eliminate packages without recent updates, unless they are still popular and highly recommended by the domain users \cite{SmithEtAl2021}.

With domain experts in the team, we value their opinions on the filtering process. For example, if a software package is not OSS and has no updates for a long while, but the domain experts identify it as a valuable product, we still consider it to keep.

\section{Grading Software}
\label{sec_grading_software}

We grade the selected software using a template (Section \ref{sec_grading_template}) and a specific empirical method (Section \ref{sec_empirical_measurements}). Some technical details for the measurements are in Section \ref{sec_technical_details}.

\subsection{Grading Template}
\label{sec_grading_template}
The full grading template can be found in Appendix \ref{ap_grading_template}. The template contains 101 questions that we use for grading software products.

We use the first section of the template to collect general information, such as the name, purpose, platform, programming language, publications about the software, the first release and the most recent change date, website, and source code repository of the product, etc. Information in this section helps us understand the projects better and may be helpful for further analysis, but it does not directly affect the grading scores.

We designed the following nine sections in the template for the nine software qualities mentioned in Section \ref{sec_software_quality}. For each quality, we ask several questions and the typical answers are among the collection of ``yes'', ``no'', ``n/a'', ``unclear'', a number, a string, a date, a set of strings, etc. Each quality needs an overall score between 1 and 10 based on all the previous questions. For some qualities, we perform surface measurements, which allow us to carry out on all packages with reasonable efforts. The surface measurements reveal some traits of a underlying quality, but may not fully represent it.

\begin{itemize}
\item \textbf{Installability} We check the existence and quality of installation instructions. The user experience is also an important factor, such as the ease to follow the instructions, number of steps, automation tools, and the prerequisite steps for the installation. If any problem interrupts the process of installation or uninstallation, we give a lower score to this quality. We also record the operating system for the installation test and whether we can verify the installation.

\item \textbf{Correctness \& Verifiability} For \textit{correctness}, we check the projects to identify the techniques to ensure this quality, such as literate programming, automated testing, symbolic execution, model checking, unit tests, etc. We also examine whether the projects use continuous integration and continuous delivery (CI/CD). For \textit{verifiability}, we go through the documents of the projects to check the requirements specifications, theory manuals, and getting started tutorials. If a getting started tutorial exists and provides expected results, we follow it and check if the outputs match.

\item \textbf{Surface Reliability} We check that whether the software break during the installations and tutorials, whether there are descriptive error messages, and if we can recover the process after the errors.

\item \textbf{Surface Robustness} We check that how the software handle unexpected/unanticipated input. For example, for software packages with the function to load image files, we prepare broken image files for them to open. We use a text file (.txt) with a modified extension name (.dcm) as an unexpected/unanticipated input. We load a few correct input files to ensure the function is working correctly before testing with the unexpected/unanticipated ones.

\item \textbf{Surface Usability} We examine the documents of the projects and consider software with a getting started tutorial and a user manual easier to use. Meanwhile, we check if users have any channels to get supports. Our impressions and user experiences when testing the software also affect the scores. For example, easy-to-use graphical user interfaces give us a better experience, thus lead to better scores.

\item \textbf{Maintainability} We search the projects' documents and identify the process of contributing and reviewing code. We believe that the artifacts of a project - including source code, documents, building scripts, etc. - can significantly affect its  \textit{maintainability}. Thus we check each project for its artifacts, such as API documentation, bug tracker, release notes, test cases, build files, version control, etc. We also check the tools supporting issue tracking and version control and the percentages of closed issues and comment lines in code.

\item \textbf{reusability} We count the total number of code files for each project. Projects with a large number of components provide more choices to reuse. Furthermore, well modularized code, which tend to have smaller parts in separated files, are easier to reuse. Thus, we consider the projects with more code files to be more reusable. We use \textit{GitStats} as a tool to count the number of text-based files for all projects, and consider the projects with more text-based files to also have more code files. We also decide that the projects with API documentation can deliver better \textit{Reusability}.

\item \textbf{Surface Understandability} We randomly examine 10 code files. We check the code’s style within each code file, such as whether the identifiers, parameters, indentation, and formatting are consistent, whether the constants (other than 0 and 1) are hardcoded into the code, and whether the developers modularized the code. We also check the descriptive information for the code, such as documents mentioning the coding standard, the comments in the code, and the descriptions or links for algorithms in the code. 

\item \textbf{Visibility/Transparency} To measure this quality, such as all of the steps of a software development process and the current status of a project, we check the existing documents. We examine the development process, current status, development environment, and release notes for each project. If any information is missing or poorly conveyed, the \textit{visibility/transparency} is not ideal.
\end{itemize}

All the last three sections are about the empirical measurements. For some qualities, the empirical measurements also affect the score. We use two command-line software tools \textit{GitStats} and \textit{scc} to extract information from the source code repositories. For projects held on GitHub, we manually collect additional metrics, such as the stars of the GitHub repository, and the numbers of open and closed pull requests. Section \ref{sec_empirical_measurements} presents more details about the empirical measurements.

\subsection{Empirical Measurements}
\label{sec_empirical_measurements}

We use two command-line tools for the empirical measurements. One is \textit{GitStats} that generates statistics for git repositories and display outputs in the format of web pages \cite{Gieniusz2019}; the other one is Sloc Cloc and Code (as known as \textit{scc}) \cite{Boyter2021}, aiming to count the lines of code, comments, etc.

Both tools measure the number of text-based files in a git repository and lines of text in these files. Based on our experience, most text-based files in a repository contain programming source code, and developers use them to compile and build software products. A minority of these files are instructions and other documents. So we roughly regard the lines of text in text-based files as lines of programming code. The two tools usually generate similar but not identical results. From our understanding, this minor difference is due to the different techniques to detect if a file is text-based or binary.

Additionally, we also manually collect information for projects held on GitHub, such as the numbers of stars, forks, people watching this repository, open pull requests, and closed pull requests.

These empirical measurements help us from two aspects. Firstly, they help us with getting a project overview faster and more accurately. For example, the number of commits over the last 12 months shows how active this project has been, and the number of stars and forks may reveal its popularity. Secondly, the results may affect our decisions regarding the grading scores for some software qualities. For example, if the percentage of comment lines is low, we double-check the \textit{understandability} of the code; if the ratio of open versus closed pull requests is high, we pay more attention to the \textit{maintainability}.

\subsection{Technical Details}
\label{sec_technical_details}
To test the software on a ``clean'' system, we create a new virtual machine (VM) for each software and only install the necessary dependencies before measuring. We make all 30 VM on the same computer, one at a time, and destroy them after measuring.

We spend about two hours grading one package, unless we find technical issues and need more time to resolve them. In most of the situation, we finish all the measurements for one software on the same day.

\section{Interview Methods}
\label{sec_interview_methods}

\subsection{Interviewee Selection}
For a software list with a length of roughly 30, we aim to interview about ten development teams. Interviewing multiple individuals from each team gives us more comprehensive information, but a single engineer well-knowing the project is also sufficient.

Ideally, we select projects after the grading measurements and prefer the ones with higher overall scores. However, if we do not find enough participants, we contact all teams on the list.

We try to find the contacts of the teams on the projects' websites, such as the official web pages, repositories, publications, and bio pages of the teams' institutions. Then, we send at most two emails to one contact asking for its participation before receiving any replies.

\subsection{Interview Question Selection}

We have a list of 20 questions to guide our interviews, which can be found in Section \ref{ch_interview} and Appendix \ref{ap_interview}.

Some questions are about the background of the software, the development teams, the interviewees, and how they organize the projects. We also ask about their understandings of the users. Some questions focus on the current and past difficulties, and the solutions the team has found or will try. We also discuss the importance and current situations of documentation. A few questions are about specific software qualities, such as \textit{maintainability}, \textit{understandability}, \textit{usability}, and \textit{reproducibility}.

The interviews are semi-structured based on the question list, and we also ask follow-up questions when necessary. Based on our experience, the interviewees usually bring up some exciting ideas that we do not expect, and it is worth expanding on these topics.

\subsection{Interview Process}
Since the members of the development teams are usually around the world, we organize these interviews as virtual meetings online with \hyperlink{https://zoom.us/}{Zoom}. After receiving consent from the interviewees, we also record our discussions to transcribe them better.

\section{Applying the Method to Medical Imaging (MI)}
\label{sec_applying_method}

Based on the principles in Section \ref{sec_domain_selection}, we selected the MI domain and the viewer sub-group. We also included MI domain experts in our team.

By using the method in Section \ref{sec_identify_software_candidates}, we identified 48 MI software projects as the candidates from publications \cite{Bjorn2017} \cite{Bruhschwein2019} \cite{Haak2015}, online articles related to the domain \cite{Emms2019} \cite{Hasan2020} \cite{Mu2019}, forum discussions related to the domain \cite{Samala2014}, etc.

Appendix \ref{ap_list_before_filtering} shows all the 48 software packages. Among them, there were several ones that we could not find in their source code, such as \textit{MicroDicom}, \textit{Aliza}, and \textit{jivex}, etc. These packages are likely to be freeware defined in Section \ref{sec_freeware} and not OSS. So following guidelines in Section \ref{sec_filter_software_list} we removed them from the list. Next, we focused on the MI software that could work as a MI viewer. Some of the software on the list were tool kits or libraries for other software to use as dependencies but not for end-users to view medical images, such as \textit{VTK}, \textit{ITK}, \textit{dcm4che}, etc. We also eliminated these from the list. After that, there were 29 software products left on the list. We still preferred projects using git and GitHub and being updated recently, but did not apply another filtering since the number of packages was already below 30. However, 27 out of the 29 software packages on the final list used git, and 24 chose GitHub. Furthermore, 27 packages had the latest updates after the year 2018, and 23 after 2020.

Then we followed the steps in Section \ref{sec_grading_software} to measure and grade the software. 27 out of the 29 packages are compatible with two or three different operating systems such as Windows, macOS, and Linux, and 5 of them are browser-based, making them platform-independent. However, in the interest of time, we only performed the measurements for each project by installing it on one of the platforms, most likely Windows.

Going through the interview process in Section \ref{sec_interview_methods}, we started with the teams with higher scores on our list, and eventually contacted all of them. As a result, developers/architects from 8 teams have participated in our interviews so far. Before contacting any interviewee candidate, we received ethics clearance from the McMaster University Research Ethics Board (Appendix \ref{ap_ethics}).
