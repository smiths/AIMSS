\chapter{Full Grading Template}
\label{ap_grading_template}

Table \ref{measurement_template} lists the measurements that we use to assess the software products. We use the first section to collect general information of software projects. The following nine sections assess the nine software qualities. The last three sections are for the empirical measurements.

\newpage

\begin{table}[H]
\centering
\caption{Measurement Template}\label{measurement_template}
\begin{tabular}{p{14cm}}
\hline
\textbf{Summary Information}\\
\hline
Software name? (string)\\
URL? (URL)\\
Affiliation (institution(s)) (string or {N/A})\\
Software purpose (string)\\
Number of developers (all developers that have contributed at least one commit to the project) (use repo commit logs) (number)\\
How is the project funded? (unfunded, unclear, funded$\ast$) where $\ast$ requires a string to say the source of funding\\
Initial release date? (date)\\
Last commit date? (date)\\
Status? (alive is defined as presence of commits in the last 18 months) ({alive, dead, unclear})\\
License? ({GNU GPL, BSD, MIT, terms of use, trial, none, unclear, other$\ast$}) $\ast$ given via a string \\
Platforms? (set of {Windows, Linux, OS X, Android, other$\ast$}) $\ast$ given via string\\
Software Category? The concept category includes software that does not have an officially released version. Public software has a released version in the public domain. Private software has a released version available to authorized users only. ({concept, public, private})\\
Development model? ({open source, freeware, commercial, unclear})\\
Publications about the software? Refers to publications that have used or mentioned the software. (number or {unknown})\\
Source code URL? ({set of url, n/a, unclear})\\
Programming language(s)? (set of {FORTRAN, Matlab, C, C++, Java, R, Ruby, Python, Cython, BASIC, Pascal, IDL, unclear, other$\ast$}) $\ast$ given via string \\
Is there evidence that performance was considered? Performance refers to either speed, storage, or throughput. ({yes$\ast$, no})\\
Additional comments? (can cover any metrics you feel are missing, or any other thoughts you have) \\
\hline
\end{tabular}
\end{table}

\begin{table}[H]
\centering
\begin{tabular}{p{14cm}}
\hline
\textbf{Installability  (Measured via installation on a virtual machine.) }\\
\hline
Are there installation instructions? ({yes, no})\\
Are the installation instructions in one place? Place referring to a single document or web-page. ({yes, no, n/a})\\
Are the installation instructions linear? Linear meaning progressing  in a single series of steps. ({yes, no, n/a})\\
Are the instructions written as if the person doing the installation has none of the dependent packages installed? ({yes, no, unclear})\\
Are compatible operating system versions listed? ({yes, no})\\
Is there something in place to automate the installation (makefile, script, installer, etc)? ({yes$\ast$, no})\\
If the software installation broke, was a descriptive error message displayed? ({yes, no, n/a})\\
Is there a specified way to validate the installation? ({yes$\ast$, no})\\
How many steps were involved in the installation? (Includes manual steps like unzipping files) Specify OS. (number, OS)\\
What OS was used for the installation? ({Windows, Linux, OS X, Android, other$\ast$ }) $\ast$given via string\\
How many extra software packages need to be installed before or during installation? (number)\\
Are required package versions listed? ({yes, no, n/a})\\
Are there instructions for the installation of required packages / dependencies? ({yes, no, n/a})\\
Run uninstall, if available. Were any obvious problems caused? ({yes$\ast$ , no, unavail})\\
Overall impression? ({1 .. 10})\\
Additional comments? (can cover any metrics you feel are missing, or any other thoughts you have)\\
\hline
\end{tabular}
\end{table}

\begin{table}[H]
\centering
\begin{tabular}{p{14cm}}
\hline
\textbf{Correctness and Verifiability}\\
\hline
Any reference to the requirements specifications of the program or theory manuals? ({yes$\ast$ , no, unclear})\\
What tools or techniques are used to build confidence of correctness? ({literate programming, automated testing, symbolic execution, model checking, assertions used in the code, Sphinx, Doxygen, Javadoc, confluence, unclear, other$\ast$}) $\ast$ given via string\\
If there is a getting started tutorial? ({yes, no})\\
Are the tutorial instructions linear? ({yes, no, n/a})\\
Does the getting started tutorial provide an expected output? ({yes, no$\ast$, n/a})\\
Does your tutorial output match the expected output? ({yes, no, n/a})\\
Are unit tests available?  ({yes, no, unclear})\\
Is there evidence of continuous integration? (for example mentioned in documentation, Jenkins, Travis CI, Bamboo, other) ({yes$\ast$, no, unclear})\\
Overall impression? ({1 .. 10})\\
Additional comments? (can cover any metrics you feel are missing, or any other thoughts you have) \\
\hline	
\textbf{Surface Reliability}\\
\hline
Did the software “break” during installation? ({yes$\ast$ , no})\\
If the software installation broke, was the installation instance recoverable? ({yes, no, n/a})\\
Did the software “break” during the initial tutorial testing? ({yes$\ast$, no, n/a})\\
If the tutorial testing broke, was a descriptive error message displayed? ({yes, no, n/a})\\
If the tutorial testing broke, was the tutorial testing instance recoverable? ({yes, no, n/a})\\
Overall impression? ({1 .. 10})\\
Additional comments? (can cover any metrics you feel are missing, or any other thoughts you have)\\
\hline
\textbf{Surface Robustness}\\
\hline
Does the software handle unexpected/unanticipated input (like data of the wrong type, empty input, missing files or links) reasonably? (a reasonable response can include an appropriate error message.) ({yes, no$\ast$ })\\
For any plain text input files, if all new lines are replaced with new lines and carriage returns, will the software handle this gracefully? ({yes, no$\ast$, n/a})\\
Overall impression? ({1 .. 10})\\
Additional comments? (can cover any metrics you feel are missing, or any other thoughts you have)\\
\hline
\end{tabular}
\end{table}

\begin{table}[H]
\centering
\begin{tabular}{p{14cm}}
\hline
\textbf{Surface Usability}\\
\hline
Is there a getting started tutorial? ({yes, no})\\
Is there a user manual? ({yes, no})\\
Are expected user characteristics documented? ({yes, no})\\
What is the user support model? FAQ? User forum? E-mail address to direct questions? Etc. (string)\\
Overall impression? ({1 .. 10})\\
Additional comments? (can cover any metrics you feel are missing, or any other thoughts you have)\\

\hline	
\textbf{Maintainability}\\
\hline
What is the current version number? (number)\\
Is there any information on how code is reviewed, or how to contribute? ({yes$\ast$, no})\\
Are artifacts available? (List every type of file that is not a code file – for examples please look at the ‘Artifact Name’ column of https://gitlab.cas.mcmaster.ca/SEforSC/se4sc/-/blob/git-svn/GradStudents/Olu/ResearchProposal/Artifacts\_MiningV3.xlsx) ({yes$\ast$, no, unclear}) $\ast$list via string\\
What issue tracking tool is employed? (set of {Trac, JIRA, Redmine, e-mail, discussion board, sourceforge, google code, git, BitBucket, none, unclear, other$\ast$}) $\ast$ given via string\\
What is the percentage of identified issues that are closed? (percentage)\\
What percentage of code is comments? (percentage)\\
Which version control system is in use? ({svn, cvs, git, github, unclear, other$\ast$}) $\ast$ given via string\\
Overall impression? ({1 .. 10})\\
Additional comments? (can cover any metrics you feel are missing, or any other thoughts you have)\\
\hline
\textbf{Reusability}\\
\hline
How many code files are there? (number)\\
Is API documented? ({yes, no, n/a})\\
Overall impression? ({1 .. 10})\\
Additional comments? (can cover any metrics you feel are missing, or any other thoughts you have)\\
\hline\end{tabular}
\end{table}

\begin{table}[H]
\begin{tabular}{p{14cm}}
\end{tabular}
\end{table}

\begin{table}[H]
\begin{tabular}{p{14cm}}
\hline
\textbf{Surface Understandability (Based on 10 random source files)}\\
\hline
Consistent indentation and formatting style? ({yes, no, n/a})\\
Explicit identification of a coding standard? ({yes$\ast$, no, n/a})\\
Are the code identifiers consistent, distinctive, and meaningful? ({yes, no$\ast$ , n/a})\\
Are constants (other than 0 and 1) hard coded into the program? ({yes, no$\ast$ , n/a})\\
Comments are clear, indicate what is being done, not how? ({yes, no$\ast$ , n/a})\\
Is the name/URL of any algorithms used mentioned? ({yes, no$\ast$ , n/a})\\
Parameters are in the same order for all functions? ({yes, no$\ast$ , n/a})\\
Is code modularized? ({yes, no$\ast$ , n/a})\\
Overall impression? ({1 .. 10})\\
Additional comments? (can cover any metrics you feel are missing, or any other thoughts you have)\\
\hline
\textbf{Visibility/Transparency}\\
\hline
Is the development process defined? If yes, what process is used. ({yes$\ast$, no, n/a})\\
Are there any documents recording the development process and status?  ({yes$\ast$, no}))\\
Is the development environment documented? ({yes$\ast$, no})\\
Are there release notes? ({yes$\ast$, no})\\
Overall impression? ({1 .. 10})\\
Additional comments? (can cover any metrics you feel are missing, or any other thoughts you have)\\
\hline
\end{tabular}
\end{table}

\begin{table}[H]
\begin{tabular}{p{14cm}}
\hline
\textbf{Raw Metrics (Measured via git\_stats)}\\
\hline
Number of text-based files. (number)\\
Number of binary files. (number)\\
Number of total lines in text-based files. (number)\\
Number of total lines added to text-based files. (number)\\
Number of total lines deleted from text-based files. (number)\\
Number of total commits. (number)\\
Numbers of commits by year in the last 5 years. (Count from as early as possible if the project is younger than 5 years.) (list of numbers)\\
Numbers of commits by month in the last 12 months. (list of numbers)\\
\hline
\textbf{Raw Metrics (Measured via scc)}\\
\hline
Number of text-based files. (number)\\
Number of total lines in text-based files. (number)\\
Number of code lines in text-based files. (number)\\
Number of comment lines in text-based files. (number)\\
Number of blank lines in text-based files. (number)\\
\hline
\textbf{Repo Metrics (Measured via GitHub)}\\
\hline
Number of people watching this repo. (number)\\
Number of stars. (number)\\
Number of forks. (number)\\
Number of open pull requests. (number)\\
Number of closed pull requests. (number)\\
Number of months on GitHub. (number)\\
Accessed date. (date)\\
\hline
\end{tabular}
\end{table}
