\addcontentsline{toc}{chapter}{Abstract}
\begin{center}
\textbf{\large Abstract}
\end{center}

\begingroup
\setstretch{1.25}
We present a general method to assess the state of the practice for Scientific Computing (SC) software and apply the method to the Medical Imaging (MI) software. This method guided us to select 29 MI software projects from 48 candidates, assess 10 software qualities (\textit{Installability}, \textit{Correctness \& Verifiability}, \textit{Reliability}, \textit{Robustness}, \textit{Usability}, \textit{Maintainability}, \textit{Reusability}, \textit{Understandability}, \textit{Visibility/Transparency}, and \textit{Reproducibility}) by answering 103 questions for each software, and interview eight of the 29 development teams. The results helped us with revealing the current status of MI software development. Based on the quantitative data for the first nine qualities, we ranked the MI software with the Analytic Hierarchy Process (AHP). The top three software products were \textit{3D Slicer}, \textit{ImageJ}, and \textit{OHIF Viewer}, which received high scores for most qualities. \textit{3D Slicer} was among the top two for all nine qualities except \textit{robustness}. By interviewing the developers, we identified three major types of pain points during their development process: i) the lack of resources; ii) the difficulty to balance between four factors: \textit{compatibility}, \textit{maintainability}, \textit{performance}, and \textit{security};
iii) the lack of access to real-world datasets for testing. We collected proven and potential solutions for these problems. The interviews also helped us to understand the status of documentation, project management, and five qualities (\textit{correctness}, \textit{maintainability}, \textit{understandability}, \textit{usability}, and \textit{reproducibility}) in the projects. We summarized the threats and strategies to these qualities. For future SC software development, we proposed recommendations on improving software qualities, dealing with limited resources, choosing a tech stack, and enriching the testing datasets. The recommendations include adopting test-driven development, using continuous integration and continuous delivery (CI/CD), using git and GitHub, maintaining good documentation, supporting third-party plugins or extensions, considering web application solutions, and establishing community collaboration in an SC domain.  \newline

\noindent\textbf{Keywords:} Medical Imaging, Scientific Computing, software engineering, software quality, Analytic Hierarchy Process, developer interview
\endgroup
