%% 
%% Copyright 2007-2020 Elsevier Ltd
%% 
%% This file is part of the 'Elsarticle Bundle'.
%% ---------------------------------------------
%% 
%% It may be distributed under the conditions of the LaTeX Project Public
%% License, either version 1.2 of this license or (at your option) any
%% later version.  The latest version of this license is in
%%    http://www.latex-project.org/lppl.txt
%% and version 1.2 or later is part of all distributions of LaTeX
%% version 1999/12/01 or later.
%% 
%% The list of all files belonging to the 'Elsarticle Bundle' is
%% given in the file `manifest.txt'.
%% 
%% Template article for Elsevier's document class `elsarticle'
%% with Harvard style bibliographic references

%\documentclass[preprint,12pt,authoryear]{elsarticle}
%\documentclass[3p, 12pt,authoryear]{elsarticle}
\documentclass[final, 3p, times, authoryear]{elsarticle}

%% Use the option review to obtain double line spacing
%% \documentclass[authoryear,preprint,review,12pt]{elsarticle}

%% Use the options 1p,twocolumn; 3p; 3p,twocolumn; 5p; or 5p,twocolumn
%% for a journal layout:
%% \documentclass[final,1p,times,authoryear]{elsarticle}
%% \documentclass[final,1p,times,twocolumn,authoryear]{elsarticle}
%% \documentclass[final,3p,times,authoryear]{elsarticle}
%% \documentclass[final,3p,times,twocolumn,authoryear]{elsarticle}
%% \documentclass[final,5p,times,authoryear]{elsarticle}
%% \documentclass[final,5p,times,twocolumn,authoryear]{elsarticle}

%% For including figures, graphicx.sty has been loaded in
%% elsarticle.cls. If you prefer to use the old commands
%% please give \usepackage{epsfig}

%% The amssymb package provides various useful mathematical symbols
\usepackage{amssymb}
%% The amsthm package provides extended theorem environments
\usepackage{amsmath,amsthm}

%% The lineno packages adds line numbers. Start line numbering with
%% \begin{linenumbers}, end it with \end{linenumbers}. Or switch it on
%% for the whole article with \linenumbers.
%% \usepackage{lineno}

\usepackage{hyperref}
\usepackage{xcolor}
\hypersetup{
    colorlinks,
    linkcolor={red!50!black},
    citecolor={blue!50!black},
    urlcolor={blue!80!black}
}

%\usepackage{tikz}
%\usetikzlibrary{mindmap}
\usepackage{multirow}
\usepackage{pgfplots}
\usepackage{arydshln}
\usepackage{paralist}
\usepackage{booktabs}

%% Comments
\newif\ifcomments\commentstrue

\ifcomments
\newcommand{\authornote}[3]{\textcolor{#1}{[#3 ---#2]}}
\newcommand{\todo}[1]{\textcolor{red}{[TODO: #1]}}
\else
\newcommand{\authornote}[3]{}
\newcommand{\todo}[1]{}
\fi

\newcommand{\wss}[1]{\authornote{blue}{SS}{#1}} %Spencer Smith
\newcommand{\jc}[1]{\authornote{red}{JC}{#1}} %Jacques Carette
\newcommand{\mn}[1]{\authornote{magenta}{MN}{#1}} %Mike Noseworthy
\newcommand{\ad}[1]{\authornote{cyan}{AD}{#1}} %Ao Dong

\newcommand{\notdone}[1]{\textcolor{red}{#1}}
\newcommand{\done}[1]{\textcolor{black}{#1}}

\newcounter{rqnum} %research question number
\newcommand{\rqtherqnum}{RQ`'\therqnum}
\newcommand{\rqref}[1]{RQ\ref{#1}}

\newcounter{pnum} %pain point number
\newcommand{\ppthepnum}{P`'\thepnum}
\newcommand{\ppref}[1]{P\ref{#1}}

\newcounter{qnum} %quality number
\newcommand{\qthepnum}{Q`'\theqnum}
\newcommand{\qref}[1]{Q\ref{#1}}

\journal{Academic Radiology}

%\pgfplotsset{compat=1.13}

\begin{document}

\begin{frontmatter}

%% Title, authors and addresses

%% use the tnoteref command within \title for footnotes;
%% use the tnotetext command for theassociated footnote;
%% use the fnref command within \author or \affiliation for footnotes;
%% use the fntext command for theassociated footnote;
%% use the corref command within \author for corresponding author footnotes;
%% use the cortext command for theassociated footnote;
%% use the ead command for the email address,
%% and the form \ead[url] for the home page:
%% \title{Title\tnoteref{label1}}
%% \tnotetext[label1]{}
%% \author{Name\corref{cor1}\fnref{label2}}
%% \ead{email address}
%% \ead[url]{home page}
%% \fntext[label2]{}
%% \cortext[cor1]{}
%% \affiliation{organization={},
%%            addressline={}, 
%%            city={},
%%            postcode={}, 
%%            state={},
%%            country={}}
%% \fntext[label3]{}

\title{State of the Practice for Medical Imaging Software}

%% use optional labels to link authors explicitly to addresses:
%% \author[label1,label2]{}
%% \affiliation[label1]{organization={},
%%             addressline={},
%%             city={},
%%             postcode={},
%%             state={},
%%             country={}}
%%
%% \affiliation[label2]{organization={},
%%             addressline={},
%%             city={},
%%             postcode={},
%%             state={},
%%             country={}}

\author[CAS]{Spencer Smith}
\author[CAS]{Ao Dong}
\author[CAS]{Jacques Carette}
\author[ECE]{Mike Noseworthy}

\affiliation[CAS]{organization={McMaster University, Computing and Software Department},%Department and Organization
            addressline={1280 Main Street West}, 
            city={Hamilton},
            postcode={L8S 4K1}, 
            state={Ontario},
            country={Canada}}

\affiliation[ECE]{organization={McMaster University, Electrical Engineering},%Department and Organization
            addressline={1280 Main Street West}, 
            city={Hamilton},
            postcode={L8S 4K1}, 
            state={Ontario},
            country={Canada}}

\begin{abstract}

We present the state of the practice for Medical Imaging software. We selected
29 medical imaging projects from 48 candidates, assessed 10 software qualities
(installability, correctness/ verifiability, reliability, robustness, usability,
maintainability, reusability, understandability, visibility/transparency and
reproducibility) by answering 108 questions for each software project, and
interviewed 8 of the 29 development teams. Based on the quantitative data for
the first 9 qualities, we ranked the MI software with the Analytic Hierarchy
Process (AHP). The top three software products were \textit{3D Slicer},
\textit{ImageJ}, and \textit{OHIF Viewer}. By interviewing the developers, we
identified three major pain points: i) lack of resources; ii) difficulty
balancing between compatibility, maintainability, performance, and security;
and, iii) lack of access to real-world datasets for testing. For future MI
software projects, we propose adopting test-driven development, using continuous
integration and continuous delivery (CI/CD), using git and GitHub, maintaining
good documentation, supporting third-party plugins or extensions, considering
web application solutions, and improving collaboration between different MI
software projects. \wss{Update after the paper has been revised.}

\end{abstract}

%%Graphical abstract
%\begin{graphicalabstract}
%\includegraphics{grabs}
%\end{graphicalabstract}

%%Research highlights
%\begin{highlights}
%\item Research highlight 1
%\item Research highlight 2
%\end{highlights}

\begin{keyword}
	medical imaging, research software, software engineering, software
	quality, Analytic Hierarchy Process, developer interviews
\end{keyword}

\end{frontmatter}

%% \linenumbers

\section{Introduction} \label{ch_intro}

We aim to study the state of software development practice for Medical Imaging
(MI) software.  MI tools use images of the interior of the body (from sources
such as Magnetic Resonance Imaging (MRI), Computed Tomography (CT), Positron
Emission Tomography (PET) and Ultrasound) to provide information for diagnostic,
analytic, and medical applications \citep{FDA2021, enwiki:1034887445,
Zhang2008}.  Figure~\ref{Fig_Example} shows an example image of the brain. Given
the importance of MI software and the high number of competing software
projects, we wish to understand the merits and drawbacks of the current
development processes, tools, and methodologies.  We aim to assess through a
software engineering lens the quality of the existing software with the hope of
highlighting standout examples, understanding current pain points and providing
guidelines and recommendations for future development.

\begin{figure}[!ht]
    \begin{center}
        \includegraphics[scale=0.25]{figures/MPR.png}        
    \end{center}
    \caption{Example brain image showing a multi-planar reformat using Horos (free open-source medical imaging/DICOM viewer for OSX, based on OsiriX)}
    \label{Fig_Example}
\end{figure}
    
\subsection{Research Questions} \label{sec_motivation}

Not only do we wish to gain insight into the state of the practice for MI
software, we also wish to understand the development of research software in
general. We wish to understand the impact of the often cited gap, or chasm,
between software engineering and scientific programming \citep{Kelly2007,
Storer2017}. Although scientists spend a substantial proportion of their working
hours on software development \citep{Hannay2009, Prabhu2011}, many developers
learn software engineering skills by themselves or from their peers, instead of
from proper training \citep{Hannay2009}. \citet{Hannay2009} observe that many
scientists showed ignorance and indifference to standard software engineering
concepts. For instance, according to a survey by \citet{Prabhu2011}, more than
half of their 114 subjects did not use a proper debugger when coding.

To gain insights, we devised 10 research questions, which can be applied to MI,
as well as to other domains, of research software \citep{SmithEtAl2021,
SmithAndMichalski2022}.  We designed the questions to learn about the
community's interest in, and experience with, artifacts, tools, principles,
processes, methodologies, and qualities.  When we mention artifacts we mean the
documents, scripts and code that constitutes a software development project.
Example artifacts include requirements, specifications, user manuals, unit
tests, system tests, usability tests, build scripts, API (Application
Programming Interface) documentation, READMEs, license documents, process
documents, and code.  Once we have learned what MI developers do, we then put
this information in context by contrasting MI software against the trends shown
by developers in other research software communities.  Our aim is to collect
enough information to understand the current pain points experienced by the MI
software development community so that we can make some preliminary
recommendations for future improvements. 

We based the structure of the paper on the research questions, so for each
research question below we point to the section that contains our answer.  We
start with identifying the relevant examples of MI software:

\begin{enumerate}
	\item[RQ\refstepcounter{rqnum}\therqnum \label{RQ_WhatProjects}:] What MI
	software projects exist, with the constraint that the source code must be
	available for all identified projects? (Section~\ref{ch_results})
	\item [RQ\refstepcounter{rqnum}\therqnum \label{RQ_HighestQuality}:] Which
	of the projects identified in \rqref{RQ_WhatProjects} follow current best
	practices, based on evidence found by experimenting with the software and
	searching the artifacts available in each project's repository?
	(Section~\ref{ch_results})
	\item [RQ\refstepcounter{rqnum}\therqnum \label{RQ_CompareHQ2Popular}:] How
	similar is the list of top projects identified in \rqref{RQ_HighestQuality}
	to the most popular projects, as viewed by the scientific community?
	(Section~\ref{Sec_VsCommunityRanking})
    \item [RQ\refstepcounter{rqnum}\therqnum \label{RQ_CompareArtifacts}:] How
	do MI projects compare to research software in general with respect to the
	artifacts present in their repositories?
	(Section~\ref{Sec_CompareArtifacts})
	\item [RQ\refstepcounter{rqnum}\therqnum \label{RQ_CompareToolsProjMngmnt}:]
	How do MI projects compare to research software in general with respect to
	the use of tools (Section~\ref{Sec_CompareTools}) for:
	\begin{enumerate} 
		\item [\rqref{RQ_CompareToolsProjMngmnt}.a] development; and,
		\item [\rqref{RQ_CompareToolsProjMngmnt}.b] project management?
	\end{enumerate}
	\item [RQ\refstepcounter{rqnum}\therqnum \label{RQ_CompareMethodologies}:]
	How do MI projects compare to research software in general with respect to
	principles, processes, and methodologies used?
	(Section~\ref{Sec_CompareMethodologies})
	\item [RQ\refstepcounter{rqnum}\therqnum \label{RQ_PainPoints}:] What are
	the pain points for developers working on MI software projects?
	(Section~\ref{painpoints})
	\item [RQ\refstepcounter{rqnum}\therqnum \label{RQ_ComparePainPoints}:] How
	do the pain points of developers from MI compare to the pain points
	for research software in general? (Section~\ref{painpoints})
	\item [RQ\refstepcounter{rqnum}\therqnum \label{RQ_Concerns}:] For MI
	developers what specific best practices are taken to address the pain points
	and software quality concerns? (Section~\ref{painpoints})
	\item [RQ\refstepcounter{rqnum}\therqnum \label{RQ_Recommend}:]
	What research software development practice could potentially address the
	pain point concerns identified in \rqref{RQ_PainPoints}).
	(Section~\ref{ch_recommendations})

\end{enumerate}

\subsection{Scope} \label{sec_scope}

To make the project feasible, we only cover MI visualization software.  As a
consequence we are excluding many other categories of MI software, including
Segmentation, Registration, Visualization, Enhancement, Quantification,
Simulation, plus MI archiving and telemedicine systems (Compression, Storage,
and Communication) (as summarized by \citet{Bankman2000} and
\citet{Angenent2006}).  We also exclude Statistical Analysis and Image-based
Physiological Modelling \citep{enwiki:1034877594} and Feature Extraction,
Classification, and Interpretation \citep{Kim2011}. Software that provides MI
support functions is also out of scope; therefore, we have not assessed the
toolkit libraries VTK \citep{SchroederEtAl2006} and ITK \citep{McCormick2014}.
Finally, Picture Archiving and Communication System (PACS), which helps users to
economically store and conveniently access images \citep{Choplin1992}, are
considered out of scope. 

\subsection{Methodology Overview}

We designed a general methodology to assess the state of the practice for SC
software \citep{SmithEtAl2021, SmithAndMichalski2022}. Details can be found in
Section~\ref{ch_methods}.  Our methodology has been applied to MI software
\citep{Dong2021} and Lattice Boltzmann Solvers \citep{Michalski2021}.  This
methodology builds off prior work to assess the state of the practice for such
domains as Geographic Information Systems \citep{smith2018state}, Mesh
Generators \citep{smith2016state}, Seismology software
\citep{Smith2018Seismology}, and Statistical software for psychology
\citep{smith2018statistical}.  In keeping with the previous methodology, we have
maintained the constraint that the work load for measuring a given domain should
be feasible for a team as small as one person, and for a short time, ideally
around a person month of effort. We consider a person month as $20$ working
days ($4$ weeks in a month, with $5$ days of work per week) at $8$ person hours
per day, or $20 \times 8 = 160$ person hours.

With our methodology, we first choose an SC domain (in the current case MI) and
identify a list of about 30 software packages. (For measuring MI we used 29
software packages.)  We approximately measure the qualities of each package by
filling in a grading template. Compared with our previous methodology, the new
methodology also includes repository based metrics, such as the number of files,
number of lines of code, percentage of issues that are closed, etc.  With the
quantitative data in the grading template, we rank the software with the
Analytic Hierarchy Process (AHP) (Section~\ref{ch_background} provides details).
After this, as another addition to our previous methodology, we interview some
development teams to further understand the status of their development process.

\section{Background} \label{ch_background}

To measure the existing MI software we need two sets of definitions: i) the
definitions of relevant software license models (Section
\ref{sec_software_categories}); and, ii) the definitions of the software
qualities that we will be assessing (Section \ref{sec_software_quality}). In our
assessment we rank the software packages for each quality; therefore, this
section also provides the background on our ranking process --- the Analytic
Hierarchy Process (Section \ref{sec_AHP}).

\subsection{Software Categories} \label{sec_software_categories}

When assessing software packages, we need to know the software's license.  In
particular, we need to know whether the source code will be available to us or
not.  We define three common software categories.  We will only assess software
that fits under the Open Source Software license.

\begin{itemize}

\item \textbf{Open Source Software (OSS)} For OSS, the source code is openly
accessible. Users have the right to study, change and distribute it under a
license granted by the copyright holder. For many OSS projects, the development
process relies on the collaboration of different contributors worldwide
\citep{Corbly2014}. Accessible source code usually exposes more ``secrets'' of a
software project, such as the underlying logic of software functions, how
developers achieve their works, and the flaws and potential risks in the final
product. Thus, it brings much more convenience to researchers analyzing the
qualities of the project.

\item \textbf{Freeware} Freeware is software that can be used free of charge.
Unlike OSS, the authors of do not allow access or modify the source code
\citep{LINFO2006}. To many end-users, the differences between freeware and OSS
may not be relevant. However, software developers who wish to modify the source
code, and researchers looking for insight into software development process may
find the inaccessible source code a problem. 

\item \textbf{Commercial Software} ``Commercial software is software developed
by a business as part of its business'' \citep{GNU2019}. Typically speaking, commercial software requires users to pay to access all of its features,
excluding access to the source code. However, some commercial software is also
free of charge \citep{GNU2019}. Based on our experience, most commercial
software products are not OSS.

\end{itemize}

\subsection{Software Quality Definitions} \label{sec_software_quality}

Quality is defined as a measure of the excellence or worth of an entity.  As is
common practice, we do not think of quality as a single measure, but rather as a
set of measures.  That is, quality is a collection of different qualities, often
called ``ilities.''  Below we list the 10 qualities of interest for this study.
The order of the qualities follows the order used in \citet{GhezziEtAl2003},
which puts related qualities (like correctness and reliability) together.
Moreover, the order is roughly the same as the order developers consider
qualities in practice.

\begin{itemize}
	\item \textbf{Installability} The effort required for the installation
    and/or uninstallation of software in a specified environment
    \citep{ISO/IEC25010, lenhard2013measuring}.

	\item \textbf{Correctness \& Verifiability} A program is correct if it
    matches its specification \citep[p.\ 17]{GhezziEtAl2003}.  The specification
    can either be explicitly or implicitly stated.  The related quality of
    verifiability is the ease with which the software components or the
    integrated product can be checked to demonstrate its correctness. 

	\item \textbf{Reliability} The probability of failure-free operation of a
	computer program in a specified environment for a specified time \citep{musa1987software}, \citep[p.\ 357]{GhezziEtAl2003}.

	\item \textbf{Robustness} Software possesses the characteristic of
	robustness if it behaves ``reasonably'' in two situations: i) when it
	encounters circumstances not anticipated in the requirements specification,
	and ii) when users violate the assumptions in its requirements specification 
	\citep[p.\ 19]{GhezziEtAl2003}, \citep{boehm2007software}.

	\item \textbf{Usability} ``The extent to which a product can be used by
	specified users to achieve specified goals with effectiveness, efficiency,
	and satisfaction in a specified context of use'' \citep{ISO/TR16982:2002}
	\citep{ISO9241-11:2018}.

	\item \textbf{Maintainability} The effort with which a software system or
	component can be modified to i) correct faults; ii) improve performance or
	other attributes; iii) satisfy new requirements
	\citep{IEEEStdGlossarySET1990}, \citep{boehm2007software}.

	\item \textbf{Reusability} ``The extent to which a software component can be
	used with or without adaptation in a problem solution other than the one for
	which it was originally developed'' \citep{kalagiakos2003non}.

	\item \textbf{Understandability} ``The capability of the software product to
	enable the user to understand whether the software is suitable, and how it
	can be used for particular tasks and conditions of use'' \citep{iso2001iec}.

	\item \textbf{Visibility/Transparency} The extent to which all the steps
	of a software development process and the current status of it are conveyed
	clearly \citep[p.\ 32]{GhezziEtAl2003}.

	\item \textbf{Reproducibility} ``A result is said to be reproducible if
	another researcher can take the original code and input data, execute it,
	and re-obtain the same result'' \citep{BenureauAndRougier2017}.
\end{itemize}

\subsection{Analytic Hierarchy Process (AHP)} \label{sec_AHP}

Thomas L.\ Saaty developed AHP in the 1970s, and people have widely used it
since to make and analyze multiple criteria decisions \citep{VaidyaEtAl2006}.
AHP organizes multiple criteria factors in a hierarchical structure and uses
pairwise comparisons between alternatives to calculate relative ratios
\citep{Saaty1990}. We use AHP to generate a ranking for a set of software
packages.

For a project with $m$ criteria, we can use an $m \times m$ matrix $A$ to record
the relative importance between factors. When comparing criterion $i$ and
criterion $j$, the value of $A_{ij}$ is decided as follows, with the value of
$A_{ji}$ generally equal to $1/A_{ij}$ \citep{Saaty1990}: $A_{ij} = 1$ if
criterion $i$ and criterion $j$ are equally important, while $A_{ij} = 9$ if
criterion $i$ is extremely more important than criterion $j$.  The natural
numbers between 1 and 9 are used to show the different levels of relative
importance between these two extremes. The above assumes that criterion $i$ is
not less important than criterion $j$.  If that is not the case, we reverse $i$
and $j$ and determine $A_{ji}$ first, then $A_{ij} = 1/A_{ji}$.

The priority vector $w$, which ranks the criteria by their importance, can be
calculated by solving the equation \citep{Saaty1990}:
\begin{equation} 
    A w = \lambda_{\text{max}} w,
\end{equation}
where $\lambda_{\text{max}}$ is the maximal eigenvalue of $A$.  In this project,
$w$ is approximated with the classic \textit{mean of normalized values} approach
\citep{AlessioEtAl2006}:

\begin{equation}
w_i = \frac{1}{m}\sum_{j=1}^{m}\frac{A_{ij}}{\sum_{k=1}^{m}A_{kj}}
\end{equation}

If there are $n$ alternatives, for criterion $i = 1, 2, ... , m$, we can create
an $n\times n$ matrix $B_i$ to record the relative preferences between these
choices for each of the $m$ criterion. The way of generating $B_i$ is similar to
the one for $A$. However, rather than comparing the importance between criteria,
we pairwise decide how much we favour one alternative over the other. We use the
same method to calculate the local priority vector for each $B_i$.  The local
priority vector in this case ranks the $n$ alternatives for criterion $i$.

In this project, the first nine software qualities mentioned in Section
\ref{sec_software_quality} are the criteria ($m = 9$), while 29 software
packages ($n = 29$) are compared for each of the $m$ criteria.
Section~\ref{sec_grading_software} shows the evaluation via the grading
template, including a subjective score from $1$ to $10$ for each quality for
each package. For each quality, for a pair of packages $i$ and $j$, with
$\mathit{score}_i >= \mathit{score}_j$, the difference between the two scores is
$\mathit{diff_{ij}} = \mathit{score}_i - \mathit{score}_j$. The mapping between
$\mathit{diff_{ij}}$ (which can vary between 0 and 9) and the values in $A_{ij}$
(which can vary between 1 and 9) is as follows:

\begin{itemize}
\item $A_{ij} = 1$ and $\mathit{diff_{ij}} = 0$ when criterion $i$ and criterion
$j$ are equally important;
\item $A_{ij}$ increases when $\mathit{diff_{ij}}$ increases;
\item $A_{ij} = 9$ and $\mathit{diff_{ij}} = 9$ when criterion $i$ is extremely
more important than criterion $j$.
\end{itemize}

\noindent Thus, we approximate the pairwise comparison result of $i$ versus $j$
by the following equation:

\begin{equation}
A_{ij} = \text{min}(\mathit{score}_i - \mathit{score}_j + 1, 9)
\end{equation}

\section{Methodology} \label{ch_methods}

We developed a methodology for evaluating the state of the practice of research
software \citep{SmithEtAl2021, SmithAndMichalski2022}.  The methodology can be
instantiated for a specific domain of scientific software, which in the current
case is medical imaging software for visualization.  Our methodology involves
and engages a domain expert partner throughout, as discussed in
Section~\ref{sec_vet_software_list}.  The four main steps of the methodology
are:

\begin{enumerate}
\item Identify list of representative software packages
(Section~\ref{sec_software_selection});
\item Measure (or grade) the selected software
(Section~\ref{sec_grading_software});
\item Interview developers (Section~\ref{sec_interview_methods});
\item Answer the research questions (as given in Section~\ref{sec_motivation}).
\end{enumerate}

In the sections below we provide additional detail on the above steps, while
concurrently giving examples of how we applied the methodology to the MI domain.

\subsection{Interaction With Domain Expert} \label{sec_vet_software_list}

The Domain Expert is an important member of the state of the practice assessment
team. Pitfalls exist if non-experts attempt to acquire an authoritative list of
software, or try to definitively rank the software. Non-experts have the problem
that they can only rely on information available on-line, which has the
following drawbacks:
\begin{inparaenum}[i)]
  \item the on-line resources could have false or inaccurate information; and,
  \item the on-line resources could leave out relevant information that is so
in-grained with experts that nobody thinks to explicitly record it.
\end{inparaenum}

Domain experts may be recruited from academia or industry.  The only
requirements are knowledge of the domain and a willingness to be engaged in the
assessment process.  The Domain Expert does not have to be a software developer,
but they should be a user of domain software.  Given that the domain experts are
likely to be busy people, the measurement process cannot put to much of a burden
on their time.  For the current assessment, our Domain Expert (and paper
co-author) is Dr.\ Michael Noseworthy, Professor of Electrical and Computer
Engineering at McMaster University, Co-Director of the McMaster School of
Biomedical Engineering, and Director of Medical Imaging Physics and Engineering
at St.\ Joseph's Healthcare, Hamilton, Ontario, Canada.  

In advance of the first meeting with the Domain Expert, the expert is asked to
create a list of top software packages in the domain.  This is done to help
the expert get in the right mind set in advance of the meeting.  Moreover,
by doing the exercise in advance, we avoid the potential pitfall of the expert
approving the discovered list of software without giving it adequate thought.

The Domain Experts are asked to vet the collected data and analysis.  In
particular, they are asked to vet the proposed list of software packages and the
AHP ranking.  These interactions can be done either electronically or with
in-person (or virtual) meetings.

\subsection{List of Representative Software} \label{sec_software_selection}

We have a two step process for selecting software packages: i) identify software
candidates in the chosen domain; and, ii) filter the list to remove less
relevant members \citep{SmithEtAl2021}.

We initially identified 48 MI candidate software projects from the literature
\citep{Bjorn2017, Bruhschwein2019, Haak2015}, online articles \citep{Emms2019,
Hasan2020, Mu2019}, and forum discussions \citep{Samala2014}.  The full list of
48 packages is available in \citet{Dong2021}.  To reduce the length of the list
to a manageable number (29 in this case, as given in Section~\ref{ch_results}),
we filtered the original list as follows:

\begin{enumerate}

\item We removed the packages that did not have source code available, such as
\textit{MicroDicom}, \textit{Aliza}, and \textit{jivex}.

\item We focused on the MI software that provides visualization functions, as
described in Section~\ref{sec_scope}. We removed seven packages that were
toolkits or libraries, such as \textit{VTK}, \textit{ITK}, and \textit{dcm4che}.
We removed another three that were for PACS.

\item We removed \textit{Open Dicom Viewer}, since it has not received any
updates in a long time (since 2011).

\end{enumerate}

The Domain Expert provided a list of his top 12 software packages.  We compared
his list to our list of 29.  We found 6 packages were on both lists: \textit{3D
Slicer}, \textit{Horos}, \textit{ImageJ}, \textit{Fiji}, \textit{MRIcron} (we
actually use the update version \textit{MRIcroGL}) and \textit{Mango} (we
actually use the web version \textit{Papaya}).  Six software packages
(\textit{AFNI}, \textit{FSL}, \textit{Freesurfer}, \textit{Tarquin},
\textit{Diffusion Toolkit}, and \textit{MRItrix}) were on the Domain Expert
list, but not on our filtered list.  However, when we examined those packages,
we found they were out of scope, since their primary function was not
visualization.  The Domain Expert agreed with our final choice of 29 packages.

\subsection{Grading Software} \label{sec_grading_software}

We grade the selected software using the measurement template summarized in
\citet{SmithEtAl2021}.  The template provides measures of the qualities listed
in Section~\ref{sec_software_quality}, except for reproducibility, which is
assessed through the developer interviews (Section~\ref{sec_interview_methods}).
For each software package, we fill-in the template questions. To stay within the
target of 160 person hours to measure the domain, we allocated between 1 to 4
hours for each package. Project developers can be contacted for help regarding
installation, if necessary, but a cap of about 2 hours is imposed on the
installation process, to keep the overall measurement time feasible. An excerpt
of the spreadsheet is shown in Figure~\ref{fg_grading_template_example}.  A
column is included for each measured software package. 

\begin{figure}[!ht]
\includegraphics[scale=0.67]{figures/template.pdf}
\caption{Grading template example}
\label{fg_grading_template_example}
\end{figure}

The full template consists of 108 questions categorized under 9 qualities.  The
questions were designed to be unambiguous, quantifiable and measurable with
limited time and domain knowledge. The measures are grouped under headings for
each quality, and one for summary information. The summary information (shown in
Figure~\ref{fg_grading_template_example}) is the first section of the template.
This section summarizes general information, such as the software name, purpose,
platform, programming language, publications about the software, the first
release and the most recent change date, website, source code repository of the
product, number of developers, etc.  We follow the definitions given by
\citet{GewaltigAndCannon2012} for the software categories.  Public means
software intended for public use.  Private means software aimed only at a
specific group, while the concept category is used for software written simply
to demonstrate algorithms or concepts. The three categories of development
models are (open source, free-ware and commercial) are discussed in
Section~\ref{sec_software_categories}.  Information in the summary section sets
the context for the project, but it does not directly affect the grading scores.

For measuring each quality, we ask several questions and the typical answers are
among the collection of ``yes'', ``no'', ``n/a'', ``unclear'', a number, a
string, a date, a set of strings, etc. Each quality is assigned an overall
score, between 1 and 10, based on all the previous questions.  Several of the
qualities use the word ``surface''.  This is to highlight that, for these
qualities in particular, the best that we can do is a shallow measure.  For
instance, we are not currently doing any experiments to measure usability.
Instead, we are looking for an indication that usability was considered by the
developers.  We do this by looking for cues in the documentation, like a getting
started manual, a user manual and documentation of expected user
characteristics.  Below is a summary of how we assess adoption of best practices
by measuring each quality.

\begin{itemize}

\item \textbf{Installability} We assess the following: 
\begin{inparaenum}[i)]
    \item existence and quality of installation instructions;
    \item the quality of the user experience via the ease of following
    instructions, number of steps, automation tools; and,
    \item whether there is a means to verify the installation.
\end{inparaenum}
If any problem interrupts the process of installation or uninstallation, we give
a lower score. We also record the Operating System (OS) used for the
installation test.

\item \textbf{Correctness \& Verifiability} We check each project to identify
any techniques used to ensure this quality, such as literate programming,
automated testing, symbolic execution, model checking, unit tests, etc. We also
examine whether the projects use Continuous Integration and Continuous Delivery
(CI/CD). For verifiability, we go through the documents of the projects to check
for the presence of requirements specifications, theory manuals, and getting
started tutorials. If a getting started tutorial exists and provides expected
results, we follow it to check the correctness of the output.

\item \textbf{Surface Reliability} We check the following: 
\begin{inparaenum}[i)]
    \item whether the software breaks during installation;
    \item the operation of the software following the getting started tutorial
    (if present);
    \item whether the error messages are descriptive; and,
    \item whether we can recover the process after an error.
\end{inparaenum}

\item \textbf{Surface Robustness} We check how the software handles
unexpected/unanticipated input. For example, we prepare broken image files for
MI software packages that load image files. We use a text file (.txt) with a
modified extension name (.dcm) as an unexpected/unanticipated input. We load a
few correct input files to ensure the function is working correctly before
testing the unexpected/unanticipated ones.

\item \textbf{Surface Usability} We examine the project's documentation,
checking for the presence of getting started tutorials and/or a user manual. We
also check whether users have channels to request support, such as an e-mail
address, or issue tracker. Our impressions of usability are based on our
interaction with the software during testing.  In general, an easy-to-use
graphical user interface will score high.

\item \textbf{Maintainability} We believe that the artifacts of a project,
including source code, documents, and building scripts, significantly influence
its maintainability. Thus we check each project for the presence of such
artifacts as API documentation, bug tracker information, release notes, test
cases, and build scripts. We also check for the use of tools supporting issue
tracking and version control, the percentages of closed issues, and the
proportion of comment lines in the code.

\item \textbf{Reusability} We count the total number of code files for each
project. Projects with a large number of components potentially provide more
choices for reuse. Furthermore, well-modularized code, which tends to have
smaller parts in separate files, is typically easier to reuse. Thus, we assume
that projects with more code files and less Lines of Code (LOC) per file are
more reusable. We also consider projects with API documentation as delivering
better reusability.

\item \textbf{Surface Understandability} Given that time is a constraint, we
cannot look at all code files for each project; therefore, we randomly examine
10 code files for their understandability. We check the code's style within each
file, such as whether the identifiers, parameters, indentation, and formatting
are consistent, whether the constants (other than 0 and 1) are hardcoded, and
whether the code is modularized. We also check the descriptive information for
the code, such as documents mentioning the coding standard, the comments in the
code, and the descriptions or links for details on algorithms in the code. 

\item \textbf{Visibility/Transparency} To measure this quality, we check the
existing documents to find whether the software development process and
current status of a project are visible and transparent. We examine the
development process, current status, development environment, and release notes
for each project.
\end{itemize}

As part of filling in the measurement template, we use freeware tools to collect
repository related data. \href{https://github.com/tomgi/git_stats}{GitStats}
\citep{Gieniusz2019} is used to measure the number of binary files as well as
the number of added and deleted lines in a repository. We also use this tool to
measure the number of commits over different intervals of time.
\href{https://github.com/boyter/scc}{Sloc Cloc and Code (scc)}
\citep{Boyter2021} is used to measure the number of text based files as well as
the number of total, code, comment, and blank lines in a repository.

Both tools measure the number of text-based files in a git repository and lines
of text in these files. Based on our experience, most text-based files in a
repository contain programming source code, and developers use them to compile
and build software products. A minority of these files are instructions and
other documents. So we roughly regard the lines of text in text-based files as
lines of programming code. The two tools usually generate similar but not
identical results. From our understanding, this minor difference is due to the
different techniques to detect if a file is text-based or binary.

For projects on GitHub we manually collect additional information, such as the
numbers of stars, forks, people watching this repository, open pull requests,
closed pull requests, and the number of months a repository has been on GitHub.
We need to take care with the project creation date, since a repository can have
a creation date much earlier than the first day on GitHub.  For example, the
developers created the git repository for \textit{3D Slicer} in 2002, but did
not upload a copy of it to GitHub until 2020. Some GitHub data can be found
using its the GitHub Application Program Interface (API) via the following url:
\textit{https://api.github.com/repos/[owner]/[repository]} where [owner] and
[repository] are replaced by the repo specific values. The number of months a
repository has been on GitHub helps us understand the average change of metrics
over time, like the average new stars per month. 

The repository measures help us in many ways. Firstly, they help us get a fast
and accurate project overview. For example, the number of commits over the last
12 months shows how active a project has been, and the number of stars and forks
may reveal its popularity (used to assess \rqref{RQ_CompareHQ2Popular}).
Secondly, the results may affect our decisions regarding the grading scores for
some software qualities. For example, if the percentage of comment lines is low,
we double-check the understandability of the code; if the ratio of open versus
closed pull requests is high, we pay more attention to maintainability.

As in \citet{SmithEtAl2016}, Virtual machines (VMs) were used to provide an
optimal testing environments for each package. VMs were used because it is
easier to start with a fresh environment without having to worry about existing
libraries and conflicts. Moreover, when the tests are complete the VM can be
deleted, without any impact on the host operating system. The most significant
advantage of using VMs is to level the playing field. Every software install
starts from a clean slate, which removes ``works-on-my-computer'' errors. When
filling in the measurement template spreadsheet, the details for each VM
are noted, including hypervisor and operating system version.

When grading the software, we found 27 out of the 29 packages are compatible
with two or three different OSes, such as Windows, macOS, and Linux, and 5 of
them are browser-based, making them platform-independent. However, in the
interest of time, we only performed the measurements for each project by
installing it on one of the platforms.  When it was an option, we selected
Windows as the host OS.

\subsection{Interview Methods} \label{sec_interview_methods}

The repository-based measurements summarize the information we can collect from
on-line resources. This information is incomplete because it doesn't generally
capture the development process, the developer pain points, the perceived
threats to software quality, and the developers' strategies to address these
threats.  Therefore, part of our methodology is to also interview developers.

We based our interviews on a list of 20 questions, which can be found in
\citet{SmithEtAl2021}. Some questions are about the background of the software,
the development teams, the interviewees, and how they organize the projects. We
also ask about the developer's understandings of the users. Some questions focus
on the current and past difficulties, and the solutions the team has found, or
will try. We also discuss the importance and current situations of
documentation. A few questions are about specific software qualities, such as
maintainability, understandability, usability, and reproducibility. The
interviews are semi-structured based on the question list; we ask follow-up
questions when necessary.  The interview process presented here was approved by
the McMaster University Research Ethics Board under the application number 
\href{https://github.com/smiths/AIMSS/blob/master/StateOfPractice/MACREM/Application.pdf}
{MREB\#: 5219}.

We sent interview requests to all 29 projects using contact information from
projects websites, code repository, publications, and from biographic pages at
the teams's institutions.  In the end nine developers from eight of the projects
agreed to participate: \textit{3D Slicer}, \textit{INVESALIUS 3}, \textit{dwv},
\textit{BioImage Suite Web}, \textit{ITK-SNAP}, \textit{MRIcroGL},
\textit{Weasis}, and \textit{OHIF}. We spent about 90 minutes for each
interview. One participant was too busy to have an interview, so they wrote down
their answers. In one case two developers from the same project agreed to be
interviewed. Meetings were held on-line using either Zoom or Teams, which
facilitated recording and automatic transcription of the meetings. The full
interview answers can be found in \citet{Dong2021}.

\section{Measurement Results} \label{ch_results}

Table \ref{tab_final_list} shows the 29 software packages that we measured,
along with summary data collected in the year 2020. We arrange the items in
descending order of LOC. We found the initial release dates (Rlsd) for most
projects and marked the two unknown dates with ``?''. We used the date of the
latest change to each code repository to decide the latest update. We found
funding information (Fnd) for only eight projects.  For the number of
contributors (NOC) we considered anyone who made at least one accepted commit as
a contributor. The NOC is not usually the same as the number of long-term
project members, since many projects received change requests and code from the
community.  With respect to the OS, 25 packages work on all three OSs: Windows
(W), macOS (M), and Linux (L). Although the usual approach to cross-platform
compatibility was to work natively on multiple OSes, five projects achieved
platform-independence via web applications. The full measurement data for all
packages is available in \citet{Dong2021-Data}.

\begin{table}[!ht]
\centering
\begin{tabular}{p{6cm}lllllllll}
\toprule
\multirow{2}{*}{Software} & \multirow{2}{*}{Rlsd} & \multirow{2}{*}{Updated} & \multirow{2}{*}{Fnd} & \multirow{2}{*}{NOC} & \multirow{2}{*}{LOC} & \multicolumn{3}{c}{OS} & \multirow{2}{*}{Web} \\ \cline{7-9}
 &  &  &  &  &  & W & M & L &  \\ \midrule
ParaView \citep{Ahrens2005} & 2002 & 2020-10 & X & 100 & 886326 & X & X & X & X \\
Gwyddion \citep{Nevcas2012} & 2004 & 2020-11 &  & 38 & 643427 & X & X & X &  \\
Horos \citep{horosproject2020} & ? & 2020-04 &  & 21 & 561617 &  & X &  &  \\
OsiriX Lite \citep{PixmeoSARL2019} & 2004 & 2019-11 &  & 9 & 544304 &  & X &  &  \\
3D Slicer \citep{Kikinis2014} & 1998 & 2020-08 & X & 100 & 501451 & X & X & X &  \\
Drishti \citep{Limaye2012} & 2012 & 2020-08 &  & 1 & 268168 & X & X & X &  \\
Ginkgo CADx \citep{Wollny2020} & 2010 & 2019-05 &  & 3 & 257144 & X & X & X &  \\
GATE \citep{Jan2004} & 2011 & 2020-10 &  & 45 & 207122 &  & X & X &  \\
3DimViewer \citep{TESCAN2020} & ? & 2020-03 & X & 3 & 178065 & X & X &  &  \\
medInria \citep{Fillard2012} & 2009 & 2020-11 &  & 21 & 148924 & X & X & X &  \\
BioImage Suite Web \citep{Papademetris2005} & 2018 & 2020-10 & X & 13 & 139699 &
X & X & X & X \\
Weasis \citep{Roduit2021} & 2010 & 2020-08 &  & 8 & 123272 & X & X & X &  \\
AMIDE \citep{Loening2017} & 2006 & 2017-01 &  & 4 & 102827 & X & X & X &  \\
XMedCon \citep{Nolf2003} & 2000 & 2020-08 &  & 2 & 96767 & X & X & X &  \\
ITK-SNAP \citep{Yushkevich2006} & 2006 & 2020-06 & X & 13 & 88530 & X & X & X &  \\
Papaya \citep{UTHSCSA2019} & 2012 & 2019-05 &  & 9 & 71831 & X & X & X &  \\
OHIF Viewer \citep{Ziegler2020} & 2015 & 2020-10 &  & 76 & 63951 & X & X & X & X \\
SMILI \citep{Chandra2018} & 2014 & 2020-06 &  & 9 & 62626 & X & X & X &  \\
INVESALIUS 3 \citep{Amorim2015} & 2009 & 2020-09 &  & 10 & 48605 & X & X & X &  \\
dwv \citep{Martelli2021} & 2012 & 2020-09 &  & 22 & 47815 & X & X & X & X \\
DICOM Viewer \citep{Afsar2021} & 2018 & 2020-04 & X & 5 & 30761 & X & X & X &  \\
MicroView \citep{ParallaxInnovations2020} & 2015 & 2020-08 &  & 2 & 27470 & X & X & X &  \\
MatrixUser \citep{Liu2016} & 2013 & 2018-07 &  & 1 & 23121 & X & X & X &  \\
Slice:Drop \citep{Haehn2013} & 2012 & 2020-04 &  & 3 & 19020 & X & X & X & X \\
dicompyler \citep{Panchal2010} & 2009 & 2020-01 &  & 2 & 15941 & X & X &  &  \\
Fiji \citep{Schindelin2012} & 2011 & 2020-08 & X & 55 & 10833 & X & X & X &  \\
ImageJ \citep{Rueden2017} & 1997 & 2020-08 & X & 18 & 9681 & X & X & X &  \\
MRIcroGL \citep{Rorden2021} & 2015 & 2020-08 &  & 2 & 8493 & X & X & X &  \\
DicomBrowser \citep{Archie2012} & 2012 & 2020-08 &  & 3 & 5505 & X & X & X &  \\ \bottomrule
\end{tabular}
\caption{Final software list (sorted in descending order of the number of Lines
Of Code (LOC))}
\label{tab_final_list}
\end{table}

Figure \ref{fig_language} shows the primary languages versus the number of
projects using them.  The primary language is the language used for the majority
of the project's code; in most cases projects also use other languages.  The
most popular language is C++, with almost 40\% of projects (11 of 29).  The two
least popular choices are Pascal and Matlab, with around 3\% of projects each
(1 of 29).

\begin{figure}[!ht]
\centering
\includegraphics[scale=0.5]{figures/PrimaryLanguages.pdf}
\centering
\caption{\label{fig_language}Primary languages versus number of projects}
\end{figure}

\subsection{Installability} \label{sec_result_installability}

Figure \ref{fg_installability_scores} lists the installability scores.  We found
installation instructions for 16 projects. Among the ones without instructions,
\textit{BioImage Suite Web} and \textit{Slice:Drop} do not need installation,
since they are web applications. Installing 10 of the projects required extra
dependencies. Five of them are web applications (as shown in
Table~\ref{tab_final_list}) and depend on a browser; \textit{dwv}, \textit{OHIF
Viewer}, and \textit{GATE} needs extra dependencies to build; \textit{ImageJ}
and	\textit{Fiji} need an unzip tool; \textit{MatrixUser} is based on Matlab;
\textit{DICOM Viewer} needs to work on a Nextcloud platform.

\begin{figure}[!ht]
\includegraphics[scale=0.48]{figures/installability_scores.pdf}
\caption{AHP installability scores}
\label{fg_installability_scores}
\end{figure}

\textit{3D Slicer} has the highest score because it had easy to follow
installation instructions, and the installation processes were automated, fast,
and frustration-free, with all dependencies automatically added. There were also
no errors during the installation and uninstallation steps. Many other software
packages also had installation instructions and automated installers, and we had
no trouble installing them, such as \textit{INVESALIUS 3}, \textit{Gwyddion},
\textit{XMedCon}, and \textit{MicroView}. We determined their scores based on
the understandability of the instructions, installation steps, and user
experience. Since \textit{BioImage Suite Web} and \textit{Slice:Drop} needed no
installation, we gave them high scores. \textit{BioImage Suite Web} also
provided an option to download cache for offline usage, which was easy
to apply.

\textit{dwv}, \textit{GATE}, and \textit{DICOM Viewer} showed severe
installation problems. We were not able to install them, even after a reasonable
amount of time (2 hours).  For \textit{dwv} and \textit{GATE} we failed to build
from the source code, but we were able to proceed with measuring other qualities
using a deployed online version for \textit{dwv}, and a VM version for
\textit{GATE}. For \textit{DICOM Viewer} we could not install the NextCloud
dependency, and we did not have another option for running the software.
Therefore, for \textit{DICOM Viewer} we could not measure reliability or
robustness.  The other seven qualities could be measured, since they do not
require installation.

\textit{MatrixUser} has a lower score because it depended on Matlab. We assessed
the score from the point of view of a user that would have to install Matlab and
acquire a license.  Of course, for users that already work within Matlab, the
installability score should be higher.

\subsection{Correctness \& Verifiability} \label{sec_result_correctness_verifiability}

The scores of correctness \& verifiability are shown in
Figure~\ref{fg_correctness_verifiability_scores}. Generally speaking, the
packages with higher scores adopted more techniques to improve correctness, and
had better documentation for us to verify against.  For instance, we looked for
evidence of unit testing, since it benefits most parts of the software's life
cycle, such as designing, coding, debugging, and optimization
\citep{Hamill2004}.  We only found evidence of unit testing for about half of
the projects. We identified five projects using CI/CD tools: \textit{3D Slicer},
\textit{ImageJ}, \textit{Fiji}, \textit{dwv}, and \textit{OHIF Viewer}.

\begin{figure}[!ht]
\includegraphics[scale=0.48]{figures/correctness_verifiability_scores.pdf}
\caption{AHP correctness \& verifiability scores}
\label{fg_correctness_verifiability_scores}
\end{figure}

Even for some projects with well-organized documents, requirements
specifications and theory manuals were still missing.  We could not identify
theory manuals for all projects and we did not find requirements specifications
for most projects. The only document we found was a road map of \textit{3D
Slicer}, which contained design requirements for upcoming changes.

\subsection{Surface Reliability} \label{sec_result_reliability}

Figure~\ref{fg_reliability_scores} shows the AHP results.  As shown in
Section~\ref{sec_result_installability}, most of the software products did not
``break'' during installation, or did not need installation; \textit{dwv} and
\textit{GATE} broke in the building stage, and the processes were not
recoverable; we could not install the dependency for \textit{DICOM Viewer}. Of
the seven software packages with a getting started tutorial and operation steps
in the tutorial, most showed no error when we followed the steps. However,
\textit{GATE} could not open macro files and became unresponsive several times,
without any descriptive error message. When assessing robustness
(Section~\ref{sec_result_robustness}), we found that \textit{Drishti} crashed
when loading damaged image files, without showing any descriptive error message.
On the other hand, we did not find any problems with the online version of
\textit{dwv}.

\begin{figure}[!ht]
\includegraphics[scale=0.48]{figures/reliability_scores.pdf}
\caption{AHP surface reliability scores}
\label{fg_reliability_scores}
\end{figure}

\subsection{Surface Robustness} \label{sec_result_robustness}

Figure \ref{fg_robustness_scores} presents the scores for surface robustness.
The packages with higher scores elegantly handled unexpected/unanticipated
inputs, typically showing a clear error message. We may have underestimated the
score of \textit{OHIF Viewer}, since we needed further customization to load
data.

\begin{figure}[!ht]
\includegraphics[scale=0.48]{figures/robustness_scores.pdf}
\caption{AHP surface robustness scores}
\label{fg_robustness_scores}
\end{figure}

Digital Imaging and Communications in Medicine (DICOM) ``defines the formats for
medical images that can be exchanged with the data and quality necessary for
clinical use'' \citep{MITA2021}. According to their documentation, all 29
software packages should support the DICOM standard. To test robustness, we
prepared two types of image files: correct and incorrect formats (with the
incorrect format created by relabelled a text file to have the ``.dcm''
extension).  All software packages loaded the correct format image, except for
\textit{GATE}, which failed for unknown reasons.  For the broken format,
\textit{MatrixUser}, \textit{dwv}, and \textit{Slice:Drop} ignored the incorrect
format of the file and loaded it regardless. They did not show any error message
and displayed a blank image. \textit{MRIcroGL} behaved similarly except that it
showed a meaningless image. \textit{Drishti} successfully detected the broken
format of the file, but the software crashed as a result.

\subsection{Surface Usability} \label{sec_result_usability}

Figure \ref{fg_usability_scores} shows the AHP scores for surface usability. The
software with higher scores usually provided both comprehensive documented
guidance and a good user experience. \textit{INVESALIUS 3} provided an excellent
example of a detailed and precise user manual. \textit{GATE} also provided a
large number of documents, but unfortunately we had difficulty understanding and
using them. We found getting started tutorials for only 11 projects, but a user
manual for 22 projects. \textit{MRIcroGL} was the only project that explicitly
documented expected user characteristics.

\begin{figure}[!ht]
\includegraphics[scale=0.48]{figures/usability_scores.pdf}
\caption{AHP surface usability scores}
\label{fg_usability_scores}
\end{figure}
 
\subsection{Maintainability} \label{sec_score_maintainability}

Figure~\ref{fg_maintainability_scores} shows the ranking results for
maintainability. We marked \textit{3D Slicer} with the highest score because
we found it to have the most comprehensive artifacts. For example, as far as we
could find, only a few of the 29 projects had a product, developer's
manual, or API documentation, and only \textit{3D Slicer}, \textit{ImageJ},
\textit{Fiji} included all three documents. Moreover, \textit{3D Slicer} has a
much higher percentage of closed issues (91.65\%) compared to \textit{ImageJ}
(52.49\%) and \textit{Fiji} (63.79\%). Table~\ref{tab_maintainability_docs}
shows which projects had these documents, in the descending order of their
maintainability scores. 

\begin{figure}[!ht]
\includegraphics[scale=0.48]{figures/maintainability_scores.pdf}
\caption{AHP maintainability scores}
\label{fg_maintainability_scores}
\end{figure}

\begin{table}[!ht]
\centering
\begin{tabular}{lccc}
\toprule
\multicolumn{1}{c}{Software} & Prod.\ roadmap & Dev.\ manual & API doc. \\ 
\midrule
3D Slicer & X & X & X \\
ImageJ & X & X & X \\
Weasis &  & X &  \\
OHIF Viewer &  & X & X \\
Fiji & X & X & X \\
ParaView & X &  &  \\
SMILI &  &  & X \\
medInria &  & X &  \\
INVESALIUS 3 & X &  &  \\
dwv &  &  & X \\
BioImage Suite Web &  & X &  \\
Gwyddion &  & X & X \\ 
\bottomrule
\end{tabular}
\caption{Software with the maintainability documents (listed in descending order of 
maintainability score)}
\label{tab_maintainability_docs}
\end{table}

27 of the 29 projects used git as the version control tool, with 24 of these
using GitHub. \textit{AMIDE} used Mercurial and \textit{Gwyddion} used
Subversion. \textit{XMedCon}, \textit{AMIDE}, and \textit{Gwyddion} used
SourceForge. \textit{DicomBrowser} and \textit{3DimViewer} used BitBucket. 

\subsection{Reusability} \label{sec_result_reusability}

Figure~\ref{fg_reusability_scores} shows the AHP results for reusability. As
described in Section~\ref{sec_grading_software}, we gave higher scores to the
projects with API documentation. As shown in
Table~\ref{tab_maintainability_docs}, seven projects had API documents. We also
assumed that projects with more code files and less LOC per code file as more
reusable. Table \ref{tab_loc_per_file} shows the number of text-based files by
projects, which we used to approximate the number of code files. The table also
lists the total number of lines (including comments and blanks), LOC, and
average LOC per file. We arranged the items in descending order of their
reusability scores.

\begin{figure}[!ht]
\includegraphics[scale=0.48]{figures/reusability_scores.pdf}
\caption{AHP reusability scores}
\label{fg_reusability_scores}
\end{figure}

\begin{table}[!ht]
\centering
\begin{tabular}{lllll}
\toprule
\multirow{2}{*}{Software} & \multirow{2}{*}{Text files} & \multirow{2}{*}{Total lines} & \multirow{2}{*}{LOC} & \multirow{2}{*}{LOC/file} \\
 &  &  &  &  \\ 
\midrule
OHIF Viewer & 1162 & 86306 & 63951 & 55 \\
3D Slicer & 3386 & 709143 & 501451 & 148 \\
Gwyddion & 2060 & 787966 & 643427 & 312 \\
ParaView & 5556 & 1276863 & 886326 & 160 \\
OsiriX Lite & 2270 & 873025 & 544304 & 240 \\
Horos & 2346 & 912496 & 561617 & 239 \\
medInria & 1678 & 214607 & 148924 & 89 \\
Weasis & 1027 & 156551 & 123272 & 120 \\
BioImage Suite Web & 931 & 203810 & 139699 & 150 \\
GATE & 1720 & 311703 & 207122 & 120 \\
Ginkgo CADx & 974 & 361207 & 257144 & 264 \\
SMILI & 275 & 90146 & 62626 & 228 \\
Fiji & 136 & 13764 & 10833 & 80 \\
Drishti & 757 & 345225 & 268168 & 354 \\
ITK-SNAP & 677 & 139880 & 88530 & 131 \\
3DimViewer & 730 & 240627 & 178065 & 244 \\
DICOM Viewer & 302 & 34701 & 30761 & 102 \\
ImageJ & 40 & 10740 & 9681 & 242 \\
dwv & 188 & 71099 & 47815 & 254 \\
MatrixUser & 216 & 31336 & 23121 & 107 \\
INVESALIUS 3 & 156 & 59328 & 48605 & 312 \\
AMIDE & 183 & 139658 & 102827 & 562 \\
Papaya & 110 & 95594 & 71831 & 653 \\
MicroView & 137 & 36173 & 27470 & 201 \\
XMedCon & 202 & 129991 & 96767 & 479 \\
MRIcroGL & 97 & 50445 & 8493 & 88 \\
Slice:Drop & 77 & 25720 & 19020 & 247 \\
DicomBrowser & 54 & 7375 & 5505 & 102 \\
dicompyler & 48 & 19201 & 15941 & 332 \\ 
\bottomrule
\end{tabular}
\caption{Number of files and lines (entries are sorted in descending order of
their reusability scores)}
\label{tab_loc_per_file}
\end{table}

\subsection{Surface Understandability} \label{sec_result_understandability}

Figure~\ref{fg_surface_understandability_scores} shows the scores for surface
understandability. All projects had a consistent coding style with parameters in
the same order for all functions; the code was modularized; the comments were
clear, indicating what is being done, not how. However, we only found explicit
identification of a coding standard for 3 out of the 29: \textit{3D Slicer},
\textit{Weasis}, and \textit{ImageJ}. We also found hard-coded constants (rather
than symbolic constants) in \textit{medInria}, \textit{dicompyler},
\textit{MicroView}, and \textit{Papaya}. We did not find any reference to the
algorithms used in projects \textit{XMedCon}, \textit{DicomBrowser},
\textit{3DimViewer}, \textit{BioImage Suite Web}, \textit{Slice:Drop},
\textit{MatrixUser}, \textit{DICOM Viewer}, \textit{dicompyler}, and
\textit{Papaya}. 

\begin{figure}[!ht]
\includegraphics[scale=0.48]{figures/understandability_scores.pdf}
\caption{AHP surface understandability scores}
\label{fg_surface_understandability_scores}
\end{figure}

\subsection{Visibility/Transparency} \label{sec_result_visibility_transparency}

Figure~\ref{fg_visibility_transparency_scores} shows the AHP scores for
visibility/transparency. Generally speaking, the teams that actively documented
their development process and plans scored higher.
Table~\ref{tab_Visibility/Transparency_docs} shows the projects that had
documents for the development process, project status, development environment,
and release notes, in descending order of their visibility/transparency
scores.

\begin{figure}[!ht]
\includegraphics[scale=0.48]{figures/visibility_transparency_scores.pdf}
\caption{AHP visibility/transparency scores}
\label{fg_visibility_transparency_scores}
\end{figure}

\begin{table}[!ht]
\centering
\begin{tabular}{lllll}
\toprule
Software & Dev process & Proj status & Dev env & Rls notes \\ 
\midrule
3D Slicer & X & X & X & X \\
ImageJ & X & X & X & X \\
Fiji & X & X & X &  \\
MRIcroGL &  &  &  & X \\
Weasis &  &  & X & X \\
ParaView &  & X &  &  \\
OHIF Viewer &  &  & X & X \\
DICOM Viewer &  &  & X & X \\
medInria &  &  & X & X \\
SMILI &  &  &  & X \\
Drishti &  &  &  & X \\
INVESALIUS 3 &  &  &  & X \\
OsiriX Lite &  &  &  & X \\
GATE &  &  &  & X \\
MicroView &  &  &  & X \\
MatrixUser &  &  &  & X \\
BioImage Suite Web &  &  & X &  \\
ITK-SNAP &  &  &  & X \\
Horos &  &  &  & X \\
dwv &  &  &  & X \\
Gwyddion &  &  &  & X \\ 
\bottomrule
\end{tabular}
\caption{Software with visibility/transparency related documents (software is
listed in descending order of visibility/transparency score)}
\label{tab_Visibility/Transparency_docs}
\end{table}

\subsection{Overall Scores}

As described in Section~\ref{sec_AHP}, for our AHP measurements, we have nine
criteria (qualities) and 29 alternatives (software packages). In the absence of
a specific real world context, we assumed all nine qualities are equally
important. Figure~\ref{fg_overall_scores} shows the overall scores in descending
order. Since we produced the scores from the AHP process, the total sum of the
29 scores is precisely 1.

\begin{figure}[!ht]
\includegraphics[scale=0.48]{figures/overall_scores.pdf}
\caption{Overall AHP scores with an equal weighting for all 9 software qualities}

\label{fg_overall_scores}
\end{figure}

The top three software products \textit{3D Slicer}, \textit{ImageJ}, and
\textit{OHIF Viewer} had higher scores in most criteria. \textit{3D Slicer}
ranked in the top two software products for all qualities except \textit{surface
robustness}; \textit{ImageJ} ranked in the top three for correctness \&
verifiability, surface reliability, surface usability, maintainability, surface
understandability, and visibility/transparency. \textit{OHIF Viewer} ranked in
the top five products for correctness \& verifiability, surface reliability,
surface usability, maintainability, and reusability. Given the installation
problems, we may have underestimated the scores on reliability and robustness
for \textit{DICOM Viewer}, but we compared it equally for the other seven
qualities.

\section{Comparison to Community Ranking} \label{Sec_VsCommunityRanking}

To address~\rqref{RQ_CompareHQ2Popular} about how our ranking compares to the
popularity of projects as judged by the scientific community, we make two
comparisons:
\begin{itemize}
\item A comparison of our ranking (from Section~\ref{ch_results}) with the
community ratings on GitHub, as shown by GitHub stars, number of forks, and
number of people watching the projects; and,
\item A comparison of top-rated software from our methodology with the top
recommendations from our domain experts (as mentioned in
Section~\ref{sec_software_selection}).
\end{itemize}

Table~\ref{tab_ranking_vs_GitHub} shows our ranking of the 29 MI projects, and
their GitHub metrics, if applicable. As mentioned in
Section~\ref{sec_score_maintainability}, 24 projects used GitHub. Since GitHub
repositories have different creation dates, we collect the number of months each
stayed on GitHub, and calculate the average number of new stars, people
watching, and forks per 12 months. The method of getting the creation date is
described in Section~\ref{sec_grading_software}.  The items in
Table~\ref{tab_ranking_vs_GitHub} are listed in descending order of the average
number of new stars per year.  The non-GitHub items are listed in the order of
our ranking.  All GitHub statistics were collected in July, 2021.  

Generally speaking, most of the top-ranking MI software projects also received
greater attention and popularity on GitHub. Between our ranking and the GitHub
stars-per-year ranking, four of the top five software projects appear in both
lists.Our top 5 packages are scattered among the first 8 positions on the GitHub
list. However, as discussed below there are discrepancies between the two lists.

In some cases projects are popular in the community, but were assigned a low
rank by our methodology.  This is the case for \textit{dwv}. The reason for the
low ranking is that, as mentioned in Section~\ref{sec_result_installability}, we
failed to build it locally, and used the test version on its websites for the
measurements. We followed the instructions and tried to run the command ``yarn
run test'' locally, which did not work. In addition, the test version did not
detect a broken DICOM file and displayed a blank image as described in
Section~\ref{sec_result_robustness}. We might underestimate the scores for
\textit{dwv} due to uncommon technical issues. We also ranked \textit{DICOM
Viewer} much lower than its popularity. As mentioned in
Section~\ref{sec_result_installability} it depended on the NextCloud platform
that we could not successfully install. Thus, we might underestimate the scores
of its surface reliability and surface robustness. In addition, we weighted all
qualities equally, which is not likely to be the case with all users. As a
result, some projects with high community popularity may have scored lower with
our method because of a relatively higher (compared to the scientific
community's implicit ranking) weighting of the poor scores for some qualities. A
further explanation for discrepancies between our measures and the star measures
may also be due to inaccuracy with using stars to approximate popularity.  Stars
are not an ideal measure because stars represent the community's feeling in the
past more than they measure current preferences \citep{Szulik2017}.  The issue
with stars is that they tend only to be added, not removed.  A final reason for
inconsistencies between our ranking and the community's ranking is that, as for
consumer products, more factors influence popularity than just quality.

\begingroup
\renewcommand{\arraystretch}{0.85}
\begin{table}[!ht]
\centering
\begin{tabular}{llllll}
\toprule
Software & Comm.\ rank & Our rank & Stars/yr & Watches/yr & Forks/yr \\ 
\midrule
3D Slicer & 1 & 1 & 284 & 19 & 128 \\
OHIF Viewer & 2 & 3 & 277 & 19 & 224 \\
dwv & 3 & 23 & 124 & 12 & 51 \\
ImageJ & 4 & 2 & 84 & 9 & 30 \\
ParaView & 5 & 5 & 67 & 7 & 28 \\
Horos & 6 & 12 & 49 & 9 & 18 \\
Papaya & 7 & 17 & 45 & 5 & 20 \\
Fiji & 8 & 4 & 44 & 5 & 21 \\
DICOM Viewer & 9 & 29 & 43 & 6 & 9 \\
INVESALIUS 3 & 10 & 10 & 40 & 4 & 17 \\
Weasis & 11 & 6 & 36 & 5 & 19 \\
dicompyler & 12 & 26 & 35 & 5 & 14 \\
OsiriX Lite & 13 & 9 & 34 & 9 & 24 \\
MRIcroGL & 14 & 18 & 24 & 3 & 3 \\
GATE & 15 & 25 & 19 & 6 & 26 \\
Ginkgo CADx & 16 & 14 & 19 & 4 & 6 \\
BioImage Suite Web & 17 & 8 & 18 & 5 & 7 \\
Drishti & 18 & 27 & 16 & 4 & 4 \\
Slice:Drop & 19 & 20 & 10 & 2 & 5 \\
ITK-SNAP & 20 & 15 & 9 & 1 & 4 \\
medInria & 21 & 7 & 7 & 3 & 6 \\
SMILI & 22 & 13 & 3 & 1 & 2 \\
MatrixUser & 23 & 28 & 2 & 0 & 0 \\
MicroView & 24 & 16 & 1 & 1 & 1 \\
Gwyddion & 25 & 11 & n/a & n/a & n/a \\
XMedCon & 26 & 19 & n/a & n/a & n/a \\
DicomBrowser & 27 & 21 & n/a & n/a & n/a \\
AMIDE & 28 & 22 & n/a & n/a & n/a \\
3DimViewer & 29 & 24 & n/a & n/a & n/a \\ 
\bottomrule
\end{tabular}
\caption{Software ranking by our methodology versus the community (Comm.)
ranking using GitHub metrics (Sorted in descending order of community
popularity, as estimated by the number of new stars per year)}
\label{tab_ranking_vs_GitHub}
\end{table}
\endgroup

As shown in Section~\ref{sec_software_selection}, our domain experts recommended
a list of top software with 12 software products.  All of the top 4 entries from
the Domain Expert's list are among the top 12 ranked by our methodology. Three
of the top four on both lists are the same: \textit{3D Slicer}, \textit{ImageJ},
and \textit{Fiji}. \textit{3D Slicer} is top project by both rankings (and by
the GithHub stars measure as well).  The Domain Expert ranked \textit{Horos} as
their second choice, while we ranked it twelfth.  Our third ranked project,
\textit{OHIF Viewer} was not listed by the Domain Expert.  Neither were the
software packages that we ranked from fifth to eleventh (\textit{ParaView},
\textit{Weasis}, \textit{medInria}, \textit{BioImage Suite Web}, \textit{OsiriX
Lite}, \textit{INVESALIUS}, and \textit{Gwyddion}).  The software mentioned by
the Domain Expert that we did not rank were the six recommended packages that
did not have visualization as the primary function (as discussed in
Section~\ref{sec_software_selection}).  The differences between the list
recommended by our methodology and the Domain Expert are not surprising.  As
mentioned above, the methodology weights all qualities equally, but that may not
be the case for the Domain Expert's impressions.  Moreover, although the Domain
Expert has significant experience with MI software, he has not used every one of
the 29 packages that were measured.

Although our ranking and the estimate of the community's ranking are not perfect
measures, they do suggest a correlation between best practices and popularity.
We do not know which comes first, the use of best practices or popularity, but
we do know that the top ranked packages tend to incorporate best practices. The
next sections will explore how the practices of the MI community compare to the
broader research software community. We will also investigate the practices from
the top projects that others within the MI community, and within the broader
research software community, can potentially adopt.

\section{Comparison Between MI and Research Software for Artifacts}
\label{Sec_CompareArtifacts}

As part of filling in the measurement template (from
Section~\ref{sec_grading_software}) we summarized the artifacts observed in each
MI package. Table~\ref{artifactspresent} groups the artifacts by frequency into
categories of common (20 to 29 ($>$67\%) packages), uncommon (10 to 19 (33-67\%)
packages), and rare (1 to 9 ($<$33\%) packages). The full measurements are
summarized in \citet{Dong2021-Data}.  The details on which projects use which
types of artifacts are summarized in Tables~\ref{tab_maintainability_docs}
and~\ref{tab_Visibility/Transparency_docs} for documents related to
maintainability and visibility, respectively.

\begin{table}[ht!]
    \begin{center}
    \begin{tabular}{ p{4.6 cm} p{5.6 cm} p{5 cm}}
    \toprule
    \textbf{Common} & \textbf{Uncommon} & \textbf{Rare} \\
    \midrule
    README (29) & Build scripts (18) & Getting Started (9)\\
    Version control (29) & Tutorials (18) & Developer's manual (8)\\
    License (28) & Installation guide (16) & Contributing (8)\\
    Issue tracker (28) & Test cases (15) & API documentation (7)\\
    User manual (22) & Authors (14) & Dependency list (7)\\
    Release info. (22) & Frequently Asked Questions (FAQ) (14) & Troubleshooting guide (6)\\
     & Acknowledgements (12) & Product roadmap (5)\\
     & Changelog (12) & Design documentation (5)\\
     & Citation (11) & Code style guide (3)\\
     & & Code of conduct (1)\\
     & & Requirements (1)\\
    \bottomrule
    \end{tabular}
    \caption{Artifacts Present in MI Packages, Classified by Frequency (The number 
    in brackets is the number of occurrences)}
    \label{artifactspresent}
    \end{center}
\end{table}

We answer~\rqref{RQ_CompareArtifacts} by comparing the artifacts that we
observed in MI repositories to those observed and recommended for research
software in general. Our comparison may point out areas where some MI software
packages fall short of current best practices. This is not intended to be a
criticism of any existing packages, especially since in practice not every
project needs to achieve the highest possible quality. However, rather than
delve into the nuances of which software can justify compromising which
practices we will write our comparison under the ideal assumption that every
project has sufficient resources to match best practices.
    
Table~\ref{Tbl_Guidelines} (based on data from \citep{SmithAndMichalski2022})
shows that MI artifacts generally match the recommendations found in nine
current research software development guidelines:
\begin{itemize}
\item United States Geological Survey Software Planning Checklist
\citep{USGS2019},
\item DLR (German Aerospace Centre) Software Engineering Guidelines
\citep{TobiasEtAl2018}, 
\item Scottish Covid-19 Response Consortium Software Checklist
\citep{BrettEtAl2021},
\item Good Enough Practices in Scientific Computing \citep{WilsonEtAl2016},
\item xSDK (Extreme-scale Scientific Software Development Kit) Community Package
Policies \citep{SmithAndRoscoe2018},
\item Trilinos Developers Guide \citep{HerouxEtAl2008},
\item EURISE (European Research Infrastructure Software Engineers') Network
Technical Reference \citep{ThielEtAl2020},
\item CLARIAH (Common Lab Research Infrastructure for the Arts and Humanities)
Guidelines for Software Quality \citep{vanGompelEtAl2016}, and
\item A Set of Common Software Quality Assurance Baseline Criteria for Research
Projects \citep{OrvizEtAl2017}.
\end{itemize}

In Table~\ref{Tbl_Guidelines} each row corresponds to an artifact.  For a given
row, a checkmark in one of the columns means that the corresponding guideline
recommends this artifact.  The last column shows whether the artifact appears in
the measured set of MI software, either not at all (blank), commonly (C),
uncommonly (U) or rarely (R).  We did our best to interpret the meaning of each
artifact consistently between guidelines and specific MI software, but the
terminology and the contents of artifacts are not standardized.  The challenge
even exists for the ubiquitous README file.  As illustrated by
\citet{PranaEtAl2018}, the content of README files shows significant variation
between projects.  Although some content is reasonably consistent, with 97\% of
README files contain at least one section describing the `What` of the
repository and 89\% offering some `How` content, other categories are more
variable.  For instance, information on `Contribution`, `Why`', and `Who`,
appear in 28\%, 26\% and 53\% of the analyzed files, respectively
\citep{PranaEtAl2018}.  

The frequency of checkmarks in Table~\ref{Tbl_Guidelines} indicates the
popularity of recommending a given artifact, but it does not imply that the most
commonly recommended artifacts are the most important artifacts. Just because an
artifact is not explicitly recommended in a given guidelines, does not mean that
the artifact is not valued by the guideline authors.  They may have excluded it
because it is out of the scope of their recommendations, or outside of their
experience.  For instance, an artifact related to uninstall is only explicitly
mentioned by \citep{vanGompelEtAl2016}), but other guideline authors would
likely see its value.  They may simply feel that uninstall is implied by
install, or they may have never asked themselves whether separate uninstall
instructions are needed.

\begin{table}[!ht]
\begin{center}
\begin{tabular}{ p{2.5cm}p{1cm}p{1cm}p{1cm}p{1cm}p{1cm}p{1cm}p{1cm}p{1.2cm}p{1cm}p{0.8cm} }
\toprule
~ \ & \cite{USGS2019} & \cite{TobiasEtAl2018} & \cite{BrettEtAl2021} &
\cite{WilsonEtAl2016} & \cite{SmithAndRoscoe2018} & \cite{HerouxEtAl2008} &
\cite{ThielEtAl2020} & \cite{vanGompelEtAl2016} & \cite{OrvizEtAl2017} & MI\\
\midrule
LICENSE & \checkmark & \checkmark & \checkmark & \checkmark & \checkmark & &
\checkmark & \checkmark & \checkmark & C\\
README &  & \checkmark & \checkmark & \checkmark & \checkmark & & \checkmark &
\checkmark & \checkmark & C\\
CONTRIBUTING &  & \checkmark & \checkmark & \checkmark & \checkmark & &
\checkmark & \checkmark & \checkmark & R\\
CITATION &  &  &  & \checkmark & & & & \checkmark & \checkmark & U\\
CHANGELOG &  & \checkmark &  & \checkmark & \checkmark & & \checkmark &  &  & U\\
INSTALL &  &  &  &  & \checkmark & & \checkmark & \checkmark & \checkmark & U\\
\midrule
Uninstall &  &  &  &  & & & & \checkmark & &  \\
Dependency List &  &  & \checkmark & & \checkmark & & & \checkmark &  & R\\
Authors &  &  &  &  &  &  & \checkmark & \checkmark & \checkmark & U\\
Code of Conduct &  &  &  &  & & & \checkmark & & & R\\
Acknowledgements &  &  &  &  &  &  & \checkmark & \checkmark & \checkmark & U\\
Code Style Guide &  & \checkmark &  &  & & & \checkmark & \checkmark & \checkmark & R\\
Release Info. &  & \checkmark &  &  & & \checkmark & \checkmark & & & C\\
Prod.\ Roadmap &  &  &  &  & & \checkmark & \checkmark & \checkmark & & R\\
\midrule
Getting started &  &  &  &  & \checkmark & & \checkmark & \checkmark & \checkmark & R\\
User manual &  &  & \checkmark &  & & & \checkmark & & & C\\
Tutorials &  &  &  &  & & & \checkmark & & & U\\
FAQ &  &  &  &  & & & \checkmark & \checkmark & \checkmark & U\\
\midrule
Issue Track &  & \checkmark & \checkmark & & \checkmark & \checkmark &
\checkmark & & \checkmark & C\\
Version Control &  & \checkmark & \checkmark & \checkmark & \checkmark &
\checkmark & \checkmark & \checkmark & \checkmark & C\\ 
Build Scripts &  & \checkmark &  & \checkmark & \checkmark & \checkmark &
\checkmark & & \checkmark & U\\
\midrule
Requirements &  & \checkmark &  &  & & \checkmark &  &  & \checkmark & R\\
Design Doc.\ &  & \checkmark  & \checkmark &  & \checkmark & & \checkmark &
\checkmark& \checkmark & R\\
API Doc. &  &  &  &  & \checkmark & & \checkmark & \checkmark & \checkmark & R\\
Test Plan &  & \checkmark &  &  & & \checkmark & & & &  \\
Test Cases & \checkmark & \checkmark & \checkmark &  & \checkmark & \checkmark &
\checkmark & \checkmark & \checkmark & U\\
\bottomrule
\end{tabular}
\caption{Comparison of Recommended Artifacts in Software Development Guidelines
to Artifacts in MI Projects (C for Common, U for Uncommon and R for Rare)}
\label{Tbl_Guidelines}
\end{center}
\end{table}

Two of the items that appear in Table~\ref{artifactspresent} do not appear in
the software development guidelines shown in Table~\ref{Tbl_Guidelines}:
Troubleshooting guide and Developer's manual.  Although these two artifacts
aren't specifically named in the guidelines that we found, the information
contained within them overlaps with the recommended artifacts.  A
Troubleshooting guideline contains information that would typically be found in
a User manual.  A Developer's guide overlaps with information from the README,
INSTALL, Uninstall, Dependency List, Release Information, API documentation and
Design documentation.  In our current analysis, we have identified artifacts by
the names given by the software guidelines and MI examples.  In the future, a
more in-depth analysis would look at the knowledge fragments that are captured
in the artifacts, rather than focusing on the names of the files that collect
these fragments together.

Although the MI community shows examples of 88\% (23 of 26) of the practices we
found in research software guidelines (Table~\ref{Tbl_Guidelines}), three
recommended artifacts were not observed: i) Uninstall, ii) Test plans, and iii)
Requirements.  Uninstall is likely an omission caused by the focus on installing
software. Given the storage capacity of current hardware, developers and users
are not generally concerned with uninstall.  Moreover, as mentioned above,
uninstall is not particularly emphasized in existing recommendations.  Test
plans were not observed for MI software, but that doesn't mean they weren't
created; it means that the plans are not under version control.  Test plans
would have to at least be implicitly created, since test cases were observed
with reasonable frequency for MI software (test cases are are categorized as
uncommon).

MI software is like other research software in its neglect of requirements
documentation.  Although requirements documentation is recommended by some
\citep{TobiasEtAl2018, HerouxEtAl2008, SmithAndKoothoor2016}, in practice
research software developers often do not produce a proper requirements
specification \citep{HeatonAndCarver2015}. \citet{SandersAndKelly2008}
interviewed 16 scientists from 10 disciplines and found that none of the
scientists created requirements specifications, unless regulations in their
field mandated such a document. \citet{Nguyen-HoanEtAl2010} showed requirements
are the least commonly produced type of documentation for research software in
general. When looking at the pain points for research software developers,
\citet{WieseEtAl2019} found that software requirements and management is the
software engineering discipline that most hurts scientific developers,
accounting for 23\% of the technical problems reported by study participants.
The lack of support for requirements is likely due to the perception that
up-front requirements are impossible for research software
\citep{CarverEtAl2007, SegalAndMorris2008}, but when the instance on
``up-front'' requirements is dropped, allowing the requirements to be written
iteratively and incrementally, requirements are feasible \citep{Smith2016}.  

Table~\ref{Tbl_Guidelines} shows several artifacts that are rarely observed in
practice.  A theme among these rare artifacts is that many of them are related
to developers more than users.  For instance, the rare artifacts include
Contributing, Developer code of conduct, Code style guide, Product roadmap,
Requirements, Design documentation, API documentation and a Test plan.
\wss{additional details are available in the LBM document for other missing
parts, like code style guides.}  The rare artifacts for MI software are similar
to the rare artifacts for Lattice Boltzmann solvers \citep{Michalski2021},
except LBM software is more likely to have developer related artifacts, like
Contributing, Dependency list, and Design documentation.

To improve MI software in the future, an increased use of checklists could help.
Checklists can be used in projects to ensure that best practices are followed by
all developers.  Some examples include checklists merging branches into master
\citep{Brown2015}, checklists for saving and sharing changes to the project
\citep{WilsonEtAl2016}, checklists for new and departing team members
\citep{HerouxAndBernholdt2018}, checklists for processes related to commits and
releases \citep{HerouxEtAl2008} and checklists for overall software quality
\citep{ThielEtAl2020, SSI2022}.  For instance, for Lattice Boltzmann solver
software, ESPResSo has a checklist for managing releases \citep{Michalski2021}. 

The above discussion shows that, taken together, MI projects fall somewhat short
of recommended best practices for research software.  However, MI software is
not alone in this.  Many, if not most, research projects fall short of best
practices.  A gap exists in scientific computing development practices and
software engineering recommendations \citep{Storer2017, Kelly2007,
OwojaiyeEtAl2021_CSE}. \citet{JohansonAndHasselbring2018} observe that the
state-of-the-practice for SCS in industry and academia does not incorporate
state-of-the-art SE tools and methods.  This causes sustainability and
reliability problems \citep{FaulkEtAl2009}. Rather than benefit from capturing
and reusing previous knowledge, projects waste time and energy ``reinventing the
wheel'' \citep{deSouzaEtAl2019}.

\section{Comparison of Tool Usage Between MI and Other Research Software}
\label{Sec_CompareTools}

Software tools are used to support the development, verification, maintenance,
and evolution of software, software processes, and artifacts \citep[p.\
501]{GhezziEtAl2003}. MI software uses tools for CI/CD, user support, version
control, documentation, and project management.  To answer
\rqref{RQ_CompareToolsProjMngmnt} we summarize the tool usage in these
categories, and compare this to the usage by the research software community.

Table~\ref{tab_user_support_model} summarizes the user support models by the
number of projects using each model (projects may use more than one support
model). We do not know whether the prevalent use of GitHub issues for user
support is by design, or whether this just naturally happens as users seek
help. The common use of GitHub by MI developers is not surprising, given that
GitHub is the largest code host in the world, with over 128 million public
repositories and over 23 million users (as of roughly February 26, 2020)
\citep{Kashyap2020}.

\begin{table}[!ht]
\centering
\begin{tabular}{lc}
\toprule
\multicolumn{1}{c}{User support model} & Num.\ projects \\
\midrule
GitHub issue & 24 \\
Frequently Asked Questions (FAQ) & 12 \\
Forum & 10 \\
E-mail address & 9 \\
GitLab issue, SourceForge discussions & 2 \\
Troubleshooting & 2 \\
Contact form & 1 \\ 
\bottomrule
\end{tabular}
\caption{\label{tab_user_support_model}User support models by number of projects}
\end{table}

From Section~\ref{sec_score_maintainability}, 27 of the 29 projects used git as
the version control tool, 1 used Mercurial and 1 used Subversion.  The hosting
is on GitHub for 24 packages, SourceForge for 3 and BitBucket for 2.  Although
teams may have a process for accepting new contributions, no one discussed this
during their interviews. However, most teams (8 of 9) mentioned using GitHub and
pull requests to manage contributions from the community. The interviewees
generally gave very positive feedback on using GitHub. Some teams previously
used a different approach to version control and eventually transferred to git
and GitHub.  The past approaches included contributions from e-mail (3 teams),
contributions from forums (1 team) and e-mailing the git repository back and
forth between developers (1 team).

The common use of version control for MI software illustrates considerable
improvement from the poor adoption of version control tools that Wilson lamented
in 2006 \citep{Wilson2006}.  The proliferation of version control tools for MI
matches the increase in the broader research software community.  A little over
10 years ago \citet{Nguyen-HoanEtAl2010} estimated that version control was used
in only 50\% of research software projects, but even at that time
\citet{Nguyen-HoanEtAl2010} noted an increase from previous usage levels. A
survey in 2018 shows 81\% of developers use a version control system
\citep{AlNoamanyAndBorghi2018}. \citet{Smith2018} has similar results, showing
version control usage for alive projects in mesh generation, geographic
information systems and statistical software for psychiatry increasing from
75\%, 89\% and 17\% (respectively) to 100\%, 95\% and 100\% (respectively) over
a four-year period ending in 2018. (For completeness the same study showed a
decrease in version control usage for seismology software over the same time
period, from 41\% down to 36\%).  A recent survey by \citet{CarverEtAl2022}
shows version control use among practitioners at over 95\%, with 83/87 survey
respondents indicating that they use it. All but one of the software guides
cited in Section~\ref{Sec_CompareArtifacts} includes the advice to use version
control. (The USGS guide \citep{USGS2019} was the only set of recommendations to
not mention version control.) The high usage of version control tools in MI
software matches the trend in research software in general.

As mentioned in Section~\ref{sec_result_correctness_verifiability}, we
identified five projects using CI/CD tools (about 17\% of the assessed
projects). We found CI/CD using projects by examining the documentation and
source code of all projects. The count of CI/CD usage may actually be higher,
since traces of CI/CD usage may not always appear in a repository.  This was the
case for a study of LBM software, where interviews with developers showed that
more projects used CI/CD than was evident from repository artifacts alone
\citep{Michalski2021}.  The 17\% utilization for MI software contrasts with the
high frequency with which research software development guidelines recommend
continuous integration \citep{BrettEtAl2021, Brown2015, ThielEtAl2020,
Zadka2018, vanGompelEtAl2016}. Although there is currently little data available
on CI/CD utilization for research software, our impression is that CI/CD is not
yet common practice, despite its recommendation.  For LBM software at least the
situation is similar to MI software, with only 12.5\% of 24 LBM packages showing
evidence of CI/CD in their repositories \citep{Michalski2021}.  The survey of
\citet{CarverEtAl2022} suggests higher use of CI/CD with 54\% (54/100)
respondents indicating that they use it.

For documentation tools and methods mentioned by the interviewees, the most
popular (mentioned by about 30\% of developers) were forum discussions and
videos.  The second most popular options (mentioned by about 20\% of developers)
were GitHub, wiki pages, workshops, and social media. The least frequently
mentioned options (about 10\% of developers) included writing books, google
forms and state management.  In contrasting MI software with LBM software, the
most significant documentation tool difference is that LBM software often uses
document generation tools, like doxygen and sphinx \citep{Michalski2021}, while
MI does not appear to use these tools. 

Some interviewees mentioned the project management tools they used. Generally
speaking, the interviewees talked about two types of tools:
\begin{inparaenum}[i)]
\item trackers, including GitHub, issue trackers, bug trackers and Jira; and,
\item documentation tools, including GitHub, Wiki page, Google Doc, and
Confluence.
\end{inparaenum}
Of the specifically named tools in the above lists, GitHub was mentioned 3
times, while each of the other tools was only mentioned once.

Based on information provided by \citet{JungEtAl2022}, tool utilization for MI
software has much in common with tool utilization for ocean modelling software.
Both use tools for editing, compiling, code management, testing, building and
project management.  From the data available, ocean modelling differs from MI
software in its use of Kanban boards for project management.

\section{Comparison of Principles, Process and Methodologies to Research Software in General} \label{Sec_CompareMethodologies}

We answer research question \rqref{RQ_CompareMethodologies} by comparing the
principles, processes and methodologies used for MI software to what can be
gleaned from the literature on research software in general. In our interviews
with developers the responses about development model were vague, with only two
interviewees following a definite development model. In some cases the
interviewees felt their process was similar to an existing development model.
Three teams (about 38\%) either followed agile, or something similar to agile.
Two teams (25\%) either followed a waterfall process, or something similar.
Three teams (about 38\%) explicitly stated that their process was undefined or
self-directed.

Our observations of an informally defined process, with elements of agile
methods, matches what has been observed for research software in general.
Scientific developers naturally use an agile philosophy \citep{AckroydEtAl2008,
CarverEtAl2007, EasterbrookAndJohns2009, Segal2005, HeatonAndCarver2015}, or an
amethododical process \citep{Kelly2013}, or a knowledge acquisition driven
process \citep{Kelly2015}.  A waterfall like process can work for research
software \citep{Smith2016}, especially if the developers work iteratively and
incrementally, but externally document their work as if a rationale design
process were followed \citep{parnas1986rational}.

No interviewee introduced any strictly defined project management process. The
most common approach was following the issues, such as bugs and feature
requests. Additionally, the \textit{3D Slicer} team had weekly meetings to
discuss the goals for the project; the \textit{INVESALIUS 3} team relied on the
GitHub process for their project management; the \textit{ITK-SNAP} team had a
fixed six-month release pace; only the interviewee from the \textit{OHIF} team
mentioned that the team has a project manager; the \textit{3D Slicer} team and
\textit{BioImage Suite Web} team do nightly builds and tests. The \textit{OHIF}
developer believes that a better project management process can improve junior
developer efficiency while also improving internal and external communication.

We identified the use of unit testing in less than half of the 29 projects. On
the other hand, the interviewees believed that testing (including usability
tests with users) was the top solution to improve correctness, usability, and
reproducibility.  This level of testing matches what was observed for LBM
software \citep{Michalski2021} and is apparently greater than the level of
testing for ocean modelling software.  \citet{JungEtAl2022} reports that ocean
modellers underemphasize testing.

As the observed artifacts in Table~\ref{artifactspresent} show, none of the 29
projects emphasize documentation. None of them had theory manuals, although we
did identify a road map in the \textit{3D Slicer} project.  We did not find
requirements specifications. Table~\ref{tab_opinion_doc} summarizes
interviewees' opinions on documentation. Interviewees from each of the eight
projects thought that documentation was essential to their projects, and most of
them said that it could save their time to answer questions from users and
developers. Most of them saw the need to improve their documentation, and only
three of them thought that their documentations conveyed information clearly
enough. Nearly half of developers also believed that the lack of time prevented
them from improving documentation.

\begin{table}[!ht]
\centering
\begin{tabular}{ll}
\toprule
Opinion on documentation & Num ans. \\ 
\midrule
Documentation is vital to the project & 8 \\
Documentation of the project needs improvements & 7 \\
Referring to documentation saves time to answer questions & 6 \\
Lack of time to maintain good documentation & 4 \\
Documentation of the project conveys information clearly & 3 \\
Coding is more fun than documentation & 2 \\
Users help each other by referring to documentation & 1 \\ 
\bottomrule
\end{tabular}
\caption{Opinions on documentation by the numbers of interviewees with the
answers}
\label{tab_opinion_doc}
\end{table}

As Table~\ref{Tbl_Guidelines} suggests, an emphasis on documentation, especially
for new developers, is echoed in research software guidelines. Multiple
guidelines recommend a document explaining how to contribute to a project, often
named CONTRIBUTING. Guidelines also recommend tutorials, user guides and quick
start examples. \citet{SmithAndRoscoe2018} suggests including instructions
specifically for on-boarding new developers. For open source software in general
(not just research software), \citet{Fogel2005} recommends providing tutorial
style examples, developer guidelines, demos, and screenshots.

\section{Developer Pain Points} \label{painpoints}

Based on interviews with nine developers (described in
Section~\ref{sec_interview_methods}), we answer three research questions (first
mentioned in Section~\ref{sec_motivation}): \rqref{RQ_PainPoints}) What are the
pain points for developers working on research software projects?;
\rqref{RQ_ComparePainPoints}) How do the pain points of developers from MI
compare to the pain points for research software in general?; and
\rqref{RQ_Concerns}) For MI developers what specific best practices are taken to
address the pain points and software quality concerns? 

Our interviews identified pain points related to a lack of time and funding,
technology hurdles, improving correctness, and improving usability.  In this
section, we go through each pain point and contrast the MI experience with
observations from other domains.  We also cover potential ways to address the
pain points, as promoted by the community.  (Later, in
Section~\ref{ch_recommendations}, we propose additional pain mitigation
strategies based on our experience.)  In addition to pain points, we summarize
MI developer strategies for improving maintainability and reproducibility.
Although the interviewees did not explicitly identify these two qualities as
pain points, they were discussed as part of our interview process
\citep{SmithEtAl2021}.  The interviewee's practices for addressing pain points
and improving quality can potentially be emulated by other MI developers.
Moreover, these practices may provide examples that can be followed by other
research software domains.

\wss{CHECK  THESE, after writing all the PPs} \citet{PintoEtAl2018} lists some pain
points that did not come up in our conversations with MI developers:
Cross-platform compatibility, interruptions while coding, scope bloat, lack of
user feedback, hard to collaborate on software projects, and aloneness.
\citet{WieseEtAl2019} repeat some previous pain points and add the following:
dependency management, data handling concerns (like data quality, data
management and data privacy) \wss{Actually mentioned by MI software},
reproducibility, and software scope determination. Although MI developers did
not mention these pain points, we cannot conclude that they are not relevant for
MI software development, since we only interviewed nine developers for about
an hour each.  \wss{Add LBM pain points not covered.}

\begin{enumerate}

\item[P\refstepcounter{pnum}\thepnum \label{P_LackDevTime}:] \textbf{Lack of
Development Time:} Many interviewees thought lack of time, along with lack of
funding (discussed next), were their most significant obstacles. Other domains
of research software also experience the lack of time pain point
\citep{PintoEtAl2018, PintoEtAl2016, WieseEtAl2019}. Our study of LBM software
\citep{SmithEtAl2022} also highlighted lack of time as a significant pain point.

Potential and proven solutions suggested by the interviewees include:

\begin{itemize}
\item Shifting from development to maintenance when the team does not have
enough developers for building new features and fixing bugs at the same time;
\item Improving documentation to save time answering users' and developers'
questions;
\item Supporting third-party plugins and extensions; and,
\item Using GitHub Actions for continuous integration and continuous
development.
\end{itemize}

\item[P\refstepcounter{pnum}\thepnum \label{P_LackFunding}:] \textbf{Lack of
Funding:} Developers felt the pain of having to attract funding to develop and
maintain their software. For instance, the interviewees from \textit{3D Slicer}
and \textit{OHIF} said getting funding for software maintenance is more
challenging than finding funding for research. The interviewee from the
\textit{ITK-SNAP} team thought more funding was a way to solve the lack of time
problem, because they could hire more dedicated developers. On the other hand,
the interviewee from the \textit{Weasis} team did not feel that funding could
solve the same problem, since they would still need time to supervise the project. 

Funding challenges have also been noted by others \citep{GewaltigAndCannon2012,
Goble2014, KaterbowAndFeulner2018, SmithEtAl2022}. Researchers that devote time
to software have the additional challenge that funding agencies do not always
count software when they are judging the academic excellence of the applicant.
\citet{WieseEtAl2019} reported developer pains related to publicity, since
publishing norms have historically made it difficult to get credit for creating
software.  As studied by \citet{HowisonAndBullard2016}, research software
(specifically biology software, but the trend likely applies to other research
software domains) is infrequently cited. \citet{PintoEtAl2018} also mentions the
lack of formal reward system for research software.

An interviewee proposed an idea for increasing funding: Licensing the software
to commercial companies to integrate it into their products.
    
\item[P\refstepcounter{pnum}\thepnum \label{P_TechnologyHurdles}:]
\textbf{Technology Hurdles:} The technology hurdles mentioned by MI developers
include: hard to keep up with changes in OS and libraries, difficult to transfer
to new technologies, hard to support multiple OSes, and hard to support lower-end
computers. Developers expressed difficulty balancing between four factors:
cross-platform compatibility, convenience to development and maintenance,
performance, and security.

The pain point survey of \citet{WieseEtAl2019} also highlighted
technical-related problems like dependency management, cross-platform
compatibility, CI, hardware issues and operating system issues. From
\citep{SmithEtAl2022} technology pain points for LBM developers include setting
up parallelization and CI. 

The solutions proposed by the MI developers include the following:

\begin{itemize}
\item Adopting a web-based approach with backend servers, to better support
lower-end computers;
\item Using memory-mapped files to consume less computer memory, to better
support lower-end computers; 
\item Using computing power from the computers GPU for web applications;
%\item Increasing funding;
\item Maintaining better documentations to ease the development and maintenance
processes;
\item Improving performance via more powerful computers, which one interviewee
pointed out has already happened to reduce the balance problem.
\end{itemize}

As the above list shows, developers perceive that web-based applications will
address the technology hurdle.  Table~\ref{tab_native_vs_web} shows the teams'
choices between native application and web application. Most of the 29 teams (24
of 29, or 83\%) chose to develop native applications. For the eight teams we
interviewed, three of them were building web applications, and the
\textit{MRIcroGL} team was considering a web-based solution.

\begin{table}[!ht]
\centering
\begin{tabular}{lll}
\toprule
Software team & Native application & Web application \\ 
\midrule
3D Slicer & X & \\
INVESALIUS 3 & X & \\
dwv & & X \\
BioImage Suite Web & & X \\
ITK-SNAP & X & \\
MRIcroGL & X & \\
Weasis & X & \\
OHIF & & X \\ 
\midrule
Total number among the eight teams & 5 & 3 \\
Total number among the 29 teams & 24 & 5 \\ 
\bottomrule
\end{tabular}
\caption{Teams' choices between native application and web application}
\label{tab_native_vs_web}
\end{table}

The advantage for native applications is higher performance, while web
applications have the advantage of cross-platform compatibility and a simpler
build process.  These web advantages mirror the native disadvantages of
difficulty with cross-platform compatibility and a complex build process.  The
lower performance disadvantage of web applications can be improved with a server
backend, but in this case there are disadvantages for privacy protection and
server costs.

\item[P\refstepcounter{pnum}\thepnum \label{P_Correctness}:]
\textbf{Ensuring Correctness:} Interviewees identified multiple threats to
correctness.  The most frequently mentioned threat was complexity.  Complexity
enters the software by various means, including the large variety of data formats,
complicated data standards, differing outputs between medical imaging machines,
and the addition of (non-viewing related) functionality.  Other threats to
correctness identified include the following:

\begin{itemize}
\item Lack of real world image data for testing, in part because of patient
privacy concerns;
\item Tests are expensive and time-consuming because of the need for huge datasets;
\item Software releases are difficult to manage;
\item No systematic unit testing; and,
\item No dedicated quality assurance team.
\end{itemize}

As implied by the above threats to correctness, testing was the most often
mentioned strategy for MI developers for ensuring correctness.  Seven teams
mentioned test related activities, including test-driven development, component
tests, integration tests, smoke tests, regression tests, self tests and
automated tests.  With the common emphasis on testing to improve correctness, MI
software is ahead of some other scientific domains.  For scientific software in
general \citet{PintoEtAl2018} mention the problem of insufficient testing and
\citet{HannayEtAl2009} show that more developers think testing is important than
the number that believe they have a sufficient understanding of testing
concepts.  Our study of LBM software suggests that this domain shares the
challenges of insufficient testing and insufficient understanding of testing
concepts \citep{SmithEtAl2022}. Automated testing is a specific challenge for
LBM software since free testing services do not offer adequate facilities for
large amounts of data \citep{SmithEtAl2022}. Although not specifically mentioned
during our interviews, the large data sets for MI likely cause a challenge for
using free testing services, like GitHub Actions.

Research software in general often struggles with the oracle problem for testing
because for many potential test cases the developer doesn't have a means to
judge the correctness of their calculated solutions \citep{HannayEtAl2009,
KanewalaAndBieman2013, KellyEtAl2011, WieseEtAl2019}.  The MI developers did not
allude to this challenge, likely because for a give image (test case) it is
possible to determine, possibly using other software, the expected analysis
results.

A frequently cited strategy for building confidence in correctness (mentioned by
3 interviewees) is a two state development process with stable releases and
nightly builds.  Other strategies for ensuring correctness that came up during
the interviews include CI/CD, using de-identified copies of medical images for
debugging, sending beta versions to medical workers who can access the data to
do the tests, and collecting/maintaining a dataset of problematic images.  Some
additional strategies used by MI developers include:

\begin{itemize}
\item Using open datasets, like the datasets for the liver \citep{BilicEtAl2019}
and brain \citep{MenzeEtAl2015} tumor segmentation benchmarks.
\item If (part of) the team belongs to a medical school or a hospital, using the
datasets they can access;
\item If the team has access to MRI scanners, self-building sample images for
testing;
\item If the team has connections with MI equipment manufacturers, asking for
their help on data format problems;
\end{itemize}

The feedback from the interviewees makes it clear that increased connections
between the development team and medical professionals/institutions could ease
the pain of ensuring correctness via testing.

\item[P\refstepcounter{pnum}\thepnum \label{P_Usability}:]
\textbf{Usability:}  

The discussion with the developers focused on usability issues for two classes
of users: the end users and other developers.  The threats to usability for end
users include an unintuitive user interface, inadequate feedback from the
interface (such as lack of a progress bar), users being unable to determine the purpose of
the software, not all users knowing if the software includes certain features, not
all users understanding how to use the command line tool, and not all users
understanding that the software is a web application. For developers the threats to
usability include not being able to find clear instructions on how to deploy the
software, and the architecture being difficult for new developers to understand.

At least to some extent the problems for MI software users are due to holes in
their background knowledge.  The survey of \citet{WieseEtAl2019} for research
software in general also mentioned that users do not always have the expertise
required to install or use the software. \citet{SmithEtAl2022} observes a
similar pattern for LBM software, with several LBM developers noting that users
sometimes try to use incorrect method combinations. Furthermore, some LBM users
think that the packages will work out of the box to solve their cases, while in
reality CFD knowledge needs to be applied to correctly modify the packages for a
new endeavour.

To improve the usability of MI software, the most common strategies mentioned by
developers are as follows:

\begin{itemize}[i)]
    \item use documentation (user manuals, mailing lists, forums) (mentioned by
    4 developers)
    \item usability tests and interviews with end users; and, (mentioned by 3
    developers)
    \item adjusting the software according to user feedback. (mentioned by 3
    developers)
\end{itemize}

Other suggested and practiced strategies include a graphical user interface,
testing every release with active users, making simple things simple and
complicated things possible, focusing on limited number of functions, icons with
clear visual expressions, designing the software to be intuitive, having a UX
designer, dialog windows for important notifications, providing an example for
users to follow, downsampling images to consume less memory, and an option to
load only part of the data to boost performance.  The last two points recognize
that an important component of usability is performance, since poor performance
frustrates users.

\end{enumerate}

Up to this point, we have covered the pain points that came up in interviews
with MI developers, along with a summary of the techniques that are currently
used to address these pain points.  Although the developers did not explicitly
identify the qualities of maintainability and reproducibility as pain points in
our interviews, as part of our interview questions
(Section~\ref{sec_interview_methods}) they did share their approaches for
improving these qualities, as discussed below.

\begin{enumerate}
\item[Q\refstepcounter{qnum}\theqnum \label{Q_Maintainability}:]
\textbf{Maintainability:} Although not explicitly highlighted as a pain point
during our interviews, \citet{Nguyen-HoanEtAl2010} rate maintainability as the
third most important software quality for research software in general. The push
for sustainable software \citep{deSouzaEtAl2019} is motivated by the pain that
past developers have had with accumulating too much technical debt
\citep{KruchtenEtAl2012}.  For LBM software, \citet{SmithEtAl2022} identifies
technical debt as one of the developer pain points.

To improve maintainability, the most popular (with five out of nine interviewees
mentioning it) strategy is to use a modular approach, with often repeated
functions in a library.  Other strategies that were mentioned for improving
maintainability include supporting third-party extensions, an easy-to-understand
architecture, a dedicated architect, starting from simple solutions, and
documentation.  The \textit{3D Slicer} team used a well-defined structure for
the software, which they named as an ``event-driven MVC pattern''. Moreover,
\textit{3D Slicer} discovers and loads necessary modules at runtime, according
to the configuration and installed extensions. The \textit{BioImage Suite Web}
team had designed and re-designed their software multiple times in the last 10+
years. They found that their modular approach effectively supports
maintainability \citep{Joshi2011}. 

\item[Q\refstepcounter{qnum}\theqnum \label{Q_Reproducibility}:]
\textbf{Reproducibility:}  Although the MI developers did not mention
reproducibility explicitly as a pain point, they did mention the need to improve
documentation.  Good documentation doesn't just address the pain points of lack
of developer time (\ppref{P_LackDevTime}), technology hurdles
(\ppref{P_TechnologyHurdles}), usability \ppref{P_Usability}, and
Maintainability.  Documentation is also necessary for reproducibility. The
challenges of inadequate documentation are a known problem for research software
\citep{PintoEtAl2018, WieseEtAl2019} and for non-research software
\citep{LethbridgeEtAl2003}. 

In our interviews, we discussed threats to reproducibility and strategies for
improving it.  The threats that were mentioned include closed-source software,
no user interaction tests, no unit tests, the need to change versions of some
common libraries, variability between CPUs, and misinterpretation of how
manufacturers create medical images. 

The most commonly cited (by 6 teams) strategy to improve reproducibility was
testing (regression tests, unit tests, having good tests). The second most
common strategy (mentioned by 5 teams) is making code, data, and documentation
available, possibly by creating open-source libraries.  Other ideas that were
mentioned include running the same tests on all platforms, a dockerized version
of the software to insulate it from the OS environment, using standard
libraries, monitoring the upgrades of the library dependencies, clearly
documenting the version information, bringing along the exact versions of all
the dependencies with the software, providing checksums of the data, and
benchmarking the software against other software that overlaps in functionality.
Specifically one interviewee suggested using \textit{3D Slicer} as the benchmark
to test their reproducibility.

\end{enumerate}

\section{Recommendations} \label{ch_recommendations}

- add in ACs, borrow from previous paper.
- mention the solutions proposed by LBM developers, like design for change

- not repeating best practices mentioned previously, like CI/CD, version
control, testing, etc.

This section presents our recommendations on MI software development. Although
our focus is on MI software, unless noted otherwise, our recommendations apply
to any scientific computing software.

In this section we provide recommendations to address the pain points from
Section~\ref{painpoints} to answer~\rqref{RQ_Recommend}.  Our recommendations
are not lists of criticisms for what should have been done in the past, or what
should be done now; they are suggestions for consideration in the future. We
will not be repeating previously discussed ideas here, such as CI, documentation
of APIs, etc.  Our aim is to mention ideas that are at least somewhat beyond
conventional best practices. The ideas listed here have the potential to become
best practices in the medium to long-term. The ideas in the following
subsections are roughly in the order of increasing implementation effort.

\subsection{Continuous Integration} \label{Sec_ContinuousIntegration}

Continuous integration is the process of building and testing software on every
push to the code repository, with pushes done very frequently \citep[p.\ 13]
{HumbleAndFarley2010}, \citep{ShahinEtAl2017, Fowler2006}.  CI consists of the
following elements:
\begin{itemize}
	\item A version control system \citep{Fowler2006}. To be effective, all
	files should be under version control, not just code files.  Anything that
	is needed to build, install and run the software should be under version
	control, including configuration files, build scripts, test harnesses, and
	operating system configurations \citep[p. 19]{HumbleAndFarley2010}.
	Fortunately for the LBM domain, as discussed above, most alive projects
	(73\% of those assessed) use version control.
	\item A fully automated build system \citep{Fowler2006}.  As \citet[p.\
	5]{HumbleAndFarley2010} point out, deploying software manually is an
	anti-pattern.  Fortunately, as discussed above, most LBM projects use build
	automation.
	\item An automated test system \citep{Fowler2006}. Building quality software
	involves creating automated tests at the unit, component, and acceptance
	test level, and executing these tests whenever someone makes a change to the
	code, its configuration, the environment, or the software stack that it runs
	on \citep[p.\ 83]{HumbleAndFarley2010}. As Table~\ref{artifactspresent}
	shows, test cases are in the uncommon category for LBM software artifacts,
	which means that some LBM projects will need to increase their testing
	automation if they wish to pursue CI.  The usual advice for CI is to keep
	the build and test process short \citep[p.\ 60]{HumbleAndFarley2010}. Given
	that LBM can be computationally expensive, the tests run with every check-in
	may need to focus on simple code interface tests, saving large tests for
	less frequent runs.  A more sophisticated option to address the bottleneck
	for merges is CIVET (Continuous Integration, Verification, Enhancement, and
	Testing), which solves this problem by intelligently pinning, cancelling,
	and if necessary, restarting jobs as merges occur \citep{SlaughterEtAl2021}.
	\item An automated system for other tasks, such as code checking,
	documentation building and web-site updating.  These other tasks are not
	essential to CI, but they can be incorporated to improve the quality
	of the code and the communication between developers and users. For
	instance, a static analysis (possibly via linters) of the code may find poor
	programming practice or lack of adherence to adopted coding standards.
	\citet{SlaughterEtAl2021} provides another example of an automated task ---
	checking that a test specification includes the test's motivation, a test
	description, and a design description for all changes. 
	\item An integration build system to pull everything together.  Every time
	there is a check-in (for instance a pull request), the integration server
	automatically checks out the sources onto the integration machine, starts a
	build, runs tests, and informs the committer of the results. 
\end{itemize}

Although CI can take some time and effort to set up and integrate into a team's
workflow, the benefits can be significant, as follows:

\begin{itemize}
	\item Elimination of headaches associated with a separate integration phase
	\citep{Fowler2006}, \citep[p.\ 20]{HumbleAndFarley2010}. If integration of
	the work of different developers, or even separate chunks of work by the
	same developer, is postponed, integration problems are inevitable.  By
	continuously integrating, problems are immediately obvious and the source of
	the problem can be isolated to the small increment that was just committed.
	\item Bugs can be quickly detected and removed \citep{Fowler2006} via
	automated testing.  To improve productivity, defects are best discovered and
	fixed at the point where they are introduced \citep[p.\
	23]{HumbleAndFarley2010}.  Code is not the only source of errors; they are
	also found in the files and scripts related to configuration management
	\citep[p.\ 18]{HumbleAndFarley2010}.
	\item Developers are always working on a stable base, since the master
	branch will always be working.  A stable base passes all tests and, if the
	CI system uses generators and linters, it will also have current
	documentation and coding standard compliant code.  A stable base improves
	developer productivity, allowing them to focus on coding, testing, and
	documentation.
\end{itemize}

Although there are still challenges, setting up a CI system has never been
easier than it is today.  The option of installing a dedicated CI server (either
physically or virtually) exists with tools such as
\href{https://www.jenkins.io/} {Jenkins}, \href{http://buildbot.net/}
{Buildbot}, \href{https://www.gocd.org/} {Go}, and
\href{http://integrity.github.io/} {Integrity}. However, installation on your
own server is unnecessary since there are many hosted CI tool solutions
available, such as: \href{https://travis-ci.org/} {Travis CI},
\href{https://github.com/features/actions} {GitHub Actions} and
\href{https://circleci.com/} {CircleCI}.  Getting started with a hosted CI is
straightforward.  All that is required to begin is editing a few lines of a YAML
configuration file in the project's root directory.

Challenges that exist for adopting CI include lack of awareness and
transparency, coordination and collaboration challenges, lack of expertise and
skills, more pressure and workload for team members, general resistance to
change, scepticism and distrust on continuous practices \citep{ShahinEtAl2017}.
The most common reason given for not adopting CI is that developers on my
project are not familiar enough with CI \citep{HiltonEtAl2016}.  These problems
can be mitigated via planning and documentation, promoting a team mindset,
adopting new rules and policies, improving testing activities, and decomposing
development into smaller units \citep{ShahinEtAl2017}.

\subsection{Employ Linters} \label{Sec_Linters}

Except for \citet{ThielEtAl2020}, the research software guidelines that we
consulted do not mention linters.  However, we believe that linters have the
potential to improve code quality at a relatively low cost.  A linter is a tool
that statically analyzes code to find programming errors, stylistic
inconsistencies and suspicious constructs \citep{Wikipedia2022_Lint}. Linters
can be used to spot check code files, or even better as part of a continuous
integration system, as discussed in Section~\ref{Sec_ContinuousIntegration}.  

A sample of the potential benefits of linters include: finding memory leaks,
finding potential bugs, standardizing code with respect to formatting, improving
performance, removing silly errors before code reviews, and catching potential
security issues \citep{SourceLevel2022_Lint}.  Linters are available for most
popular programming languages.  For instance Python has the options of PyLint,
flake8 and Black \citep{Zadka2018}.

Linters can address the pain points for LBM developers.  For instance, a linter
can decrease the amount of development time (\ppref{P_LackDevTime}) by
decreasing the number of mundane mistakes programmers have to catch.  Since the
linter can include rules that capture the wisdom of senior programmers, it can
help newer developers avoid common mistakes. Although there are not many linters
for checking parallel programming mistakes (\ppref{P_TechnologyHurdles}),
Parallel Lint is an option for OpenMP. Consistently using a linter can guard
against some technical debt (\ppref{Q_Maintainability}). Although linters are
tools for code analysis, similar ideas can be applied to ``lint'' documentation
to ensure adherence to basic rules.  The use of tools to check documentation is
a partial explanation for the relatively higher quality of statistical tools
that are part of the Comprehensive R Archive Network (CRAN)
\citep{SmithEtAl2018_StatSoft}.

\subsection{Conduct Rigorous Peer Reviews} \label{Sec_PeerReview}

Peer review is a strategy already adopted by some LBM developers to improve
confidence in their software, but the benefits of peer review can be improved by
making its application more rigorous. \citet{Jones2008} shows that rigorous
inspection finds 60-65\% of latent defects on average, and often tops 85\% in
defect removal efficiency.  By way of comparison, most forms of testing average
between 30 and 35\% for defect removal efficiency \citep{EbertAndJones2009,
Jones2008}.  A formal code inspection involves asking the reviewer to either
follow a review checklist (check consistency of variable names, look for
terminating loops, etc.), or perform specific review tasks (like summarize the
purpose of the code, create a data dictionary for a given module,
cross-reference the code to the technical manual, etc.) The task based
inspection approach has been effectively used for research software, as
described by \citet{KellyAndShepard2000}. Task based inspection is an ideal fit
with an issue tracking system, like GitHub.  The review tasks can be issues, so
that they can be easily assigned, monitored and recorded.  Other potential
issues for the tracker include assigning junior team members to test
installation instructions and getting-started tutorials.

Rigorous peer review addresses the same pain points as linters
(Section~\ref{Sec_Linters}): \ppref{P_LackDevTime}, \ppref{P_TechnologyHurdles},
and \ppref{Q_Maintainability}. Peer review addresses these points by efficiently
searching for defects and problems.  Finding misunderstandings in of how the
code implements the required theory improves the software's correctness
(\ppref{P_Correctness}).

Peer review --  One of the developers (ESPResSo) that was interviewed noted that
an ad hoc peer review process is used to assess major changes and additions.
Using peer review (also called technical review) matches with recommended
practice for research software \citep{HerouxEtAl2008, Givler2020, OrvizEtAl2017,
USGS2019}.

\subsection{Move To Web Applications} \label{sec_recommendations_tech_stack}

A tech stack refers to a set of technologies used by a team to build software
and manage the project. Section~\ref{painpoints} lists the advantages and
disadvantages between native and web applications. We give further suggestions
on the choice of a tech stack to improve the four priority factors identified by
developers: compatibility, maintainability, performance, and security.  The
suggestions are intended to provide ideas and avenues for exploration; not all
of the suggestions will be the right fit for all projects and all teams.

\begin{itemize}
\item \textbf{Identify the priorities between the factors.} Simultaneously
achieving high levels for all four factors (compatibility, maintainability,
performance and security) is difficult. A team needs to prioritize its
objectives according to its resource and experience.

\item \textbf{Be open-minded about new technologies.} Web applications with only
a frontend are known for worse performance than native applications. However,
new technologies may ease this difference. For example, some JavaScript
libraries can help the frontend harness the power of the computer's GPU and
accelerate graphical computing. In addition, there are new frameworks helping
developers with cross-platform compatibility. For example, the
\href{https://flutter.dev/}{Flutter} project enables support for web, mobile,
and desktop OS with one codebase.

\item \textbf{Web applications can also deliver high performance.} Web
applications with backend servers may perform even better than native
applications. If a team needs to support lower-end computers, it is good to use
back-end servers for heavy computing tasks.

\item \textbf{Backend servers can have low costs.} Serverless solutions from
major cloud service providers may be worth exploring. Serverless still uses a
server, but the team is only charged when they use it. The solution is
event-driven, and costs the team by the number of requests it processes. Thus,
serverless can be very cost-effective for less intensively used functions.

\item \textbf{Web transmission may diminish security.} Transferring sensitive
data on-line can be a problem for projects requiring high security. Regulations
for some MI applications may forbid doing web transmissions. In this case, a web
application with a backend may not be an option.

\end{itemize}

\subsection{Recommendations on Enriching the Testing Datasets} 
\label{sec_recommendations_testing_dataset}

As described in Section~\ref{painpoints}, it is difficult for software
development teams to access real-world medical imaging datasets. This problem
restricts their capability and flexibility for testing. We provide some
suggestions as follows:

\begin{itemize}
\item \textbf{Build and maintain good connections to datasets.} A team can build
connections with professionals working in the medical domain, who may have
access to private datasets and can perform tests for the team. If a team has
such professionals as internal members, the process can be simplified.

\item \textbf{Collect and maintain datasets over time.} A team may face problems
caused by various unique inputs over the years of software development. This
data should be collected and maintained over time to form a good, comprehensive,
dataset for testing.

\item \textbf{Search for open data sources.} In general, there are many open MI
datasets.  For instance, there are
\href{https://nihcc.app.box.com/v/ChestXray-NIHCC}{Chest X-ray Datasets} by
National Institute of Health \citep{WangEtAl2017},
\href{https://www.cancerimagingarchive.net/}{Cancer Imaging Archive}
\citep{PriorEtAl2017}, and \href{https://medpix.nlm.nih.gov/home}{MedPix} by
National Library of Medicine \citep{Smirniotopoulos2014}. A team developing MI
software should be able to find more open datasets according to their needs.

\item \textbf{Create sample data for testing.} If a team can access tools
creating sample data, they may also self-build datasets for testing. For
example, an MI software development team can use an MRI scanner to create images
of objects, animals, and volunteers. The team can build the images based on
specific testing requirements.

\item \textbf{Remove privacy from sensitive data.} For data with sensitive
information, a team can ask the data owner to remove such information or add
noise to protect privacy. One example is using de-identified copies of medical
images for testing.

\item \textbf{Establish community collaboration in the domain.} During our
interviews with developers in the MI domain, we heard many stories of asking for
supports from other professionals or equipment manufacturers. However, we
believe that broader collaboration between development teams can address this
problem better. Some datasets are too sensitive to share, but if the community
has some kind of ``group discussion'', teams can better express their needs, and
professionals can better offer voluntary support for testing. Ultimately, the
community can establish a nonprofit organization as a third-party, which
maintains large datasets, tests OSS in the domain, and protects privacy. 

\end{itemize}

\subsection{Follow Advice From the Open-Source Community to Grow the Number of Contributors} \label{Sec_GrowContributors}

In our interviews with developers we heard they would like to have additional
software contributors.  More developers would help with addressing the lack of
development time pain point (\ppref{P_LackDevTime}).  In investigating advice on
increasing the number of contributors, we found that the advice usually starts
with following best practices, including providing artifacts like those
discussed in Section~\ref{Sec_CompareArtifacts}, such as a contributor
guidelines, code style guidelines, a clear code of conduct and high quality code
and documentation. This advice is logical, but in our assessment of LBM software
we found examples of projects that already follow many best practices, but still
have small development teams. Attracting developers apparently requires more
than just building a good product.

Some potential additional ideas for growing the number of contributors are
as follows:

\begin{itemize}
	\item Clearly identify issues that are an appropriate starting points for
	new developers \citep{Garcia2016, Jalan2016, Proffitt2017}.
	\item Given that in most projects developers were once users
	\citep{McQuaid2018}, recruit future developers from the current set of
	users.
	\item Create templates for pull requests and issues \citep{Jalan2016}.
    \item Welcome all kinds of contributions, not just code.  Non-code
    contributions include documentation, fixing typos, issue reporting and test
    cases \citep{Jalan2016, Proffitt2017}.
    \item Reward and recognize new contributors, via small rewards like stickers
    or shirts, or even just a simple shoutout or mention in a blog post or on
    social media \citep{Jalan2016, Proffitt2017}.
    \item Recognize that open source is more about people than it will ever be
    about code \citep{Jalan2016}.
    \item Look beyond just recruiting online and seek new developers at
    conferences or other user meet ups \citep{Garcia2016}.
    \item Follow the advice of \citet{Kuchner2012} to adapt ideas from marketing
    to promote science.  Specific ideas that could be applied to open source
    projects include the following: 
    \begin{itemize}
		\item Storytelling to motivate interest in the project, where the story
		is a sequence of events and pauses for reflection \citep[p.\
		21--22]{Kuchner2012}.
		\item Invest effort in building relationships with users and with potential future developers.
		\item Brand your project with what makes it unique and special.
	\end{itemize}
	\item Consider combining existing projects to pool collective resources, and
	reduce competition between different open source projects.
\end{itemize}

\subsection{Design For Change} \label{Sec_DesForChange}

To address technical debt (\ppref{Q_Maintainability}), the top LBM developers show how
modularization can be used to design research software for future change.
Although the advice to modularize research software to handle complexity is
common \citep{WilsonEtAl2014, StewartEtAl2017, Storer2017}, specific guidelines
on how to divide the software into modules is less prevalent.  Not every
decomposition is a good design for supporting change, as shown by
\citet{Parnas1972a}.  A design with low cohesion and high coupling \citep[p.\
48]{GhezziEtAl2003} will make change difficult. Especially in research software,
where change is inevitable, designers need to produce a modularization that
supports change. Ocean modelling software is currently feeling the pain of
not emphasizing modularization in legacy code \citep{JungEtAl2022}.

Interviewed LBM developers highlighted cases where their modularizations
anticipated future changes.  The developer of pyLBM mentioned that the
geometries and models of their system had been ``decoupled'', using abstraction
and modularization of the source code, to make it ``very easy to add [new]
features''.  The pyLBM design allows for independent changes to the geometry and
the model.  We also learned that the package pyLBM redeveloped data structures
to ease future changes. The developer of TCLB noted that their design allows for
the addition of some LBM features, but changes to major aspects of the system
would be difficult. For example, ``implementing a new model will be an easy
contribution'', but changes to the ``Cartesian mesh … will be a nightmare''. The
design of TCLB highlights that not every conceivable change needs to be
supported, only the likely changes.  

As the LBM developers illustrate, they accomplish design for change by first
identifying likely changes, either implicitly or explicitly, and second by
hiding each likely change behind a well-defined module interface.  Although it
is unclear whether the developers are aware of Parnas's work, the approach
mirrors his recommendations \citep{Parnas1972a}. \citet{SmithEtAl2022} (Section
Documentation as Part of the Development Process) lists ideas for how to
document the design, including the likely changes, so that they are more visible
to others.

\subsection{Assurance Case} \label{AssuranceCases}

From \cite{RinehartEtAl2015}, an assurance case is ``[a] reasoned and compelling
argument, supported by a body of evidence, that a system, service or
organization will operate as intended for a defined application in a defined
environment.''  Assurance cases have been successfully employed for safety
critical systems, but using this technique for SC software is a new idea.  The
Goal Structuring Notation (GSN) \cite{Spriggs2012} seems like a good framework
for initial investigation.

Scientific software, such as medical software, is often subject to
standardization and regulatory approval. While applying such approvals and
standards has had a beneficial effect on system quality, it does not provide
good tracking of the development stages, as the compliance with the standards
is mostly checked after the system development. Once a system is implemented,
its documentations must be approved by the regulators. This process is lengthy
and expensive. In contrast, assurance case development usually occurs in
parallel with the system construction, resulting in a traceable, detailed
argument for the desired property.  Moreover, since the assurance case is
integrated into the system construction, the domain experts are involved from
the start.

Putting the argument in the hands of the experts means that they will work to
convince themselves, along with the regulators.  They will use the expertise
that the regulators do not have and they will see the value of documentation (as
mentioned in section on pain points with methods to mitigate); they will be
engaged.  This engagement will hopefully help bridge the current chasm between
software engineering and scientific computing~\cite{Kelly2007}, by motivating
scientists toward documentation and correcting the problem of software engineers
failing to meet scientists' expectations~\cite{Segal2008}.  Significant
documentation will still likely be necessary, but now the developers control the
documentation.  What is created will be relevant and necessary.

For SCS a top level claim could read ``Program X delivers correct outputs when
used for its intended use/purpose in its intended environment.''  The next step
would be to decompose this claim into sub-claims that will be easier to prove.
The sub-claims will likely also be further divided until the bottom of the
graph, where the measurable evidence is provided.  Typical evidence will consist
of documents, expert reviews, test case results etc.

Using the specific example of preparing an assurance case for pre-existing
medical image analysis software (3dfim+)~\cite{SmithEtAl2018_ICSEPoster}, the
top level can be decomposed into four sub-goals, as shown in
Figure~\ref{TopGoal}.  This example follows the same pattern as used for medical
devices~\cite{Wassyng2015}.  The first sub-goal (GR) argues for the quality of
the documentation of the requirements (SRS).  The second sub-goal (GD) says that
the design complies with the requirements and the third proposes that the
implementation also complies with the requirements.  The fourth sub-goal (GA)
claims that the inputs to 3dfim+ will satisfy the operational assumptions, since
we need valid input to make an argument for the correctness of the output.

\begin{figure*}[!h]
\centering
\includegraphics[width=1.0\textwidth]{./figures/TopGoal.pdf}
\caption{Top Goal of the assurance case and its sub-goals}
\label{TopGoal}
\end{figure*}

Preparing an assurance case for the pre-existing 3dfim+ software justifies the
value of assurance cases for the certification of SCS~\cite{SmithEtAl2018}.
Although no errors were found in the output of the existing software, the rigour
of the proposed approach did lead to finding ambiguities and omissions in the
existing documentation, such as missing information on the coordinate system
convention.  In addition, a potential concern for the software itself was
identified from the GA argument: running the software does not produce any
warning about the obligation of the user to provide data that matches the
parametric statistical model employed for the correlation calculations.

Potential objectives for research on assurance cases applied to SCS are as
follows:

\begin{enumerate}
\item Creation of an assurance case template will require looking at a variety
  of SCS projects, which will require engaged project partners; therefore, the
  initial approach should be to target domains where certification is required,
  or domains where verification is critical, but challenging, like the High
  Energy Physics (HEP). \wss{reference the paper about domain specific
  methodology - maybe I already did?}
\item Evidence is provided at the bottom of an assurance case, but what is
  the best form for this evidence?  As mentioned for the SRS, we need an
  approach for verification.  We also need an approach to verify the other
  software artifacts, the traceability between them, the team of developers, the
  team of reviewers, etc.  The question for the future is what combination of
  tools and processes will provide convincing evidence?
\item Creation of assurance cases is challenging with existing tools.  In the
  future, the goal should be automatic generation (see
  \citet{Smith2018}, Section SecAutoGenerate) of the visual assurance case documentation from
  the arguments and evidence
\item Creating a revelation on the value of documentation \citep{Smith2018} may
   require a pseudo adversarial approach.  SCS developers should be explicitly
   challenged to present their verification efforts so that their work can be
   independently verified.  Although SCS developers have developed many
   successful theories, techniques, and testing procedures for verification, the
   evidence is generally presented in an ad hoc manner.  Developers should be
   asked to connect all the pieces of their evidence in a coherent argument.  As
   a bonus, the act of creating the assurance case may also lead developers to
   discover subtle edge cases, which would not have been noticed with a less
   rigorous and systematic approach.

\end{enumerate}

\subsection{Generate All Things} \label{Sec_GenAllThings}

We propose automatically generating LBM code and its documentation, using a
Domain Specific Modelling (DSM) approach. DSM means creating a knowledge base of
models for physics, computing, mathematics, documentation, and certification and
then writing explicit ``recipes'' that weave together this knowledge to generate
the desired code, documentation, test cases, inspection reports and build
scripts. This definition of DSM is more general than usual; in this
recommendation DSM implies generation of all software artifacts, not just the
code. DSM moves development to a higher level of abstraction; domain experts can
work without concern for low-level implementation details. Using DSM, we
optimally generate code and documentation, eliminate redundancy, reduce the
likelihood of errors, and automate maintenance. Moreover, a generative approach
facilitates the inevitable exploration of LBM modelling assumptions.

DSM provides a transformative technology for documentation, design, and
verification \citep{JohansonAndHasselbring2018, Smith2018}. DSM allows
scientists to focus on their science, not software.  A generative approach
removes the maintenance nightmare of documentation duplicates and near
duplicates \citep{LucivEtAl2018}, since developers capture knowledge once and
transform it as needed.  Code generation has previously been applied to improve
research software.  For instance, ATLAS (Automatically Tuned Linear Algebra
Software) \citep{WhaleyEtAl2001} and Blitz++ \citep{Veldhuizen1998} produces
efficient and portable linear algebra software.  Spiral \citep{Pueschel2001}
uses software/hardware generation for digital signal processing.
\citet{Carette2008} shows how to generate a family of efficient, type-safe
Gaussian elimination algorithms.  FEniCS (Finite Element and Computational
Software) uses code generation when solving differential equations
\citep{LoggEtAl2012}. \citet{OberEtAl2018} and \citet{MatkerimEtAl2013} apply
DSM to High Performance Computing (HPC), using UML (Unified Modelling Language)
for their domain models. Unlike previous DSM work, the current recommendation
focuses on generating all software artifacts (requirements, design, etc.), not
just code.  \citet{SzymczakEtAl2016} presents initial work on this ``generate
all things'' approach, \citet{SmithAndCarette2021-BRIC} presents a motivating
example, and \citep{CaretteEtAl2021-Drasil} provides a prototype.

A DSM approach addresses multiple pain points.  For instance, once the
infrastructure is in place, a DSM can decrease the amount of development time
(\ppref{P_LackDevTime}) by automation.  Since a DSM approach allows scientists
to focus on their science, rather than software, the lack of software
development experience will be less of an issue. The DSM approach can capture
computing knowledge to mitigate the technology related pain points
(\ppref{P_TechnologyHurdles}).  For ensuring correctness (\ppref{P_Correctness})
a generative approach can aim to be correct by construction.  If there are
mistakes, DSM has the advantage that they are propagated throughout the
generated artifacts, which greatly increases the chance that someone will notice
the mistake.  Technical debt (\ppref{Q_Maintainability}) is not a concern with a
DSM approach since developers write the recipes used for generation at a high
level making them relatively easy to change.  With respect to pain points on
usability (\ppref{P_Usability}), the generative approach we are proposing
addresses these directly, since documentation is a primary concern, rather than
an afterthought.

\section{Threats to Validity} \label{sec_threats_to_validity}

Below we categorize and list the threats to validity that we have identified.
The categories come from \citet{AmpatzoglouEtAl2019} and \citet{ZhouEtAl2016}.

\wss{observed artifacts (Section~\ref{Sec_CompareArtifacts}) - human judgement,
not always given the expected name, but the content is there.}

\begin{description}
    \item[Construct Validity:] ``Defines how effectively a test or experiment measures
    up to its claims. This aspect deals with whether or not the researcher measures
    what is intended to be measured'' \citep{AmpatzoglouEtAl2019}.
\end{description}

\begin{itemize}
\item For practical considerations the time spent measuring each package had to
be limited.  The time limit may have caused an assessment to have missed
something something relevant.
\item Our ranking is partly based on surface (shallow) measurement, which may
not fully reveal the underlying qualities.
\item The questions in the measurement template may not actually measure the
qualities they are associated with.  For instance, we have assumed that
maintainability is improved if a high percentage of identified issues are
closed, but it is possible for a project with a wealth of ideas to have many
open issues, and still be maintainable.
\item With the exception of the interview data for 8 projects, we collected all
of the information for each project from the artifacts available on the
Internet. In some cases this source of data may mean we did not find evidence of
something, like unit testing, not because the project didn't do it, but because
no artifacts of this activity remained in the publicly available repository.
\item As mentioned in Section~\ref{painpoints}, one interviewee was too busy
to participate in a full interview, so they provided a version of written answers
to us. Since we did not have the chance to explain our questions or ask them
follow-up questions, there is a possibility of misinterpretation of the
questions or answers.
\item As mentioned in Section~\ref{sec_result_installability}, we could not
install or build \textit{dwv}, \textit{GATE}, and \textit{DICOM Viewer}. We used
a deployed online version for \textit{dwv}, a VM version for \textit{GATE}, but
no alternative for \textit{DICOM Viewer}. We might underestimate their rank due
to an uncommon technical issue.
\end{itemize}

\begin{description}
    \item[Internal Validity:] ``This aspect relates to the examination of causal
    relations. Internal validity examines whether an experimental
    treatment/condition makes a difference or not, and whether there is evidence
    to support the claim'' \citep{AmpatzoglouEtAl2019}.
\end{description}

\begin{itemize}
    \item Many of the measurement template questions look for the presence of
    certain artifacts, like a user manual, or a getting started tutorial.  We
    have implicitly assumed that the presence of these artifacts is an
    indication of a given quality, but this is an indirect measure.  It may be
    possible to achieve qualities without the artifacts we sought.  A direct
    measure of quality, like an usability experiment, would not have this
    problem.
    \item We compared our rankings to the rankings by the community and the
    Domain Expert, but we have assumed all qualities are equally weighted.  The
    community and the Domain Expert likely have a more complex weighting between
    qualities.
    \item We assumed there was a casual relationship between the number of files
    and reusability, since we assumed multiple files implies modularity.  Of
    course the code can be modular, and not divided into separate files.
    Moreover, having many files does not immediately in itself imply a
    decomposition based on sound design principles, like information hiding.
\end{itemize}

\begin{description}
    \item[External Validity:] ``Define the domain to which a study's findings
    can be generalized'' \citep{ZhouEtAl2016}.
\end{description}

\begin{itemize}
\item We interviewed eight teams, which is a good proportion of the 29. However,
there is still a risk that the subset of the entire group does not represent the
whole MI software community.
\item We identified qualities that we believe the community will be interested
in, and we weighted these qualities equally, but the qualities we chose, and
their weighting, may not match the external reality.
\item The number of GitHub stars, watches, and forks are not ideal measures of
popularity.  \wss{Can we find a reference for this?}
\end{itemize}

\begin{description}
    \item[Conclusion Validity:] ``Demonstrate that the operations of a study
    such as the data collection procedure can be repeated, with the same
    results'' \citep{ZhouEtAl2016}.
\end{description}

\begin{itemize}
\item The grading template included an entry for the reviewer's impression.  We
aimed for objectivity, but there is a risk that some scores may be subjective
and biased.
\item The measurements are made at an instant in time, and the instant differs
by a few weeks between projects, due to the time needed to measure them. For the
most part, the projects are living and continually changing, which means if we
measured at a different time, our results may change.
\end{itemize}

\section{Conclusions} \label{ch_conclusions}

We analyzed the state of the practice for the MI domain with the goal of
understanding current practice, answering our six research questions
(Section~\ref{sec_motivation}) and providing recommendations for current and
future projects.  Our methods in Section~\ref{ch_methods} form a general process
to evaluate domain-specific software, that we apply to the specific domain of MI
software. We identified 48 MI software candidates, then, with the help of the
Domain Expert selected 29 of them to our final list. Section~\ref{ch_results}
lists our measurements to nine software qualities for the 29 projects, and
Section~\ref{painpoints} contains our interviews with eight of the 29 teams,
discussing their development process and five software qualities.  We answered
our research questions. In addition, Section~\ref{ch_recommendations} presents
our recommendations on SC software development.

\subsection{Key Findings}

With the measurement results in Section~\ref{ch_results}, we summarized the
current status of MI software development. We ranked the 29 software projects in
nine qualities.  Based on on the grading scores \textit{3D Slicer},
\textit{ImageJ}, and \textit{OHIF Viewer} are the top three software packages.

The interview results in Section~\ref{painpoints} show some merits, drawbacks,
and pain points within the development process. The three primary categories of
pain points are:
\begin{itemize}
\item the lack of fundings and time;
\item the difficulty to balance between four factors: cross-platform
compatibility, convenience to development \& maintenance, performance, and
security;
\item the lack of access to real-world datasets for testing.
\end{itemize}
We summarized the solutions from the developers to address these problems,
including developing a web-based approach with backend servers and maintaining
better documentation. We also collected the status of documentation.  We found
that for all 8 interviewed teams that documentation is felt to be vital to a
project, with the most popular form of documentation being forum discussions and
videos.  With respect to project management almost all teams used GitHub and
pull requests to manage contributions.  Very few teams used a specific
development model.  It appears that the development process is more ad hoc than
planned for the majority of projects.

Our answers to the research questions are based on the above findings. We
identified the existing artifacts, tools, principles, processes, and
methodologies in the 29 projects. By comparisons with the implied popularity of
existing projects we found: 1) four of the top five software projects in our
ranking were also among the top five ones receiving the most GitHub stars per
year (Table~\ref{tab_ranking_vs_GitHub}); 2) three of the top four in our
ranking were among the top four provided by the domain experts.

Section~\ref{ch_recommendations} presents our recommendations on improving
software qualities and easing pain points during development. Some highlighted
recommendations are as follows:
\begin{itemize}
\item adopting test-driven development with unit tests, integration tests, and
nightly tests;
\item maintaining good documentation (e.g., installation instructions,
requirements specifications, theory manuals, getting started tutorials, user
manuals, project plan, developer’s manual, API documentation, requirements on
coding standards, development process, project status, development environment,
and release notes);
\item using CI/CD;
\item using git and GitHub;
\item modular approach with the design principle proposed by
\citet{ParnasEtAl2000};
\item considering newer technologies (e.g.\ web application and serverless
solution);
\item various ways of enriching the testing datasets, such as using existing
open data sources and establishing greater community collaboration in the MI
domain (Section~\ref{sec_recommendations_testing_dataset}).
\end{itemize}

\subsection{Future Works}

With learnings from this project, we summarized recommendations for the future
state of the practice assessments:
\begin{itemize}
    \item we can make the surface measurements less shallow. For example:
    \begin{itemize}
        \item surface reliability: our current measurement relies on
        the processes of installation and getting started tutorials. However,
        not all software needs installation or has a getting started tutorial.
        We can design a list of operation steps, perform the same operations
        with each software, and record any errors.
        \item surface robustness: we used damaged images as inputs for
        this measuring MI software. This process is similar to fuzz testing
        \citep{enwiki:1039424308}, which is one type of fault injection
        \citep{enwiki:1039005082}. We may adopt more fault injection methods, and
        identify tools and libraries to automate this process.
        \item surface usability: we can design usability tests and test
        all software projects with end-users. The end-users can be volunteers
        and domain experts.
        \item surface understandability: our current method does not require
        understanding the source code. As software engineers, perhaps we can
        select a small module of each project, read the source code and
        documentation, try to understand the logic, and score the ease of the
        process.  Ideas for getting started are available in
        \citet{SmithEtAl2021}.
        \item measure modifiability as part of the measurement of
        maintainability.  An experiment could be conducted asking participants
        to make modifications, observing the study subjects during the
        modifications, testing the resulting software and surveying the
        participants \citep{SmithEtAl2021}.
    \end{itemize}
	\item we can further automate the measurements on the grading template. For
	example, with automation scripts and the GitHub API, we may save significant
	time on retrieving the GitHub metrics through a GitHub Metric Collector.
	This Collector can take GitHub repository links as input, automatically
	collect metrics from the GitHub API, and record the results.
	\item the rubric for the grading standard can be made more explicit.
	\item we can improve some interview questions. Some examples are:
	\begin{itemize}
	    \item in {Q14}, ``Do you think improving this process can tackle the
	    current problem?'' is a yes-or-no question, which is not informative
	    enough. As mentioned in Section~\ref{Sec_CompareMethodologies}, most
	    interviewees ignored it. We can change it to ``By improving this
	    process, what current problems can be tackled?''; 
	    \item in {Q16}, we can ask for more details about the modular
	    approach, such as ``What principles did you use to divide code into
	    modules? Can you describe an example of using your principles?'';
	    \item {Q17} and {Q18} should respectively ask understandability to
	    developers and usability to end-users, since there was confusion during
	    the interviews as to which group was being discussed.
	\end{itemize}
	\item we can better organize the interview questions. Since we use audio
	conversion tools to transcribe the answers, we should make the transcription
	easier to read. For example, we can order them together for questions about
	the five software qualities and compose a similar structure for each.
	\item we can mark the follow-up interview questions with keywords. For
	example, say ``this is a follow-up question'' every time asking one. Thus, we
	record this sentence in the transcription, and it will be much easier to
	distinguish the follow-up questions from the 20 designed questions.
\end{itemize}

%% The Appendices part is started with the command \appendix;
%% appendix sections are then done as normal sections
%% \appendix

%% \section{}
%% \label{}

%% If you have bibdatabase file and want bibtex to generate the
%% bibitems, please use
%%
%%  \bibliographystyle{elsarticle-harv} 
%%  \bibliography{<your bibdatabase>}

%% else use the following coding to input the bibitems directly in the
%% TeX file.

%% Bibliography
\bibliographystyle{ACM-Reference-Format}
\bibliography{MedImageSoft_SOP}

% \begin{thebibliography}{00}

% %% \bibitem[Author(year)]{label}
% %% Text of bibliographic item

% \bibitem[ ()]{}

% \end{thebibliography}

\end{document}

\endinput
%%
%% End of file `elsarticle-template-harv.tex'.

%%%%%%%%%%%

% Overview of scientific computing
We define Scientific Computing (SC) as ``the use of computer tools to analyze or
simulate mathematical models of real world systems of engineering or scientific
importance so that we can better understand and predict the system's behaviour''
\citep{Smith2006}. Many researchers consider SC as the third pillar of science
and engineering, along with theory and experiment \citep{Landau2005}. Almost all
areas in science and engineering use computers for modelling \citep{Golub2014},
and software plays an essential role in modern scientific research
\citep{Hannay2009, Wilson2014}.  Software development in SC depends on three
fields of knowledge: engineering or scientific domain knowledge, mathematical
algorithm knowledge, and computational algorithm knowledge \citep{Landau2005,
Mehta2015}. Thus, most SC software developers are scientists in SC domains
\citep{Wilson2014}. However, they do not always use modern software development
techniques, tools, and methods \citep{Wilson2014}. Therefore, we developed a
methodology for assessing the state of the practice for SC software
\citep{SmithEtAl2021, SmithAndMichalski2022}. We apply this process to Medical
Imaging (MI) software that belongs to a specific domain of SC.

TO DO

- CarverEtAl2022 has discussion of requirements etc. that we could also contrast (seems that people overestimate their use of requirements.)
- add more on comparison to existing literature surveys
- ask Mike:
    - for journal version ask for more figures?
    - for journal double blind solution
    - can he submit
    - potential reviewers
    - feedback
- Radiology - should not exceed 6000 words, 50 refs, 8 figures and 4 tables

- add discussion of sensitivity analysis (borrow from LBM)