\documentclass[12pt]{casletter}

\oddsidemargin -10mm
\evensidemargin -10mm
\textwidth 180mm
\textheight 200mm
%\renewcommand\baselinestretch{0.85}

\usepackage{xcolor}
\usepackage{hyperref}

\pagestyle {plain}
\pagenumbering {arabic}

\email{smiths@mcmaster.ca}
\telephone{(905) 525-9140 ext.\ 27929}
\www{{http://}www.cas.mcmaster.ca/\~\@smiths}

\signature{Spencer Smith}

\address{Dr.\ Spencer Smith, Associate Professor\\
Computing and Software Department\\
Faculty of Engineering\\
McMaster University\\
1280 Main Street West\\
Hamilton, Ontario, Canada L8S 4K1}%  \\ Phone: 525-9140 Ext.~27929 \\E-mail:
                                  %  smiths@mcmaster.ca}

\begin{document}

\begin{letter}{~\\
    ~\\
    ~\\
    ~\\
    ~\\
    Dr.\ Anthony McGoron\\
    Editor-in-Chief\\
    Critical Reviews in Biomedical Engineering\\
    ~\\
    {\bf Re: Revisions to CRB-57015, State of the Practice for Medical Imaging
    Software Based on Open Source Repositories}}

  \opening {Dear Dr.\ McGoron:}

  Thank you and the reviewers for the feedback on our submission.  The reviewers
  provided thoughtful and constructive comments.  In response to your e-mail,
  dated January 3, 2025, we have revised the paper to incorporate the requested
  revisions.  We provide a summary below.  The revised submission also includes
  a ``diff'' version of the paper showing the additions and deletions.

  Some of the reviewer's questions are best answered by citing our related work.
  Therefore, the new version undoes the redaction originally added for the
  double blind review.

  \textbf{Reviewer \#1}

  \begin{enumerate}
  \item The relative priorities of each software quality were set equal.
  Couldn't some different weighting have been considered? Perhaps one that comes
  from interviews with developers or cited in papers? \medskip

  \emph{The reviewer is correct that other weightings could be considered.  We
  left this out of the scope of the original submission because of space
  considerations and because determining a meaningful alternative weighting is
  difficult.  Although Nguyen-Hoan et al.\ (2010) (A Survey of Scientific
  Software Development) provides survey data on the relative importance of
  various software qualities, their list of qualities differs from ours.  We are
  not aware of a recent survey that assesses the relative importance of
  qualities for scientific software. Moreover, even if we have quality
  priorities for all scientific software, our paper focuses only on medical
  imaging software.  The priorities for medical imaging software likely differ
  from the priorities for other scientific applications.  Furthermore, the
  priorities likely differ between different medical imaging applications.
  Rather than speculate on the relative priorities, we used the simplest case of
  equal priority. For any readers that would like to apply their own priorities,
  we can direct them to the spreadsheet used to generate the results shown in the
  paper.  With some simple modifications the plots can regenerated with whatever
  priority ranking a user wishes to see.  To make this clear, we have added the
  following text to the revised paper:  } \smallskip
  
  \textcolor{blue}{(Alternative priority schemes can be investigated by changing
the Prioritization Matrix in
\href{https://github.com/smiths/AIMSS/blob/master/StateOfPractice/AHP2021/MedicalImaging/AHP_Template.xlsx}
{AHP\_Template.xlsx} available on GitHub.)}

  \item In addition to github, could any other reference have been used, such as
  the number of citations in papers or relevant websites in the field? \medskip

  \emph{Although the idea of including citations in papers is a good one, we
  have excluded this data because we feel it is unreliable.  Unfortunately,
  software is infrequently and inconsistently cited even when it is used (as
  observed by Smith et al (2016) (Software Citation Principles)).  Similar
  inconsistency problems exist for data on websites.  For other readers who have
  the same observation as the reviewer, we have added the following text to the
  revised version: \smallskip}

  \textcolor{blue}{We considered also comparing software citations, but this
  measure of popularity would be unreliable since software is infrequently and
  inconsistently cited (SmithEtAl2016).}

  \item Where are the quality ratings for each software? \medskip

  \emph{The quality ratings for each software are summarized in Figure 2
  (Overall AHP scores with an equal weighting for all 9 software qualities). To
  keep the length of the paper reasonable, we do not have figures showing the
  ranking of the software packages for each quality, but this information is
  available through the citations given in the now unredacted revised version of the
  paper.  Some of the relevant citations include Dong (2021) (Assessing the
  state of the practice for medical imaging software, MEng thesis), and Smith et
  al (2024) (State of the practice for medical imaging software, extended report
  on arXiv). \smallskip}

  \item Caption tables above them, not below them. \medskip

  \emph{This has been fixed in the revised manuscript. \smallskip}

  \item What are the possible reasons for the differences between the ranking
  found and that of GitHub, in addition to the impossibility of installing some
  of them? \medskip

  \emph{Other possible reasons for the differences in ranking are covered in the
  original submitted paper in the following text: ``We weighted all qualities
  equally, which is not the likely the same weighting that users implicitly use.
  To properly assess this would require a broad user study.  Furthermore our
  measures of popularity (like stars) are only \emph{proxies} which are biased
  towards past rather than current preferences (Szulik2017), as these are
  monotonically increasing quantities. Finally there are often more factors than
  just quality that influence the popularity of products.''  Given that this
  wasn't obvious to the reviewer, we strengthened the topic sentence that
  introduced the reasons with the following text: \smallskip}

  \textcolor{blue}{Besides the installation problem, another possible reason for
  discrepancies between our ranking and GitHub popularity is that we weighted
  all qualities equally, ....}

  \item Some software could not be installed. What problems were found? Why not
  remove it from the list? \medskip

  \emph{In the interest of space, we have chosen to not include the details in
  on installation problems for \textit{GATE}, \textit{dwv}, and \textit{DICOM
  Viewer} in the paper.  Instead, the additional details are given in the
  extended version of the paper on arXiv (Smith et al (2024)) and in the
  associated Masters thesis (Dong (2021)).  The software with installation
  problems was not removed from our review because we aimed for as complete an
  overview as possible.   Moreover, the installation problems only affected the
  score on installability, reliability and robustness.  The other qualities are
  based on the artifacts in the repository, not on running the software.  To
  clarify this for readers of the paper, we have added text to the methodology
  for grading in the revised manuscript: \smallskip}

  \textcolor{blue}{We explicitly note any software that cannot be installed,
  but for completeness we still measure it for the remaining qualities, since
  only reliability and robustness require running the software.}

  \end{enumerate}

  \textbf{Reviewer \#2}

  \begin{enumerate}
  
  \item The introduction section should contain the scope, significance of the
  research by summarizing current understanding and background information,
  stating the purpose of the work, and highlighting the potential outcomes.
  Also, there's no need to divide various subsection in the introduction.
  \medskip

  \emph{We believe that our introduction covers the scope (in Section 1.2 of the
  submitted version) the significance (in the first paragraph).  We also believe
  that the current understanding, background information and potential outcomes
  are covered by the research questions. There is not a considerable amount of
  background information because there isn't a previous study (that we are aware
  of) that assesses the state of the practice for medical imaging software. The
  potential outcomes are the answers to the research questions. \smallskip}

  \emph{The criticism of the introduction, including the comment about not
  having subsections, is likely due to our covering the methodology in the
  introduction section. In the revised version, we have made the methodology its
  own section and removed the subsections from the introduction. \smallskip}

  \item The title of the table should be placed above the table. \medskip

  \emph{This has been fixed in the revised manuscript.  \smallskip}

  \item It is not recommended to include tables in the conclusion. \medskip

  \emph{Although including a table in the conclusions is rare, we think it is
  appropriate for this paper.  A table is a quickest way to summarize the
  software package quality.  Summarizing the same information in words would be
  hard to follow.  A conclusion shouldn't introduce new information, but the
  table is not new information; it is an alternate view of the information
  presented in Section 4 of the paper.  This new view makes some conclusions
  about the state of the practice clear, so it feels appropriate for the
  conclusion section. \smallskip}

  \end{enumerate}

  \closing{Best regards,~\newline} \vspace{-29mm}
  \includegraphics[scale=0.75]{Signature.pdf}

\end {letter}

\end{document}