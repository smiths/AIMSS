% This is samplepaper.tex, a sample chapter demonstrating the
% LLNCS macro package for Springer Computer Science proceedings;
% Version 2.21 of 2022/01/12
%
\documentclass[runningheads]{llncs}
%
\usepackage[T1]{fontenc}
% T1 fonts will be used to generate the final print and online PDFs,
% so please use T1 fonts in your manuscript whenever possible.
% Other font encondings may result in incorrect characters.
%
\usepackage{graphicx}
% Used for displaying a sample figure. If possible, figure files should
% be included in EPS format.
%
% If you use the hyperref package, please uncomment the following two lines
% to display URLs in blue roman font according to Springer's eBook style:
% \usepackage{color}
% \renewcommand\UrlFont{\color{blue}\rmfamily}
%
\usepackage{hyperref}
\usepackage{paralist}
\usepackage{booktabs}
\usepackage{totcount}
\usepackage{amssymb}

\newcounter{totHours} %to track the estimate of total hours to assess SOP
\setcounter{totHours}{0}
\regtotcounter{totHours}

\newcounter{rqnum} %research question number
\newcommand{\rqtherqnum}{RQ`'\therqnum}
\newcommand{\rqref}[1]{RQ\ref{#1}}

\newcounter{pnum} %pain point number
\newcommand{\ppthepnum}{P`'\thepnum}
\newcommand{\ppref}[1]{P\ref{#1}}

\newcommand{\CC}{C\nolinebreak\hspace{-.05em}\raisebox{.4ex}{\small\bf
+}\nolinebreak\hspace{-.10em}\raisebox{.4ex}{\small\bf +}}

\begin{document}
%
% \title{Digging Deep to Assess the State of the Practice for Different Research Software Domains}
\title{Digging Deeper Into the State of the Practice for Domain Specific Research Software}
\titlerunning{Digging Deeper}
% If the paper title is too long for the running head, you can set
% an abbreviated paper title here
%
\author{Spencer Smith\inst{1}\orcidID{0000-0002-0760-0987} \and
Peter Michalski\inst{1}}%\orcidID{1111-2222-3333-4444} \and
%Third Author\inst{3}\orcidID{2222--3333-4444-5555}}
%
\authorrunning{S.\ Smith and P.\ Michalski}
% First names are abbreviated in the running head.
% If there are more than two authors, 'et al.' is used.
%
\institute{McMaster University, 1280 Main Street West, Hamilton ON L8S 4K1, Canada
\email{smiths@mcmaster.ca}\\
\url{http://www.cas.mcmaster.ca/~smiths/}}
%
\maketitle              % typeset the header of the contribution
%
\begin{abstract}

	To improve software development methods and tools for research software, we
	first need to understand the current state of the practice.  Therefore, we
	have developed a methodology for assessing the state of the software
	development practices for a given research software domain.  The methodology
	is applied to one domain at a time in recognition that software development in
	different domains is likely to have adopted different best practices.
	Moreover, providing a means to measure different domains facilitates
	comparison of development practices between domains.  For each domain we wish
	to answer questions such as: 
  \begin{inparaenum}[i)]
    \item What artifacts (documents, code, test cases, etc.) are present?
    \item What tools are used?
    \item What principles, process and methodologies are used?
    \item What are the pain points for developers?
    \item What actions are used to improve qualities like maintainability and
    reproducibility?
  \end{inparaenum} 
  To answer these questions, our methodology prescribes the following steps: 
  The abstract should briefly summarize the contents of the paper in 150--250 words.

\keywords{First keyword  \and Second keyword \and Another keyword.}
\end{abstract}

\section{Introduction} \label{SecIntroduction}

Research software is critical for tackling problems in areas as diverse as
manufacturing, financial planning, environmental policy and medical diagnosis
and treatment.  However, developing reliable, reproducible, sustainable and fast
research software to address new problems is challenging because of the
complexity of the physical models and the nuances of floating point and parallel
computation. The importance of research software and the difficulty with its
development have prompted multiple researchers to investigate how developing
this software differs from other classes of software.  Previous studies have
focused on surveying developers
\cite{HannayEtAl2009,Nguyen-HoanEtAl2010,PintoEtAl2018}, developer interviews
\cite{Kelly2013} and case studies \cite{CarverEtAl2007,Segal2005}.  A
valuable source of information that has received less attention is the data in
publicly available software repositories.  We propose digging deeper into how
well projects are applying best practices by using a new methodology that
combines manual and automated techniques to extract repository-based
information.

The surveys used in previous studies have tended to recruit participants from
all domains of research software, possibly distinguishing them by programming
language (for instance, R developers \cite{PintoEtAl2018}), or by the role of
the developers (for instance postdoctoral researchers \cite{UditAndKatz2017}).
Case studies, on the other hand, have focused on a few specific examples at a
time, as is the nature of case studies.  For our new methodology, we propose a
scope between these two extremes.  Rather than focus on all research software,
or just a few examples, we will focus on one domain of research software at a
time. The practical reason for defining a domain scope is that digging deep into
repository data takes time, making a broad scope infeasible. We have determined
that a practical constraint for the work load is for a team as small as one
individual doing one person month worth of work per domain.\footnote{A person
month is considered to be $20$ working days ($4$ weeks in a month, with $5$ days
of work per week) at $8$ person hours per day, or $20 \cdot 8 = 160$ person
hours.} Focusing on one domain at a time has more than just practical
advantages.  By restricting ourselves to a single domain we can bring domain
knowledge and domain experts into the mix.  The domain customized insight
provided by the assessment has the potential to help a specific domain as they
adopt and develop new development practices.  Moreover, measuring different
domains facilitates uncovering domain specific practices, which can later be
compared and contrasted.

Our methodology is built around 10 research questions.  Assuming that the domain
has been identified (Section~\ref{SecIdentifyDomain}), the first question is:

\begin{enumerate}
	\item[RQ\refstepcounter{rqnum}\therqnum \label{RQ_WhatProjects}:] What
	software projects exist in the domain, with the constraint that the source
	code must be available for all identified projects?
	(Sections~\ref{identifysoftware},~\ref{filtersoftware})
\end{enumerate}

We next wish to assess the representative software to determine how well they
apply current software development best practices.  At this point in the
process, to remove potential user/developer bias, we will base our assessment
only on publicly available artifacts, where artifacts are the documents, scripts
and code that we find in a project's public repository. Example artifacts
include requirements, specifications, user manuals, unit test cases, system
tests, usability tests, build scripts, API (Application Programming Interface)
documentation, READMEs, license documents, process documents, and code.
Following best practices does not guarantee popularity, so we will also compare
our ranking to how the user community itself ranks the identified projects.

\begin{enumerate}
	\item [RQ\refstepcounter{rqnum}\therqnum \label{RQ_HighestQuality}:] Which
	of the projects identified in \rqref{RQ_WhatProjects} follow current best
	practices, based on evidence found by experimenting with the software and
	searching the artifacts available in each project's repository? (Sections~\ref{empiricalmeasures})
	\item [RQ\refstepcounter{rqnum}\therqnum \label{RQ_CompareHQ2Popular}:] How
	similar is the list of top projects identified in \rqref{RQ_HighestQuality}
	to the most popular projects, as viewed by the scientific community? (Section~\ref{repmetrics})
\end{enumerate}

To understand the state of the practice we wish to learn the frequency with
which different artifacts appear, the types of development tools used and the
methodologies used for software development.  With this data, we can ask
questions about how the domain software compares to other research software.

\begin{enumerate}
  \item [RQ\refstepcounter{rqnum}\therqnum \label{RQ_CompareArtifacts}:] How
	do domain projects compare to research software in general with respect to the
	artifacts present in their repositories? (Section~\ref{Sec_CompareArtifacts})
	\item [RQ\refstepcounter{rqnum}\therqnum \label{RQ_CompareToolsProjMngmnt}:]
	How do domain projects compare to research software in general with respect to
	the use of tools (Section~\ref{Sec_CompareTools}) for:
	\begin{enumerate} 
		\item [\rqref{RQ_CompareToolsProjMngmnt}.a] development; and,
		\item [\rqref{RQ_CompareToolsProjMngmnt}.b] project management?
	\end{enumerate}
	\item [RQ\refstepcounter{rqnum}\therqnum \label{RQ_CompareMethodologies}:]
	How do domain projects compare to research software in general with respect to
	the processes used? (Section~\ref{Sec_CompareMethodologies})
\end{enumerate}	

Only so much information can be gleaned by digging into software repositories.
To gain additional insight, we need to interview developers
(Section~\ref{SecSurvey}).  We need to learn their concerns and how they deal
with these concerns; we need to learn their pain points.

\begin{enumerate}
	\item [RQ\refstepcounter{rqnum}\therqnum \label{RQ_PainPoints}:] What are
	the pain points for developers working on domain software projects?
	(Section~\ref{painpoints})
	\item [RQ\refstepcounter{rqnum}\therqnum \label{RQ_ComparePainPoints}:] How
	do the pain points of domain developers compare to the pain points
	for research software in general? (Section~\ref{painpoints})
\end{enumerate}

Our methodology answers the research question through inspecting repositories,
using the Analytic Hierarch Process (AHP) for ranking software, interviewing
developers and interacting with at least one domain expert.  We leave the
measurement of the performance, for instance using benchmarks, to other projects
\cite{KaagstromEtAl1998}. The current methodology updates the approach used in
prior assessments of domains like Geographic Information Systems
\cite{SmithEtAl2018_arXivGIS}, Mesh Generators \cite{SmithEtAl2016}, Seismology
software \cite{SmithEtAl2018}, and statistical software for psychology
\cite{SmithEtAl2018_StatSoft}.  Initial tests of the new methodology have been
done for medical image analysis software \cite{Dong2021} and for Lattice
Boltzmann Method (LBM) software \cite{Michalski2021}.  The LBM example will be
used throughout this paper to illustrate the steps in the methodology.  The
paper ends with a summary of potential threats to validity
(Section~\ref{threats}) and our final conclusions and recommendations
(Section~\ref{SecConcludingRemarks}).

\section{Methodology} \label{methodology}

The assessment is conducted via the following steps.  The steps depend on
interaction with a Domain Expert partner, as discussed in
Section~\ref{sec_vet_software_list}.

\begin{enumerate}
  \item Identify the domain of interest. (Section~\ref{SecIdentifyDomain})
	\item List candidate software packages for the domain.
	(Section~\ref{identifysoftware})
	\item Filter the software package list. (Section~\ref{filtersoftware})
	\item Gather the source code and documentation for each software package.
	\item Collect quantitative measures from the project repositories.
	(Section~\ref{empiricalmeasures})
	\item Measure using the measurement template.  The full measurement template
	can be found in \cite{SmithEtAl2021}. (Section~\ref{empiricalmeasures})
	\item Use AHP to rank the software packages. (Section~\ref{empiricalmeasures})
	\item Interview the developers. (Section~\ref{SecSurvey})
    \item Domain analysis.  (Section~\ref{SecDomainAnalysis})
	\item Analyze the results and answer the research questions. (Sections~\ref{repmetrics}---\ref{painpoints}) %,~\ref{Sec_CompareArtifacts},~\ref{Sec_CompareTools},~\ref{Sec_CompareMethodologies},~\ref{painpoints},~\ref{Sec_AddressConcerns},
\end{enumerate}

Close to our goal of 160 person hours, we estimate 173 hours to complete the
assessment of a given domain~\cite{SmithEtAl2021}.  This estimate assumes that
the domain has already been decided and the Domain Expert has been recruited.
The estimate assumes that the template spreadsheets and developed AHP
tool~\cite{SmithEtAl2021}, will be employed, rather than developing new tools.
The person hours given are a rough estimate, based on our experience completing
assessments for medical image analysis software \cite{Dong2021} and for Lattice
Boltzmann Method (LBM) software \cite{Michalski2021}.  The estimate assumes 30
software packages will be measured; the numbers will need to be adjusted if the
total packages changes.

\subsection{How to Identify the Domain?} \label{SecIdentifyDomain} 

A domain of research software must be identified to begin the assessment. To be
applicable for the methodology described in this document, the chosen domain
must have the following properties:

\begin{enumerate}
\item The domain must have well-defined and stable theoretical underpinning.  A
  good sign of this is the existence of standard textbooks, preferably
  understandable by an upper year undergraduate student.
\item There must be a community of people studying the domain.
\item The software packages must have open source options.
\item A preliminary search, or discussion with experts, suggests that there will
  be numerous, close to 30, candidate software packages in the domain that
  qualify as `research software'.
\end{enumerate}	

\subsection{How to Identify Candidate Software from the Domain?}
\label{identifysoftware}

To answer~\rqref{RQ_WhatProjects} we needed to identify existing projects. The
candidate software should be found through search engine queries targeting
authoritative lists of software.  Candidate places to search include
\href{https://github.com/} {GitHub} and \href{https://swmath.org/} {swMATH}, as
well as through scholarly articles. The Domain Expert
(Section~\ref{sec_vet_software_list}) should also be engaged in selecting the
candidate software.  The following properties were considered when creating the
list and reviewing the candidate software:

\begin{enumerate}
	\item The software functionality must fall within the identified domain.
	\item The source code must be viewable.
	\item The repository based measures should be available, which implies a
	preference for GitHub-style repositories.
	\item The software cannot be marked as incomplete or in an initial state of
	development.
\end{enumerate}

\subsection{How to Filter the Software List?} \label{filtersoftware}

If the list of software is too long (over around 30 packages), then steps need
to be taken to create a more manageable list to answer~\rqref{RQ_WhatProjects}.
The following filters were applied in the priority order listed. Copies of both
the initial and filtered lists, along with the rationale for shortening the
list, should be kept for traceability purposes.

\begin{enumerate}
	\item Scope: Software is removed by narrowing what functionality is
	considered to be within the scope of the domain.
	\item Usage: Software packages were eliminated if their installation
	procedure was missing or not clear and easy to follow.
	\item Age: The older software packages (age being measured by the last date
	when a change was made) were eliminated, except in the cases where an older
	software package appears to be highly recommended and currently in use. (The
	Domain Expert should be consulted on this question as necessary.)
\end{enumerate}

For the LBM example the initial list had 46 packages~\cite{Michalski2021}. This
list was filtered by scope, usage, and age to decrease the length to 24. Many of
the removed packages could not be tested as there was no installation guide,
they were incomplete, source code was not publicly available, a license was
needed, or the project was out of scope or not up to a standard that would
support incorporating them into this study~\cite{Michalski2021}.

\subsection{Quantitative Measures} \label{empiricalmeasures}

We rank the projects by how well they follow best practices
(\rqref{RQ_HighestQuality}) via a measurement template~\cite{SmithEtAl2021}. For
each software package (each column in the template), we fill-in the rows. This
process takes about 2 hours per package, with a cap of 4 hours.  An excerpt of
the template is shown in Figure~\ref{measurement_template_image}.

\begin{figure}[!ht]
	\begin{center}
	  \includegraphics[width=1.0\textwidth]{./figures/measurement_template.pdf}
	  \caption{Excerpt of the Top Sections of the Measurement Template}
	  \label{measurement_template_image}
	\end{center}
\end{figure}

The full template~\cite{SmithEtAl2021} consists of 108 questions categorized
under 9 qualities:
\begin{inparaenum}[(i)]
	\item installability;
	\item correctness and verifiability;
	\item surface reliability;
	\item surface robustness;
	\item surface usability;
	\item maintainability;
	\item reusability;
	\item surface understandability; and,
	\item visibility/transparency. 
\end{inparaenum} 

The questions were designed to be unambiguous, quantifiable and measurable with
limited time and domain knowledge. The measures are grouped under headings for
each quality, and one for summary information
(Figure~\ref{measurement_template_image}).   The summary section provides
general information, such as the software name, number of developers, etc.
Several of the qualities use the word ``surface''.  This is to highlight that,
for these qualities in particular, the best that we can do is a shallow measure.
For instance, we do not conduct experiments to measure usability. Instead, we
are looking for an indication that usability was considered by the developers by
looking for cues in the documentation, such as getting started instructions, a
user manual or documentation of expected user characteristics.

Tools are used to find some of the measurements, such as the number of files,
number of lines of code (LOC), percentage of issues that are closed, etc. The
tool \href{https://github.com/tomgi/git_stats}{GitStats} is used to measure
each software package's GitHub repository for the number of binary files, the
number of added and deleted lines, and the number of commits over varying time
intervals. The tool \href{https://github.com/boyter/scc}{Sloc Cloc and Code
(scc)} is used to measure the number of text based files as well as the number
of total, code, comment, and blank lines in each GitHub repository.

Virtual machines (VMs) are used to provide an optimal testing environments for
each package~\cite{SmithEtAl2016}. VMs are used because it is easier to start
with a fresh environment so as to not worry about existing libraries and
conflicts. Moreover, when the tests are complete the VM can be deleted, without
any impact on the host operating system. The most significant advantage of using
VMs is to level the playing field. Every software install starts from a clean
slate, which removes ``works-on-my-computer'' errors.

Once we have measured each package, we still need to rank them to answer
\rqref{RQ_HighestQuality}.  To do this, we used the Analytical Hierarchy Process
(AHP), a decision-making technique that is used to compare multiple options by
multiple criteria \cite{Saaty1980}. AHP performs a pairwise analysis using a
matrix and generates an overall score as well as individual quality scores for
each software package. The advantage of pair-wise comparisons is they facilitate
a separation of concerns.  Rather than worry about the entire problem, the
decision maker can focus on one comparison at a time.  In our work AHP performs
a pairwise analysis between each of the 9 quality options for each of the
(approximately) 30 software packages.  This results in a matrix, which is used
to generate an overall score for each software package for the given
criteria~\cite{SmithEtAl2016}.

\begin{figure}[h!]
	\centering
		\includegraphics[width=1.0\textwidth]{./figures/finalscore_chart.pdf}
		\caption{AHP Overall Score}
		\label{Fig_OverallScore}
\end{figure}

Figure~\ref{Fig_OverallScore} shows the overall ranking of the LBM software
packages. In the absence of other information on priorities, the overall ranking
was calculated by assuming an equal weighting between all qualities. If the
qualities were weighed differently, the overall software package ranking would
change.

\subsection{Interview Developers} \label{SecSurvey}

Several of the research question (\rqref{RQ_CompareToolsProjMngmnt},
\rqref{RQ_CompareMethodologies} and \rqref{RQ_PainPoints}) require going beyond
the quantitative data from the measurement template. To gain the required
insight, we interview developers using a list of 20 questions
\cite{SmithEtAl2021}. The questions cover the background of the development
teams, the interviewees, and the software itself. We ask the developers how they
organize their projects. We also ask them about their understanding of the
users. Some questions focus on the current and past difficulties, and the
solutions the team has found, or will try. We also discuss the importance of,
and the current situation for, documentation. A few questions are about specific
software qualities, such as maintainability, understandability, usability, and
reproducibility. The interviews are semi-structured based on the question list;
we ask follow-up questions when necessary.  Each interview should take about 1
hour.

The interviewees should follow a process where they can make informed consent.
The interviews should follow standard ethics guideline of asking for consent
before interviewing, recording, and including participant details in the report.
The interview process presented here was approved by the McMaster University
Research Ethics Board under the application number 
\href{https://github.com/smiths/AIMSS/blob/master/StateOfPractice/MACREM/Application.pdf}
{MREB\#: 5219}.  For LBM we were able to recruit 4 developers to participate in
our study.

\subsection{Interaction With Domain Expert} \label{sec_vet_software_list}

Our methodology relies on engaging a Domain Expert to vet the list of projects
(\rqref{RQ_WhatProjects}) and the AHP ranking (\rqref{RQ_HighestQuality}).  The
Domain Expert is an important member of the assessment
team. Pitfalls exist if non-experts attempt to acquire an authoritative list of
software, or try to definitively rank the software. Non-experts have the problem
that they can only rely on information available on-line, which has the
following drawbacks:
\begin{inparaenum}[i)]
  \item the on-line resources could have false or inaccurate information; and,
  \item the on-line resources could leave out relevant information that is so
in-grained with experts that nobody thinks to explicitly record it.
\end{inparaenum}
Domain experts may be recruited from academia or industry.  The only
requirements are knowledge of the domain and a willingness to be engaged in the
assessment process.  The Domain Expert does not have to be a software developer,
but they should be a user of domain software.  Given that the domain experts are
likely to be busy people, the measurement process cannot put to much of a burden
on their time.

\subsection{Domain Analysis} \label{SecDomainAnalysis}

Since each domain we will study will have a reasonably small scope, we will be
able to view the software as constituting a program family. Studying the common
properties within a family of related programs is termed a domain analysis.  For
the current methodology time constraints necessitate a shallow analysis. For
each domain a table should be constructed that distinguishes the programs under
study by their variabilities.  In research software the variabilities are often
related to assumptions.  Table~\ref{tbl_features} shows the variabilities for
the LBM software example~\cite{Michalski2021}.

\begin{table}[h!]
	\begin{center}
		\begin{tabular}{ p{3cm}p{1.25cm}p{2.25cm}llllll}
			\toprule
			Name & Dim & Pll & Com & Rflx & MFl & Turb & CGE & OS\\
			\midrule
			DL\_MESO (LBE) & 2, 3 & MPI/OMP & Y & Y & Y & Y & Y & W, M, L\\
			ESPResSo & 1, 2, 3 & CUDA/MPI & Y & Y & Y & Y & Y & M, L\\
			ESPResSo++ & 1, 2, 3 & MPI & Y & Y & Y & Y & Y & L\\
			HemeLB & 3 & MPI & Y & Y & Y & Y & Y & L\\
			laboetie & 2, 3 & MPI & Y & Y & Y & Y & Y & L\\
			LatBo.jl & 2, 3 & - & Y & Y & Y & N & Y & L\\
			LB2D-Prime & 2 & MPI & Y & Y & Y & Y & Y & W, L\\
			LB3D & 3 & MPI & N & Y & Y & Y & Y & L\\
			LB3D-Prime & 3 & MPI & Y & Y & Y & Y & Y & W, L\\
			lbmpy & 2, 3 & CUDA & Y & Y & Y & Y & Y & L\\
			lettuce & 2, 3 & CUDA & Y & Y & Y & Y & Y & W, M, L\\
			LIMBES & 2 & OMP & Y & Y & N & N & Y & L\\
			Ludwig & 2, 3 & MPI & Y & Y & Y & Y & Y & L\\
			LUMA & 2, 3 & MPI & Y & Y & Y & Y & Y & W, M, L\\
			MechSys & 2, 3 & - & Y & Y & Y & Y & Y & L\\
			MP-LABS & 2, 3 & MPI/OMP & N & Y & Y & N & N & L\\
			Musubi & 2, 3 & MPI & Y & Y & Y & Y & Y & W, L\\
			OpenLB & 1, 2, 3 & MPI & Y & Y & Y & Y & Y & W, M, L\\
			Palabos & 2, 3 & MPI & Y & Y & Y & Y & Y & W, L\\
			pyLBM & 1, 2, 3 & MPI & Y & Y & N & Y & Y & W, M, L\\
			Sailfish & 2, 3 & CUDA & Y & Y & Y & Y & Y & M, L\\
			SunlightLB & 3 & - & Y & Y & N & N & Y & L\\
			TCLB & 2, 3 & CUDA/MPI & Y & Y & Y & Y & Y & L\\
			waLBerla & 2, 3 & MPI & Y & Y & Y & Y & Y & L\\
			\bottomrule
		\end{tabular}
		\caption{Features of Software Packages (Dim for Dimension (1, 2, 3), Pll
			for Parallel (CUDA, MPI, OpenMP (OMP)), Com for Compressible (Yes or
			No), Rflx for Reflexive Boundary Condition (Yes or No), MFl for
			Multi-fluid (Yes or No), Turb for Turbulent (Yes or No), CGE for
			Complex Geometries (Yes or No), OS for Operating System (Windows
			(W), macOS (M), Linux (L)))} \label{tbl_features}
	\end{center}
\end{table}

\section{Comparison to Community Ranking} \label{repmetrics}

To address \rqref{RQ_CompareHQ2Popular} we need to compare the ranking by best
practices to the communities ranking.  Our best practices ranking comes from the
results of the AHP ranking (Section~\ref{empiricalmeasures}).  We estimate the
communities ranking by repository stars and watches.  The comparison will
provide insight on whether best practices are rewarded by popularity.  However,
inconsistencies between the AHP ranking and the communities ranking are in
inevitable for the following reasons: 
\begin{inparaenum}[i)]
	\item the overall quality ranking via AHP makes the unrealistic assumption
	of equal weighting between the different quality factors;
	\item stars are known to not be a particularly good measure of popularity,
	because of how people use stars and since young projects have less time to
	accumulate stars~\cite{Szulik2017}; 
	\item and, as for consumer products, there are more factors that determine
	popularity than quality alone.
\end{inparaenum}

Table~\ref{repometrics} compares the AHP ranking of the LBM package versus their
popularity in the research community.  Nine packages do not use GitHub, so they
do not have a measure of repository stars. Looking at the repository stars of
the other 15 packages, we can observe a pattern where packages that have been
highly ranked by our assessment tend to have more stars than lower ranked
packages. The best ranked package by AHP (ESPResSo) has the second most stars,
while the ninth ranked package (Sailfish) has the highest number of stars. The
same correlation is observed in the repository watch column, although this
column contains less data, since two of the packages (Palabos, waLBerla) use
GitLab, which does not track watches. Although the AHP ranking and the community
popularity estimate are not perfect measures, they do suggest a correlation
between best practices and popularity.

\begin{table}[!h]
	\begin{center}
		\begin{tabular}{ p{3cm}p{1.25cm}p{1.75cm}p{1.75cm}p{1.75cm}p{1.75cm} }
			\toprule
			Name & Our Ranking & Repository Stars & Repository Star Rank &
			Repository Watches & Repository Watch Rank\\
			\midrule
			ESPResSo & 1 & 145 & 2 & 19& 2\\
			Ludwig & 2 & 27 & 8 & 6& 7\\
			Palabos & 3 & 34 & 6 & GitLab& GitLab\\
			OpenLB & 4 & No Git & No Git & No Git& No Git\\
			LUMA & 5 & 33 & 7 & 12& 4\\
			pyLBM & 6 & 95 & 3 & 10& 5\\
			DL\_MESO (LBE) & 7 & No Git & No Git & No Git & No Git\\
			Musubi & 8 & No Git & No Git & No Git & No Git\\
			Sailfish & 9 & 186 & 1 & 41& 1\\
			waLBerla & 10 & 20 & 9 & GitLab& GitLab\\
			laboetie & 11 & 4 & 13 & 5& 8\\
			TCLB & 12 & 95 & 3 & 16& 3\\
			MechSys & 13 & No Git & No Git & No Git& No Git\\
			lettuce & 14 & 48 & 4 & 5& 8\\
			ESPResSo++ & 15 & 35 & 5 & 12& 4\\
			MP-LABS & 16 & 12 & 11 & 2& 9\\			
			SunlightLB & 17 & No Git & No Git & No Git& No Git\\
			LB3D & 18 & No Git & No Git & No Git& No Git\\			
			LIMBES & 19 & No Git & No Git & No Git& No Git\\
			LB2D-Prime & 20 & No Git & No Git & No Git& No Git\\		
			HemeLB & 21 & 12 & 11 & 12& 4\\
			lbmpy & 22 &  11 & 12 & 2& 9\\	
			LB3D-Prime & 23 & No Git & No Git & No Git& No Git\\	
			LatBo.jl & 24 & 17 & 10 & 8& 6\\			
			\bottomrule
		\end{tabular}
		\caption{Repository Ranking Metrics} \label{repometrics}
	\end{center}
	\end{table}

\section{Comparison of Artifacts to Other Research Software}
\label{Sec_CompareArtifacts}

In this step of the methodology we answer \rqref{RQ_CompareArtifacts} by
comparing the artifacts that we observed in the domain repositories to those
observed and recommended for research software in general.  While filling in the
measurement template (Section~\ref{empiricalmeasures}), the domain software is
examined for the presence of artifacts, which are then categorized by frequency.
We suggest grouping them into the categories: common (more than 2/3 of
projects), less common (between 1/3 and 2/3), and rare (less than 1/3 of
projects). The observed frequency of artifacts should then be compared to the
artifacts recommended by research software guidelines, as summarized in
Table~\ref{Tbl_Guidelines}.  

\begin{table}[!h]
\begin{center}
\begin{tabular}{ p{3cm}p{1cm}p{1cm}p{1cm}p{1cm}p{1cm}p{1cm}p{1cm}p{1cm}p{1cm} }
\toprule
~ \ & \cite{USGS2019} & \cite{TobiasEtAl2018} & \cite{BrettEtAl2021} & \cite{WilsonEtAl2016} & \cite{SmithAndRoscoe2018} & \cite{HerouxEtAl2008} & \cite{ThielEtAl2020} & \cite{vanGompelEtAl2016} & \cite{OrvizEtAl2017}\\
\midrule
LICENSE & \checkmark &  & \checkmark & \checkmark & \checkmark & & \checkmark & \checkmark & \checkmark\\
README &  &  &  &  & & & & \\
CONTRIBUTING &  &  &  &  & & & & & \\
CITATION &  &  &  &  & & & & & \\
CHANGELOG &  &  &  &  & & & & & \\
INSTALL &  &  &  &  & & & & & \\
Uninstall &  &  &  &  & & & & & \\
Contributor Guide &  &  &  &  & & & & & \\
Getting started &  &  &  &  & & & & & \\
User manual &  &  &  &  & & & & & \\
Tutorials &  &  &  &  & & & & & \\
FAQ &  &  &  &  & & & & & \\
Dependency List &  &  &  & & & & & & \\
Issue Track &  &  &  &  & & & & & \\
Version Control &  &  &  &  & & & & &\\ 
Requirements &  &  &  &  & & & & & \\
Design &  &  &  &  & & & & & \\
API Doc. &  &  &  &  & & & & & \\
Build Scripts &  &  &  &  & & & & & \\
Unit Tests &  &  &  &  & & & & & \\
Test Plan &  &  &  &  & & & & & \\
Integ. Tests &  &  &  &  & & & & &  \\
System Tests &  &  &  &  & & & & & \\
Acceptance Tests &  &  &  &  & & & & &  \\
Regression Tests &  &  &  &  & & & & & \\
Code Style Guide &  &  &  &  & & & & & \\
Release Info. &  &  &  &  & & & & & \\
Product Roadmap &  &  &  &  & & & & & \\
Code of Conduct &  &  &  &  & & & & & \\
\midrule
\end{tabular}
\caption{Commonly Recommended Artifacts in Software Development Guidelines} \label{Tbl_Guidelines}
\end{center}
\end{table}

The results for the LBM example are shown below.  The majority of LBM generated
artifacts correspond to general recommendations from research software
developers.  For LBM, a union of the three categories mostly corresponds to
recommendations made by the research software community, as shown in
Table~\ref{Tbl_Guidelines}. Although the LBM community participates in most of
the practices we found listed in the general research software guidelines, some
recommended practices were not observed or rarely observed as follows: API
documentation, a roadmap, a code of conduct, programming style guide, uninstall
instructions.

\begin{description}
	\item[Common:] Developer List, Issue Tracker, Dependency List, Installation
	Guide, Theory Notes, Related Publications, Build Files, README File,
	License, Tutorial, Version Control
	\item[Less Common:] Change Log, Design Doc., Functional Spec., Performance
	Info., Test Cases, User Manual
	\item[Rare:] API Doc., Developer Manual, FAQ, Verification Plan, Video Guide,
	Requirements Spec.
\end{description}

% \begin{table}[h!]
% \begin{center}
% \begin{tabular}{ p{12 cm}}
% %\toprule
% \textbf{Common}\\
% \midrule
% Developer List, Issue Tracker, Dependency List, Installation Guide, Theory
% Notes, Related Publications, Build Files, README File, License, Tutorial,
% Version Control\\
% %\midrule
% \textbf{Less Common}\\
% \midrule
% Change Log, Design Doc., Functional Spec., Performance Info., Test Cases, User
% Manual\\
% %\midrule
% \textbf{Rare}\\
% \midrule
% API Doc., Developer Manual, FAQ, Verification Plan, Video Guide, Requirements
% Spec.\\
% %\bottomrule
% \end{tabular}
% \caption{Artifacts Present in LBM Packages, Classified by Frequency}
% \label{artifactspresent}
% \end{center}
% \end{table}

\section{Comparison of Tools to Other Research Software}
\label{Sec_CompareTools}

To answer \rqref{RQ_CompareToolsProjMngmnt} we need to summarize tools that are
visible in the repositories and that are mentioned during the developer
interviews. Software tools are used to support the development, verification,
maintenance, and evolution of software, software processes, and
artifacts~\cite[p.\ 501]{GhezziEtAl2003}. Development tools support the
development of end products, but do not become part of them, unlike dependencies
that remain in the application once it is released \cite[p.\
506]{GhezziEtAl2003}. As an example, the tools found for LBM software packages
are as follows:

\begin{description}
	\item[Development Tools:] Continuous Integration, Code Editors, Development
	Environments, Runtime Environments, Compilers, Unit Testing Tools,
	Correctness Verification Tools
	\item[Dependencies:] Build Automation Tools, Technical Libraries, Domain
	Specific Libraries
	\item[Project Management Tools:] Collaboration Tools, Email, Change Tracking
	Tools, Version Control Tools, Document Generation Tools
\end{description}

Once the data on tools is collected, the use of two specific tools should be
compared to research software norms: version control and continuous integration.
The poor adoption of version control tools that Wilson lamented in 2006
\cite{Wilson2006} has greatly improved in the intervening years.  A little over
10 years ago version control was estimated to be used in only 50\% of research
software projects \cite{Nguyen-HoanEtAl2010}, but even at that time
\cite{Nguyen-HoanEtAl2010} noted an increase from previous usage levels. A
survey in 2018 shows 81\% of developers use a version control system
\cite{AlNoamanyAndBorghi2018}.  \cite{Smith2018} has similar results, showing
version control usage for alive projects in mesh generation, geographic
information systems and statistical software for psychiatry increasing from
75\%, 89\% and 17\% (respectively) to 100\%, 95\% and 100\% (respectively) over
a four year period ending in 2018.  (For completeness the same study showed a
decrease in version control usage for seismology software over the same time
period, from 41\% down to 36\%).  Almost every software guide cited in
Section~\ref{Sec_CompareArtifacts} includes the advice to use version control.
For LBM packages 67\% of LBM packages use version control (GitHub, GitLab or
CVS). The high usage of version control tools in LBM software matches the trend
in research software in general.

Continuous integration is rarely used in LBM (3 of 24 packages or 12.5\%). This
contrasts with the frequency with which continuous integration is recommended in
research software development
guidelines~\cite{BrettEtAl2021,vanGompelEtAl2016,ThielEtAl2020}. In this case,
it seems likely that the recommendations are ahead of common practice.

\section{Comparison of Processes to Other Research Software} \label{Sec_CompareMethodologies}

The interview data on development processes is used to answer research question
\rqref{RQ_CompareMethodologies}.  This data should be contrasted with the
development process used by research software in general. The literature
suggests that scientific developers naturally use an agile philosophy
\cite{CarverEtAl2007,Segal2005}, or an amethododical process \cite{Kelly2013}.
Another point of comparison should be on the use of peer review, since peer
review is frequently recommended in software development
guidelines~\cite{HerouxEtAl2008,OrvizEtAl2017,USGS2019}.

The running example of LBM confirms an informal process, with elements of agile
methods. The software development process is not explicitly indicated in the
artifacts for most of the packages. However, during interviews one developer
(ESPResSo) told us their non-rigorous development model is like a combination of
agile and waterfall. Employing a loosely defined process makes sense for LBM
software, given that the teams are generally small and self-contained. One of
the developers (ESPResSo) that was interviewed also noted that an ad hoc peer
review process is used to assess major changes and additions, matching the
common recommendation.

\section{Developer Pain Points} \label{painpoints}

To answer \rqref{RQ_PainPoints}, we ask developers about their pain points and
compare their responses to pain points identified in the
literature~\cite{WieseEtAl2019,PintoEtAl2018}.  Pain points to watch for
include: lack of development time, Cross-platform compatibility, interruptions
while coding, scope bloat, lack of user feedback, hard to collaborate on
software projects, aloneness, dependency management, data handling concerns
(like data quality, data management and data privacy), reproducibility, and
software scope determination. For the LBM example, two sample pain points are:

\begin{enumerate}

	\item[P\refstepcounter{pnum}\thepnum \label{P_LackDevTime}:] \textbf{Lack of
	Development Time} A developer of pyLBM noted that their small development
	team has a lack of time to implement new features. Small development teams
	are common for LBM software packages (as shown in the measurement table
	excerpt in Figure~\ref{measurement_template_image}). Lack of time is also
	highlighted as a pain point by other research software developers
	\cite{PintoEtAl2018,WieseEtAl2019}.

	\item[P\refstepcounter{pnum}\thepnum \label{P_LackSoftDevExp}:] \textbf{Lack
	of Software Development Experience} A lack of software development
	experience was noted by the developer of TCLB, and others noted a need for
	improving software engineering education. Many of the team members on their
	project are domain experts, not computer scientists or software engineers.
	This same trend is noted by \cite{Nguyen-HoanEtAl2010}, which showed only
	23\% of research software survey respondents having a computing-related
	background. Similarly, \cite{UditAndKatz2017} show that the majority (54\%)
	of postdocs have not received training in software development.  The LBM
	developer suggesting an increasing role for formal software education
	matches the trend observed by \cite{PintoEtAl2018}, where their replication
	of a previous study \cite{HannayEtAl2009}, shows a growing interest in
	formal training (From 13\% of respondents in 2009 to 22\% in 2018).
	\cite{PintoEtAl2018} found that some developers feel there is a mismatch
	between coding skills and subject-matter skills. 
	
\end{enumerate}

\section{Threats To Validity} \label{threats}

This section examines potential threats to the validity of this state of the
practice assessment. These can be categorized into methodology and data
collection issues. The goal of this assessment isn't to rank the software, but
to use the ranking exercise as a means to understand the state of the practice
of LBM software development.

The measures listed in our measurement template may not be broad enough to
accurately capture some qualities. For example, there are only two measures of
surface robustness. The measurement of robustness could be expanded, as it
currently only measures unexpected input. Other faults could be introduced, but
could require a large investment of time to develop, and might not be a fair
measure for all packages. Similarly, reusability is assessed along the number of
code files and LOC per file. While this measure is indicative of modularity, it
is possible that some packages have many files, with few LOC, but the files do
not contain source code that is easily reusable. The files may be poorly
formatted, or the source code may be vague and have ambiguous identifiers.
Furthermore, the measurement of understandability relies on 10 random source
code files. It is possible that the 10 files that were chosen to represent a
software package may not be a good representation of the understandability of
that package.

Regarding data collection, a risk to the validity of this assessment is missing
or incorrect data. Some software package data may not have been measured due to
technology issues like broken links. This issue arose with the measurement of
Palabos, which had a broken link to its user manual, as noted in
Section~\ref{Sec_CompareArtifacts}. 

Some pertinent data may not have been specified in public artifacts, or may be
obscure within an artifact or web-page. The use of unit testing and continuous
integration was mentioned in the artifacts of only three (ESPResSo, Ludwig,
Musubi) packages. However, interviews suggested a more frequent use of both unit
testing and continuous integration in the development processes than what was
observed from the initial survey of the artifacts. For example, OpenLB, pyLBM,
and TCLB use such methods during development despite this not being explicitly
clear from an analysis of the material available online. 

Furthermore, design documentation was measured to be a ``less common'' artifact
in this assessment, but it is probable that such documentation is part of all
LBM packages. After all, developing SCS is not a trivial endeavour. It is likely
that many packages have such documentation but did not make it public, and due
to this the measured data is not a true reflection of software package quality.

Interviews with the developers revealed a potentially more frequent use of both
unit testing and continuous integration, compared to what was observed from
studying the repository artifacts.

[*{There are more threats to validity.  Brainstorm.  Watch for them while 
editing.  Look at Ao paper.}

Problems with the measurement of qualities:

\begin{itemize}
\item The assumption that more code files is an indicator of reusability
(Section~\ref{reusabilityresults}) may not be a valid assumption.
\item Understandability was measured using only 10 random files
(Section~\ref{Sec_SurfUnderstandability}).  This could be improved.
\end{itemize}

Overall ranking with equally weighted qualities isn't realistic
(Section~\ref{Sec_OverallQuality}).

Threat to validity - the gap between measures of everything else and Musubi
(Section~\ref{identifysoftware}), and then the gap between the manual measures
and the automated measures (Section~\ref{repmetrics}).

From Section~\ref{Sec_CompareTools}: ``Interviews with the developers revealed a
potentially more frequent use of both unit testing and continuous integration,
compared to what was observed from studying the repository artifacts.''

Interviews suggested a more frequent use of both unit testing and continuous
integration in the development processes than what was observed from the initial
survey. For example, OpenLB, pyLBM, and TCLB use such methods during
development, despite this not being explicitly clear from an analysis of the
material available online.  The correctness and verifiability of such packages
is not measured well using surface analysis.

\section{Concluding Remarks} \label{SecConcludingRemarks}

Next step to recommend ways to improve software (previously section 9).  We wish
to know what practices are used by the top domain projects, so that others can
potentially emulate these practices. We also wish to identify new practices by
borrowing successful ideas from other domains. 

Our comparison may point out areas where some LBM software
packages fall short of current best practices. This is not intended to be a
criticism of any existing packages, especially since in practice not every
project needs to achieve the highest possible quality.  However, rather than
delve into the nuances of which software can justify compromising which
practices we will write our comparison under the ideal assumption that every
project has sufficient resources to match best practices.

For each domain we wish to highlight success stories that can be shared amongst
domain community, and with the broader research software community, while at the
same time watching for areas for potential future improvement.

We have outlined a methodology for assessing the state of the practice for any
given research software domain.  (Although the scope of the current work has been
on research software, there is little in the methodology that is specific to
research software, except for the interview question related to the quality of
reproducibility.)  When applying the methodology to a given domain, we provide a
means to answer the following questions:
\begin{inparaenum}[i)]
\item What artifacts (documents, code, test cases, etc.) are present?
\item What tools are used?
\item What principles, process and methodologies are used?
\item What are the pain points for developers?
\item What actions are used to improve qualities like maintainability and
reproducibility?
\item What specific actions are taken to achieve the qualities of usability,
traceability, modifiability, maintainability, correctness, understandability,
unambiguity, reproducibility and visibility/transparency?
\item How does software designated as high quality by this methodology compare
  with top rated software by the community?
\end{inparaenum} 

The methodology depends on the engagement of a Domain Expert.  The Domain
Expert's role is to ensure that the assessment is consistent with the culture of
the community of practitioners in the domain.  The Domain Expert also has an
important role to play with the domain analysis.  For each domain we 
conduct a domain analysis to look at the commonalities, variabilities and
parameters of variation, for the family of software in the domain.  The domain
analysis means that software can be compared not just based on its quality, but
also based on its functionality.

The methodology follows a systematic procedure that begins with identifying the
domain and ends with answering the research questions posed above.  In between
we collect an authoritative list of about 30 software packages.  For each
package in the list we fill in our measurement template.  The template consists
of repository related data (like number of open issues, number of lines of code,
etc.) and 108 measures/questions related to 9 qualities: installability,
correctness/verifiability, reliability, robustness, usability, maintainability,
reusability, understandability and visibility/transparency. Filling in the
template requires installing the software, running simple tests (like completing
the getting started instructions (if present)), and searching the code,
documentation and test files.

The data for each domain is used to rank the software package according to each
quality dimension using AHP.  The ranking is not intended to identify a single
best software package.  Instead the ranking is intended to provide insights on
the top set of software for each quality.  The top performers can be contrasted
with the lesser performers to gain insight into what practices in the domain are
working.  Deeper insight can be obtained by combining this data with the
interview data from asking each recruited developer 20 questions.

Combining the quantitative data from the measurement template with the interview
results, along with the domain experts knowledge, we can determine the current
state of the practice for domain X.  Using our methodology, spreadsheet
templates and AHP tool, we estimate (based on our experience with using the
process) the time to complete an assessment for a given domain at 173 person
hours.

With the wealth of data from assessing the state of practice for multiple
domains, the next step is a meta-analysis.  We would look at how the different
domains compare. What lessons from one domain could be applied in other domains?
What (if any) differences exist in the pain points between domains?  Are there
differences in the tools, processes, and documentation between domains?

The current methodology is constrained by limited resources.  A 4 hour cap on
the measurement time for each software package limits what can be assessed.
Within this limit, we can't measure some important qualities, like usability and
modifiability.  In the future, we propose a more time-consuming process that
would capture these other quality measures.  To improve the feasibility, the
more time consuming measurements would not have to be completed for all 30
packages. Instead, a short list could be identified using the output of the AHP
ranking to select the top projects, or to select a sample of interesting
projects across the quality spectrum.

\subsubsection{Acknowledgements} Please place your acknowledgments at
the end of the paper, preceded by an unnumbered run-in heading (i.e.
3rd-level heading).

%
% ---- Bibliography ----
%
% BibTeX users should specify bibliography style 'splncs04'.
% References will then be sorted and formatted in the correct style.
%
\bibliographystyle{splncs04}
\bibliography{DiggingDeeper}
%
% \begin{thebibliography}{8}
% \bibitem{ref_article1}
% Author, F.: Article title. Journal \textbf{2}(5), 99--110 (2016)

% \bibitem{ref_lncs1}
% Author, F., Author, S.: Title of a proceedings paper. In: Editor,
% F., Editor, S. (eds.) CONFERENCE 2016, LNCS, vol. 9999, pp. 1--13.
% Springer, Heidelberg (2016). \doi{10.10007/1234567890}

% \bibitem{ref_book1}
% Author, F., Author, S., Author, T.: Book title. 2nd edn. Publisher,
% Location (1999)

% \bibitem{ref_proc1}
% Author, A.-B.: Contribution title. In: 9th International Proceedings
% on Proceedings, pp. 1--2. Publisher, Location (2010)

% \bibitem{ref_url1}
% LNCS Homepage, \url{http://www.springer.com/lncs}. Last accessed 4
% Oct 2017
% \end{thebibliography}
\end{document}