\documentclass[12pt]{article}

\usepackage[round]{natbib}
\usepackage{booktabs}

\usepackage{hyperref}
\hypersetup{
    colorlinks=true,       % false: boxed links; true: colored links
    linkcolor=red,          % color of internal links (change box color with linkbordercolor)
    citecolor=blue,       % color of links to bibliography
    filecolor=magenta,   % color of file links
    urlcolor=cyan           % color of external links
}

%% Comments
\newif\ifcomments\commentstrue

\ifcomments
\newcommand{\authornote}[3]{\textcolor{#1}{[#3 ---#2]}}
\newcommand{\todo}[1]{\textcolor{red}{[TODO: #1]}}
\else
\newcommand{\authornote}[3]{}
\newcommand{\todo}[1]{}
\fi

\newcommand{\wss}[1]{\authornote{blue}{SS}{#1}} %Spencer Smith
\newcommand{\jc}[1]{\authornote{red}{JC}{#1}} %Jacques Carette
\newcommand{\oo}[1]{\authornote{magenta}{OO}{#1}} %Olu Owojaiye
\newcommand{\pmi}[1]{\authornote{green}{PM}{#1}} %Peter Michalski
\newcommand{\ad}[1]{\authornote{cyan}{AD}{#1}} %Ao Dong

%\oddsidemargin 0mm
%\evensidemargin 0mm
%\textwidth 160mm
%\textheight 200mm

\begin{document}

\title{Empirical Study of Scientific Computer Software: Objectives and Research Questions} 
\author{Spencer Smith}
\date{\today}
	
%\maketitle

\subsection* {Overall Objective}

\textbf{What is needed to produce \{software + artifacts\} sustainably?}\\

To fully explain this research objective, we will decompose each constituent
part in turn.  In the above objective, \emph{software} consists of either
programs, which run (execute) on a computer, or libraries, which provide
services to be used by programs.  Both programs and libraries are created using
computer code, written in a programming language.  Many developers place most,
if not all, of their emphasis on the computer code that makes up programs and
libraries.  However, we consider the outcome of the development process to be
more than just software and the associated computer code; we consider the
products of the development process to be \{software + artifacts\}.  The
\emph{artifacts} include code, but also all the other potential work products,
such as documentation, build scripts, test cases, contributor's guides, etc.
From this point onward, we will use the term \emph{softifacts} when we are
referencing the combination of software and the associated artifacts.  As we
will explain below, assessing the quality of sustainability is not possible by
only considering the computer code; it depends on all of the artifacts, and on
the processes and methodologies used during software development.  \wss{Can we
  consider software as both the computer code and the executable (program or
  library), or should the term software just mean the executable?  That is, when
  we mean code, should we be careful to not instead say software?}

We define sustainable softifacts, as follows: \emph{Sustainable softifacts
  satisfy, with a reasonable amount of effort, the software requirements for the
  present, while also being maintainable, reusable and reproducible for the
  future.}  This definition is a specialization of the general definition of
sustainable development from \citet{Brundtland1987}, as discussed in
Section~\ref{Sec_Context}.  The notion of ``reasonable effort'' is part of the
definition of sustainability, but reasonable cannot be completely defined
because what is reasonable is going to depend on the context.  Problems that are
large, complex and/or critical are going to have a greater ``budget'' of time
and energy.  When comparing different alternatives for processes and
methodologies, preference would be given to the alternative that costs less
effort, assuming that this alternative achieves the same (or a higher) level of
sustainability as its competition.  Achieving a reasonable effort means that
sustainability depends on the productivity the adopted development and
maintenance process.  The assumptions listed in the next section
(Section~\ref{Sec_Context}) outline the typical space of problems where it
sustainable softifacts are necessary, and what problem constraints suggest the
possibility of a reasonable level of effort.

The overall objectives asks what is needed for produce softifacts sustainably.
We consider the \emph{what} to be the principles, processes and methodologies
used for software development.  We exclude tools from this list because,
although tools are needed to support processes and methodologies, tools change
rapidly.  Our goal is to be general.  If the right principles, processes and
methodologies are found, the tools will adapt to match.  However, understanding
the current state of the practice will involve looking at tools, as shown in the
later list of research questions (Section~\ref{Sec_ResearchQuestions}).  As
mentioned above, for sustainability to be achieved, the productivity of the
adopted process must be considered.

what else do we need - knowledge - knowledge captured in softifacts.  - overlap
plus duplication - required and desired, but still want to be productive.  To be
productive, want to say once and view as needed.  Suggests generation - maybe
put this elsewhere?

The objective is not phrased to target a specific software domain, but one of
the research assumptions (Section~\ref{Sec_Context}) is that we will focus on
Research Software (RS) examples and applications.  This focus is because
creating sustainable softifacts is more feasible when the software domain is
well understood, which is the case for RS.  Moreover, RS is an important,
often neglected, domain of software.

Traditional approach struggles.  Try generative.  Does generative supply what is
needed?  Is there an ideal process?  - what is needed to produce softifacts
sustainably motivates our ideal process (sequence of choices).

Further information on the relevant definitions and context is given in the next
section.  Roadmap.

\subsection*{Context, Definitions and Assumptions} \label{Sec_Context}

The overall objective is too ambiguous as given.  It requires additional
information/interpretation.  Therefore this section provides definitions of the
relevant terms and a list of the assumptions that define the potential contexts
where sustainable softifacts development is reasonable.

\subsubsection*{Definitions}

\begin{description}
\item[Knowledge] Facts, definitions, theories, assumptions, etc. \wss{needs a
    proper definition, with citation(s)}

``Awareness or familiarity gained by experience of a fact or situation.'' \citep{OxfordKnowledge2020}
\ad{Definition from Lexico.com (Oxford) dictionary}

\item[Process] Organization of the software development activities.  \wss{needs a
    proper definition, with citation(s)}

``A series of actions or steps taken in order to achieve a particular end.''\citep{OxfordProcess2020}
\ad{Definition from Lexico.com (Oxford) dictionary}

\item[Principles] Principles are ``general and abstract statements describing
  desirable properties of software processes and products.''  \citep[p.\
  41]{GhezziEtAl2003}
\item[Methods] ``Methods are general guidelines that govern the execution of some activity;
  they are rigorous, systematic, and disciplined approaches.''  \citep[p.\
  41]{GhezziEtAl2003}
\item[Techniques] ``Techniques are more technical and mechanical than methods.''
  \cite[p.\ 41]{GhezziEtAl2003}
\item[Methodology] Combination of methods and techniques.  \citep[p.\
  41]{GhezziEtAl2003}
\item[Tools] ``Tools ... are developed to support the application of techniques,
  method and methodologies.'' \cite[p.\ 41]{GhezziEtAl2003}

\item[Scientific Computing] the use of computational tools to analyze or
  simulate (continuous) mathematical models of real world systems of engineering
  or scientific importance so that we can better understand and (potentially)
  predict the system's behaviour. \citep{SmithAndLai2005}
\item[Sustainable] We start with the general definition:

``Sustainable development is development that meets the needs
  of the present without compromising the ability of future generations to meet
  their own needs.'' \citep{Brundtland1987}

  This definition is general; it is not specific to software and it is silent on
  the specific needs of the present and the future.  The definition can be made
  less abstract by introducing the context of softifacts and thinking
  separately about the needs of the present and the future.  For the needs of
  the present, the software should meet its requirements specification.
  \wss{highlight correctness? reliability?}  For the
  needs of the future, given uncertainty, this means supporting change from
  relevant concerns (political, economic, social, technical, legal and
  environmental) should be possible with a reasonable level of resources.
  Supporting the future then means producing softifacts that are
  maintainable, reusable and reproducible.  

  Summarizing the above, the proposed definition for sustainability is:

  \emph{Sustainable softifacts satisfy, for a reasonable amount of energy, the
    software requirements for the present, while also being maintainable,
    reusable and reproducible for the future.}

This definition has brought in several other qualities.  These qualities are
defined in the ``Quality  Definitions of Qualities'' document.
\wss{Reproducibility above might be changed to replicability, or possibly both
  terms should be included?}
\wss{Should performance be included in the list of qualities?}
\ad{Almost every SCS needs to be correct, reliable and usable, but many of them
may not need to provide edge performance. How do we choose which qualities to be
included here?}

\item[Software] Application and/or library.
\item[Application (App)] ``Software designed to fulfill specific needs of a
  user.'' \citep{IEEEStdGlossarySET1990}

``Application software is a program or group of programs designed for end users.
These programs are divided into two classes: system software and application
software. While system software consists of low-level programs that interact
with computers at a basic level, application software resides above system
software and includes applications such as database programs, word processors
and spreadsheets. Application software may be bundled with system software or
published alone.''\citep{TechopediaApplication2018} \ad{Definition from
Techopedia.com}

``A program or piece of software designed and written to fulfill a particular
purpose of the user.''\citep{OxfordApplication2020}\ad{Definition from
Lexico.com (Oxford) dictionary}

\item[Library] Services that are available to other programs, but not an
  executable program itself. \wss{needs a
    proper definition, with citation(s)}

``A software library is a suite of data and programming code that is used to
develop software programs and applications. It is designed to assist both the
programmer and the programming language compiler in building and executing
software.''\citep{TechopediaLibrary2016}\ad{Definition from Techopedia.com}

``A collection of programs and software packages made generally available, often
loaded and stored on disk for immediate
use.''\citep{OxfordLibrary2020}\ad{Definition from Lexico.com (Oxford)
dictionary}

\item[Artifacts] Work products generated during the process of creating
  software.  The work products include requirements documentation, design
  documentation, verification and validation plans, verification and validation
  reports, contributor's guides, user guides, build scripts, code, test cases,
  etc.
\end{description}

\subsubsection*{Assumptions}

Producing sustainable softifacts is not a trivial undertaking.  Not
every project needs to aim for sustainability.  Our position is that the time
and energy is justified when one or more of the following assumptions applies.

- number or name assumptions

\begin{enumerate}
\item The software project is going to be long lived, where long life means at
  least 10 years.
\item The softifacts should interest multiple stakeholders (not just the
  original developer), with different interests and backgrounds.
\item The software is safety relevant, such as software for nuclear safety,
  medical imaging or computational medicine.
\item \wss{this one is about feasibility - separate this out} The software explicitly, or implicitly, is part of a program family of
  related software.  When the program family assumption means that there are
  related programs that are less effort to design and build together than to
  design and build as separate projects.  Program family development depends on
  satisfying three hypotheses \citep{Weiss1997}:
\begin{itemize}
\item Redevelopment hypothesis – most software development involved in producing
  an individual family members should be redevelopment because the family
  members have so much in common.
\item Oracle Hypothesis – the changes that are likely to occur during the soft-
  ware’s \ad{software's} lifetime are predictable.
\item Organizational Hypothesis – designers can organize the software and the
  development effort so that predicted changes can be made independently.
\end{itemize}
\item The project recognizes that software artifacts other than the code are
  relevant.
\item The domain of the software project is well understood.  This assumption is
  related to the oracle hypothesis. \wss{related to feasibility}
\end{enumerate}

- program family comes up to make it economically feasible, not necessary, but
important (JC gave example of space probe - not a family, but still makes sense
to do it right.)
- sustainable - part of reasonable effort
Although this study is not specifically restricted to open-source software, for
practical reasons (especially the lack of access to the code), commercial
software will not be strongly emphasized.

\subsubsection*{Current State of the Practice Research Questions} \label{Sec_ResearchQuestions}

A good starting point for understanding how to develop sustainable SCS softifacts is
to look at how SCS is currently developed.  By measuring relevant qualities and
relating them to development practices, we can form an idea of what is currently
working well.

Following the key points of the overall objective, the research questions are
based on investigating the following: i) knowledge, ii) principles, processes
and methodologies, iii) software qualities and iv) the necessary investment of
time and energy.  The scope of the research questions are not trying to cover
everything, but rather to focus on the most important aspects for SCS.

Define what is meant by meaningful (what criteria need to be satisfied), since
this term is used several times in the research questions.

\begin{enumerate}
\item What knowledge is currently captured in the artifacts generated by
  existing open source SCS projects?
\item What principles, processes, methodologies and tools are used by existing
  open source SCS projects?
\item What are the ``pain points'' for developers working on SCS projects?  What
  aspects of the existing processes, methodologies and tools do they consider
  could potentially be improved?
\item For a given software product (softifacts), how can we produce a
  reasonable measure of the software qualities (correctness, reliability,
  usability etc.) with a few hours of effort?
\item For a given software product (softifacts), how can we produce a
  meaningful measure of usability with a few days worth of effort?
\item For a given software product (softifacts), how can we produce a
  meaningful measure of maintainability (with respect to likely changes) with a
  few days worth of effort?
\item For a given software product (softifacts), how can we produce a
  meaningful measure of reproducibility/replicability with a few days worth of
  effort?
\item With a reasonable effort and time, how can we compare the productivity
  between different processes, methodologies and tools?
\item Can a correlation be identified between software projects that use best 
  practices for the development process and methodologies and improvement in the
  qualities of usability, performance and reproducibility/replicability?
\end{enumerate}

\subsubsection*{Ideal Process}

Some empirical research studies of SCS stop at the previous step.  They aim to
identify what is currently working well, and then implicitly assume that the
identified process/methodologies/tools provide the best guidelines for others to
follow.  However, identifying the best current approach to developing SCS does
not imply that the best possible approach has been found.  Our instinct is that
there is plenty of room for improvement with SCS development.  The following
research questions explore this possibility.

\begin{enumerate}
\item Ignoring the time and effort required, what knowledge is necessary to
  capture for sustainable SCS?  How can this knowledge best be documented in
  software artifacts?
\item How can Model Driven Engineering and Code Generation be fit in the
  development process for SCS to remove as many repetitive, tedious and error
  prone tasks as possible?  We will call this the ``ideal process.''
\item How well do existing modelling and code generation tools do at
  implementing the ``ideal process''?  Candidates for comparison include Drasil
  and Epsilon.
\item How well does a model driven approach to compared to a traditional
  approach for usability, productivity and sustainability?
\item What are the attitudes and preferences of typical users towards
  an implementation of the ``ideal process''?
\end{enumerate}

\bibliographystyle{plainnat}
\bibliography {../CommonFiles/ResearchProposal}

\end{document}

NOTES

- measuring process aimed at 'producing softifacts sustainably'. part of it could be measuring effort.